\documentclass[9pt,xcolor={dvipsnames}]{beamer}
\usetheme{Boadilla}

\makeatother
\setbeamertemplate{footline}
{
	\leavevmode%
	\hbox{%
		\begin{beamercolorbox}[wd=.4\paperwidth,ht=2.25ex,dp=1ex,center]{author in head/foot}%
			\usebeamerfont{author in head/foot}\insertshortauthor
		\end{beamercolorbox}%
		\begin{beamercolorbox}[wd=.6\paperwidth,ht=2.25ex,dp=1ex,center]{title in head/foot}%
			\usebeamerfont{title in head/foot}\insertshorttitle\hspace*{3em}
			\insertframenumber{} / \inserttotalframenumber\hspace*{1ex}
	\end{beamercolorbox}}%
	\vskip0pt%
}
\makeatletter
\setbeamertemplate{navigation symbols}{}


\usepackage{lipsum}
\usepackage{appendixnumberbeamer}

\usepackage[authoryear]{natbib}
\usepackage[latin1]{inputenc}
\usepackage[T1]{fontenc}
\usepackage{caption}
\usepackage{amsmath, amssymb}
\usepackage{epstopdf}
\usepackage{graphicx}
\usepackage{lmodern}
%\usepackage[dvipsnames]{xcolor}
\usepackage{xpatch}
\usepackage{multirow}
\usepackage{tikz}

\usepackage{amsmath,theorem,amssymb,graphicx, pgfplots, tabularx, placeins}
\usepackage{dsfont}
\usepackage{caption}
%\usepackage{subcaption}
%\usepackage{subcaption}
\setbeamertemplate{caption}{\raggedright\insertcaption\par}
%\setbeamertemplate{footline}[frame number]
\usepackage{csquotes}
\usepackage{bm}
\bibliographystyle{econometrica}
\usepackage[normalem]{ulem}
\usepackage{setspace}


\definecolor{gray(x11gray)}{rgb}{0.75, 0.75, 0.75}


\newcommand{\bit}{\begin{itemize}}
	\newcommand{\eit}{\end{itemize}}
\newcommand{\ben}{\begin{enumerate}}
	\newcommand{\een}{\end{enumerate}}

\newcommand{\bc}{\color{blue}}
\newcommand{\black}{\color{black}}
\newcommand{\rc}{\color{red}}
\newcommand{\gc}{\color{ForestGreen}}
\newcommand{\wc}{\color{white}}

\newcommand{\lb}{\label}
\newcommand{\re}{\eqref}

\title[RBC: investment dynamics]{Macroeconomics II, Lecture III:\\
	  RBC: Investment Dynamics}
\author{Erik {\"O}berg}
\date{}

\begin{document}

\begin{frame}
\maketitle
\end{frame}

\section{Introduction}


\begin{frame}{Last time}

\bit
\setlength\itemsep{1.5em} 

\item We worked harder on confronting the RBC model with data

\item Learned the importance of generating a large fluctuations in efficiency and labor wedge $\Rightarrow$ led us to consider
\bit
\setlength\itemsep{0.5em} 
\item Variable capacity utilization

\item Extensive-margin models of labor supply

\item GHH preferences/rigid wage contracts
\eit

\item Investment wedge, however, small - does that mean basic RBC model has a fully satisfactory theory of investment?
\bit
\setlength\itemsep{0.5em} 
	\item BC accounting is just one (although a very nice one) measure of empirical fit
	
	\item The basic RBC model had too little persistence, and one might think this has to do with the very jumpy response of investment
\eit

\item Today, we'll dig deeper into the theory of investment

\eit

\end{frame}

%\begin{frame}{Is investment important}
%
%\bit
%\setlength\itemsep{1.5em} 
%
%\item Before we start, we should ask: Is investment at all important to the RBC transmission mechanism?
%
%\item ff
%
%\item Easy to find out: take our vanilla RBC model and set $\alpha=0$ in production function $F(K, N)=K^{\alpha}N^{1-\alpha}$
%\bit
%	\item Capital is useless $\Rightarrow$ investment is always 0
%\eit
%\eit
%
%\end{frame}
%
%\begin{frame}{IRFs to TFP shock with $\alpha=0$}
%
%\bit
%\setlength\itemsep{1.5em} 
%
%\item ff
%\eit
%
%\end{frame}
%
%\begin{frame}{Comments}
%
%\bit
%\setlength\itemsep{1.5em} 
%
%\item ff
%\eit
%
%\end{frame}


\begin{frame}{Agenda}

\ben
\setlength\itemsep{1.5em} 

\item RBC setup with firm ownership of capital

\item Neoclassical theory vs. Q theory of investment

\een

\end{frame}

\begin{frame}

\begin{center}
	\huge RBC setup with firm ownership of capital \normalfont
\end{center}

\end{frame}

\begin{frame}{An alternative, but equivalent, setup}

\bit
\setlength\itemsep{1.5em} 

\item In the basic RBC model we've studied, household owned and rented out the capital stock to the firm
\bit
	\item Convenient because all dynamics of the model became encapsulated in household problem; firm problem was static
\eit

\item To get us started thinking about investment, let's consider a more realistic setup where firms own the capital stock

\item Households still own the firm equity, and therefore, indirectly, the firm capital stock

\item Because there are no frictions, we'll see that the two setups are equivalent
\eit

\end{frame}


\begin{frame}{Household problem}

\bit
\setlength\itemsep{1.5em} 

\item Program of the representative household 
\begin{eqnarray}
\max_{\{C_t, N_t, B_{t+1}\}} && E_O \sum_{t=0}^{\infty} \beta^t \left[U(C_t)- V(N_t) \right]\nonumber \\
\text{s.t} && C_t + Q_t B_{t+1} \leq W_t N_t + B_t + T_t \nonumber \\
&& C_t, N_t, B_{t+1} \geq 0 \nonumber
\end{eqnarray}

\item Note:
\bit
\setlength\itemsep{0.5em}
\item $T_t$ are firm profits transfered back to the household (=0 in equilibrium)

\item $Q_t$ is the price risk-free bonds that pay 1 unit of consumption goods in $t+1$ in terms of consumption goods in period $t$

\item $R_{t}=\frac{1}{Q_t}$ is the gross real return on bonds that pay in period $t+1$

\item In contrast to the $t+1$ return to capital investments in period $t$, $R^{r}_{t+1}$, $R_{t}$ is known in period $t$
\eit

\eit

\end{frame}

\begin{frame}{Household optimality conditions}

\bit
\setlength\itemsep{1.5em} 
\item Set up the Langrangian, take the F.O.C. to find
\begin{eqnarray}
U'(C_t)W_t &=& V'(N_t) \nonumber \\
U'(C_t) &=& \beta\frac{1}{Q_t} E_t U'(C_{t+1}) \nonumber 
\end{eqnarray}
or we can write the second equation as
\begin{eqnarray}
U'(C_t) &=& \beta R_{t} E_t U'(C_{t+1}) \nonumber 
\end{eqnarray}
\bit
 \item Note: in steady state $Q=\beta$ $\Rightarrow$ $R=\frac{1}{\beta}$
\eit

\item Contrast with the optimality conditions in household-ownership setup
\begin{eqnarray}
U'(C_t)W_t &=& V'(N_t) \nonumber \\
U'(C_t) &=& \beta E_t(R^{r}_{t+1}+(1-\delta)) U'(C_{t+1}) \nonumber 
\end{eqnarray}



\eit

\end{frame}

\begin{frame}{Asset pricing implications}

\bit
\setlength\itemsep{1.5em}

\item The Euler equation is also an asset valuation equation:
\begin{eqnarray}
Q_t = \mathbb E_t \left[\frac{\beta U'(C_{t+1}) }{U'(C_{t}) }\right] \nonumber
\end{eqnarray}

\item Define $Q_{t,t+s}$ as 
\begin{eqnarray}
Q_{t, t+s} &=& Q_t \times Q_{t+1} \times  .... \times Q_{t+s-1} \nonumber\\
 &=& \mathbb E_t \left[\frac{\beta U'(C_{t+1}) }{U'(C_{t}) }\right] \times \mathbb E_{t+1} \left[\frac{\beta U'(C_{t+2}) }{U'(C_{t+1}) }\right] \times  .... \times \mathbb  E_{t+s-1} \left[\frac{\beta U'(C_{t+s}) }{U'(C_{t+s-1}) }\right] \nonumber\\
&=& \mathbb E_t M_{t,t+s} \nonumber
\end{eqnarray}
where 
\begin{eqnarray}
M_{t,t+s}\equiv \beta^s\frac{U'(C_{t+s}) }{U'(C_{t})} \nonumber
\end{eqnarray}

\eit

\end{frame}


\begin{frame}{Asset pricing implications}

\bit
\setlength\itemsep{1.5em}

\item We label $M_{t,t+s}$ the {\bc stochastic discount factor}

\item The SDF measures the households' willingness to forego consumption in period $t$ to have more consumption in a particular state in period $t+s$

\item In asset market equilibrium, $M_{t,t+s}$ prices assets that pays off in a particular state in period $t+s$

\item $\mathbb E M_{t,t+s}$ prices risk-free assets that pays off in period $t+s$

\item $M_{t,t+s}$ is the key object of interest in much of {\bc macro finance}

\eit

\end{frame}


\begin{frame}{Firm problem}

\bit
\setlength\itemsep{1em} 
\item A representative firm can choose investment and labor hirings, taking prices as given

\item It can finance investment using internal funds (=equity) or risk-free debt

\item The household owns the firm: firm therefore discounts future profits using household stochastic discount factor $M_{t,t+s}$

\item Program
\begin{eqnarray}
\max_{N_t, I_t, B_{t+1}, K_{t+1}} && \mathbb E_0 \sum_{t=0}^{\infty} M_{0, t} \left(A_t F(K_t, N_t) - W_t N_t - I_t + Q_tB_{t+1} - B_t \right) \nonumber \\
\text{s.t.} && K_{t+1} \leq I_t + (1-\delta) K_t \nonumber
\end{eqnarray}

\item Note: 
\begin{eqnarray}
V_0 = \sup \left[\mathbb E_0 \sum_{t=0}^{\infty} M_{0, t} \left(A_t F(K_t, N_t) - W_t N_t - I_t + Q_tB_{t+1} - B_t \right)\right] \nonumber
\end{eqnarray}
is the value of the firm in period 0
\bit
	\item Basic asset pricing result: value of firm $=$ discounted NPV of future cash flows
\eit


\eit

\end{frame}


\begin{frame}{Firm optimality conditions}

\bit
\setlength\itemsep{1.5em} 
\item Set up the Lagrangian, take the F.O.C. to find: {\rc (Do on whiteboard)}
\begin{eqnarray}
W_t &=& A_t F_N(K_t, N_t)  \\
q_t &=& 1  \\
q_t &=&  \mathbb E_t \left[M_{t,t+1}\left[A_{t+1}F_K(K_{t+1}, N_{t+1}) + (1-\delta)q_{t+1} \right] \right]  \\
Q_t &=& \mathbb E_t M_{t,t+1} 
\end{eqnarray}
where $q_t$ is the Lagrange multiplier on the firm constraint
 
\item Optimality conditions 2-3, together with definition of $M_{t,t+1}$, implies
\begin{eqnarray}
U'(C_t) &=& \beta E_t(R^{r}_{t+1}+(1-\delta)) U'(C_{t+1}) \nonumber 
\end{eqnarray}
where $R^{r}_{t+1}= A_{t+1}F_K(K_{t+1}, N_{t+1})$ - which we recognize!

\item Note: Equation (4) is satisfied whenever households' are optimizing, what does this mean? {\wc Miller-Modgliani (AER 1963): the capital structure of the firm is irrelevant if the household can trade in the same assets as the firm}

\eit

\end{frame}

\begin{frame}{Firm optimality conditions}

\bit
\setlength\itemsep{1.5em} 
\item Set up the Lagrangian, take the F.O.C. to find: {\rc (Do on whiteboard)}
\begin{eqnarray}
\setcounter{equation}{1}
W_t &=& A_t F_N(K_t, N_t)  \\
q_t &=& 1  \\
q_t &=&  \mathbb E_t \left[M_{t,t+1}\left[A_{t+1}F_K(K_{t+1}, N_{t+1}) + (1-\delta)q_{t+1} \right] \right]  \\
Q_t &=& \mathbb E_t M_{t,t+1} 
\end{eqnarray}
where $q_t$ is the Lagrange multiplier on the firm constraint

\item Optimality conditions 2-3, together with definition of $M_{t,t+1}$, implies
\begin{eqnarray}
U'(C_t) &=& \beta E_t(R^{r}_{t+1}+(1-\delta)) U'(C_{t+1}) \nonumber 
\end{eqnarray}
where $R^{r}_{t+1}= A_{t+1}F_K(K_{t+1}, N_{t+1})$ - which we recognize!

\item Note: Equation (4) is satisfied whenever households' are optimizing, what does this mean? {\bc Miller-Modgliani (AER 1963): the capital structure of the firm is irrelevant if the household can trade in the same assets as the firm}

\eit

\end{frame}



\begin{frame}{Equivalence}

\bit
\setlength\itemsep{1.5em} 


\item Combining this with household optimaility conditions and resource constraints, the equilibrium is characterized by
\begin{eqnarray}
\text{HH intertemporal optimality:} && {\bc U'(C_t) = \beta\frac{1}{Q_t} E_t U'(C_{t+1})} \nonumber \\
\text{HH intratemporal optimality:} && U'(C_t)W_t = V'(N_t) \nonumber \\
\text{Firm optimality 1:} && U'(C_t) = \beta E_t\left[(R^{r}_{t+1}+(1-\delta)) U'(C_{t+1})\right] \nonumber \\
\text{Resource constraint:} && C_t + I_t = A_t F(K_t, N_t) \nonumber \\
\text{Producation function:} && Y_t = A_t F(K_t, N_t) \nonumber \\
\text{Capital LOM:} && K_{t+1} = (1-\delta) K_t + I_t \nonumber \\
\text{Firm optimality 2:} && R^{r}_t = A_t F_k (K_t, N_t) \nonumber \\
\text{Firm optimality 3:} && W_t = A_t F_n (K_t, N_t) \nonumber \\
\text{TFP process:} && A_t = A_{t-1}^{\rho_a}exp(\epsilon_t) \nonumber
\end{eqnarray}

\item Which, apart, from first equation is exactly the same set of equations characterizing the RBC model with household ownership of capital
\bit
	\item One more unkown $Q_t$ - one more equation
\eit

\eit

\end{frame}

\begin{frame}{Firm optimality conditions: interpretation I}

\bit
\setlength\itemsep{1.5em} 

\item Let's go back to the investment decision - firm optimality conditions:
\begin{eqnarray}
W_t &=& A_t F_N(K_t, N_t) \nonumber \\
q_t &=& 1 \nonumber \\
q_t &=&  \mathbb E_t \left[M_{t,t+1}\left[A_{t+1}F_K(K_{t+1}, N_{t+1}) + (1-\delta)q_{t+1} \right] \right] \nonumber \\
Q_t &=& \mathbb E_t M_{t,t+1} \nonumber
\end{eqnarray}

\item How to interpret $q_t?$
\bit
\setlength\itemsep{0.5em} 
	\item Lagrange multiplier $=$ shadow value of relaxing constraint $=$ shadow value of having one more unit of installed capital $K_{t+1}$
	
	\item Supposed the firm has optimized and then, out of the sky, it gains some extra $\partial K_{t+1}$ - what will it do?
	
	\item Optimal choice of $K_{t+1}$ has not changed, so it just lowers investment by $\partial K_{t+1}$ and uses proceeds to increase current profits 
	
	\item Along the optimal path, we therefore have $q_t = \frac{\partial V_t}{\partial K_{t+1}}$ (recall {\bc envelope theorem})
	
%	\item This argument is an application of the {\bc envelope theorem}
	
	\item Implication 1:  $q_t$ is the price of capital in terms of goods - why?
	
	\item Implication 2:  $q_t=1$ - why?
	
%	\item Firm value in period $t$:
%	\begin{eqnarray}
%	V_t = E_t \sum_{s=0}^{\infty} Q_{t, t+s} \left(A_{t+s} F(K_{t+s}, N_{t+s}) - W_{t+s} N_{t+s} - (K_{t+s+1}-(1-\delta)K_{t+s}) \right)\nonumber
%\end{eqnarray}
%	Suppose the firm already picked $K_{t+1}$ in period $t$ and then, out of the sky, it gains some extra $K_{t+1}$ - what is the value of this extra capital?
%	\begin{eqnarray}
%	\frac{\partial V_t}{\partial K_{t+1}} = Q_t E_t \left[A_{t+1}F_K(K_{t+1}, N_{t+1}) + (1-\delta) \right] = q_t. \nonumber
%	\end{eqnarray}
	
\eit

\eit

\end{frame}


\begin{frame}{Firm optimality conditions: interpretation II}

\bit
\setlength\itemsep{2em} 

\item Firm optimality conditions:
\begin{eqnarray}
W_t &=& A_t F_N(K_t, N_t) \nonumber \\
q_t &=& 1 \nonumber \\
q_t &=&  \mathbb E_t \left[M_{t,t+1}\left[A_{t+1}F_K(K_{t+1}, N_{t+1}) + (1-\delta)q_{t+1} \right] \right] \nonumber \\
Q_t &=& \mathbb E_t M_{t,t+1} \nonumber
\end{eqnarray}

\item Optimality condition 3 can be iterated forward:
\begin{eqnarray}
q_t = \frac{1}{1-\delta} \mathbb E_t \sum_{s=1}^{\infty} M_{t,t+s}(1-\delta)^s \left(A_{t+s}F_K(K_{t+s}, N_{t+s})\right) \nonumber
\end{eqnarray}
\bit
\setlength\itemsep{0.5em} 
\item RHS $=$ marginal benefit of having one more unit of installed capital $K_{t+1}$

\item LHS $=$ price of having one more unit of installed capital
\eit

\eit

\end{frame}


\begin{frame}{Firm optimality conditions: interpretation III}

\bit
\setlength\itemsep{2em} 

\item Firm optimality conditions:
\begin{eqnarray}
W_t &=& A_t F_N(K_t, N_t) \nonumber \\
q_t &=& 1 \nonumber \\
q_t &=&  \mathbb E_t \left[M_{t,t+1}\left[A_{t+1}F_K(K_{t+1}, N_{t+1}) + (1-\delta)q_{t+1} \right] \right] \nonumber \\
Q_t &=& \mathbb E_t M_{t,t+1} \nonumber
\end{eqnarray}


\item With $q_t=1$, optimality condition 3 can be rewritten
\begin{eqnarray}
1  &=&  (1-\delta) \mathbb E_t M_{t,t+1} + \mathbb E_t M_{t,t+1} \mathbb E_t MPK_{t+1} + CoV(MPK_{t+1}, M_{t,t+1}) \nonumber 
\end{eqnarray}
where $MPK_{t+1} = A_{t+1}F_K(K_{t+1}, N_{t+1})$

\item To a first order, we thus have
\begin{eqnarray}
r_{t} + \delta = E_t A_{t+1}F_K(K_{t+1}, N_{t+1}) \nonumber
\end{eqnarray}
where 
\begin{eqnarray}
r_{t}=R_{t}-1=\frac{1}{Q_t}-1 \nonumber
\end{eqnarray}

\eit

\end{frame}



\begin{frame}{The neoclassical theory of investment}

\bit
\setlength\itemsep{1.5em}

\item The vanilla RBC model embeds the {\bc neoclassical theory of investment}
\bit
	\item Pioneered by Jorgenson (AER 1963); Hall-Jorgenson (AER 1967) 
\eit

\item Key idea: firms should invest until $E_t MPK_{t+1}$ equals {\bc user cost}

\item {\bc User cost} = alternative cost $r_{t}$ + direct cost $\delta$

\item Leaving the first-order approximation, the user cost also reflects investment risk (measured in terms of the covariance between the payoff and the discount factor)

\item Very intuitive, but this hinges on that the real price of capital goods is always $1$

\eit

\end{frame}


\begin{frame}{Equilibrium characterization including $q_t$}

\bit
\setlength\itemsep{1.5em} 

\item Adding the price of capital $q$ to the model means that we split the firm optimality condition 1 into two pieces:
\begin{eqnarray}
\text{HH intertemporal optimality:} && U'(C_t) = \beta\frac{1}{Q_t} E_t U'(C_{t+1}) \nonumber \\
\text{HH intratemporal optimality:} && U'(C_t)W_t = V'(N_t) \nonumber \\
\text{Firm optimality 1:} && {\bc q_t = E_t \frac{\beta U'(C_{t+1})}{U'(C_{t})} \left[R^r_{t+1} + (1-\delta)q_{t+1} \right]}  \nonumber \\ 
\text{Firm optimality 2:} && {\bc q_t = 1} \nonumber \\
\text{Resource constraint:} && C_t + I_t = A_t F(K_t, N_t) \nonumber \\
\text{Producation function:} && Y_t = A_t F(K_t, N_t) \nonumber \\
\text{Capital LOM:} && K_{t+1} = (1-\delta) K_t + I_t \nonumber \\
\text{Firm optimality 3:} && R^{r}_t = A_t F_k (K_t, N_t) \nonumber \\
\text{Firm optimality 4:} && W_t = A_t F_n (K_t, N_t) \nonumber \\
\text{TFP process:} && A_t = A_{t-1}^{\rho_a}exp(\epsilon_t) \nonumber
\end{eqnarray}

\item Not very interesting, but allows better comparison to the next model that we introduce

\eit

\end{frame}



\begin{frame}

\begin{center}
	\huge Neoclassical theory vs. Q theory of investment \normalfont
\end{center}

\end{frame}

\begin{frame}{The Q theory of investment}

\bit
\setlength\itemsep{1em}
\item The neoclassical theory of investment comes with the prediction that the price of capital is constant

\item This is very much at oods with the data

\item A more reasonable theory of investment has
\bit
	\item Fluctuations in the price of capital, and
	
	\item that fluctuations in the price of capital matter for investment decisions
\eit

\item Tobin (JMCB 1969): given fluctuations in the price of capital, a reasonable theory of investment would say: invest if  
\begin{eqnarray}
\frac{\text{Market value of firm capital}}{\text{Replacement cost of capital}} >1\nonumber 
\end{eqnarray}
\bit
\setlength\itemsep{0.5em}
	\item Th left-hand side ratio is called {\bc Tobin's Q}

	\item This idea has guided much empirical research on investment
\eit

\item Note, in the notation of our firm problem:
\begin{eqnarray}
\text{Tobin's Q} = \frac{V}{k} \hspace{2mm} \text{while} \hspace{2mm} q_t = \frac{\partial V_t}{\partial k_{t+1}} \nonumber
\end{eqnarray}

\item Refining Tobin's intuition: what ought to matter for an optimizing firm is the \emph{marginal value of firm capital}, i.e, $q_t$

\eit

\end{frame}


\begin{frame}{Operationalizing the Q theory of investment}

\bit
\setlength\itemsep{1em}

\item Some version of the (marginal) Q theory naturally comes out when adding investment costs to the firm problem

\item Such costs are also very plausible - think about installing a new machine, building a new plant etc.

\item Under some conditions of the investment cost function, the Q theory comes out exactly

\item Suppose that the firm faces investment costs of the form
\begin{eqnarray}
C = C(I_t, K_t) \nonumber
\end{eqnarray}

\item Q theory arises with the following assumptions
\ben
	\item $C(\cdot)$ is convex in investment size, i.e., $C_I(I_t, K_t) \geq 0 $, $C_{II}(I_t, K_t) \geq 0 $

	\item $C(\cdot)$ is homogeneous of degree 1
	
	\item $C(\delta K_t, K_t) = 0$
	
	\item $C_{I}(\delta K_t, K_t) = 0$

	\item $C_K(I_t, K_t) < 0 $
\een


\eit

\end{frame}


\begin{frame}{Firm problem}

\bit
\setlength\itemsep{2em}

\item Popular cost function that satisfies these assumptions
\begin{eqnarray}
C(I_t, K_t) = \frac{\phi}{2} \left(\frac{I_t}{K_t}-\delta\right)^2K_t \nonumber
\end{eqnarray}

\item Consider a firm problem that faces such a cost function
\begin{eqnarray}
\max_{N_t, I_t, K_{t+1}} && \mathbb E_0 \sum_{t=0}^{\infty} M_{0, t} \left(A_t F(K_t, N_t) - W_t N_t - I_t -C(I_t, K_t) \right) \nonumber \\
\text{s.t.} && K_{t+1} \leq I_t + (1-\delta) K_t \nonumber
\end{eqnarray}

\item I've taken out debt financing $B_{t+1}$ since it doesn't matter anyway

\item Note: if $\phi=0$ $\Rightarrow$ we're back to vanilla RBC

\eit

\end{frame}

\begin{frame}{Firm optimality conditions}

\bit
\setlength\itemsep{1.5em} 
\item Set up the Lagrangian, take the F.O.C. to find: 
\begin{eqnarray}
W_t &=& A_t F_N(K_t, N_t) \nonumber \\
q_t &=& 1+C_I(I_t, K_t) \nonumber \\
q_t &=& \mathbb E_t M_{t,t+1} \left[A_{t+1}F_K(K_{t+1}, N_{t+1}) - C_K(I_{t+1}, K_{t+1}) + (1-\delta)q_{t+1} \right] \nonumber  \nonumber
\end{eqnarray}
where $q_t$ is, again, the Lagrange multiplier on the firm constraint

\item Observations:
\bit
\setlength\itemsep{0.5em} 
\item $q_t \geq 1 $ - why?

%\item  It is still true that
%\begin{eqnarray}
%q_t = \frac{\partial V_t}{\partial K_{t+1}} \nonumber
%\end{eqnarray}
%Remember: $C_K(I_t, K_t) < 0 $ - why does $q_t$ fall if $K_{t+1}$ rise?

\item As before, we can iterate on third condition to find
\begin{eqnarray}
q_t = \frac{1}{1-\delta} \sum_{s=1}^{\infty} M_{t,t+s}(1-\delta)^s \left(A_{t+s}F_K(K_{t+s}, N_{t+s}) - C_K(I_{t+s}, K_{t+s})\right) \nonumber
\end{eqnarray}
\eit

\eit

\end{frame}


\begin{frame}{q and investment}

\bit
\setlength\itemsep{1.5em} 
\item With our functional form $C(I_t, K_t) = \frac{\phi}{2} \left(\frac{I_t}{K_t}-\delta\right)^2K_t$, optimality condition 2 becomes
\begin{eqnarray}
q_t &=& 1+C_I(I_t, K_t) \nonumber \\
&=& 1 + \phi\left(\frac{I_t}{K_t} - \delta\right) \nonumber
\end{eqnarray}
or
\begin{eqnarray}
\frac{I_t}{K_t} = \frac{1}{\phi}(q_t-1)+\delta \nonumber
\end{eqnarray}

\item Predictions: 
\ben
\setlength\itemsep{0.5em} 
\item investment rate $>1$ if $q_t>1$

\item $q_t$ is a {\bc sufficienct statistic} for investment

\een

%\item Smells like progress, but third prediction maybe seems a bit too sharp

\eit

\end{frame}


\begin{frame}{Taking the Q-theory to the data}

\bit
\setlength\itemsep{1em} 
\item In the data, it is easy to observe the \emph{average q} = $\frac{V}{K}$
\bit
\setlength\itemsep{0.5em} 
	\item $V$ could be stock market valuation of firm
	
	\item $K$ is the net worth on the firm balance sheet
\eit

\item The model tells us we should relate investment to \emph{marginal q}= $\frac{\partial V}{\partial K}$

\item Hayashi (Ecmtra, 1982): if both $C(\cdot)$ and $F(\cdot)$ are homogeneous of degree 1, then average q = marginal q
\bit
	\item Take-home exercise: show that this is true with our quadratic $C(\cdot)$-function and Cobb-Douglas $F(\cdot)$!
\eit

\item {\bc Hayashi's theorem} provides rationale for estimating regression
\begin{eqnarray}
\frac{I_{it}}{K_{it}} = \alpha + \beta (\text{Average Q}_{it}-1) + \sum \gamma_{k} X_{kit} + \epsilon_{it} \nonumber
\end{eqnarray}
using firm level micro data

\item A few key papers: Summers (BPEA, 1981); Fazzari-Hubbard-Petersen (BPEA, 1988); Cummins-Hassett-Hubbard (BPEA 1994); Kaplan-Zingales (QJE 1997)

\eit

\end{frame}

\begin{frame}{Cummins-Hassett-Hubbard (BPEA 1994): Compustat data 1962-1988}

\begin{figure}
	\centering
	\includegraphics[scale=0.45]{Figures/Hassett_table3.pdf}
\end{figure}

\bit
	\item Estimating the equation reduced-form tends to produce small coefficients (implying unreasonbly large adjustment costs)
	
	\item Treatment effect of firm cash flow is seemingly much larger 
\eit

\end{frame}

\begin{frame}{Cummins-Hassett-Hubbard (BPEA 1994): Compustat data 1962-1988}

\begin{figure}
	\centering
	\includegraphics[scale=0.4]{Figures/Hassett_table4.pdf}
\end{figure}

\bit
\item Using tax reforms with heterogeneous treatment effect as instrumental variable: estimates much more reasonable 

\item Still, financial variables, e.g., firm cash flow tend to show up as large and significant $\Rightarrow$ sufficient statistic hypothesis rejected

\item This motivates introducing \emph{financial frictions} in firm investment decisions

\eit

\end{frame}


\begin{frame}{Integrating the Q theory in our RBC model}

\bit
\setlength\itemsep{1.5em} 

\item Replacing the firm optimality conditions, and also the resource constraint in our equilibrium characterization, we have
%\begin{eqnarray}
%U'(C_t) &=& \beta\frac{1}{Q_t} E_t U'(C_{t+1}) \nonumber \\
%U'(C_t)W_t &=& V'(N_t) \nonumber \\
% q_t & =& Q_t E_t \left[R^r_{t+1} + (1-\delta)q_{t+1} \right] \nonumber  \nonumber \\
%{\bc q_t} &{\bc =}& {\bc 1+C_I(I_t, K_t)} \nonumber \\
%{\bc C_t + I_t + C(I_t, K_t)} &{\bc =}& {\bc A_t F(K_t, N_t)} \nonumber \\
%Y_t &=& A_t F(K_t, N_t) \nonumber \\
%K_{t+1} &=& (1-\delta) K_t + I_t \nonumber \\
%{\bc R^r_t} &{\bc =}& {\bc A_t F_k (K_t, N_t) - C_K(I_{t}, K_{t})} \nonumber \\
%W_t &=& A_t F_n (K_t, N_t) \nonumber \\
%A_t &=& A_{t-1}^{\rho_a}exp(\epsilon_t) \nonumber
%\end{eqnarray}
\begin{eqnarray}
\text{HH intertemporal optimality:} && U'(C_t) = \beta\frac{1}{Q_t} E_t U'(C_{t+1}) \nonumber \\
\text{HH intratemporal optimality:} && U'(C_t)W_t = V'(N_t) \nonumber \\
\text{Firm optimality 1:} && q_t = E_t \frac{\beta U'(C_{t+1})}{U'(C_{t})} \left[R^r_{t+1} + (1-\delta)q_{t+1} \right] \nonumber  \\ 
\text{Firm optimality 2:} && {\bc q_t = 1+C_I(I_t, K_t)} \nonumber \\
\text{Resource constraint:} && {\bc C_t + I_t + C(I_t, K_t) = A_t F(K_t, N_t)} \nonumber \\
\text{Producation function:} && Y_t = A_t F(K_t, N_t) \nonumber \\
\text{Capital LOM:} && K_{t+1} = (1-\delta) K_t + I_t \nonumber \\
\text{Firm optimality 3:} && {\bc R^{r}_t = A_t F_k (K_t, N_t) - C_K(I_{t}, K_{t})} \nonumber \\
\text{Firm optimality 4:} && W_t = A_t F_n (K_t, N_t) \nonumber \\
\text{TFP process:} && A_t = A_{t-1}^{\rho_a}exp(\epsilon_t) \nonumber
\end{eqnarray}

\item $\Rightarrow$ Log-linearize and Dynare it

\eit

\end{frame}


\begin{frame}{IRF to TFP shock}

\begin{figure}
	\centering
	\includegraphics[scale=0.40,trim= 0 120 0 120, clip]{Figures/rbc_irf_adjcost.pdf}
\end{figure}


\end{frame}


\begin{frame}{Mechanism}

\bit
\setlength\itemsep{1.5em} 

\item TFP up $\Rightarrow$ $q_t$ up $\Rightarrow$ investment up

\item With adjustment costs, large investment jumps are especially costly

\item Investment response therefore more smooth $\Rightarrow$ increases persistence
\bit
	\item To get persistence of both investment and overall GDP right, you typically want to include inbestment adjustment costs
\eit

\item Flip side: on impact, consumption jump larger, wage jump smaller $\Rightarrow$ labor supply response smaller
 

\eit

\end{frame}


\begin{frame}{Comment: Non-convex adjustment costs}

\bit
\setlength\itemsep{1em} 

\item Quadratic adjustment cost implies that investment dynamics is smooth

\item However, it is clearly seen in micro data that investment is \emph{lumpy}: sometimes you invest nothing, sometimes you invest a lot

\item Lumpy investment dynamics follow from \emph{non-convex adjustment costs}, e.g., fixed costs:
\begin{eqnarray}
C(I_t, K_t) = \left\{
\begin{array}{cc}
0 & \text{if } I_t = \delta K_t \\
\zeta & \text{otherwise} 
\end{array} \right. \nonumber
\end{eqnarray}

\item Non-convex decision problems $\rightarrow$ optimum cannot be solved with F.O.C.s, we need a computer to characterize firm problem

\item Models that seriously try to get micro-level dynamics right typically find that both convex and non-convex adjustment costs are needed, see, e.g., Ottonello-Winberry (Ecmtra 2020)

\item Type of cost matters a lot for some macro applications (e.g. uncertainaty shocks, see Bloom Ecmtra 2007), little for others

\eit

\end{frame}


\begin{frame}{Comment: Financial frictions}

\bit
\setlength\itemsep{1em} 

\item Micro evidence: firm financial variables (like cash flow) predict investment

\item Macro evidence: many severe crisis epsiodes linked to Financial shocks (e.g. the Great Depression and the Great Recession) 

\item Big literature on the effect of financial frictions for firm investment decisions

\item Applied macro reseach intertwined with micro-theory research: Financial frictions always originate from an {\bc agency problem}, i.e., for some reason, a creditor cannot trust that a debtor will repay his/her debt

\item Canonical models: Bernanke-Gertler (AER 1989); Kiyotaki-Moore (JPE 1997)

\item KM (1997) show that collaterized debt can overcome a problem of limited enforcement

\item A reduced-form collateral constraint in our firm problem (with within-period debt):
\begin{eqnarray}
\max_{N_t, I_t, K_{t+1}} && E_O \sum_{t=0}^{\infty} M_{0, t} \left(A_t F(K_t, N_t) - W_t N_t - I_t -C(I_t, K_t) \right) \nonumber \\
\text{s.t.} && K_{t+1} \leq I_t + (1-\delta) K_t \nonumber \\
&& I_t \leq \xi q_t K_{t} \nonumber
\end{eqnarray}


\eit

\end{frame}


\begin{frame}{Optimality}

\bit
\setlength\itemsep{1em}

\item Lagrangian
\begin{eqnarray}
L =  E_O \sum_{t=0}^{\infty}&& M_{0, t} \Big[ \left(A_t F(K_t, N_t) - W_t N_t - I_t -C(I_t, K_t) \right)  \nonumber\\
&&+ q_t\left(I_t + (1-\delta)K_t-K_{t+1} \right) + \mu_t \left(\xi q_t K_{t}-I_t\right)\Big] \nonumber
\end{eqnarray}

\item F.O.C.
\begin{eqnarray}
W_t &=& A_t F_N(K_t, N_t) \nonumber \\
q_t &=& 1+C_I(I_t, K_t)+\mu_t \nonumber \\
q_t &=& E_t M_{t,t+1}\left[A_{t+1}F_K(K_{t+1}, N_{t+1}) - C_K(I_{t+1}, K_{t+1}) + \left(\mu_{t+1} \xi  + (1-\delta)\right)q_{t+1} \right] \nonumber  \nonumber
\end{eqnarray}

\item Complementary slackness
\begin{eqnarray}
\mu_t &\geq& 0 \nonumber \\
I_t &=& \xi q_t K_{t} \hspace{2mm} \text{iff } \mu>0 \nonumber 
\end{eqnarray}
\bit
\item If constraint lax $\Rightarrow$ $\mu_t=0$, back to standard model 
\item It constraint binds $\Rightarrow$ $I_t = \xi q_t K_{t}$ and $\mu_t>0$
\eit

\item One can show that (see your problem set): constraint binds in steady state iff $\xi < \delta$
\eit

\end{frame}


\begin{frame}{IRF to TFP shock with $\phi=2$}

\begin{figure}
	\centering
	\includegraphics[scale=0.40,trim= 0 120 0 120, clip]{Figures/rbc_irf_finconstraint.pdf}
\end{figure}


\end{frame}



\begin{frame}{Summing up}

\bit
\setlength\itemsep{1.5em}

\item Without any market frictions - capital ownership doesn't matter for business cycle dynamics

\item Vanilla RBC predicts a constant price of capital

\item Investment adjustment costs predicts procyclical price of capital, and a ``Q theory of invesment''

\item This closes our investigation of the RBC approach to studying business cycle dynamics

\item Next up: Nominal rigidities and monetary policy (the ``New-Keynesian'' model)
\eit

\end{frame}




\begin{frame}{RBC: what have we not covered?}

\bit
\setlength\itemsep{1.5em}
\item Other shocks
\bit
\setlength\itemsep{0.5em}
\item Investment-specific shocks: see, e.g., Greenwood-Hersowitz-Krusell (EER 2000)

\item Uncertainty shocks: see, e.g., Bloom (Ecmtra 2007)

\item News shocks: see, e.g., Beaudry-Portier (AER 2006; Koslyk 2023)
\eit

\item International business cycle models
\bit
\setlength\itemsep{0.5em}
\item see, e.g., Backus-Kehoe-Kydland (JPE 1992); Baxter (HBintecon 1995); Schmitt-Grohe-Uribe (Book 2017)
\eit

\eit

\end{frame}



\end{document}















