\documentclass[9pt]{beamer}
\usetheme{Boadilla}

\makeatother
\setbeamertemplate{footline}
{
	\leavevmode%
	\hbox{%
		\begin{beamercolorbox}[wd=.4\paperwidth,ht=2.25ex,dp=1ex,center]{author in head/foot}%
			\usebeamerfont{author in head/foot}\insertshortauthor
		\end{beamercolorbox}%
		\begin{beamercolorbox}[wd=.6\paperwidth,ht=2.25ex,dp=1ex,center]{title in head/foot}%
			\usebeamerfont{title in head/foot}\insertshorttitle\hspace*{3em}
			\insertframenumber{} / \inserttotalframenumber\hspace*{1ex}
	\end{beamercolorbox}}%
	\vskip0pt%
}
\makeatletter
\setbeamertemplate{navigation symbols}{}

\usepackage{tikz}
\usetikzlibrary{positioning}

\usepackage{lipsum}
\usepackage{appendixnumberbeamer}

\usepackage[authoryear]{natbib}
\usepackage[latin1]{inputenc}
\usepackage[T1]{fontenc}
\usepackage{caption}
\usepackage{amsmath, amssymb}
\usepackage{epstopdf}
\usepackage{graphicx}
\usepackage{lmodern}
\usepackage{xcolor}
\usepackage{xpatch}
\usepackage{multirow}

\usepackage{amsmath,theorem,amssymb,graphicx, pgfplots, tabularx, placeins}
\usepackage{dsfont}
\usepackage{caption}
%\usepackage{subcaption}
%\usepackage{subcaption}
\setbeamertemplate{caption}{\raggedright\insertcaption\par}
%\setbeamertemplate{footline}[frame number]
\usepackage{csquotes}
\usepackage{bm}
\bibliographystyle{econometrica}
\usepackage[normalem]{ulem}

\usepackage{setspace}


\definecolor{gray(x11gray)}{rgb}{0.75, 0.75, 0.75}


\newcommand{\bit}{\begin{itemize}}
	\newcommand{\eit}{\end{itemize}}
\newcommand{\ben}{\begin{enumerate}}
	\newcommand{\een}{\end{enumerate}}

\newcommand{\bc}{\color{blue}}
\newcommand{\rc}{\color{red}}


\newcommand{\lb}{\label}
\newcommand{\re}{\eqref}

\title[The Buffer-Stock Savings Model]{Macroeconomics II, Lecture XII:\\
		The Buffer-Stock Savings Model}
\author{Erik {\"O}berg}
\date{}

\begin{document}

\maketitle

\begin{frame}{Introduction}

\bit
\setlength\itemsep{2em}

\item Last lecture: theoretical implications of uninsurable income risk for consumption-savings dynamics
\ben
\setlength\itemsep{1em}
\item With incomplete markets, ex-ante homogeneous households will be \bf ex-post heterogeneous \normalfont in terms of $\{C,A,Y\}$ $\Rightarrow$ no aggregation into representative agent

\item With incomplete markets, consumption dynamics influenced by a \bf precautionary savings \normalfont motive
\een

\item Today: using incomplete-markets consumption-savings theory for quantitative empirical analysis
\bit
\setlength\itemsep{0.5em}
	\item Research program pioneered by Deaton, Zeldes and Carroll 
	
	\item Builds on foundational work by Friedman, Modigliani, Hall and others
\eit

\eit

\end{frame}


\begin{frame}{Agenda }

\ben
\setlength\itemsep{1.5em}

\item Formulating and solving the canonical buffer-stock savings model
\bit
\setlength\itemsep{0.5em}
	\item Recursive formulation
	\item Solution algorithm
	\item Calibration
\eit

\item Consumption-savings dynamics in the buffer-stock savings model
\bit
\setlength\itemsep{0.5em}
	\item Consumption-savings dynamics without income risk
	\item Consumption-savings dynamics with income risk
\eit

\item 2 famous applications
\ben
\setlength\itemsep{0.5em}
	\item Blundell-Preston-Pistaferri (AER 2008): 

	\item Gourinchas-Parker (Ecmtra 2002): Consumption over the Life Cycle

	\item Kaplan-Violante (Ecmtra 2014): A Model of the Consumption Response to Fiscal Stimulus Payments
\een

\een

\end{frame}

\begin{frame}{The canonical buffer-stock savings model}

\bit
\setlength\itemsep{1em}

\item Buffer-stock savings model $=$ income-fluctuations problem with persistent shocks
\bit
	\item Key papers: Zeldes (JPE 1989; QJE 1989), Deaton (Ecmta 1991, Book 1992), Carroll (QJE 1997)
\eit

\item Household problem:
\begin{eqnarray}
\max_{C_t, A_{t+1}} && E_0 \sum_{t=0}^{\infty} \beta^t \frac{C_t^{1-\sigma}}{1-\sigma} \nonumber \\
\text{s.t.} && C_t + A_{t+1} = Y_t + R A_t \nonumber \\
&& Y_{t} = P_{t} e^{\epsilon_{t}} \nonumber \\
&& P_{t} = P_{t-1}G e^{\nu_{t}} \nonumber \\
&& A_{t+1} \geq - \underbar A \nonumber \\
&& C_{t} \geq 0 \nonumber
\end{eqnarray}

\item Terminology:
\bit
	\item $P_t$ --- permanent income
	\item $\nu_t$ --- permanent income shocks
	\item $\epsilon_t$ --- transitory income shocks
\eit

\item We assume that $\epsilon_{t}, \nu_t$ are known at the time of choosing $C_t, A_{t+1}$

\eit

\end{frame}


\begin{frame}{Comments}

\bit
\setlength\itemsep{1.5em}

\item $G$ is a constant for simplicity
\bit
\setlength\itemsep{0.5em}
	\item It can be any deterministic function
	
	\item In many applications, it is a function of age, education, ability etc.
\eit

\item Why a permanent-transitory formulation of the income process?
\bit
\setlength\itemsep{0.5em}
\item In the data, we see that some changes to residualized income are very persistent, whereas others are very short-lived

\item Permanent-transitory formulation provides a parsimonious parameterization of household income process that captures these features

\item For some applications, estimating the persistence of income shocks can be important
\bit
	\item E.g., business-cycle applications (typically quarterly models) with unemployment shocks
\eit
\eit

\item CRRA utility: reasonable baseline + big tractability gains
\bit
\item allows us to normalize the problem w.r.t. to permanent income
\eit

\eit

\end{frame}


\begin{frame}{Recursive formulation}

\bit
\setlength\itemsep{1.5em}

\item The model cannot be solved analytically

\item To solve the problem, and to investigate the properties of the solution, we need to recast the problem on its recursive form

\item Introduce cash on hand $M_t = Y_t + RA_t$

\item Recursive formulation
\begin{eqnarray}
V({\color{white}M, P}) &=& \max_{C, A'} U(C) + \beta E V({\color{white}M', P'}) \nonumber \\
&\text{s.t.}& A'=M-C \nonumber \\
&& M' = P'e^{\epsilon'} + RA' \nonumber \\
&& P' = GPe^{\nu'} \nonumber \\
&& A'\geq -\underbar A \nonumber \\
&& C\geq 0 \nonumber
\end{eqnarray}

\item Which are the state variables?
\bit
\item What information, known at time $t$, is useful for the household in choosing $C, A'$ and compute $U(C)+\beta E V'()$?
\eit

\eit

\end{frame}

\begin{frame}{Recursive formulation}

\bit
\setlength\itemsep{1.5em}

\item The model cannot be solved analytically

\item To solve the problem, and to investigate the properties of the solution, we need to recast the problem on its recursive form

\item Introduce cash on hand $M_t = Y_t + RA_t$

\item Recursive formulation
\begin{eqnarray}
V({\color{black}M, P}) &=& \max_{C, A'} U(C) + \beta E V({\color{black}M', P'}) \nonumber \\
&\text{s.t.}& A'=M-C \nonumber \\
&& M' = P'e^{\epsilon'} + RA' \nonumber \\
&& P' = GPe^{\nu'} \nonumber \\
&& A'\geq -\underbar A \nonumber \\
&& C\geq 0 \nonumber
\end{eqnarray}

\item Which are the state variables?
\bit
\item What information, known at time $t$, is useful for the household in choosing $C, A'$ and compute $U(C)+\beta E V'()$?
\eit

\eit

\end{frame}

\begin{frame}{Normalization w.r.t. permanent income}

\bit
\setlength\itemsep{1.5em}

\item It appears that the state variables are cash on hand $M$ and permanent income $P$

\item With $U(C)=\frac{C^{1-\sigma}}{1-\sigma}$ being CRRA, permanent income is actually not a state variable

\item Define
\bit
\setlength\itemsep{0.3em}
	\item $m=\frac{M}{P}$
	\item $c=\frac{C}{P}$
	\item $a'=\frac{A'}{P}$
	\item $\underbar a=\frac{\underbar A}{P}$
	\item $v(M, P)=\frac{V(M, P)}{P^{1-\sigma}}$
\eit


\eit

\end{frame}

\begin{frame}{Normalization w.r.t. permanent income II}

\bit
\setlength\itemsep{1.5em}

\item Household problem
\begin{eqnarray}
V(M, P) &=& \max_{C, A'} \frac{C^{1-\sigma}}{1-\sigma} + \beta E V(M', P') \nonumber \\
&\text{s.t.}& A'=M-C \nonumber \\
&& M' = P'e^{\epsilon'} + RA' \nonumber \\
&& P' = GPe^{\nu'} \nonumber \\
&& A'\geq -\underbar A \nonumber \\
&& C\geq 0 \nonumber
\end{eqnarray}

\item Using our definitions
\begin{eqnarray}
P^{1-\sigma}v(M, P) &=& \max_{c, a'} \frac{(cP)^{1-\sigma}}{1-\sigma} + \beta E P'^{1-\sigma} v(M', P') \nonumber \\
&\text{s.t.}& a'P = mP - cP \nonumber \\
&& m'P' = P'e^{\epsilon'} + Ra'P \nonumber \\
&& P' = GPe^{\nu'} \nonumber \\
&& a'P \geq -\underbar a P \nonumber \\
&& c P' \geq 0 \nonumber
\end{eqnarray}

\eit

\end{frame}

\begin{frame}{Normalization w.r.t. permanent income III}

\bit
\setlength\itemsep{0.5em}

\item Dividing through
\begin{eqnarray}
v(M, P) &=& \max_{c, a'} \frac{c^{1-\sigma}}{1-\sigma} + \beta E \left(\frac{P'}{P}\right)^{1-\sigma} v(M', P') \nonumber \\
&\text{s.t.}& a' = m - c \nonumber \\
&& m' = e^{\epsilon'} + Ra'\frac{P}{P'} \nonumber \\
&& \frac{P'}{P} = Ge^{\nu'} \nonumber \\
&& a' \geq -\underbar a \nonumber \\
&& c \geq 0 \nonumber
\end{eqnarray}

\item Substituting the law-of-motion for permanent income:
\begin{eqnarray}
v(M, P) &=& \max_{c, a'} \frac{c^{1-\sigma}}{1-\sigma} + \beta E \left(Ge^{\nu'}\right)^{1-\sigma} v(M', P') \nonumber \\
&\text{s.t.}& a' = m - c \nonumber \\
&& m' = e^{\epsilon'} + \frac{Ra'}{Ge^{\nu'}} \nonumber \\
&& a' \geq -\underbar a \nonumber \\
&& c \geq 0 \nonumber
\end{eqnarray}

\eit

\end{frame}

\begin{frame}{Normalized recursive program}

\bit
\setlength\itemsep{0.5em}

\item Normalized household problem:
\begin{eqnarray}
v(M, P) &=& \max_{c, a'} \frac{c^{1-\sigma}}{1-\sigma} + \beta E_t \left(Ge^{\nu'}\right)^{1-\sigma} v(M', P') \nonumber \\
&\text{s.t.}& a' = m - c \nonumber \\
&& m' = e^{\epsilon'} + \frac{Ra'}{Ge^{\nu'}} \nonumber \\
&& a' \geq -\underbar a \nonumber \\
&& c \geq 0 \nonumber
\end{eqnarray}

\item Which are the state variables?

\eit

\end{frame}

\begin{frame}{Normalized recursive program}

\bit
\setlength\itemsep{0.5em}

\item Normalized household problem:
\begin{eqnarray}
v({\color{black}m}) &=& \max_{c, a'} \frac{c^{1-\sigma}}{1-\sigma} + \beta E_t \left(Ge^{\nu'}\right)^{1-\sigma} v({\color{black}m'}) \nonumber \\
&\text{s.t.}& a' = m - c \nonumber \\
&& m' = e^{\epsilon'} + \frac{Ra'}{Ge^{\nu'}} \nonumber \\
&& a' \geq -\underbar a \nonumber \\
&& c \geq 0 \nonumber
\end{eqnarray}

\item Which are the state variables?

\eit

\end{frame}

\begin{frame}{What did we just learn?}

\bit
\setlength\itemsep{1.5em}

\item The sufficient state variable is $m = \frac{M}{P}$


\item Economics:
\bit
\setlength\itemsep{0.5em}
	\item Decision functions in normalized problem: $c=c(m), a'=a'(m)$
	
	\item Un-normalized decision functions $C=Pc(m)$, $A'=Pa'(m)$
	
	\item Implication: A permanent-income rich household will consume the same amount of an increase in his normalized cash-on-hand as permanent-income poor household  	
\eit


\item Computations:
\bit
\setlength\itemsep{0.5em}
\item We have reduced a two-dimensional function equation to a one-dimensional functional equation

\item Big computational gain of dimension reduction when solving the problem numerically
\eit

\item Key assumptions: CRRA utility (or, more generally, a power-function utility) and linear constraints


\eit

\end{frame}

\begin{frame}{Properties of solution}

\bit
\setlength\itemsep{1.5em}

\item Solution given by a consumption function $c(m)$ and a savings function $a'(m)=m-c(m)$

\item As usual, an interior solution $c(m)$ must satisfy the Euler equation
\begin{align*}
c^{-\sigma} = \beta R E \left(\left(Ge^{\nu'}c'\right)^{-\sigma}\right)
\end{align*}

\item Else, the credit constraint is binding and $c(m)=m+\underbar a$

\item Given some parametric restrictions, the Bellman equation defines a contraction mapping, and we can solve the equation using value function iteration

%\item Just as in the simplified problem considered in previous class, one can show that there is a cutoff point $m^*$, below of which the constraint always binds, above of which it never binds

%\item Two parametric restrictions in order to have a well-defined problem
%\bit
%\setlength\itemsep{0.5em}
%	\item Impatience condition: sufficiently low $\beta R$ to ensure that assets converge assymptotically (remember last lecture...)
%	\bit
%		\item Sufficient condition: $\beta R E\left[(Ge^{\nu})^{-\sigma}\right]<1$ (Deaton, Ecmtra 1991)
%	\eit
%	
%	\item $R>G$ to ensure that NPV of human wealth is finite
%	\bit
%		\item Otherwise, solution is unbounded without income risk
%	\eit
%\eit
\eit


\end{frame}

%
%\begin{frame}{Solution algorithm}
%
%\bit
%\setlength\itemsep{1em}
%\item Given some parametric restrictions, the Bellman equation defines a contraction mapping, and we can solve the equation using value function iteration
%
%%\item Note: in each iteration, you loop over $N$ states. With one more state variable with $M$-size grid, you must loop over $M\times N$ states ({\bc ``curse of dimensionality''})
%
%\eit
%
%\end{frame}


\begin{frame}{Value function iteration}

\bit
\setlength\itemsep{1em}
\item Construct a discrete grid $\hat m = \{\hat m_1, \hat m_2, ..., \hat m_N\}$

\item Guess a discrete value function $\hat v^{0} = \{\hat v^{0}_1, \hat v^{0}_2, ..., \hat v^{0}_N\}$

\item For each $\hat m_k$, $k=1,...,N$, solve the decision problem
\begin{eqnarray}
\max_{c, a'} && \frac{c^{1-\sigma}}{1-\sigma} + \beta \sum_{i} \sum_{j} \pi(\eta_i) \pi(\epsilon_j) \left(Ge^{\nu_i'}\right)^{1-\sigma} \hat v^0\left(e^{\epsilon_j} + \frac{R a'}{Ge^{\nu_i}}\right) \nonumber \\
\text{s.t. } && a' = \hat m_k - c \nonumber \\
&& a' \geq -\underbar a \nonumber \\
&& c \geq 0 \nonumber
\end{eqnarray}
using a standard numerical maximization method (e.g. Newton-Raphson)

\item If $\hat m_k<e^{\epsilon_j} + \frac{R a'}{Ge^{\nu_i}}<\hat m_{k+1}$ for some $k<N$ $\Rightarrow$ interpolate $v^0_k, v^0_{k+1}$ to compute $\hat v^0\left(e^{\epsilon_j} + \frac{R a'}{Ge^{\nu_i}}\right)$

\item The maximum of the objective function is your new guess $\hat v^1_k$

\item Repeat until convergence
\eit

%\item Note: in each iteration, you loop over $N$ states. With one more state variable with $M$-size grid, you must loop over $M\times N$ states ({\bc ``curse of dimensionality''})
%
%\eit

\end{frame}

%\begin{frame}{VFI: comment}
%
%\bit
%\setlength\itemsep{1em}
%\item {\bc ``Curse of dimensionality''}: In each iteration, you loop over $N$ states. With one more state variable with $M$-size grid, you must loop over $N\times M$ states
%
%\eit
%
%\end{frame}


\begin{frame}{Parameterization}

\bit
\setlength\itemsep{1em}

\item Parameters: $\beta, \sigma, G, R, \underbar a$ and the distribution of $\epsilon', \nu'$

\item How to select parameter values?

\item $G$ and $R$ can be directly observed in the data

\item Data at hand: panels of income and consumption
\bit
\setlength\itemsep{0.5em}
\item Survey data, e.g., PSID (US)
\item Register data, e.g., Swedish income and wealth registers
\eit

\item The distribution of $\epsilon', \nu'$ can be estimated using panel data on household income


\item Calibrate $\beta, \sigma, \underbar a$ typically done by method of moments
\bit
\setlength\itemsep{0.5em}
\item Given preferences, our model defines a mapping from an income process to an allocation of assets and consumption at the household level

\item Populate an economy with, say 1000 households, and simulate the economy using the stochastic shock processes and the decision functions retrieved from solving the household problem

\item Set parameters so implied model moments fit data moments
\bit
\item Example moments: mean asset holdings, share with negative assets, consumption-income profiles...
\eit
\eit
\eit

\end{frame}

\begin{frame}{Income process estimation}

\bit
\setlength\itemsep{0.5em}

\item Standard practice: parametric assumption + method of moments

\item In logs ($\tilde x = \log X$), our process is
\begin{eqnarray}
& Y_{t} = P_{t} e^{\epsilon_{t}}, & y_{t} = p_{t} + \epsilon_{t}  \nonumber \\
& P_{t} = P_{t-1}G e^{\nu_{t}}, &  p_{t} =  p_{t-1} +  g + \nu_{t} \nonumber 
\end{eqnarray}
which give us
\begin{eqnarray}
&& \Delta  y_{t} =  g + \nu_{t}  + \epsilon_{t}-\epsilon_{t-1} \nonumber \\
&& \Delta  y_{t-1} =  g + \nu_{t-1}  + \epsilon_{t-1}-\epsilon_{t-2} \nonumber 
\end{eqnarray}
which means that
\begin{eqnarray}
Var(\Delta y_{t}) = Var(\nu_t) + 2Var(\epsilon_{t}) \nonumber \\
CoV(\Delta y_{t}, \Delta \tilde y_{t-1}) = -Var(\epsilon_{t}) \nonumber 
\end{eqnarray}

\item Typically, we apply this estimator to income growth residuals in household panel data

\item This method can be generalized in several dimensions (and recent admin data has taught us a lot!):
\bit
	\item Guvenen-Karahan-Ozkan-Song (Ecmtra 2021): income growth residuals exhibits fat tails and significant skewness (normality is a bad assumption)

	\item Carter Braxton-Herkenhoff-Rothbaum-Schmidt (AER 2025): permanent earnings risk has evolved differently across the skill distribution over time
	
	\item Harmenberg-Lizarraga (2025): a generalized square root process capture earnings dynamics at the top of the distribution very well
\eit


 

\eit

\end{frame}
%
%
%\begin{frame}{Calibration}
%
%\bit
%\setlength\itemsep{1em}
%
%\item To calibrate $\beta, \sigma, \underbar a$, we could either use external evidence (e.g. experiments), {\bc estimation} or {\bc calibration}
%
%\item Estimation and calibration usually done by the method of moments:
%\bit
%	\setlength\itemsep{0.5em}
%	\item Given preferences, our model defines a mapping from an income process to an allocation of assets and consumption at the household level
%
%	\item Populate an economy with, say 1000 households, and simulate the economy using the stochastic shock processes and the decision functions retrieved from solving the household problem
%	
%	\item Set parameters so implied model moments fit data moments
%		\bit
%			\item Example moments: mean asset holdings, share with negative assets, consumption-income profiles...
%		\eit
%\eit
%
%\item In practice, there are many more moments than parameters $\Rightarrow$ the model is {\bc overidentified}
%
%\item Choice of calibration moments depends on question of interest
%%
%%\item If outcome variables does not align well with external evidence (e.g. experiments) and common sense, we should be worried
%
%\item We will talk more about this when looking at the applied papers
%
%\eit
%
%\end{frame}

\begin{frame}{Consumption-savings dynamics in the buffer-stock savings model}

\bit
\setlength\itemsep{1.5em}

\item Now we know how to solve and parameterize our model

\item Next step: look at how households behave in the model

\item Recall:
\bit
\setlength\itemsep{0.5em}
	\item state variable: $m=\frac{M}{P} = \frac{Y+RA}{P}$
	
	\item law of motion: $m'=e^{\epsilon'} + \frac{Ra'(m)}{G e^{\nu'}}$
	
	\item decisions: $c=c(m)$ and $a'=a'(m) =m-c(m)$
	
	\item unnormalized decisions: $C=c(m)P$ and $A' = a'(m) P$
\eit

\item Defining features of the model: uninsurable income risk and potentially binding credit constraint

\item To understand the implications of these features, we first remind ourselves about consumption-savings dynamics in a frictionless perfect-foresight model

\eit

\end{frame}

\begin{frame}{Perfect-foresight model}

\bit
\setlength\itemsep{1.5em}

\item Set $\epsilon=\nu=0$ and $\underbar a = \underbar a^{nbl}$, where $\underbar a^{nbl}$ is the natural borrowing limit
%\bit
%\setlength\itemsep{0.5em}
%\item $\underbar a^{nlb}$ ensures that all borrowing is risk-free 
%\item $\underbar a^{nlb}$ determined by the condition where the household needs to set $c=0$ for all future periods to pay off its debt
%\item Therefore, the credit constraint will never bind in the solution to the household problem
%\item See problem set
%\eit

\item Perfect-foresight Bellman equation:
\begin{eqnarray}
v({\color{black}m}) &=& \max_{c, a'} \frac{c^{1-\sigma}}{1-\sigma} + \beta G^{1-\sigma} v({\color{black}m'}) \nonumber \\
&\text{s.t.}& a' = m - c \nonumber \\
&& m' = 1 + \frac{Ra'}{G} \nonumber \\
&& a' \geq -\underbar a^{nlb} \nonumber \\
&& c \geq 0 \nonumber
\end{eqnarray}

\eit
\end{frame}


\begin{frame}{Perfect-foresight model}

\bit
\setlength\itemsep{1.5em}

\item As we will iterate on the budget constraint, I now reintroduce time subscripts

\item As $\underbar a = \underbar a^{nbl}$, we will always have an interior solution

\item Interior solution satisfies
\begin{align*}
c_t^{-\sigma} = \beta R \left(Gc_{t+1}\right)^{-\sigma}
\end{align*}

\item For simplicity, set $\sigma$ so that $\left(\beta R\right)^{-\frac{1}{\sigma}} G=1$ $\Rightarrow$
\begin{align*}
c_t = c_{t+1}
\end{align*}


\eit
\end{frame}

\begin{frame}{Perfect-foresight model: solution}

\bit
\setlength\itemsep{1.5em}

\item To solve for $c=c(m)$, iterate on the budget constraint and the law-of-motion for $m$:
\begin{eqnarray}
&& a_{t+1} = m_t - c_t \nonumber \\
&& m_{t+1} = 1 + \frac{Ra_{t+1}}{G} \nonumber 
\end{eqnarray}
which gives us 
\begin{eqnarray}
&& m_{t+1} = 1 + \frac{R}{G}(m_t-c_t) \nonumber \\
\Rightarrow &&  m_t = c_t - \frac{G}{R} + \frac{G}{R}m_{t+1}	\nonumber \\
\Rightarrow &&  m_t = c_t - \frac{G}{R} + \frac{G}{R}\left(c_{t+1} - \frac{G}{R} + \frac{G}{R}m_{t+2} \right)	\nonumber \\
...&& 	\nonumber \\
\Rightarrow &&  m_t = \sum_{k=0}^{\infty} \left(\frac{G}{R}\right)^k c_{t+k} - \sum_{k=1}^{\infty} \left(\frac{G}{R}\right)^k  + \lim\limits_{T \to \infty} \left(\frac{G}{R}\right)^T m_{t+T} \nonumber \\
\Rightarrow &&  m_t = c_t \frac{1}{1-\frac{G}{R}} - \frac{G}{R} \frac{1}{1-\frac{G}{R}} \nonumber
\end{eqnarray}
where we have used that $c_{t+k}=c_t$ for all $k$ and the transversality condition $\lim\limits_{T \to \infty} \left(\frac{G}{R}\right)^T m_{t+T}=0$


\eit
\end{frame}


\begin{frame}{Perfect-foresight model: solution II}

\bit
\setlength\itemsep{1.5em}

\item Therefore, we get
\begin{align*}
c_t = \Bigg(1-\underbrace{\frac{G}{R}}_{\text{eff. disc. rate}}\Bigg)\Bigg(\underbrace{m_t}_{\text{c.o.h.}} + \underbrace{\frac{G}{R} \frac{1}{1-\frac{G}{R}}}_{\text{NPV future norm. inc.}} \Bigg) \nonumber
\end{align*}

\item $\Rightarrow$ household behave according to permanent-income hypothesis (PIH)
\eit
\end{frame}

%\begin{frame}{Perfect-foresight model: solution II}
%
%\bit
%\setlength\itemsep{1.5em}
%
%\item Using the equations on the previous slide, we have that
%\begin{eqnarray}
%c_t && = m_t - a_{t+1}  \nonumber \\
%&& = m_t - \frac{G}{R}(m_{t+1}-1) \nonumber \\
%&& = m_t  +\frac{G}{R} - \frac{G}{R} \left(c_{t+1} \frac{1}{1-\frac{G}{R}} - \frac{1}{1-\frac{G}{R}} \right) \nonumber \\
%&& = m_t  +\frac{G}{R} - \frac{G}{R} \left(c_{t} \frac{1}{1-\frac{G}{R}} - \frac{1}{1-\frac{G}{R}} \right) \nonumber
%\end{eqnarray}
%and solving for $c_t$, we get
%\begin{align*}
%c_t = \Bigg(1-\underbrace{\frac{G}{R}}_{\text{eff. disc. rate}}\Bigg)\Bigg(\underbrace{m_t}_{\text{c.o.h.}} + \frac{G}{R}\underbrace{\left(1+ \frac{1}{1-\frac{G}{R}}\right)}_{\text{NPV future income}} \Bigg) \nonumber
%\end{align*}
%
%\item $\Rightarrow$ household behave according to permanent-income hypothesis (PIH)
%\eit
%\end{frame}

\begin{frame}{Perfect-foresight model: MPC out of transitory income shocks}

\bit
\setlength\itemsep{1em}

\item Consider a marginal change in $M$, holding $P$ constant:
\bit
	\item I.e., consider a transitory shock to current income
\eit
\begin{align*}
\frac{\partial C}{\partial M} &= \frac{\partial \left(c(m)P\right)}{\partial \left(mP\right)} \\
&= \frac{P \partial \left(c(m)\right)}{P \partial m}  \\
&= \frac{\partial c(m)}{ \partial m}  
\end{align*}
which gives us
\begin{align*}
\frac{\partial C}{\partial M} = \left(1-\frac{G}{R}\right)
\end{align*}


\item Insights:
\bit
\setlength\itemsep{0.5em}
	\item Marginal propensity to consume (MPC) out of current income is constant
	\item For reasonable parameters: MPC out of current income is very small
\eit





\eit
\end{frame}


%\begin{frame}{Perfect-foresight model: MPC out of permanent income shocks}
%
%
%\bit
%\setlength\itemsep{1em}
%
%\item Consider marginal change in permanent income P:
%\begin{align*}
%\frac{\partial C}{\partial P} &= \frac{\partial \left(c(m)P\right)}{\partial P} \\
%&=  c(m)+ P\frac{\partial c(m)}{\partial m} \frac{\partial (M/P)}{\partial P}    \\
%&=  c(m)+ P\frac{\partial c(m)}{\partial m} \left[\frac{1}{P}\frac{\partial M}{\partial P} +M\frac{\partial (1/P)}{\partial P} \right]    \\
%&=  c(m)+ P\frac{\partial c(m)}{\partial m} \left[\frac{1}{P} -M\frac{1}{ P^2} \right]    \\
%&=  c(m)+ \frac{\partial c(m)}{\partial m} \left[1 -m \right]
%\end{align*}
%which gives us
%\begin{align*}
%\frac{\partial C}{\partial P} &= \left(1-\frac{G}{R}\right)\left(m+\frac{G}{R}\frac{1}{1-\frac{G}{R}}\right) + \left(1-\frac{G}{R}\right)(1-m) = 1
%\end{align*}
%
%\item Insight:
%\bit
%	\item MPC out of permanent income is $=$ 1
%\eit
%
%\item Now: lets look at what happens in the buffer-stock savings model
%
%\eit
%\end{frame}


\begin{frame}{Consumption dynamics in the buffer-stock savings model}

\begin{figure}
	\centering
	\includegraphics[scale=0.3, trim = 0cm 1.7cm 0cm 2cm, clip]{figures/consumption_function.pdf}
\end{figure}

\bit
	\item Parameter values: $\beta = 0.95$. $\sigma = 1.5$, $R=1.04$, $G=1.03$, $\underbar a = 0$, $\sigma_{\epsilon}, \sigma_{\nu}$ taken from Gourinchas-Parker (Ecmtra, 2002)
	\item Code available by email
\eit

\end{frame}

\begin{frame}{Consumption dynamics in the buffer-stock savings model}

\begin{figure}
	\centering
	\includegraphics[scale=0.3, trim = 0cm 1.7cm 0cm 2cm, clip]{figures/consumption_function.pdf}
\end{figure}

\bit
	\item Result 1: with income risk, households consume less than if no income risk
		\bit
			\item with income risk, having assets have insurance value
			\item without income risk, household run down current assets quickly due to impatience
		\eit
\eit

\end{frame}

\begin{frame}{Consumption dynamics in the buffer-stock savings model}

\begin{figure}
	\centering
	\includegraphics[scale=0.3, trim = 0cm 1.7cm 0cm 2cm, clip]{figures/consumption_function.pdf}
\end{figure}

\bit
\item Result 2: with income risk, consumption function is concave
\bit
	\item Equivalently: with income risk, MPC out of current income is larger and decreasing
	\item As household approaches constraint, MPC out of current income grows to $1$
	\item With a lot of assets, households behaves as if no income risk and $MPC \to \left(1-\frac{G}{R}\right)$
\eit
\eit

\end{frame}

%\begin{frame}{Consumption dynamics in the buffer-stock savings model}
%
%\begin{figure}
%	\centering
%	\includegraphics[scale=0.3, trim = 0cm 1.7cm 0cm 2cm, clip]{figures/consumption_function.pdf}
%\end{figure}
%
%\bit
%\item Result 3: With income risk, MPC out of permanent income is lower for $m>1$
%\bit
%	\item MPC out of permanent income (with $\epsilon=0$) $= c(m)+ \frac{\partial c(m)}{\partial m} \left[1 -m \right]$
%	\item With income risk, $c(m)$ is lower and $\frac{\partial c(m)}{\partial m}$ is larger 
%\eit
%\eit
%
%\end{frame}


\begin{frame}{Consumption dynamics in the buffer-stock savings model}

\begin{figure}
	\centering
	\includegraphics[scale=0.3, trim = 0cm 1.7cm 0cm 2cm, clip]{figures/consumption_function.pdf}
\end{figure}

\bit
\item Result 3: because of concavity, there exists a target buffer stock of assets
	\bit
		\item To insure themselves, households seek to hold a certain amount of money relative to their permanent income
		\item With a lot of assets, households dissave
	\eit
\eit

\end{frame}


\begin{frame}{Using the buffer-stock savings model}

\bit
\setlength\itemsep{1.5em}
\item Now we understand the basic properties of the buffer-stock savings model

\item Move on to analyze how to use to model to interpret the data

\item Early literature (Zeldes, Deaton, Carroll and others) focused on testing buffer-stock model against PIH model (description does not fit all papers)
\bit
\setlength\itemsep{0.5em}
	\item Perhaps ex post unsurprisingly, PIH was rejected in several dimensions
	
	\item In late 80's/early 90's, numerical solutions were still difficult $\Rightarrow$ calibrated/estimated models were not used much
\eit

\item We will focus on second-generation literature

\item Two influential applications:
\ben
\setlength\itemsep{0.5em}
	\item Gourinchas-Parker (Ecmtra 2002): Consumption over the Life Cycle
	
	\item Blundell-Pistaferri-Preston (AER 2008): Consumption Inequality and Partial Insurance
	
	\item Kaplan-Violante (Ecmtra 2014): A Model of the Consumption Response to Fiscal Stimulus Payments
\een

\item I use my notation and will make some simplifications when discussing these papers


\eit

\end{frame}


\begin{frame}{Gourinchas-Parker: overview}

\bit
\setlength\itemsep{1.5em}
\item Q: how does consumption-savings dynamics evolve over the life-cycle?
\bit
	\item Does buffer-stock savings behavior or permament-income hypothesis provide better fit of the data? Does it depend on age?
\eit

\item Method: Estimate structural life-cycle model with idiosyncratic income risk
\bit
\setlength\itemsep{0.5em}
	\item Estimate income processes in PSID data 
	
	\item Construct consumption-income age profiles using CEX data
	
	\item Estimate model to match profiles
	
	\item Decompose savings behavior over the life cycle using estimated model
\eit

\eit

\end{frame}


\begin{frame}{Gourinchas-Parker: Model}

\bit
\setlength\itemsep{1.5em}
\item Life-cycle version of the model we have described:
\begin{eqnarray}
\max_{C_t, A_{t+1}} && E_0 \sum_{t=0}^{T} \beta^t \frac{C_t^{1-\sigma}}{1-\sigma} \nonumber \\
\text{s.t.} && C_t + A_{t+1} = Y_t + R A_t \nonumber \\
&& A_{t+1} \geq 0 \nonumber \\
&& C_{t} \geq 0 \nonumber
\end{eqnarray}

\item For $0 \leq t \leq N$, households are working and income process is
\begin{eqnarray}
Y_{t} &=& P_{t} e^{\epsilon_{t}} \nonumber \\
P_{t} &=& P_{t-1}G_t e^{\nu_{t}} \nonumber 
\end{eqnarray}
note that $G_t$ has $t$ subscript

\item For $N+1 \leq t \leq T$, households are retired and $Y_t$ is deterministic

\eit

\end{frame}

\begin{frame}{Gourinchas-Parker: Calibration}

\bit
\setlength\itemsep{1em}
\item $R=1.034$ taken from average return on safe bond assets

\item For other parameters, use micro data
\bit
	\item PSID (panel with rich info on income but small sample) 
	\item CEX (rich consumption data and larger sample, but weak panel dimension)
\eit

\item Income process:
\bit
\setlength\itemsep{0.5em}
	\item Estimate age-income profile $G_t$ by regressing income on age dummies with various controls
	
	\item Estimate permanent-transitory income shocks from income growth residuals

\eit 

\item Consumption-income profiles
\bit
\setlength\itemsep{0.5em}
%	\item At retirement, households behave according to permanent-income hypothesis and 
%	\begin{align*}
%		c_t = \gamma_0 + \gamma_1 m_t
%	\end{align*}
%	where $m_t$ is cash-on-hand, and $\gamma_0, \gamma_1$ are functions of $\beta, R, \sigma$
%	
%	\item The retirement rule can be estimated directly in the data

	\item Estimate consumption-income age profiles in the data
	
	\item Set $\beta$ and $\sigma$ so that model mimics (as close as possible) this profile
\eit

\item GP do this procedure for 16 different educational-occupational groups, but we only focus on the results for the total population here


\eit

\end{frame}


\begin{frame}{Gourinchas-Parker: Estimated consumption-income profile}

\begin{figure}
	\centering
	\includegraphics[scale=0.6]{figures/gourinchas_fig2.pdf}
\end{figure}

\bit
\setlength\itemsep{0.5em}
	\item Hump-shape in consumption directly rejects that working-age hosueholds have full insurance - why?
\eit

\end{frame}

\begin{frame}{Gourinchas-Parker: Matching model to estimated consumption-income profile}

\begin{figure}
	\centering
	\includegraphics[scale=0.5, trim = 0cm 1.5cm 0cm 0cm, clip]{figures/gourinchas_fig5.pdf}
\end{figure}

\bit
\setlength\itemsep{0.5em}
\item Matching consumption-income profile pins down $\beta$ and $\sigma$
\eit

\end{frame}


\begin{frame}{Gourinchas-Parker: Results}

\begin{figure}
\centering
\includegraphics[scale=0.5, trim = 0cm 1.5cm 0cm 0cm, clip]{figures/gourinchas_fig7.pdf}
\end{figure}

\bit
\item By turning on and off income risk in the model, GP decompose households savings behavior into its \emph{life-cycle} and \emph{precautionary} components
\eit

\end{frame}

\begin{frame}{Gourinchas-Parker: Results}

\bit
\setlength\itemsep{2em}
\item Young households are buffer-stock savers, old households are similar to PIH savers 

\item Young households: start with no buffer and and retirement is far away $\Rightarrow$ saving primarily reflect precautionary behavior

\item Middle-age housholds: already have a buffer and retirement is near $\Rightarrow$ saving primarily reflect consumption-smoothing w.r.t. retirement

%\item This tells us where we should expect to find high MPC agents
%\bit
%	\item Why might this be important?
%\eit

\eit

\end{frame}


\begin{frame}{Gourinchas-Parker: Discussion}

\bit
\setlength\itemsep{2em}
\item Early paper estimating a consumption-savings model to match key moments in micro data

\item Large literature has followed

\item Two semi-recent (job market) papers:
\bit
\setlength\itemsep{1em}

\item Boar (ReStud 2021): Estimate OLG-version of the model using PSID data 
	\bit
		\item Finds that significant part of household buffers is explained by childrens' income risk
	\eit

\item Druedahl and Martinello (ReStat 2020): Use Swedish register data to estimate consumption response to exogenous bequest shocks 
\bit
\setlength\itemsep{0.5em}
	\item Find that households quickly consume these bequests $\Rightarrow$ suggests low $\beta$
	
	\item Still, these household typically have sizable wealth holdings.

 	\item Matching these moments suggests that risk aversion is much higher than previously estimated
\eit

\eit
\eit

\end{frame}


\begin{frame}{Blundell-Pisaterri-Preston (AER 2008): The question}

\begin{figure}
	\centering
	\includegraphics[scale=0.7, trim = 0cm 1cm 0cm 0cm, clip]{figures/blundell_fig1.pdf}
\end{figure}

\bit
\setlength\itemsep{0.5em}
\item In the 80's, US income inequality grew a lot
\item Consumption inequality also grew, but at lower pace and the gap between consumption and income inequality increased significantly in the late 80's/early 90's
\item What can explain this divergence?
\eit

\end{frame}

\begin{frame}{BPP: The idea}

\bit
\setlength\itemsep{1em}
\item Buffer-stock savings theory teaches us that consumption responds very differently to permanent vs. transitory income shocks

\item With micro panel data on consumption and income, we can estimate
\ben
	\item The evolution of permanent income risk/inequality
	\item The evolution of transitory income risk/inequality
	\item The evolution of the pass-through of permanent and transitory income to consumption (the extent of insurance)
\een

\item For the third part, BPP suggest to estimate the effect using a linear consumption rule
\bit
\setlength\itemsep{0.5em}
	\item People sometimes say this is a ``semi-structural'' approach, I don't know what ``semi'' means other than ``linear''
\eit

\item The paper makes two contributions
\ben
	\item Show how to impute consumption in PSID (panel data on income) using a food Engel curves
	\item Use consumption-augmented PSID to estimate to evolution of permanent/transitory income risk and pass-through the consumption
\een

\item I focus on the second contribution here

\eit

\end{frame}

\begin{frame}{BPP: Setup}

\bit
\setlength\itemsep{1em}
\item Suppose we have panel data on log consumption and log income $\{c_{it}, y_{it}\}$

\item First, take out deterministic component by regressing $c_{it}, y_{it}$ on various demographic variables

\item Second, assume residualized income evolves according to (slightly more general process in the paper, but it doesn't make any difference)
\begin{align*}
y_{it} &= p_{it} + \epsilon_{it} \\
p_{it} &= p_{it-1} + \nu_{it} 
%\epsilon_{it} &= \sum_{j=0}^{q} \theta_j \zeta_{i,t-q}
\end{align*}
%Note: $MA(q)$ instead of $MA(1)$ specification of transitory-income process

\item Third, assume consumption responds to shocks according to
\begin{align*}
\Delta c_{it} &= \phi_{t} \nu_{it} + \psi_{t} \epsilon_{it} + \xi_{it}
\end{align*}
\bit
\item Note 1: treatment effect is time-varying 

\item Note 2: this consumption rule is postulated, not derived
\eit
\eit

  \end{frame}

\begin{frame}{BPP: Strategy}

\bit
\setlength\itemsep{1.5em}
\item With the postulated consumption rule, we have
\begin{align*}
Var(\Delta c_{it}) &= \phi^2_{t} Var(\nu_{it}) + \psi^2_{t} Var(\epsilon_{it}) + Var(\xi_{it})
\end{align*}
Assuming that the error term is stationary, consumption growth inequality may grow because 
\ben
	\item Inequality $Var(\nu_{it}), Var(\epsilon_{it})$ increase, or
	\item Pass-through parameters $\phi_t, \psi_t$ increase
\een

\item Goal: estimate the time series of $Var(\nu_{it}), Var(\epsilon_{it}), \phi_t, \psi_t$

\item Method: Estimation by method of moments (we need at least four moments)

\eit

\end{frame}


\begin{frame}{BPP: Identification}

\bit
\setlength\itemsep{1em}
\item Income process implies 
\begin{eqnarray}
Var(\Delta y_{it}) = Var(\nu_{it}) + 2 Var(\epsilon_{it}) \nonumber \\
CoV(\Delta y_{it}, \Delta y_{it-1}) = -Var(\epsilon_{it}) \nonumber 
\end{eqnarray}

\item $\Rightarrow$ A rise in income growth inequality without a corresponding decline in the autocovariance means that permanent, and not transitory, income variance has increased

\item We also have that
\begin{align*}
CoV(\Delta c_{it}, \Delta y_{it}) &= CoV(\phi_{t} \nu_{it} + \psi_{t} \epsilon_{it} + \xi_{it}, \nu_{it} + \epsilon_{it}- \epsilon_{it-1}) \nonumber \\
&= \phi_{t} Var(\nu_{it}) + \psi_{t} Var(\epsilon_{it})
\end{align*}
which, together with
\begin{align*}
\Delta Var(\Delta c_{it}) &= Var(\nu_{it})\Delta \phi^2_{t} + \phi^2_{t-1} \Delta Var(\nu_{it}) +  Var(\epsilon_{it}) \Delta \psi^2_{t} + \psi^2_{t} \Delta Var(\epsilon_{it}) 
\end{align*}
can be used to estimate $\phi_t, \psi_t$ given the income variances

%\item Similarly, they use covariances between $t$ and $t-2$ to estimate the lag components in the MA(q) process (but it turns out $q=1$ provides best fit)
%	\bit
%		\item Also, details regarding measurement error which I do not treat here
%	\eit

\eit

\end{frame}


\begin{frame}{BPP: Income moments in the data}

\begin{figure}
	\centering
	\includegraphics[scale=0.7, trim = 0cm 0cm 0cm 0cm, clip]{figures/blundell_tab3.pdf}
\end{figure}


\bit
\setlength\itemsep{1em}
\item Both variance and autocovariance of income growth grew during the eigthies, but autocovariance grew relatively more
\eit

\end{frame}

\begin{frame}{BPP: Consumption moments in the data}

\begin{figure}
	\centering
	\includegraphics[scale=0.7, trim = 0cm 0cm 0cm 0cm, clip]{figures/blundell_tab4.pdf}
\end{figure}


\bit
\setlength\itemsep{1em}
\item Variance of consumption growth grew in the early eigthies, declined thereafter
\eit

\end{frame}

\begin{frame}{BPP: Consumption-income moments in the data}

\begin{figure}
	\centering
	\includegraphics[scale=0.7, trim = 0cm 1.9cm 0cm 0cm, clip]{figures/blundell_tab5.pdf}
\end{figure}


\bit
\setlength\itemsep{1em}
\item No trend in consumption-income correlation
\eit

\end{frame}

\begin{frame}{BPP: Estimated evolution of permanent-income inequality}

\begin{figure}
	\centering
	\includegraphics[scale=0.7, trim = 0cm 0cm 0cm 0cm, clip]{figures/blundell_tab6_part1.pdf}
\end{figure}


\bit
\setlength\itemsep{1em}
\item Permanent-income inequality grew a lot in the eigthies, declined thereafter
\eit

\end{frame}

\begin{frame}{BPP: Estimated evolution of transitory-income inequality}

\begin{figure}
	\centering
	\includegraphics[scale=0.7, trim = 0cm 0cm 0cm 0cm, clip]{figures/blundell_tab6_part2.pdf}
\end{figure}


\bit
\setlength\itemsep{1em}
\item Transitory-income inequality declined in the early eigthies, grew from mid-eighties and onwards
\eit

\end{frame}

\begin{frame}{BPP: Estimated insurance}

\begin{figure}
	\centering
	\includegraphics[scale=0.7, trim = 0cm 0cm 0cm 0cm, clip]{figures/blundell_tab6_part3.pdf}
\end{figure}


\bit
\setlength\itemsep{1em}
\item Consumption responds a lot more to permanent income than to transitory income shocks, but no evidence of change over time
\eit

\end{frame}

\begin{frame}{BPP: Back to the question}

\begin{figure}
	\centering
	\includegraphics[scale=0.7, trim = 0cm 1cm 0cm 0cm, clip]{figures/blundell_fig1.pdf}
\end{figure}

\bit
\setlength\itemsep{0.5em}
\item In the early 80's, US consumption inequality tracked income inequality because income inequality growth was driven by permanent-income inequality growth

\item The series diverged as from the mid 80's and onwards, income inequality growth stem trom transitory inequality growth, against which households are much more insured (meaning that pass-through to consumption is low)
\eit

\end{frame}

\begin{frame}{BPP: Discussion}

\bit
\setlength\itemsep{1em}
\item Why do we care?
\bit
	\item Important to understand the nature of income inequality in order to assess welfare implications
	
	\item An increase in income inequality driven by transitory-income shocks is much less severe compared to it being driven by permanent-income shocks
\eit

\item See Heathcote-Storesletten-Violante (RED 2010) for documentation of similar facts for following decades

\item Semi-recent (job market) papers:
\bit
\setlength\itemsep{0.5em}
	\item Commault (AEJmacro 2021): Natural experiment evidence suggest much less insurance to transitory income shocks, how does this square with BPP and other semi-structural estimation approaches? 
		\bit
		\setlength\itemsep{0.5em}
			\item Shows that BPP's postulated consumption rule is not consistent with buffer-stock savings model
			
			\item Using the model-derived rule and a similar identification strategy, she finds much less insurance to transitory income shocks
		\eit
	
	\item Straub (2019): Augmented BPP-like estimation procedure suggest consumption does not respond linearly to permanent-income shocks
		\bit
		\setlength\itemsep{0.5em}
			\item Poor household respond more, rich less
		
			\item Shows that a calibrated buffer-stock model can account for this if adding non-homothetic preferences
			
			\item Implication 1: the rise in income inequality can explain the secular decline in the risk-free interest rate
			
			\item Implication 2: Secular increase in debt may imply secular decline in aggregate consumption demand, causing ``secular stagnation'' (Mian-Straub-Sufi, QJE 2021)
		\eit
\eit
\eit

\end{frame}

\begin{frame}{Kaplan-Violante (Ecmtra 2014): Motivation}

\bit
\setlength\itemsep{1em}
\item Natural experiments suggest that the within-a-quarter \emph{aggregate} MPC out of transitory income shocks is quite large
\bit
\setlength\itemsep{0.5em}
	\item In 2001 and 2007-2008, US government provided cash transfers to households as part of stimulus program, with random timing of the transfer
	
	\item Johnson-Parker-Souleles (AER 2006) and Parker-Souleles-Johnson-McClelland (AER 2011) exploits random timing to estimate MPCs of approximately $0.25$
	
	\item See also Misra-Surico (AEJmacro, 2014) and Jappelli-Pistaferri (AEJmacro, 2014)
\eit

\item We know from buffer-stock theory that individual MPC can be large among wealth-poor households

\item However, households with little net assets account for small share of total consumption $\Rightarrow$ cannot explain why aggregate MPC is large

\item How to make sense of this?

\item Aggregate MPC is a key moment for understanding business cycles and the macroeconomic effect of stablization policies
\eit

\end{frame}

\begin{frame}{Kaplan-Violante: Overview}

\bit
\setlength\itemsep{2em}
\item Idea: even though many households have substantial wealth holdings, much of this wealth is \emph{illiquid}
\bit
	\item Housing wealth, retirement accounts...
\eit

\item It is primarily the access to liquid wealth that should determine the ability of households to insure against transitory income shocks

\item Introduce the concept of ``wealthy hand-to-mouth'' (wealthy HtM) households: households with a lot of assets but little liquid assets

\item Two contributions:
\ben
\setlength\itemsep{0.5em}
	\item Document that up to $1/4$ of US households can be classified as wealthy HtM households
	
	\item Extend buffer-stock savings model to have both liquid and illiquid savings
		\bit
		\setlength\itemsep{0.5em}
			\item Calibrate to match documented evidence
			
			\item Assess whether model can explain quasi-experimental evidence on aggregate MPC
		\eit
\een

\eit

\end{frame}

\begin{frame}{Kaplan-Violante: Descriptive evidence on asset holdings}

\begin{figure}
	\centering
	\includegraphics[scale=0.7, trim = 0cm 0cm 0cm 0cm, clip]{figures/kaplan_tab3.pdf}
\end{figure}

\bit
\setlength\itemsep{0.5em}
\item Define HtM household as a household with liquid assets $<1/2$ of monthly earnings
\item Define wealthy HtM as a HtM households with net illiquid assets $>x$ dollars ($x=0, 1000, 3000$)
\item Depending on $x$ and the definition of illiquid assets, you find that wealthy HtM households make up $7-24$\% of the population
\eit

\end{frame}

\begin{frame}{Kaplan-Violante: (simplified) Model}

\bit
\setlength\itemsep{0.5em}
\item A simplified version of KV's life-cycle model
\begin{eqnarray}
\max_{C_t, A_{t+1}} && E_0 \sum_{t=0}^{T} \beta^t U(c_t) \nonumber \\
\text{s.t.} && C_t + A^{\text{liq}}_{t+1} +  A^{\text{ill}}_{t+1} = Y_t + R^{\text{liq}} A^{\text{liq}}_{t} +  R^{\text{ill}}A^{\text{ill}}_{t} - AC(A^{\text{ill}}_{t+1}, A^{\text{ill}}_{t}) \nonumber \\
&& AC(A^{\text{ill}}_{t+1}, A^{\text{ill}}_{t}) = \left\{ \begin{array}{cc}
0 & \text{if } A^{\text{ill}}_{t+1} = R A^{\text{ill}}_{t} \\
\kappa & \text{if } A^{\text{ill}}_{t+1} \neq R A^{\text{ill}}_{t}
\end{array} \right. \nonumber \\
&& Y_{t} = P_{t} e^{\epsilon_{t}} \nonumber \\
&& P_{t} = P_{t-1}G e^{\nu_{t}} \nonumber \\
&& A^{\text{ill}}_{t+1}, C_t \geq 0 \nonumber \\
&& A^{\text{liq}}_{t+1} \geq -\bar A \nonumber 
\end{eqnarray}

\item Their full model also include
\bit
	\item Utility flow from illiquid assets (think housing)
	\item Different interest rate on borrowing compared to saving
	\item Taxes and transfers
	\item Richer income process
\eit

\item When taking model to the data, the key difficulty lies in calibrating $\kappa$

\eit

\end{frame}

%\begin{frame}{Kaplan-Violante: Behavior of a poor HtM household}
%
%\begin{figure}
%	\centering
%	\includegraphics[scale=0.7, trim = 0cm 0cm 0cm 0cm, clip]{figures/kaplan_fig1.pdf}
%\end{figure}
%
%\bit
%\setlength\itemsep{0.5em}
%\item Fixed cost $\kappa$ means that you want to make as few transactions in illiquid assets as possible
%\bit
%	\item When not making transacation, the household behaves as in a standard buffer-stock model
%	\item When making transaction, consumption and liquid assets jump
%\eit
%
%\item If $R^{\text{liq}}$ is sufficiently high compared to $R^{\text{ill}}$, the household will start to accumulate liquid assets immediately after the transaction
%
%\item Until age 40 (quarterly model starting at age 20), this household is poor HtM
%
%\item Except directly after adjustment, this household is never wealthy HtM
%\eit
%
%\end{frame}

\begin{frame}{Kaplan-Violante: Behavior of a wealthy HtM household}

\begin{figure}
	\centering
	\includegraphics[scale=0.7, trim = 0cm 0cm 0cm 0cm, clip]{figures/kaplan_fig2.pdf}
\end{figure}

\bit
\setlength\itemsep{0.5em}

\item A model period is a quarter, period $0$ is age 20.

\item Since $R^{\text{ill}}>R^{\text{liq}}$, saving in illiquid assets maximizes life-time consumption

\item Fixed cost $\kappa$ means that you want to make as few transactions in illiquid assets as possible $\Rightarrow$ illiquid assets provide poor insurance
%\bit
%	\item When not making transacation, the household behaves as in a standard buffer-stock model
%	\item When making transaction, consumption and liquid assets jump
%\eit

\item If return difference is large enough, household put in all cash-on-hand into illiquid asset account when investing, becoming liquidity constrained for several periods thereafter 

\eit

\end{frame}


\begin{frame}{Kaplan-Violante: Implications for the aggregate MPC}

\begin{figure}
	\centering
	\includegraphics[scale=0.7, trim = 0cm 0cm 0cm 0cm, clip]{figures/kaplan_fig5.pdf}
\end{figure}

\bit
\setlength\itemsep{0.5em}
\item Matching average returns on liquid and illiquid savings, KV finds that even with modest fixed cost, the model can explain high aggregate MPC

\item KV also show that this model can match cross-sectional evidence on the distribution of MPC across income and wealth
\eit

\end{frame}


%\begin{frame}{Kaplan-Violante: Size-dependency of MPC}
%
%\begin{figure}
%	\centering
%	\includegraphics[scale=0.85, trim = 0cm 0cm 0cm 0cm, clip]{figures/kaplan_fig8.pdf}
%\end{figure}
%
%\bit
%\setlength\itemsep{0.5em}
%\item Large shock $\Rightarrow$ optimal to pay the transaction cost to tap into illiquid account $\Rightarrow$ households can smooth shock $\kappa$ $\Rightarrow$ MPC down
%
%\item The model has several other interesting implications that we as of yet have not been able to test using natural experiments
%
%\eit
%
%\end{frame}

\begin{frame}{Kaplan-Violante: Discussion}

\bit
\setlength\itemsep{1em}
\item In conventional (rep-agent) macro models, MPC is low and fluctuations in income therefore plays little role for fluctuations in aggregate demand
\bit
\setlength\itemsep{0.5em}
	\item This is counterintuitive and at odds with micro-level evidence

	\item KV provides a theory for why income fluctuations may matter

	\item Important for the development of Heterogeneous-Agent New-Keynesian (HANK) models (more about this later...)
\eit

\item An alternative theory to explain high aggregate MPC: some households are just not very optimizing and consume directly what they get independently of their financial situation
\bit
\setlength\itemsep{0.5em}
	\item See, e.g., Campbell-Mankiw (NBER annual, 1989)
	
	\item Fagereng-Holm-Natvik (AEJmacro 2021): Using Norwegian register data on consumption response to lottery winnings, they show aggregate MPC is smoothly declining with time
	
	\item Auclert-Rognlie-Straub (JPE 2025): Alternative behavioral theory has a hard time explaining the time pattern in MPC, but KV's two-asset model can 
\eit

\item Recent explorative approaches to understanding heterogeneity in MPC 
\bit
	\item Aguiar-Bils-Boar (REStud 2025), Colarieti-Mei-Stantcheva (2025), Carlsson-D'Amico-�berg-Skans-Walentin (2025)
\eit
\eit

\end{frame}

\begin{frame}{Summing up}

\bit
\setlength\itemsep{1em}
\item Buffer-stock savings model provides a powerful framework for quantitative and empirical research of consumption-savings dynamics

\item Big literature - we've only glanced at some  applications today. 

\item Accompied with big literature on estimating household earnings dynamics
%\bit
%\item Some important contributors: Meghir, Pistaferri, De Nardi, Guvenen, Heathcote, Storesletten
%\item See papers by Domeij, Klein, Druedahl, Jorgensen for studies using Scandinavian registers
%\eit

\item Research in this area operates at the intersection of theory and micro data, often using both structural and reduced-form approaches

%Some additional inspiration: 
%\bit
%\setlength\itemsep{0.5em}	
%	\item Krueger-Perri (ReStud, 2006): BS model with endogeneous credit constraint can tightly fit the evolution of income and consumption inequality in the US
%
%	\item Berger-Vavra (Ecmtra, 2015): BS model with irreversible durable investments suggest that aggregate MPC w.r.t. durable goods declines in bad times
%	
%	\item Fagereng-Guiso-Pistaferri (ReStud, 2019): Norwegian matched employer-employee data to identify shocks to household income risk suggest big effects on portfolio rebalancing, consistent with BS model
%\eit



\item So far, our investigation has been focused on microeconomics: how households behave taking prices, shocks and policy as given

\item Next lecture: how does incomplete-market households interact with the aggregate economy?
\bit
	\item We need to develop a general equilibrium model
\eit

\eit


\end{frame}




\end{document}





