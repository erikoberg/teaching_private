\documentclass[9pt,xcolor={dvipsnames}]{beamer}
\usetheme{Boadilla}

\makeatother
\setbeamertemplate{footline}
{
	\leavevmode%
	\hbox{%
		\begin{beamercolorbox}[wd=.4\paperwidth,ht=2.25ex,dp=1ex,center]{author in head/foot}%
			\usebeamerfont{author in head/foot}\insertshortauthor
		\end{beamercolorbox}%
		\begin{beamercolorbox}[wd=.6\paperwidth,ht=2.25ex,dp=1ex,center]{title in head/foot}%
			\usebeamerfont{title in head/foot}\insertshorttitle\hspace*{3em}
			\insertframenumber{} / \inserttotalframenumber\hspace*{1ex}
	\end{beamercolorbox}}%
	\vskip0pt%
}
\makeatletter
\setbeamertemplate{navigation symbols}{}


\usepackage{lipsum}
\usepackage{appendixnumberbeamer}

\usepackage[authoryear]{natbib}
\usepackage[latin1]{inputenc}
\usepackage[T1]{fontenc}
\usepackage{caption}
\usepackage{amsmath, amssymb}
\usepackage{epstopdf}
\usepackage{graphicx}
\usepackage{lmodern}
%\usepackage[dvipsnames]{xcolor}
\usepackage{xpatch}
\usepackage{multirow}
\usepackage{tikz}
\usepackage{subfig}

\usepackage{amsmath,theorem,amssymb,graphicx, pgfplots, tabularx, placeins}
\usepackage{dsfont}
\usepackage{caption}
%\usepackage{subcaption}
%\usepackage{subcaption}
\setbeamertemplate{caption}{\raggedright\insertcaption\par}
%\setbeamertemplate{footline}[frame number]
\usepackage{csquotes}
\usepackage{bm}
\bibliographystyle{econometrica}
\usepackage[normalem]{ulem}
\usepackage{setspace}


\definecolor{gray(x11gray)}{rgb}{0.75, 0.75, 0.75}


\newcommand{\bit}{\begin{itemize}}
	\newcommand{\eit}{\end{itemize}}
\newcommand{\ben}{\begin{enumerate}}
	\newcommand{\een}{\end{enumerate}}

\newcommand{\black}{\color{black}}
\newcommand{\bc}{\color{blue}}
\newcommand{\rc}{\color{red}}
\newcommand{\gc}{\color{ForestGreen}}
\newcommand{\wc}{\color{white}}

\newcommand{\lb}{\label}
\newcommand{\re}{\eqref}

\title[NK: Shocks \& Propagataion]{Macroeconomics II Part II, Lecture V:\\
	  The New-Keynesian Model: Shocks \& Propagataion}
\author{Erik {\"O}berg}
\date{}

\begin{document}

\begin{frame}
\maketitle
\end{frame}

\section{Introduction}


\begin{frame}{Recap}

\bit
\setlength\itemsep{1.5em} 

\item Basic NK model = RBC model + monopolistic competition and sticky prices

\item Last lecture: Setup, derivations and determinacy

\item Today: What are the predictions of the NK model?
\eit

\end{frame}


\begin{frame}{Agenda}

\ben
\setlength\itemsep{1.5em} 

\item Monetary Policy shocks

\item TFP shocks

\item Cost-push shocks
\een

\end{frame}


\begin{frame}

\begin{center}
	\huge Monetary Policy shocks \normalfont
\end{center}

\end{frame}


\begin{frame}{The 3-equation representation}

\bit
\setlength\itemsep{1.5em}
\item The log-linearized equilibrium can be characterized by
\begin{eqnarray}
\text{DIS curve:} && \hat y_t = - (\hat i_t-E_t \pi_{t+1}) + E_t \hat y_{t+1} \nonumber \\
\text{Phillips curve:} && \pi_t = \beta E_t \pi_{t+1} + \kappa \hat y_t \nonumber \\
\text{Policy rule:} && \hat i_t = \phi \pi_t + \nu_t \nonumber 
\end{eqnarray}

\item 3 equations in 3 unknowns: $\{\hat y_t, \hat i_{t}, \pi_t\}$!	

\item This is how the model is usually presented in the literature

\item \bf Warning!: \normalfont Although very convenient, these equations mix multiple equilibrium relationships $\Rightarrow$ hard to extract a precise intuition about model mechanisms
\eit

\end{frame}




\begin{frame}{Calibration}

\bit
\setlength\itemsep{1.5em}
\item Quarterly frequency

\item For the recurrent parameters, let's stick with we used for our RBC model
\bit
\setlength\itemsep{0.5em}
\item $\beta = 0.99$

\item $\varphi = 1$
\eit

\item The new ones:
\bit
\setlength\itemsep{0.5em}

\item $\theta=2/3$ to match average price duruation of three quarters (Gal\'{i}-Lop\'{e}z-Salido EER 2001)

\item $\epsilon=6$ to match average markup of $20 \%$

\item $\phi = 1.5$ to match Fed reaction function during Greenspan era
\eit

\item The shock process:
\begin{eqnarray}
\nu_t = \rho_{\nu} \nu_{t-1} + \epsilon_t \nonumber
\end{eqnarray}
with $\rho_{\nu} = 0.5$ to generate a ``moderately'' persistent shocks, as we saw in the empirical IRFs

\eit

\end{frame}


\begin{frame}{IRFs to monetary policy shock}

\begin{figure}
\centering
\includegraphics[scale=0.5,trim= 0 200 0 200, clip]{figures/nk_monshock_4variables.pdf}
\end{figure}


\end{frame}


\begin{frame}{Comments I}

\bit
\setlength\itemsep{1em}
\item In contrast to the vanilla RBC model, there are no humps and bumps here

\item This is the because the model has no state variable - dynamics are completely forward-looking:
\begin{eqnarray}
\text{DIS curve:} && \hat y_t = - (\hat i_t-E_t \pi_{t+1}) + E_t \hat y_{t+1} \nonumber \\
\text{Phillips curve:} && \pi_t = \beta E_t \pi_{t+1} + \kappa \hat y_t \nonumber \\
\text{Policy rule:} && \hat i_t = \phi \pi_t + \nu_t \nonumber 
\end{eqnarray}

\item This means that the policy function for any variable $x_t \in \{\hat y_t, \pi_t, \hat i_t \}$
\begin{eqnarray}
x_t = \sum_{s=0}^{t} a^x_{t-s} \nu_s \nonumber
\end{eqnarray}
has $a_{t-s}=0$ for all $s<t$, i.e.,
\begin{eqnarray}
x_t = a^x_0 \nu_t. \nonumber
\end{eqnarray}

\item An implication is that you can solve for the IRFs analytically, see problem set 6.

\eit

\end{frame}


\begin{frame}{Comments II}

\bit
\setlength\itemsep{1.5em}
\item Basic NK model qualitatively matches the evidence: a surprise increase in the policy rate leads to a fall in $y$

\item Quantitatively, it is way off, we'll get back to this in Lecture VI

\item To understand the mechanism, let's go back to the full 8-equation system, where we have more clean interpretations of the equilibrium equations
\eit

\end{frame}


\begin{frame}{Reminder: Full system looks like...}

\bit
\setlength\itemsep{1.5em}
\item The log-linearized equilibrium is characterized by
\begin{eqnarray}
\text{Intratemporal hh optimality:} && \hat \omega_t = \hat c_t + \varphi \hat n_t  \nonumber \\
\text{Intertemporal hh optimality:} && \hat c_t =  - (\hat i_t - E_t \pi_{t+1} ) + E_t \hat c_{t+1}  \nonumber \\
\text{Firm optimality:} && \pi_t = \beta E_t \pi_{t+1} + \lambda \widehat{mc}_t \nonumber \\
\text{Marginal cost:} && \widehat{mc}_t = \hat \omega_t \nonumber \\
\text{Goods clearing:} && \hat c_t = \hat y_t \nonumber \\
\text{Labor clearing:} && \hat y_t = \hat n_t \nonumber \\
\text{Policy:} && \hat i_t = \phi \pi_t + \nu_t \nonumber 
\end{eqnarray}
where $\hat \omega_t = \hat w_t-p_t$ is log deivations in the real wage

\item Note: Real interest rate given by $\hat r_t =  \hat i_t - E_t \pi_{t+1}$
\eit

\end{frame}

\begin{frame}{IRFs to monetary policy shock: full system}

\begin{figure}
\centering
\includegraphics[scale=0.5,trim= 0 200 0 200, clip]{figures/nk_monshock_9variables.pdf}
\end{figure}


\end{frame}

\begin{frame}{IRFs to monetary policy shock: mechanism}

\bit
\setlength\itemsep{1em}
\item How can we explain the equilibrium responses?

\item Simplify, simplify, simplify

\item Let's start with flexible prices: $\theta \to 0$

\eit


\end{frame}


\begin{frame}{IRFs to monetary policy shock under flexible prices}

\begin{figure}
	\centering
	\includegraphics[scale=0.5,trim= 0 200 0 200, clip]{figures/nk_monshock_9variables_flexprices.pdf}
\end{figure}


\end{frame}


\begin{frame}{Mechanism under flexible prices}

\bit
\setlength\itemsep{1em}
\item Last lecture, we showed that under flexible prices, system collapses to
\begin{eqnarray}
\text{DIS curve:} &&  \hat i_t = E_t \pi_{t+1}  \nonumber \\
\text{Policy rule:} && \hat i_t = \phi \pi_t + \nu_t \nonumber 
\end{eqnarray}

\item 0 response of all real variables, i.e., {\bc monetary neutraility}

\item We also showed that the unique solution has
\begin{eqnarray}
\pi_t = -\sum_{s=0}^{\infty} \frac{1}{\phi^{s+1}} \nu_{t+s}  \nonumber
\end{eqnarray}

\item Inflation falling in response to ``contractionary'' monetary policy shock has nothing to do with sticky prices or other frictions, simply a consequence of a Taylor and rational expectations

\item Inflation response so strong that interest rate falls in response to positive shock! 


\eit


\end{frame}

\begin{frame}{Back to sticky prices}

\begin{figure}
	\centering
	\includegraphics[scale=0.5,trim= 0 200 0 200, clip]{figures/nk_monshock_9variables.pdf}
\end{figure}


\end{frame}



\begin{frame}{Mechanism}

\bit
\setlength\itemsep{1em}
\item Sticky price $\rightarrow$ smaller inflation response

\item With smaller inflation response, nominal interest rate actually increases

\item As a consequence, the real interest rate increases:
\begin{eqnarray}
\hat r_t = \hat i_t - E_t \pi_{t+1} \nonumber
\end{eqnarray}

\item {\bc Monetary non-neutraility!}

\item Given that the real interest rate increases, we can back out the response of all other variables

\item In so doing, recall that we have assumed that we only search for bounded solutions!


\eit



\end{frame}


\begin{frame}{Mechanism cont'd}

\bit
\setlength\itemsep{1em}
\item Take as given that $\hat r_t$ increases, then
\begin{eqnarray}
\text{Intertemporal hh optimality:} && \hat c_t =  - (\hat r_t) + E_t \hat c_{t+1} \nonumber
\end{eqnarray}
implies $\Delta E_t c_{t+1}$ is positive

\item $\Delta E_t c_{t+1}>0$ + Bounded solution $\Rightarrow$ $\{\hat c_{t+s}\}$ must converge to some steady state

\item But the steady state is unique, so we know $\{\hat c_{t+s}\}$ must converge to $\hat c_{t+s}=0$

\item Therefore, we must have that $\hat c_t<0$!

\eit


\end{frame}



\begin{frame}{Mechanism cont'd}

\bit
\setlength\itemsep{1em}
\item Take as given that $\hat c_t<0$

\item Market clearing $\hat c_t = \hat y_t = \hat n_t < 0$
\bit
\item $\Rightarrow$ we may think of the output drop as being caused by drop in {\bc aggregate demand}
\eit

\item How is this consistent with optimal labor supply? Intratemporal optimality condition:
\begin{eqnarray}
\text{Intratemporal hh optimality:} && \hat \omega_t = \hat c_t + \varphi \hat n_t  \nonumber
\end{eqnarray}

\item $\hat n_t<0$ only if $\hat \omega_t < \hat y_t < 0$
\bit
\setlength\itemsep{0.5em}
\item Wages need to respond more than output (and profits less)!
\eit

\item Is the fall in inflation consistent with the Phillips curve?
\bit
\setlength\itemsep{0.5em}
\item $\hat \omega_t <0$ $\Rightarrow$ $\hat mc_t<0$ $\Rightarrow$ $\beta E_t \pi_{t+1}-\pi_t \approx \Delta \pi_{t+1} >0$ from the Phillips curve

\item Again, bounded solution + unique steady state $\Rightarrow$ $\hat \pi_t<0$
\eit

\eit


\end{frame}




\begin{frame}{Monetary policy shock equivalent to demand shock}

\bit
\setlength\itemsep{1em}
\item Recall: DIS curve stems from household Euler equation
\begin{eqnarray}
c_t &=& - (i_t-E_t \pi_{t+1}-\xi) + E_t c_{t+1} \nonumber
\end{eqnarray}
where $\xi = -\log \beta$

\item Suppose there are shocks to discount factor $\beta$ 
\bit
\setlength\itemsep{0.5em}
\item Specifically, assume $\xi_t = \rho_{\xi} \xi_{t-1} + \epsilon_{\xi, t}$

\item $=$ shock to the marginal value of current consumption - a ``demand shock''
\eit

\item Then, 3-equation system becomes
\begin{eqnarray}
\text{DIS curve:} && \hat y_t = - (\hat i_t-E_t \pi_{t+1} - \xi_t) + E_t \hat y_{t+1} \nonumber \\
\text{Phillips curve:} && \pi_t = \beta E_t \pi_{t+1} + \kappa \hat y_t \nonumber \\
\text{Policy rule:} && \hat i_t = \phi \pi_t + \nu_t \nonumber 
\end{eqnarray}
or
\begin{eqnarray}
\text{DIS curve + Policy rule:} && \hat y_t = - (\phi \pi_t-E_t \pi_{t+1}) + E_t \hat y_{t+1} + \xi_t - \nu_t \nonumber \\
\text{Phillips curve:} && \pi_t = \beta E_t \pi_{t+1} + \kappa \hat y_t \nonumber 
\end{eqnarray}

\item $\Rightarrow$ Positive monetary policy shocks are equivalent to negative demand shocks

\eit


\end{frame}


\begin{frame}

\begin{center}
	\huge TFP shocks \normalfont
\end{center}

\end{frame}

\begin{frame}{TFP shocks}

\bit
\setlength\itemsep{1em}
\item So far: focus on monetary policy shocks

\item Naturally, we are interested in how sticky prices and the behavior of monetary policy affect response to non-policy shocks

\item Let's consider TFP shocks

\item Assume that $a_t \equiv \log A_t$ follows
\begin{eqnarray}
a_t = \rho_a a_{t-1} + \epsilon^a_t \nonumber
\end{eqnarray}

\item Which equations in our equilibrium system are affected?
\begin{eqnarray}
\text{Intratemporal hh optimality:} && \hat \omega_t = \hat c_t + \varphi \hat n_t  \nonumber \\
\text{Intertemporal hh optimality:} && \hat c_t =  - (\hat i_t - E_t \pi_{t+1} ) + E_t \hat c_{t+1}  \nonumber \\
\text{Firm optimality:} && \pi_t = \beta E_t \pi_{t+1} + \lambda \widehat{mc}_t \nonumber \\
\text{{\black Marginal cost:} } && {\black \widehat{mc}_t = \hat \omega_t}  \nonumber \\
\text{Goods clearing:} && \hat c_t = \hat y_t \nonumber \\
\text{{\black Labor clearing:}} && {\black \hat y_t = \hat n_t} \nonumber \\
\text{Policy:} && \hat i_t = \phi \pi_t + \nu_t \nonumber 
\end{eqnarray}

\eit


\end{frame}


\begin{frame}{TFP shocks}

\bit
\setlength\itemsep{1em}
\item So far: focus on monetary policy shocks

\item Naturally, we are interested in how sticky prices and the behavior of monetary policy affect response to non-policy shocks

\item Let's consider TFP shocks

\item Assume that $a_t \equiv \log A_t$ follows
\begin{eqnarray}
a_t = \rho_a a_{t-1} + \epsilon^a_t \nonumber
\end{eqnarray}

\item Which equations in our equilibrium system are affected?
\begin{eqnarray}
\text{Intratemporal hh optimality:} && \hat \omega_t = \hat c_t + \varphi \hat n_t  \nonumber \\
\text{Intertemporal hh optimality:} && \hat c_t =  - (\hat i_t - E_t \pi_{t+1} ) + E_t \hat c_{t+1}  \nonumber \\
\text{Firm optimality:} && \pi_t = \beta E_t \pi_{t+1} + \lambda \widehat{mc}_t \nonumber \\
\text{{\bc Marginal cost:} } && {\bc \widehat{mc}_t = \hat \omega_t}  \nonumber \\
\text{Goods clearing:} && \hat c_t = \hat y_t \nonumber \\
\text{{\bc Labor clearing:}} && {\bc \hat y_t = \hat n_t} \nonumber \\
\text{Policy:} && \hat i_t = \phi \pi_t + \nu_t \nonumber 
\end{eqnarray}

\eit


\end{frame}


\begin{frame}{Incorporating TFP shocks}

\bit
\setlength\itemsep{2em}

\item Marginal cost:
\begin{eqnarray}
MC_t = \frac{W_{t}}{A_tP_t} \hspace{2mm} \Rightarrow \hat mc_t = \hat \omega_t - a_t \nonumber
\end{eqnarray}

\item Labor market clearing:
\begin{eqnarray}
N_t = \frac{Y_{t}}{A_t} D_t \hspace{2mm} \Rightarrow \hat n_t = \hat y_t - a_t \nonumber
\end{eqnarray}

\item Again, here we have that $\hat a_t = a_t$

\eit

\end{frame}


\begin{frame}{Full system with TFP shocks}

\bit
\setlength\itemsep{1.5em}
\item The log-linearized equilibrium is characterized by
\begin{eqnarray}
\text{Intratemporal hh optimality:} && \hat \omega_t = \hat c_t + \varphi \hat n_t  \nonumber \\
\text{Intertemporal hh optimality:} && \hat c_t =  - (\hat i_t - E_t \pi_{t+1} ) + E_t \hat c_{t+1}  \nonumber \\
\text{Firm optimality:} && \pi_t = \beta E_t \pi_{t+1} + \lambda \widehat{mc}_t \nonumber \\
\text{Marginal cost:} && \widehat{mc}_t = \hat \omega_t - a_t \nonumber \\
\text{Goods clearing:} && \hat c_t = \hat y_t \nonumber \\
\text{Labor clearing:} && \hat y_t = \hat n_t + a_t \nonumber \\
\text{Policy:} && \hat i_t = \phi \pi_t + \nu_t \nonumber 
\end{eqnarray}
where $\hat \omega_t = \hat w_t-p_t$ is log deivations in the real wage

\item Also, the law of motion for exogenous shocks:
\begin{eqnarray}
\nu_t &=& \rho_{\nu} \nu_{t-1} + \epsilon_t \nonumber \\
a_t &=& \rho_a a_{t-1} + \epsilon^a_t \nonumber
\end{eqnarray}
\eit

\end{frame}

\begin{frame}{IRFs to a TFP shock}

\begin{figure}
\centering
\includegraphics[scale=0.5,trim= 0 200 0 200, clip]{figures/nk_tfpshock_9variables.pdf}
\end{figure}



\end{frame}


\begin{frame}{Mechanism}

\bit
\setlength\itemsep{1em}
\item Again, no hump shapes like in the vanilla RBC model. 
\bit
	\item Again, without capital, there is no state variable.
\eit

\item Recall, in RBC without capital, hours worked did not respond at all

\item Here, hours worked falls
\bit
	\item Gal\'{i} (AER 1999) and Basu-Fernald-Kimball (AER 2006) argue this is consistent with the data
\eit

\item What role does sticky prices and monetary policy play for this response?

\item Again, let's go back to flexible prices (which is just the RBC model with no capital (and monopolistic competition))

\item With flexible prices, firm optimality is
\begin{eqnarray}
P_{it} &=& M \psi_t \nonumber \\
\frac{P_{it}}{P_t} &=& M \frac{\psi_t}{P_t} \hspace{2mm} \Rightarrow \widehat{mc}_t = 0 \nonumber
\end{eqnarray}

\item All other relationships are unaffected
\eit

\end{frame}

\begin{frame}{Equilibrium system with flexible prices}

\bit
\setlength\itemsep{1.5em}
\item The log-linearized equilibrium is characterized by
\begin{eqnarray}
\text{Intratemporal hh optimality:} && \hat \omega_t = \hat c_t + \varphi \hat n_t  \nonumber \\
\text{Intertemporal hh optimality:} && \hat c_t =  - (\hat i_t - E_t \pi_{t+1}) + E_t \hat c_{t+1}  \nonumber \\
\text{\bc Firm optimality:} && {\bc \widehat{mc}_t = 0} \nonumber \\
\text{Marginal cost:} && \widehat{mc}_t = \hat \omega_t - a_t \nonumber \\
\text{Goods clearing:} && \hat c_t = \hat y_t \nonumber \\
\text{Labor clearing:} && \hat y_t = \hat n_t + a_t \nonumber \\
\text{Policy:} && \hat i_t = \phi \pi_t + \nu_t \nonumber 
\end{eqnarray}
where $\hat \omega_t = \hat w_t-p_t$ is log deivations in the real wage

\item Also, the law of motion for exogenous shocks:
\begin{eqnarray}
\nu_t &=& \rho_{\nu} \nu_{t-1} + \epsilon_t \nonumber \\
a_t &=& \rho_a a_{t-1} + \epsilon^a_t \nonumber
\end{eqnarray}

\eit

\end{frame}

\begin{frame}{Equilibrium system with flexible prices: solution}

\bit
\setlength\itemsep{1.5em}
\item The log-linearized equilibrium is characterized by
\begin{eqnarray}
\text{Intratemporal hh optimality:} && \hat \omega_t = \hat c_t + \varphi \hat n_t  \nonumber \\
\text{Intertemporal hh optimality:} && \hat c_t =  - (\hat i_t - E_t \pi_{t+1}) + E_t \hat c_{t+1}  \nonumber \\
\text{Firm optimality:} && {\widehat {mc}_t = 0} \nonumber \\
\text{Marginal cost:} && \widehat{mc}_t = \hat \omega_t - a_t \nonumber \\
\text{Goods clearing:} && \hat c_t = \hat y_t \nonumber \\
\text{Labor clearing:} && \hat y_t = \hat n_t + a_t \nonumber \\
\text{Policy:} && \hat i_t = \phi \pi_t + \nu_t \nonumber 
\end{eqnarray}

\item Work it through to find {\bc the natural interest rate}: {\rc (Do on whiteboard)}
\begin{eqnarray}
\hat r^n_t &=& \hat i_t - E_t \pi_{t+1} \nonumber \\
&=& - (1-\rho_a) a_t \nonumber
\end{eqnarray}

\item Key result: positive TFP shocks cause a decline in the natural real interest rate

\item Mechanism: $a_t>0$ $\Rightarrow$ $\Delta E_t a_{t+1} <0$ $\Rightarrow$ $\Delta E_t c_{t+1} <0$ $\Rightarrow$ $ \hat r^n_{t} < 0$

\eit

\end{frame}

\begin{frame}{Sticky-price equilibrium in gaps}

\bit
\setlength\itemsep{1.5em}
\item Let's turn back to the {\bc sticky-price equilibrium}
%\begin{eqnarray}
%\text{DIS curve:} && \hat y_t = - (\hat i_t-E_t \pi_{t+1}) + E_t \hat y_{t+1} \nonumber \\
%\text{Phillips curve:} && \pi_t = \beta E_t \pi_{t+1} + \kappa \hat y_t \nonumber \\
%\text{Policy rule:} && \hat i_t = \phi \pi_t + \nu_t \nonumber 
%\end{eqnarray}

\item We linearized the equilibrium around the steady state $\hat x = x_t - x_{ss}$, for any $x$

\item Let's instead consider a linearization around the {\bc flexible-price equilibrium} or, using another terminology, the {\bc natural equlibrium}

\item We'll go directly at the 3-equation system

\item Define ``gaps'' as $\tilde x_t = x_t-x_t^n$

\item Note $\tilde x_t = (x_t-x_{ss})- (x_t^n-x_{ss}) = \hat x_t - \hat x^n_t$


\eit

\end{frame}


\begin{frame}{Sticky-price equilibrium in gaps II}

\bit
\setlength\itemsep{1.5em}
\item DIS curve: not affected by $a_t$, simply subtract natural equilibrium to find
\begin{eqnarray}
&& \hat y_t = - (\hat i_t-E_t \pi_{t+1}) + E_t \hat y_{t+1} \nonumber \\
\Leftrightarrow && \hat y_t-\hat y^n_t = - (\hat i_t-E_t \pi_{t+1}-\hat r^n_t) + E_t \hat y_{t+1}-E_t \hat y^n_{t+1}  \nonumber \\
\Leftrightarrow && \tilde y_t = - (\hat i_t-E_t \pi_{t+1}-\hat r^n_t) + E_t \tilde y_{t+1}  \nonumber
\end{eqnarray}

%\item Recall that the natural interest rate is given by
%\begin{eqnarray}
%\hat r^n_t = - (\rho_a-1) a_t \nonumber
%\end{eqnarray}

\item Phillips curve: 
\bit
\setlength\itemsep{0.5em}
	\item Start with firm optimality:
	\begin{eqnarray}
	\pi_t = \beta E_t \pi_{t+1} + \lambda \widehat{mc}_t \nonumber
	\end{eqnarray}

	\item Flexible prices: $\widehat{mc}^n_t = 0$ $\Rightarrow$ $\widehat{mc}_t = \widetilde{mc}_t $
	
	\item Work through the flex-price equations to find $\widetilde{mc}_t = (1+\varphi) \tilde y_t$, hence
	\begin{eqnarray}
	\pi_t = \beta E_t \pi_{t+1} + \kappa \tilde y_t \nonumber 
	\end{eqnarray}
	where $\kappa = \lambda (1+\varphi)$
\eit

\eit

\end{frame}


\begin{frame}{Sticky-price equilibrium in gaps III}

\bit
\setlength\itemsep{1.5em}
\item Putting it together, the equilibrium can be summarized as
\begin{eqnarray}
\text{DIS curve:} && \tilde y_t = - (\hat i_t-E_t \pi_{t+1}-\hat r^n_t) + E_t \tilde y_{t+1}  \nonumber \\
\text{Phillips curve:} && \pi_t = \beta E_t \pi_{t+1} + \kappa \tilde y_t \nonumber \\
\text{Policy rule:} && \hat i_t = \phi \pi_t + \nu_t \nonumber \\
\text{Natural real interest rate:} && \hat r^n_t = - (\rho_a-1) a_t \nonumber 
\end{eqnarray}

\item $\tilde y_t$ measures fluctuations that are {\bc inefficient}

\item Model captures two central assertions of the prevailing {\bc Neo-Wicksellian} view of business cycles
\ben
\setlength\itemsep{0.5em}
\item Monetary factors does not affect the natural real interest

\item Inefficient fluctuations are linked by deviations in the real interest rate from the natural real interest rate
\een

\eit

\end{frame}


\begin{frame}{Intermezzo: The Great Knut Wicksell}


\begin{figure}
	\centering
		\begin{subfigure}{\textwidth}
		\centering
		\includegraphics[width=.4\linewidth]{figures/wicksell_bok.jpg}
	\end{subfigure}%
	\begin{subfigure}{\textwidth}
		\centering
		\includegraphics[width=.4\linewidth]{figures/wicksell.jpg}
	\end{subfigure}
\end{figure}

\end{frame}


\begin{frame}{Intermezzo: is the natural-rate hypothesis correct?}

\begin{figure}
	\centering
	\includegraphics[scale=0.45]{figures/Hillenbrand_figure.pdf}
\end{figure}

From Hillenbrand (2023)

\end{frame}


\begin{frame}{IRFs to a TFP shock: Gaps}

\begin{figure}
\centering
\includegraphics[scale=0.5,trim= 0 200 0 200, clip]{figures/nk_tfpshock_gaps_4variables.pdf}
\end{figure}


\end{frame}

\begin{frame}{IRFs to a TFP shock: comments}

\bit
\setlength\itemsep{1.5em}
\item A positive TFP shocks leads to ``gap'' recession

\item This reflects that monetary policy is not doing a good job

\item Mehcanism: TFP $\uparrow$ $\Rightarrow$ Marginal cost $\downarrow$ $\Rightarrow$ inflation $\downarrow$ $\Rightarrow$ Interest rate $\downarrow$
\bit
\setlength\itemsep{0.5em}

	\item Monetary policy ``stimulates'' the economy...
	
	\item ... but not enough to raise consumption to its efficient level
	
	\item In the ``natural equilibrium'', real interest rate is even lower, and hours worked are constant 
	
	\item Therefore, hours decline in the observed equilibrium
\eit

\item Ineffiency not surprising: the policy rule was specified in a completely ad hoc manner

\eit

\end{frame}


\begin{frame}

\begin{center}
	\huge Cost-push shocks \normalfont
\end{center}

\end{frame}

\begin{frame}{Cost-push shocks}

\bit
\setlength\itemsep{1.5em}
\item TFP shocks is an example of a \emph{supply shock}

\item Due to sticky prices, TFP shocks create a wedge (``deviations'') between the {\bc natural equilibrium} and the actual equilibrium

\item Because the fluctuations in the natural equilibrium are efficient, the policy problem tends be organized around how to undo the effect of price stickiness 

\item Another class of supply shocks creates a wedge between the {\bc natural equilibrium} and the {\bc efficient equilibrium}, so called {\bc cost-push shocks}

\item Examples: shocks to firm's desired markup (greedflation?), shocks to distorionary taxes
\eit

\end{frame}


\begin{frame}{Rewriting the system once more}

\bit
\setlength\itemsep{1.5em}
\item Written in deviations from the natural equilibrium, our system is
\begin{eqnarray}
\text{DIS curve:} && \tilde y_t = - (\hat i_t-E_t \pi_{t+1}-\hat r^n_t) + E_t \tilde y_{t+1}  \nonumber \\
\text{Phillips curve:} && \pi_t = \beta E_t \pi_{t+1} + \kappa \tilde y_t \nonumber \\
\text{Policy rule:} && \hat i_t = \phi \pi_t + \nu_t \nonumber 
\end{eqnarray}

\item Define $x_t = y_t-y^e_t$, where $y^e_t$ is the efficient equibrium path. 

\item We have that  $\tilde y_t = x_t + (y^e_t-y^n_t)$

\item We can write our system as
\begin{eqnarray}
\text{DIS curve:} && x_t = - (\hat i_t-E_t \pi_{t+1}-\hat r^e_t) + E_t x_{t+1}  \nonumber \\
\text{Phillips curve:} && \pi_t = \beta E_t \pi_{t+1} + \kappa x_t + u_t \nonumber \\
\text{Policy rule:} && \hat i_t = \phi \pi_t + \nu_t \nonumber 
\end{eqnarray}
where $u_t = \kappa (y^e_t-y^n_t)$ and $\hat r^e_t = E_t \Delta y^e_{t+1}$

\item $u_t$ is, in \emph{reduced form}, a cost-push shock

\eit

\end{frame}


\begin{frame}{IRFs to a cost-push shocks}

\begin{figure}
	\centering
	\includegraphics[scale=0.5,trim= 0 200 0 200, clip]{figures/nk_costpushshock_4variables.pdf}
\end{figure}

\end{frame}


\begin{frame}{Comments}

\bit
\setlength\itemsep{1.5em}
\item Cost-push shocks are interesting from a policy perspective

\item These shocks increase inflation, but also cause an inefficient recession
\bit
	\item A monetary authority can combat inflation, but this will lead an even deeper recession
\eit

\item Contrast with a positive TFP shock: inefficient recession, but lowers inflation 
\bit
	\item Here, a monetary authority can seemingly adress two problems simoultaneously by stimulating the economy
\eit

\item It seems like we need to think more systematically about the policy problem...
\eit

\end{frame}


\begin{frame}{Summing up}

\bit
\setlength\itemsep{1.5em}
\item Basic NK model = RBC model + monopolistic competition and sticky prices

\item In contrast to RBC, NK predicts inefficient fluctations, and a role for monetary policy

\item In response to a positive shock, the NK model predicts that
\vspace{5mm}
\begin{table}[h]
	\centering
	\begin{tabular}{|l|l|l|l|l|}
		\hline
		Shock & $\hat y_t$ & $\tilde y_t$ & $x_t$ & $\pi_t$ \\
		\hline
		Monetary policy & down & down & down & down \\
		TFP & up & down & down & down \\
		Cost-push & --- & --- & down & up \\
		\hline
	\end{tabular}
	\label{tab:shocks}
\end{table}



\item In line with the evidence on monetary policy (qualitatively, but not quantitatively)

\item Next lecture, optimal policy in the NK model


\eit

\end{frame}





\end{document}














