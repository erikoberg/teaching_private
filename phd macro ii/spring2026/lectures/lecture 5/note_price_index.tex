\documentclass[9pt,xcolor={dvipsnames}]{beamer}
\usetheme{Boadilla}

\makeatother
\setbeamertemplate{footline}
{
	\leavevmode%
	\hbox{%
		\begin{beamercolorbox}[wd=.4\paperwidth,ht=2.25ex,dp=1ex,center]{author in head/foot}%
			\usebeamerfont{author in head/foot}\insertshortauthor
		\end{beamercolorbox}%
		\begin{beamercolorbox}[wd=.6\paperwidth,ht=2.25ex,dp=1ex,center]{title in head/foot}%
			\usebeamerfont{title in head/foot}\insertshorttitle\hspace*{3em}
			\insertframenumber{} / \inserttotalframenumber\hspace*{1ex}
	\end{beamercolorbox}}%
	\vskip0pt%
}
\makeatletter
\setbeamertemplate{navigation symbols}{}


\usepackage{lipsum}
\usepackage{appendixnumberbeamer}

\usepackage[authoryear]{natbib}
\usepackage[latin1]{inputenc}
\usepackage[T1]{fontenc}
\usepackage{caption}
\usepackage{amsmath, amssymb}
\usepackage{epstopdf}
\usepackage{graphicx}
\usepackage{lmodern}
%\usepackage[dvipsnames]{xcolor}
\usepackage{xpatch}
\usepackage{multirow}
\usepackage{tikz}
\usepackage{subfig}

\usepackage{amsmath,theorem,amssymb,graphicx, pgfplots, tabularx, placeins}
\usepackage{dsfont}
\usepackage{caption}
%\usepackage{subcaption}
%\usepackage{subcaption}
\setbeamertemplate{caption}{\raggedright\insertcaption\par}
%\setbeamertemplate{footline}[frame number]
\usepackage{csquotes}
\usepackage{bm}
\bibliographystyle{econometrica}
\usepackage[normalem]{ulem}
\usepackage{setspace}


\definecolor{gray(x11gray)}{rgb}{0.75, 0.75, 0.75}


\newcommand{\bit}{\begin{itemize}}
	\newcommand{\eit}{\end{itemize}}
\newcommand{\ben}{\begin{enumerate}}
	\newcommand{\een}{\end{enumerate}}

\newcommand{\black}{\color{black}}
\newcommand{\bc}{\color{blue}}
\newcommand{\rc}{\color{red}}
\newcommand{\gc}{\color{ForestGreen}}
\newcommand{\wc}{\color{white}}

\newcommand{\lb}{\label}
\newcommand{\re}{\eqref}

\title[NK: Topics]{Macroeconomics II: A Note on Price Indices}
\author{Erik {\"O}berg}
\date{February 16, 2022}

\begin{document}

\begin{frame}
\maketitle
\end{frame}


\begin{frame}{The price level and its relation to individual prices}

\bit
\setlength\itemsep{1.5em} 

\item In the NK model presented yesterday, $P_t$ referred to the price of ``final goods'' and $P_{it}$ to the price of ``intermediate goods''

\item With a CES production function in the final goods producer problem, we showed that they are related through
\begin{eqnarray}
P_t =  \left(\int_{0}^{1} P_{it}^{1-\epsilon}  di\right)^{\frac{1}{1-\epsilon}} \nonumber
\end{eqnarray}

\item Hence, we can interpret the final good price $P_t$ as a ``Producer Price Index'' (PPI)

\eit

\end{frame}


\begin{frame}{Another setup}

\bit
\setlength\itemsep{1.5em} 

\item An alternative, but equivalent, setup is to assume that there exist only one firm layer, which produce differentiated consumer goods

\item In such a setup, the household problem becomes
\begin{eqnarray}
\max_{\{C_{it}, N_t, B_{t+1}\}} && E_O \sum_{t=0}^{\infty} \beta^t \left[U(C_t)- V(N_t) \right]\nonumber \\
\text{s.t} && \int_{i=0}^{1} P_{it} C_{it} di + Q_t B_{t+1} \leq W_t N_t + B_{t} + T_t \nonumber 
\end{eqnarray}
with
\begin{eqnarray}
C_t = \left(\int_{0}^{1} C_{it}^{\frac{\epsilon-1}{\epsilon}} di \right)^{\frac{\epsilon}{\epsilon-1}} \nonumber
\end{eqnarray}


\eit

\end{frame}


\begin{frame}{Another setup II}

\bit
\setlength\itemsep{1.5em} 

\item By first solving the within-period cost-minimization problem:
\begin{eqnarray}
\min_{C_{it}} && \int_{i=0}^{1} P_{it} C_{it} di \nonumber \\
\text{s.t} &&  \int_{i=0}^{1} P_{it} C_{it} \leq E \nonumber \\
&& \left(\int_{0}^{1} C_{it}^{\frac{\epsilon-1}{\epsilon}} di \right)^{\frac{\epsilon}{\epsilon-1}} = C_t \nonumber
\end{eqnarray}
one will similarly find that
\begin{eqnarray}
C_{it} = \left(\frac{P_{it}}{P_t}\right)^{-\epsilon} C_t \nonumber
\end{eqnarray}
with
\begin{eqnarray}
P_t =  \left(\int_{0}^{1} P_{it}^{1-\epsilon}  di\right)^{\frac{1}{1-\epsilon}} \nonumber
\end{eqnarray}

\item In this setting, $P_t$ can be interpreted as a ``Consumer Price Index'' (CPI)


\eit

\end{frame}


\begin{frame}{Another setup III}

\bit
\setlength\itemsep{1.5em} 

\item After having solved the static expenditure allocation problem, the household problem becomes
\begin{eqnarray}
\max_{C_t, N_t, B_{t+1}\}} && E_O \sum_{t=0}^{\infty} \beta^t \left[U(C_t)- V(N_t) \right]\nonumber \\
\text{s.t} && P_t C_t + Q_t B_{t+1} \leq W_t N_t + B_{t} + T_t \nonumber 
\end{eqnarray}
where $C_t$ now is an optimally chosen consumption bundle

\eit

\end{frame}


\begin{frame}{Models vs data}

\bit
\setlength\itemsep{1.5em} 

\item $\Rightarrow$ In these models, not really any distinction between PPI and CPI

\item In the data, they can differ a lot

\begin{figure}
	\centering
	\includegraphics[scale=0.25]{fredgraph.png}
\end{figure}

\eit

\end{frame}




\end{document}