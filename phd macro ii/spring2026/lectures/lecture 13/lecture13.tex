\documentclass[9pt]{beamer}
\usetheme{Boadilla}

\makeatother
\setbeamertemplate{footline}
{
	\leavevmode%
	\hbox{%
		\begin{beamercolorbox}[wd=.4\paperwidth,ht=2.25ex,dp=1ex,center]{author in head/foot}%
			\usebeamerfont{author in head/foot}\insertshortauthor
		\end{beamercolorbox}%
		\begin{beamercolorbox}[wd=.6\paperwidth,ht=2.25ex,dp=1ex,center]{title in head/foot}%
			\usebeamerfont{title in head/foot}\insertshorttitle\hspace*{3em}
			\insertframenumber{} / \inserttotalframenumber\hspace*{1ex}
	\end{beamercolorbox}}%
	\vskip0pt%
}
\makeatletter
\setbeamertemplate{navigation symbols}{}

\usepackage{tikz}
\usetikzlibrary{positioning}

\usepackage{lipsum}
\usepackage{appendixnumberbeamer}

\usepackage[authoryear]{natbib}
\usepackage[latin1]{inputenc}
\usepackage[T1]{fontenc}
\usepackage{caption}
\usepackage{amsmath, amssymb}
\usepackage{epstopdf}
\usepackage{graphicx}
\usepackage{lmodern}
\usepackage{xcolor}
\usepackage{xpatch}
\usepackage{multirow}

\usepackage{amsmath,theorem,amssymb,graphicx, pgfplots, tabularx, placeins}
\usepackage{dsfont}
\usepackage{caption}
%\usepackage{subcaption}
%\usepackage{subcaption}
\setbeamertemplate{caption}{\raggedright\insertcaption\par}
%\setbeamertemplate{footline}[frame number]
\usepackage{csquotes}
\usepackage{bm}
\bibliographystyle{econometrica}
\usepackage[normalem]{ulem}

\usepackage{setspace}


\definecolor{gray(x11gray)}{rgb}{0.75, 0.75, 0.75}


\newcommand{\bit}{\begin{itemize}}
	\newcommand{\eit}{\end{itemize}}
\newcommand{\ben}{\begin{enumerate}}
	\newcommand{\een}{\end{enumerate}}

\newcommand{\bc}{\color{blue}}
\newcommand{\rc}{\color{red}}


\newcommand{\lb}{\label}
\newcommand{\re}{\eqref}

\title[Incomplete Markets in General Equilibrium]{Macroeconomics II, Lecture XIII:\\
	Incomplete Markets in General Equilibrium}
\author{Erik {\"O}berg}
\date{}

\begin{document}

\maketitle

\begin{frame}{Recap}

\bit
\setlength\itemsep{2em}

\item Recap:
\ben
\setlength\itemsep{1em}
\item With incomplete markets, ex-ante homogeneous households will be \bf ex-post heterogeneous \normalfont in terms of $\{c,a,y\}$ $\Rightarrow$ no aggregation into representative agent

\item With incomplete markets, consumption dynamics influenced by a \bf precautionary savings \normalfont motive

%\item With incomplete markets, \bf R has to be lower \normalfont for assets not to diverge
\een

\item Today: we will study the interplay between incomplete markets and the aggregate economy
\bit
	\item That is, we will set up a general-equilibrium model
\eit

\eit

\end{frame}

\begin{frame}{Why general equilibrium? A reminder.}

\bit
\setlength\itemsep{1.5em}

\item PE enables us to explore how households consumption-saving decisions respond to income, risk, price changes, credit availability etc...

\item But, since prices and income processes are endogenous objects, PE does not permit exploring how economy-wide objects respond to fiscal policy, technical change, labor market shocks etc.

\item Economy-wide objects: the distribution of income and wealth, aggregate savings, aggregate demand, the equity premium etc.

\item First-generation GE models focused on steady state: Bewley (1986), Imrohoro\v{g}lu (JPE 1989), Huggett (JEDC 1993) and Aiyagari (QJE 1994)

\item Second generation added aggregate shocks: Krusell-Smith (JPE 1998)

\item Currently, research using business-cycle models with heterogeneous agents grows exponentially

\eit

\end{frame}

\begin{frame}{Agenda}

\ben
\setlength\itemsep{2em}

\item The static Aiyagari model
\ben
	\item Model setup and definition of stationary recursive equilibrium
	\item The Aiyagari diagram
\een

\item Applications
\ben
	\item Can the model explain US wealth inequality?
%	\item Is the equilibrium allocation (constrained) efficient?
	\item Efficiency and Fiscal policy
\een

%\item Adding aggregate dynamics (briefly)
%\ben
%	\item Deterministic transition dynamics with unexpected schocks
%\een

%\item A dynamic application: HANK

\een

\end{frame}


\begin{frame}

\begin{center}
	\huge The Aiyagari model \normalfont
\end{center}

\end{frame}


\begin{frame}{Overview}

\bit
	\setlength\itemsep{2em}
	\item The Aiyagari model $=$ Neoclassical growth model with continuum of households that face uninsurable idiosyncratic income risk 
	
	\item Three ingredients

\ben
\setlength\itemsep{1.5em}

\item Households that solves an income-fluctuations problem
\bit
	\item Determines supply of labor and assets, demand for consumption
\eit

\item Competitive firms that maxmize production using a production function
\bit
	\item Determines demand for labor and assets, supply of consumption
\eit

\item Market clearing conditions
\bit
	\setlength\itemsep{0.5em}
	\item Goods demand $=$ Goods supply
	\item Labor demand $=$ Labor supply
	\item Asset demand $=$ Asset supply
\eit

\een
\eit


\end{frame}


\begin{frame}{Household prefences}

\bit
\setlength\itemsep{2em}

\item A unit mass of households, indexed by $i$, who solve an income-fluctations problem

\item Preferences: households seek to maximize the objective
\begin{eqnarray}
E_0 \sum_{t=0}^{\infty} \beta^t U(C_{it}) \nonumber
\end{eqnarray} 

\item $U(C)$ satisifies the usual regularity conditions

\item Labor supply is exogenous, no disutility from labor

\eit

\end{frame}

\begin{frame}{Household income}

\bit
\setlength\itemsep{2em}

\item Household income: $\epsilon_{it} W_t$

\item Household effective labor supply $\epsilon$ varies stochastically

\item $\epsilon$ has finite support $\epsilon \in \mathcal E= \{\epsilon_0, \epsilon_1, ..., \epsilon_N\}$ and follows a \bf Markov process \normalfont with transititions governed by
\begin{eqnarray}
\pi(\epsilon', \epsilon) = Pr(\epsilon_{t+1}= \epsilon' | \epsilon_{t} = \epsilon) \nonumber
\end{eqnarray}

\item The transition probabilities are the same for all households

\item This formulation of the income process allows for an arbitrary level of persistence (in contrast to transitory-permanent income process consider in last lecture)
\bit
	\item See Tauchen (EL 1986) and Tauchen-Hussey (Ecmtra 1991) for how to map any continuous AR process into a discrete-space Markov process
\eit

\eit

\end{frame}

\begin{frame}{Markov processes I}

\bit
\setlength\itemsep{1.5em}

\item Let $\mathcal E = \{0, 1\}$ (unemployed and employed)

\item Collect transition probabilities in matrix $T$:
\begin{eqnarray}
T = \left( \begin{array}{cc}
\pi(0,0) & \pi(0,1) \\
\pi(1,0) & \pi(1,1)
\end{array}\right) \nonumber
\end{eqnarray}

\item Suppose $t$ is a quarter. What is the interpretation of $\pi(1,0)$? $\pi(0,1)$?

\item Let $\pi_{it}(\epsilon)$ be the probability distribution over an individual household $i$'s state in period $t$
\bit
	\item If we know the household i is unemployed in period $t$, then $ \pi_{it}(\epsilon) = \left( \begin{array}{c}
	1 \\
	0
	\end{array}\right)$
\eit

\item $\pi_{it+s}(\epsilon)$ is given by
\begin{eqnarray}
\pi_{it+s}(\epsilon) = T^s \pi_{it}(\epsilon)\nonumber
\end{eqnarray}
\bit
	\item If we know the household is unemployed in period $t$, then 
\begin{eqnarray}
\bm \pi_{it+1}(\epsilon) = T \left( \begin{array}{c}
1 \\
0
\end{array}\right) = \left( \begin{array}{c}
\pi(0,0) \\
\pi(1,0)
\end{array}\right) \nonumber
\end{eqnarray}
	
\eit

\eit

\end{frame}

\begin{frame}{Markov processes II}

\bit
\setlength\itemsep{1.5em}

\item Let $\Pi_{t}(\epsilon)$ be the distribution of all households across states $\epsilon$ in period $t$. $\Pi_{t+s}(\epsilon)$ is similarly given by
\begin{eqnarray}
\Pi_{t+s}(\epsilon) = T^s \Pi_{t}(\epsilon)\nonumber
\end{eqnarray}

\item Stationary distribution $\Pi^*(\epsilon)$ satisfies
\begin{eqnarray}
\Pi^{*}(\epsilon) = T \Pi^{*}(\epsilon) \nonumber
\end{eqnarray}
%$\Rightarrow$ $\Pi^{*}(\epsilon)$ is the eigenvector of $T$ associated with eigenvalue $1$

\item If $T$ is {\bc regular} (all entries of $T^n$ for some $n>0$ is positive), $\Pi^{*}(\epsilon)$ is given by
\begin{eqnarray}
\Pi^{*}(\epsilon) = \lim_{s \rightarrow \infty} T^s \Pi_{t}(\epsilon)\nonumber
\end{eqnarray}
for any starting distribution $\Pi_{0}$.


\item Implication: there is a unique steady state value of aggregate labor supply, given by
\begin{eqnarray}
L = \sum_{\epsilon \in \mathcal E} \epsilon \Pi^{*}(\epsilon) \nonumber
\end{eqnarray}
\eit

\end{frame}


\begin{frame}{Household constraints}

\bit
\setlength\itemsep{2em}

\item Households face budget constraint:
\begin{eqnarray}
C_{it} + A_{it+1} \leq \epsilon_{it} W_t + R_tA_{it} \nonumber
\end{eqnarray}
\item and credit constraint
\begin{eqnarray}
A_{it+1} \geq -\bar A \nonumber
%\geq - \sum_{t=0}^{\infty} \left[\prod_{s=0}^{t} \frac{1}{(1+r_t)} \right] \epsilon_0\nonumber
\end{eqnarray}
%\item where the parametric restriction in the second inequality ensures that all borrowing is risk-free (see problem set)

\item Notes:
\bit
\setlength\itemsep{0.5em}
\item The savings instrument in this economy is capital, just as in lecture I-II

\item To make the determination of the real interest rate transparent, we assume here that depreciation happens at the firm 

\item $R_t$ is the real gross rate of return on savings in period $t$
\eit

\eit

\end{frame}

\begin{frame}{Firms}

\bit
\setlength\itemsep{1.5em}

\item The representive firm rents labor and capital to produce consumption goods $Y$ using a Cobb-Douglas production function:
\begin{eqnarray}
Y_t = ZF(K_t, L_t) = ZK_t^{\alpha}L_t^{1-\alpha} \nonumber
\end{eqnarray}

\item Firm's problem
\begin{eqnarray}
&& \max_{K_t, L_t} ZF(K_t, L_t)+(1-\delta)K_t - W_t L_t - R_t K_t \nonumber \\
=&& \max_{K_t, L_t} ZF(K_t, L_t) - W_t L_t - (R_t - (1-\delta)) K_t \nonumber \\
=&& \max_{K_t, L_t} ZF(K_t, L_t) - W_t L_t - (r_t + \delta) K_t \nonumber
\end{eqnarray}
where $r_t = R_t-1$

\item In the optimum:
\begin{eqnarray}
r_t +\delta &=&  F_K(K, L) = \alpha \left(\frac{K_t}{L_t}\right)^{\alpha-1} \nonumber \\
W_t &=& F_L(K, L) = (1-\alpha) \left(\frac{K_t}{L_t}\right)^{\alpha} \nonumber
\end{eqnarray}

%\item Resource constraint:
%\begin{eqnarray}
%Y_t = C_t + K_{t+1} - (1-\delta)K_t \nonumber
%\end{eqnarray}
\eit

\end{frame}

\begin{frame}{Market clearing}

\bit
\setlength\itemsep{2em}

\item The goods market clears when
\begin{eqnarray}
\int_{i=0}^{1}C_{it} di + K_{t+1}-(1-\delta)K_t = Y_t \nonumber
\end{eqnarray}

\item The asset market clears when
\begin{eqnarray}
K_t = \int_{i} A_{it} di \nonumber
\end{eqnarray}

\item The labor market clears when
\begin{eqnarray}
L_t = \sum_{\epsilon \in \mathcal E} \epsilon \Pi_t(\epsilon) \nonumber
\end{eqnarray}


\eit

\end{frame}


\begin{frame}{Rational expectations with heterogenous households}

\bit
\setlength\itemsep{1.5em}

\item Note: $R_{t+1}$ and $W_{t+1}$, depends on aggregate savings $A_{it+1}$, which, via household decision functions, in turn depends on the joint distribution of $A_{it}, \epsilon_{it}$ 

\item Denote the PDF of this distribution with $\gamma(A_{it}, \epsilon_{it})$

\item To form rational expectations of $R_{t+1}, W_{t+1}$, {\bc households need to know $\gamma(A_{it}, \epsilon_{it})$} 

\item \bf $\Rightarrow$ $\gamma$ is, in general, a state variable in the household problem \normalfont
\bit
\setlength\itemsep{0.5em}
	\item $\gamma$ is a function, an infinite-dimensional object
	\item Appears bothersome, recall ``curse of dimensionality''
\eit

\item In a stationary equilibrium, however, $\gamma$ is constant and $R_{t}=R$ and $W_{t}=W$ for all $t$

\item \bf $\Rightarrow$ $\gamma$ is not a state variable assuming that we start and remain in steady state forever \normalfont

\eit

\end{frame}

\begin{frame}{Stationary recursive household problem}

\bit
\setlength\itemsep{2em}

\item Defining, computing and illustrating the equilibrium is easier with a recursive setup

\item In a stationary environment, the recursive household problem is
\begin{eqnarray}
V(A,\epsilon) = &\max_{C, A'}& u(C) + \beta \sum_{\epsilon' \in \mathcal E} \pi(\epsilon', \epsilon) V(A',\epsilon') \nonumber \\
&\text{s.t.} & C + A' =  W\epsilon + (1+r)A  \nonumber \\
&&	A'\geq -\bar A  \nonumber
\end{eqnarray}

\item $r, W$ constant $\Rightarrow$ solution functions $V(A,\epsilon), C(A,\epsilon), A'(A,\epsilon)$ are time-invariant

\eit

\end{frame}

\begin{frame}{Stationary distribution}

\bit
\setlength\itemsep{1.5em}

\item A stationary equilibrium features a constant distribution $\gamma(A, \epsilon)$ with support $\bm{\mathcal A} \times \bm{\mathcal E}$

\item Define the transition function $Q((A',\epsilon'), (A, \epsilon))$, Q is given by
\begin{eqnarray}
Q((A',\epsilon'), (A, \epsilon)) =  I_{A'=A'(A,\epsilon)} \pi(\epsilon', \epsilon) \nonumber
\end{eqnarray}

\item Importantly, {\bc $Q$ is generated by the household policy functions}

\item $\gamma'$ is then given by
\begin{eqnarray}
\gamma'(A', \epsilon') = \int_{\bm{\mathcal A} \times \bm{\mathcal E}} Q((A',\epsilon'), (A, \epsilon)) d\gamma \nonumber
\end{eqnarray}
where {\bc $d\gamma = \gamma(A, \epsilon) dA d\epsilon$}

\item A stationary distribution $\gamma^*$ maps itself onto itself:
\begin{eqnarray}
\gamma^* = \int_{\bm{\mathcal A} \times \bm{\mathcal E}} Q((A',\epsilon'), (A, \epsilon)) d\gamma^* \nonumber
\end{eqnarray}

\eit

\end{frame}



\begin{frame}{Stationary Recursive Competitive Equilibrium}

\bit
\setlength\itemsep{1.5em}

\item A SRCE is a value function $V(A, \epsilon)$, policy functions $C(A, \epsilon), A'(A, \epsilon)$, a distribution $\gamma(A,\epsilon)$, aggregate capital and labor demand $K, L$, and prices $W,R$, s.t.

\ben
\setlength\itemsep{1.5em}

\item Given $W,R$: the value function $V(A, \epsilon)$ and policy functions $C(A, \epsilon), A'(A, \epsilon)$ solves the household Bellman equation

\item Given $W,R$: $K, L$ solves the firm problem

\item The market for capital and labor clear:
\begin{eqnarray}
K &=& \int_{\bm{\mathcal A} \times \bm{\mathcal E}} A d\gamma \nonumber \\
L &=& \int_{\bm{\mathcal A} \times \bm{\mathcal E}} \epsilon d\gamma \color{red} = \sum_{\epsilon \in E} \epsilon \Pi^{*}(\epsilon) \nonumber
\end{eqnarray}

\item $\gamma(\cdot)$ satisfies
\begin{eqnarray}
\gamma(A', \epsilon) = \int_{\bm{\mathcal A} \times \bm{\mathcal E}} Q((A',\epsilon'), (A, \epsilon)) d\gamma \nonumber
\end{eqnarray}
where $Q$ is defined in the previous slide
\een
\eit

\end{frame}

\begin{frame}{Does an SCRE exist?}

\bit
\setlength\itemsep{2em}

\item Firm and household problem are well-defined: solutions exist and policy functions are continuous in $R$ and $W$

\item Labor supply is constant: Solving for $W$ that clears the labor market is done by invoking firm optimality given a market clearing level of capital

\item The question boils down to: does there exist an $R$ s.t. the asset market clears?

\item Restated: does asset demand and asset supply curves cross at least once?

\eit

\end{frame}


\begin{frame}{Does an SCRE exist?}

\bit
\setlength\itemsep{1.5em}

\item To show existence, we will draw the \bf Aiyagari diagram \normalfont in $r,K$-space {\rc (Do on whiteboard)} 
\bit
	\item $r = R -1$
\eit

\item Capital demand $K(r)$ is given by firm F.O.C.: $\alpha \left(\frac{K(r)}{L}\right)^{\alpha-1} = r+\delta $
\ben
\setlength\itemsep{0.5em}
	\item decreasing
	\item $r \to -\delta: K(r) \to \infty$
	\item $r \to \infty: K(r) \to  0$
\een

\item Capital supply $A(r)$ is given by household savings: $A(r) = \int_{\bm{\mathcal A} \times \bm{\mathcal E}} A'(A,\epsilon; r) d\gamma $
\ben
\setlength\itemsep{0.5em}
\item The stationary state replicates itself forever: $A(r) = $ long-run mean level of assets  
\item At the lower end, we have that $\lim_{r \to -1} A(r) = - \bar A$
\item What about the upper end?
\een

\item {\color{white} Therefore: Asset supply and asset demand crosses at least once}

\eit

\end{frame}


\begin{frame}{Asset convergence: No income risk}

\bit
\setlength\itemsep{1.5em}

\item Without any income risk, $\epsilon=0$, it is straightforward to show that if
\ben
\setlength\itemsep{0.5em}
\item $\beta (1+r) > 1$, consumption and savings grow indefinetly
\item $\beta (1+r) = 1$, consumption is constant, savings is bounded
\item $\beta (1+r) < 1$, consumption and savings decrease until borrowing constraint binds, in the limt $C=\bar Y$
\een

\item Why? If credit constraint does not bind, then the household problem is characterized by
\begin{eqnarray}
U_c(C) = \beta (1+r) U_c(C') \nonumber
\end{eqnarray}

\item Implication: with complete markets (or no income risk), steady state interest rate $R=\frac{1}{\beta}$

\eit

\end{frame}


\begin{frame}{Asset convergence: With income risk}

\bit
\setlength\itemsep{1.5em}

\item With uninsurable income risk, household have a stronger savings motive

\item Implication: assets and consumption grow without bound for lower levels of interest rate $R$

\item For example, consider the case with $\beta R = 1$

\item Optimality conditions
\begin{eqnarray}
U_c(C) &=&  E\left[V_a(A', \epsilon')\right] + \mu \nonumber \\
V_a(A', \epsilon') &=& U_c(C') \nonumber
\end{eqnarray}
or
\begin{eqnarray} 
V_a(A, \epsilon) &\geq &  E\left[V_a(A', \epsilon')\right] \nonumber
\end{eqnarray}

\eit

\end{frame}

\begin{frame}{Asset convergence: With income risk}

\bit
\setlength\itemsep{1.5em}

\item Recall: $\epsilon_N$ is the maximum of $\epsilon$

\item Suppose that the solution is bounded: the maximum of $A'$ is $A_{max}$

\item Evaluating optimality condition at point $A_{max}, \epsilon_N$:
\begin{eqnarray}
V_a(A_{max}, \epsilon_N) &\geq& E\left[V_a(A'(A_{max}, \epsilon), \epsilon')\right]\nonumber\\
&>& E\left[V_a(A_{max}, \epsilon_N)\right]\nonumber \text{ using that } V_{aa}<0 \nonumber\\
&=& V_a(A'_{max}, \epsilon_N) \nonumber 
\end{eqnarray}
which is a contradiction. Ergo $A'$ does not have an upper bound $\Rightarrow$ asset holdings diverge to infinity

\eit

\end{frame}

\begin{frame}{Does an SCRE exist?}

\bit
\setlength\itemsep{1.5em}

\item To show existence, we will draw the \bf Aiyagari diagram \normalfont in $r,K$-space {\rc (Do on whiteboard)} 
\bit
\item $r = R -1$
\eit

\item Capital demand $K(r)$ is given by firm F.O.C.: $\alpha \left(\frac{K(r)}{L}\right)^{\alpha-1} = r+\delta $
\ben
\setlength\itemsep{0.5em}
\item decreasing
\item $r \to -\delta: K(r) \to \infty$
\item $r \to \infty: K(r) \to  0$
\een

\item Capital supply $A(r)$ is given by household savings: $A(r) = \int_{\bm{\mathcal A} \times \bm{\mathcal E}} A'(A,\epsilon; r) d\gamma $
\ben
\setlength\itemsep{0.5em}
\item The stationary state replicates itself forever: $A(r) = $ long-run mean level of assets  
\item At the lower end, we have that $\lim_{r \to -1} A(r) = - \bar A$
\item What about the upper end?
\een

\item {\color{white} Therefore: Asset supply and asset demand crosses at least once}

\eit

\end{frame}



\begin{frame}{Does an SCRE exist?}

\bit
\setlength\itemsep{1.5em}

\item To show existence, we will draw the \bf Aiyagari diagram \normalfont in $r,K$-space {\rc (Do on whiteboard)} 
\bit
\item $r = R -1$
\eit

\item Capital demand $K(r)$ is given by firm F.O.C.: $\alpha \left(\frac{K(r)}{L}\right)^{\alpha-1} = r+\delta $
\ben
\setlength\itemsep{0.5em}
\item decreasing
\item $r \to -\delta: K(r) \to \infty$
\item $r \to \infty: K(r) \to  0$
\een

\item Capital supply $A(r)$ is given by household savings: $A(r) = \int_{\bm{\mathcal A} \times \bm{\mathcal E}} A'(A,\epsilon; r) d\gamma $
\ben
\setlength\itemsep{0.5em}
\item The stationary state replicates itself forever: $A(r) = $ long-run mean level of assets  
\item At the lower end, we have that $\lim_{r \to -1} A(r) = - \bar A$
\item {\bc At the upper end, $\lim_{r \to \frac{1}{\beta}-1} A(r) = \infty$}
\een

\item {\color{white} Therefore: Asset supply and asset demand cross at least once}
\eit

\end{frame}



\begin{frame}{Does an SCRE exist?}

\bit
\setlength\itemsep{1.5em}

\item To show existence, we will draw the \bf Aiyagari diagram \normalfont in $r,K$-space {\rc (Do on whiteboard)} 
\bit
\item $r = R -1$
\eit

\item Capital demand $K(r)$ is given by firm F.O.C.: $\alpha \left(\frac{K(r)}{L}\right)^{\alpha-1} = r+\delta $
\ben
\setlength\itemsep{0.5em}
\item decreasing
\item $r \to -\delta: K(r) \to \infty$
\item $r \to \infty: K(r) \to  0$
\een

\item Capital supply $A(r)$ is given by household savings: $A(r) = \int_{\bm{\mathcal A} \times \bm{\mathcal E}} A'(A,\epsilon; r) d\gamma $
\ben
\setlength\itemsep{0.5em}
\item The stationary state replicates itself forever: $A(r) = $ long-run mean level of assets  
\item At the lower end, we have that $\lim_{r \to -1} A(r) = - \bar A$
\item {\bc At the upper end, $\lim_{r \to \frac{1}{\beta}-1} A(r) = \infty$}
\een

\item Therefore: {\bc Asset supply and asset demand cross at least once}
\eit
	
\end{frame}



\begin{frame}{Incomplete vs complete markets}

\bit
\setlength\itemsep{2em}

\item Consider the steady state of a representative-agent (complete-market) economy

\item Euler equation implies:
\begin{eqnarray}
&& u'(C_t) = \beta (1+r) u'(C_{t+1}) \nonumber \\
&\Rightarrow& r=\frac{1}{\beta}-1 \nonumber
\end{eqnarray}

\item Asset demand still given by the same firm F.O.C.: $\alpha \left(\frac{K(r)}{L}\right)^{\alpha-1} = r+\delta $

\item Result: {\bc The addition of uninsurable income risk raises $K$ and depresses $r$}
\eit

\end{frame}


%\begin{frame}{Is the SCRE unique?}
%
%\bit
%\setlength\itemsep{2em}
%
%\item Under the assumptions made, there does not exist results on whether $E(a(r))$ is monotonically increasing
%
%\item Increasing $r$ has both income and substitution effects whose relative strength may depend on the level of $r$ 
%
%\item Uniqueness is, at this stage, not guaranteed, although I know of no examples where two stationary states have been a problem in practice
%\bit
%	\item Ben Moll has recently teamed up with some mathematicians and made some progress on uniqueness, see Achdou et al (2017)
%\eit
%
%\eit
%
%\end{frame}


\begin{frame}{Stationary state: computational algorithm}

\ben
\setlength\itemsep{1.5em}

\item Set initial $r_0$ which satisfies $-\delta<r_0<\frac{1}{\beta}-1$

\item Solve for $K$ and then $W$ from firm F.O.C.

\item Solve household problem given $\{r_0,W\}$, e.g., by value function iteration
\bit
	\item This gives you $A'(A, \epsilon; r_0)$
\eit

\item Solve for $\gamma_{T}$ for some large $T$ using the law of motion
\begin{eqnarray}
\gamma_{t+1} = \int_{\bm{\mathcal A} \times \bm{\mathcal E}} Q((A',\epsilon'), (A, \epsilon)) d\gamma_t \nonumber
\end{eqnarray}
\bit
	\item Note: this involves picking initial distribution $\gamma_0$ and then simulating using the computed policy functions
\eit

\item Compute $A_{agg}=\int_{\bm{\mathcal A} \times \bm{\mathcal E}} A'(a,\epsilon; r) d\gamma_T$

\item If $A_{agg}>K$, set $r_1<r_0$ and repeat. If $A_{agg}<K$, set $r_1>r_0$ and repeat until $A_{agg}$ is sufficiently close to $K$

\een

\end{frame}

%\begin{frame}{Computational algorithm: comment}
%
%\bit
%\setlength\itemsep{1.5em}
%
%\item The algorithm builds on the assumption that the economy converges to a stationary state
%
%\item We have not prooved this here, but practical experience suggest that this is usually the case
%
%\eit
%
%\end{frame}


\begin{frame}{A note: Aiyagari vs Huggett}

\bit
\setlength\itemsep{1.5em}

\item An incomplete-markets model without physical capital is sometimes referred to as a Huggett model, after Huggett (JEDC 1993)

\item Example production function:
\begin{eqnarray}
Y_t = F(L_t) = Z L_t^{1-\alpha} \nonumber
\end{eqnarray}

\item In the case of zero net asset suppy (e.g., no government debt, money, foreign borrowing etc), asset market clearing is simply
\begin{eqnarray}
\int_{\bm{\mathcal A} \times \bm{\mathcal E}} A d\gamma \nonumber = 0 \nonumber
\end{eqnarray}
\bit
	\item Zero net asset supply: households can borrow and save in partial equilibrium, but in net, aggregate savings are zero 
\eit

\item Aiyagari diagram the same, except that aggregate demand curve is vertical at zero
\bit
	\item See your problem set
\eit

\eit

\end{frame}

\begin{frame}

\begin{center}
	\huge Applications \normalfont
\end{center}

\end{frame}

\begin{frame}{The Aiyagari model put to work}

\bit
\setlength\itemsep{2em}

\item Now we have an incomplete markets stationary equilibrium, lets put it to work

\item Questions:
\bit
\setlength\itemsep{0.5em}

	\item Does the equilibrium wealth distribution match the data?
	
	\item Can fiscal policy be welfare-improving? Redistributive taxation? Public debt management?
\eit

\eit

\end{frame}

\begin{frame}{Wealth inequality}

\bit
\setlength\itemsep{2em}

\item Taken a realistic income process as given, can the model explain the observed wealth inequality?

\item This is theory of wealth inequality entirely based on luck

\item Some agents will experience lucky income draws and accumulate assets to insure against bad draws in the future

\item Some agents will experience not so lucky income draws and deaccumulate assets/borrow to smooth consumption

\item Some agents will experience very unlucky income draws and hit the borrowing constraint

\eit

\end{frame}

\begin{frame}{US wealth inequality}

\begin{figure}
	\centering
	\includegraphics[scale=0.4]{figures/dgr_tab1.pdf}
	\caption*{\footnotesize Numbers measured in \$$*10^3$. From \{Diaz-Giminez\}-Glover-\{Rios-Rull\} (FED Minneapolis 2011). Data from SCF.}
\end{figure}

\begin{figure}
	\centering
	\includegraphics[scale=0.4]{figures/dgr_tab2.pdf}
	\caption*{\footnotesize From \{Diaz-Giminez\}-Glover-\{Rios-Rull\} (FED Minneapolis 2011). Data from SCF.}
\end{figure}

\bit
	\item Main take-away: wealth distribution a lot more concentrated than income distribution
\eit

\end{frame}

%\begin{frame}{US top income share using tax data}
%
%\begin{figure}
%	\centering
%	\includegraphics[scale=0.4]{figures/piketty_inc_ineq.pdf}
%\end{figure}
%
%
%\end{frame}
%
%\begin{frame}{US top wealth share using tax data}
%
%\begin{figure}
%	\centering
%	\includegraphics[scale=0.4]{figures/piketty_w_ineq.pdf}
%\end{figure}
%
%\end{frame}

\begin{frame}{A (standard) parameterization}

\bit
\setlength\itemsep{2em}

	\item $\underbar A=0$ or $\underbar A=$ natural borrowing constraint
	
	\item Income process: 
	\ben
	\item Run Mincer regression
	\begin{eqnarray}
	\log y_{it} = \alpha + \beta X_{it} + \nu_{it}  \nonumber
	\end{eqnarray}
	where $X$ controls for changes in earnings that are predetermined
	
	\item Find a process $\mathcal E = \{\epsilon_0, ..., \epsilon_1\}$, $\pi(\epsilon, \epsilon')$ that matches autocorrelation and variance of $\nu_{it}$ 
	\een

\item $\sigma \in [1,5]$, consistent with experimental evidence on risk aversion
\bit
	\item $\sigma=1$ $\Leftrightarrow$ log utility
\eit

\item Calibrated parameters:
\bit
\item $\alpha$ to match labor share $\frac{WL}{Y}$
\item $\beta$ to match risk-free interest rate $r$
\item $\delta$ to match outside estimates
\eit

\eit

\end{frame}

\begin{frame}{Results}

\bit
\setlength\itemsep{2em}

\item Calibrated model typically has wealth Gini coefficient around 0.4
\bit
	\item US wealth Gini coefficient is 0.8
\eit

\item Suggests that precautionary savings can explain some, but not all, of US wealth inequality
\bit
	\item would be very surprising if it could explain everything...
\eit

\item Failure due to problems at both ends of wealth distribution:
\ben
\setlength\itemsep{0.5em}
\item Very few households close to borrowing constraint
\bit
	\item The welfare loss of not being able ot smooth at all is large
\eit
\item Very few households with savings several multiples of income at top of distribution
\bit
	\item Only rational if you face substantial risk loosing all your income for a long time
\eit
\een

\eit

\end{frame}

\begin{frame}{What is the model missing?}

\bit
\setlength\itemsep{1.5em}

\item Suggestions 
\bit
\setlength\itemsep{1em}
\item Adding means-tested social insurance eliminates saving motive among income-poor (Skinner-Hubbard-Zeldes, JPE 1995)

\item Adding bequest savings as a luxury good generates additional savings among the income-rich (De Nardi, ReStud 2004) 

\item $\beta$-heterogeneity: given some $r$, households with low $\beta$ deaccumulate, households with high $\beta$ accumulate $\rightarrow$ more wealth heterogeneity within income groups (Krusell-Smith, MD 1997; JPE 1998)

\item $r-$heterogeneity:
\bit
	\setlength\itemsep{0.5em}
	\item Similar logic to $\beta$-heterogeneity
	
	\item In the data, large fraction of top wealth is held by entreprenuers. Quadrini (RED 2000) and Cagetti-De Nardi (JPE 2006) model entreprenuers as households with access to high-return investments
	
	\item See also Krusell-Hubmer-Smith (NBERannual 2020); Paulie-Ridder-Westergren (2021)
	
\eit

\item Non-zero probability of very rich people to loose it all (\{Diaz-Giminez\}-Glover-\{Rios-Rull\}, JPE 2003)
\bit
	\item Authors argue that we miss this in standard calibrations due to top coding in survey income data 
\eit

\eit

\eit

\end{frame}


\begin{frame}{Return heterogeneity in Norway}

\begin{figure}
	\centering
	\includegraphics[scale=0.75]{figures/fgmp_1.pdf}
	\caption*{\footnotesize From Fagereng-Guiso-Malacrino-Pistaferri (Ecmtra 2020). Norwegian registry data.}
\end{figure}

\end{frame}

\begin{frame}{Return heterogeneity in Norway}

\begin{figure}
	\centering
	\includegraphics[scale=0.6]{figures/fgmp_2.pdf}
	\caption*{\footnotesize From Fagereng-Guiso-Malacrino-Pistaferri (Ecmtra 2020). Norwegian registry data.}
\end{figure}

\end{frame}


\begin{frame}{Efficiency and policy I}

\bit
\setlength\itemsep{1.5em}

\item Recap: the incomplete-markets equilibrium features a lower $r$ and higher $K$ (and higher $Y$) than the corresponding complete markets equilibrium in which $\beta (1+r) = 1$

\item Unsurprisingly, this reflects that the market allocation is \bf inefficient \normalfont

\item An unconstained social planner would insure all households against idiosyncratic income risk $\Rightarrow$ Constant consumption for all households in the stationary state $\Rightarrow$ $\beta (1+r) =1$
\bit
	\item An unconstained social planner effectively reintroduce the missing market for a complete set of Arrow securities  
\eit

\item Is the stationary equilibrium allocation \bf constrained efficient\normalfont?
\bit
	\setlength\itemsep{0.5em}
	\item Constrained efficiency: Would a social planner, who can use the same savings instruments as the households, choose different policy functions $A'(A, \epsilon), C(A, \epsilon)$ than households choose in the decentralized equilibrium?
	
	\item Davila-Hong-Krusell-\{R�os-Rull\} (Ecmtra 2012): The Aiyagari model is not constrained efficient, the planner wants to instruct well-off households to save more, so to raise the wage of the poor households
	
	\item $\Rightarrow$ K(constrained efficient equilibrium) > K(decentralized equilibrium) > K(efficient equilibrium)
\eit

\eit

\end{frame}


\begin{frame}{Efficiency and policy II}

\bit
\setlength\itemsep{1.5em}

\item Lack of efficiency opens many doors for welfare-improving policy

\item Flod�n-Linde (RED 2001) studies redistributive taxation
\bit
	\setlength\itemsep{0.5em}
	\item A flat labor tax $\tau$ to finance lump sum payments $T$ (``basic income''), provides insurance and so raises welfare
	
	\item At the same time, it depresses incentives to work (labor supply decision easily added to the model)
	
	\item Floden and Linde adds labor supply decision to Aiyagari model and finds optimal level of labor taxes $\tau \approx 0.27$ for US calibration  
\eit

\item A progressive income tax schedule is a even more effective insurance device
\bit
	\setlength\itemsep{0.5em}
	\item Optimal progressivity studied by B�nabou (Ecmtra 2002); Conesa-Krueger (RED 2006); Krueger-Ludwig (AER 2013); Heathcote-Storesletten-Violante (QJE 2017)
\eit


\item Aiyagari-McGrattan (JME 1998) studies government debt management
\bit
	\setlength\itemsep{0.5em}
	\item A higher level of debt $B$ increases the set of liquid assets $\rightarrow$ enables more insurance
	
	\item Higher $B$ crowds out savings in the capital stock $K$ $\rightarrow$ reduces output and consumption
	
	\item Aiyagari and McGrattan find that these effects largely offset eachother, and the welfare effects from varying the overall debt level is small
\eit

\eit

\end{frame}


\begin{frame}{What about dynamics?}
\bit
\setlength\itemsep{1.5em}

\item So far, our analysis has been static: no aggregate shocks


\item Generally, analyzing dynamics requires computational techniques which we do not practice in this course (covered in first second-year course)


\item Central problem: the entire wealth distribution is a state-variable, infinite-dimensional object

\item Useful to know: two computational approaches:
\ben
\setlength\itemsep{1em}

\item Approximate equilibrium (Krusell-Smith JPE, 1998): exploit that knowing just a few moments of the wealth distribution is sufficient to make very good forecast of prices

\item Linearization (Boppart-Krusell-Mitman, JEDC 2018; Auclert-Bardoczy-Rognlie-Straub, Ecmtra 2021): transition dynamics (IRFs) to one-time unexpected shock is a sufficient statistic for solving a first-order approximation, just like in rep-agent RBC
\bit
	\item Computing transition dynamics to one-time unexpected shock is bigger system of equations, byt conceptually the same as computing the steady state 
	
\eit

\een

\eit

\end{frame}

%
%\begin{frame}{Transition dynamics to an MIT schock}
%\bit
%\setlength\itemsep{1.5em}
%
%\item Consider our Ayaigari model which rests in steady state at period $-1$, but where TFP $Z_t$ is unexpectedly hit by a shock $\xi$ in period 0 and then evolves auto-regressively:
%\begin{eqnarray*}
%	\log Z_t = \left\{\begin{array}{cc}
%		0 & \text{if } t<0 \\
%		\xi & \text{if } t=0 \\
%		\rho \log(Z_{t-1}) & \text{if } t > 0
%	\end{array} \right.
%\end{eqnarray*}
%
%\item Suppose we want to know the welfare effect of this shock
%
%\item In the long run, welfare is unaffected
%
%\item But what about the short-run dynamics?
%	
%\item To evaluate welfare effect of reform, we need to calculate the entire transition path
%
%\eit
%
%\end{frame}
%
%%\begin{frame}{Transition dynamics to MIT schock}
%%\bit
%%\setlength\itemsep{1.5em}
%%
%%\item Consider our Ayaigari model, but where we add a tax-and-transfer sytem
%%\bit
%%\item Labor income tax $\tau_t$
%%\item Lump sum transfer $T_t$
%%\item Balanced-budget government constraint: $\tau_tw_tL_t = T_t$
%%\eit
%%
%%\item Assume economy rests in steady state associated with tax level $\tau$, but there is an unexpected tax reform in period $0$:
%%\begin{eqnarray*}
%%	\tau_t = \left\{ \begin{array}{cc}
%%		\tau & \text{if } t<0 \\
%%		\tau' & \text{if } t \geq 0
%%	\end{array} \right.
%%\end{eqnarray*}
%%
%%\item How does economy respond to the shock?
%%\bit
%%\item Transition from old steady state $K(\tau)$ to new steady state $K(\tau')$
%%\eit
%%
%%\item To evaluate welfare effect of reform, we need to calculate the entire transition path
%%
%%\eit
%%
%%\end{frame}
%
%\begin{frame}{Dynamic household problem}
%\bit
%\setlength\itemsep{1.5em}
%
%\item Recursive household problem:
%\begin{eqnarray}
%V_t(A,\epsilon) = &\max_{C, A'}& U(C) + \beta \sum_{\epsilon' \in \mathcal E} \pi(\epsilon', \epsilon) V_{t+1}(A',\epsilon') \nonumber \\
%&\text{s.t.} & C + A' =  W_t\epsilon + R_t A  \nonumber \\
%&&	a'\geq - \bar A  \nonumber
%\end{eqnarray}
%\item Solution no longer time-invariant
%
%\item Important: aggregate state is not stochastic, so time $t$ is a sufficient state variable for aggregate state (not true when we have aggregate risk as opposed to a deterministic shock)
%
%\eit
%
%\end{frame}
%
%
%\begin{frame}{Recursive Competitive Equilibrium}
%
%\bit
%\setlength\itemsep{1.5em}
%
%\item A RCE is a \emph{sequence} of value functions $\{V_t(A, \epsilon)\}_{t=0}^{\infty}$, policy functions $\{C_t(A, \epsilon), A_{t+1}(A, \epsilon)\}_{t=0}^{\infty}$, distributions $\{\gamma_t(A,\epsilon)\}_{t=0}^{\infty}$, aggregate capital and labor demand $\{K_t, L_t\}_{t=0}^{\infty}$ and prices $\{W_t,R_t\}_{t=0}^{\infty}$ s.t.
%
%\ben
%\setlength\itemsep{0.5em}
%
%\item Given $\{W_t,R_t\}_{t=0}^{\infty}$: the value function $V_t(A, \epsilon)$ and policy functions $C_t(A, \epsilon), A_{t+1}(A, \epsilon)$ solve the household Bellman equation
%
%\item Given $\{W_t,R_t\}_{t=0}^{\infty}$: $K_t, L_t$ solve the firm problem
%
%\item The market for capital and labor clear:
%\begin{eqnarray}
%K_{t+1} &=& \int_{\bm{\mathcal A} \times \bm{\mathcal E}} A_{t+1}(A, \epsilon) d\gamma_t(A,\epsilon) \nonumber \\
%L_t &=&  \int_{\bm{\mathcal A} \times \bm{\mathcal E}} \epsilon d\gamma_t = \sum_{\epsilon \in \bm{\mathcal E}} \epsilon \Pi^{*}(\epsilon) = L \nonumber
%\end{eqnarray}
%
%\item $\gamma_t(A,\epsilon)$ satisfies
%\begin{eqnarray}
%\gamma_{t+1} =  \int_{\bm{\mathcal A} \times \bm{\mathcal E}} Q_t((A',\epsilon'), (A, \epsilon)) d\gamma_t \nonumber
%\end{eqnarray}
%where $Q_t$ is defined as 
%\begin{eqnarray}
%Q_t((A',\epsilon'), (A, \epsilon)) =  I_{A'=A_{t+1}(a,\epsilon)} \pi(\epsilon', \epsilon) \nonumber
%\end{eqnarray}
%\een
%\eit
%
%\end{frame}
%
%\begin{frame}{Computational algorithm}
%
%\ben
%\setlength\itemsep{0.5em}
%
%\item Calculate the steady state following previous recipe
%
%\item Fix $T$ (e.g., $T=t+200$) and assume that the economy has converged back to the steady state after $T$ periods
%
%\item Guess a sequence of capital $\{\hat K_s\}_{s=t}^{T}$ (e.g., take the solution from corresponding representative agent economy) and calculate $\{\hat W_s, \hat R_s\}_{t=t}^{T}$ from firm F.O.C. 
%
%\item Solve household problem by backwards induction. Solve for period $T-1$ given $T$:
%\begin{eqnarray}
%\hat V_{T-1}(A,\epsilon) = &\max_{C, A'}& U(C) + \beta \sum_{\epsilon' \in \bm{\mathcal E}} \pi(\epsilon', \epsilon) \hat V_{T}(A',\epsilon') \nonumber \\
%&\text{s.t.} & C + A' =  \hat W_t\epsilon + \hat R_t A \nonumber \\
%&&	A'\geq - \bar A  \nonumber
%\end{eqnarray}
%and similarly for period $T-2$ and so on...
%
%\item Solve for $\gamma_{t}$ using the law of motion
%\begin{eqnarray}
%\hat \gamma_{t+1} =  \int_{\bm{\mathcal A} \times \bm{\mathcal E}} \hat Q_t((A',\epsilon'), (A, \epsilon)) d \hat \gamma_t \nonumber
%\end{eqnarray}
%
%\item Compute $\hat A_{agg, t+1}=\int_{\mathcal A \times \mathcal E} A_{t+1}(A_t,\epsilon; R_t) d\gamma_t$
%
%\item Calculate $\max_{0<t<T} |\hat A_{agg, t+1}-\hat K_{t+1}|$. If sufficently small, stop. If not sufficiently small, repeat with new guess $\hat K_{t+1, new} = \frac{1}{2} (\hat K_{t+1} + \hat A_{agg, t+1})$
%
%\een
%
%\end{frame}
%
%
%\begin{frame}{Transition dynamics: Comments}
%
%\bit
%\setlength\itemsep{2em}
%
%\item Main point: computing transition dynamics to one-time MIT shock is conceptually the same as computing the steady state 
%
%\item We used the TFP shock as an  example, can be applied to any other shock
%
%
%\eit
%
%\end{frame}
%
%
%
%\begin{frame}{From transition dynamics to economy with aggregate risk I}
%
%\bit
%\setlength\itemsep{1.5em}
%\item Having computed deterministic transition dynamics, computing the first-order approximation of an economy with aggregate shock process is straightforward (Boppart-Krusell-Mitman, JEDC 2018)
%
%
%\item Consider a shock $\xi_t$ to TFP $z_t$:
%\begin{eqnarray}
%\log(z_t) = \rho \log(z_{t-1}) + \xi_t \nonumber
%\end{eqnarray}
%
%\item Transition dynamics gives you the time path of aggregate consumption in response to a particular shock: $C_{t+s}(\xi_t)$ for $s\geq 0$
%
%\item In the first-order approximation, the aggregate evolution of $C_t$ is just the sum of the deterministic responses to the sequence of past shocks $C_t = \sum_{k=0}^{t} C_{t}(\xi_{t-k})$
%
%\item This is exactly like the simulation of the stochastic evolution of our rep-agent economies in Lectures I-IV
%
%\item Auclert-Bardoczy-Rognlie-Straub (Ecmtra 2021) generalize this insight
%
%\eit
%
%\end{frame}
%
%
%\begin{frame}{From transition dynamics to economy with aggregate risk II}
%
%\bit
%\setlength\itemsep{1.5em}
%\item Although your model might have a lot of non-linearities at the micro level, aggregate variables might still be close to linear in response to aggregate shocks, making first-order approximation a valid approach
%
%\item As indicated, linearization methods are a quite recent phenonomon (quite surprisingly)
%
%\item ``Traditional'' method: Krusell-Smith (JPE, 1998), which also allows computing non-linear solutions
%\bit
%\setlength\itemsep{0.5em}
%	\item Key insight: although the whole wealth-income distribution is relevant, knowing a few moments might be enough to make a very accurate forecast of future prices
%\eit
%\eit
%
%\end{frame}


\begin{frame}{Summary}
\bit
\setlength\itemsep{2em}

\item Ayiagari model = Neoclassical growth model with uninurable earnings risk

\item Key prediction: equilibrium real interest rate lower, due to precautionary savings motive

\item Due to incomplete markets, equilibrium is neither efficient nor constrained efficient

\item Very general framework for exploring  the joint distribution of earnings, consumption and wealth, and their interaction with macroeconomic dynamics
\bit
\setlength\itemsep{0.5em}
	\item The literature using incomplete-market models for investigating macroeconomic dynamics has exploded

	\item To conclude the course, I will briefly talk in next class about one area of rapidly evolving research: Heterogeneous-Agent New-Keynesian (HANK) models
	\bit
		\item In particular, we will look at a HANK model that also has a frictional labor market, integrating all elements from this course
	\eit
\eit

\eit

\end{frame}


\end{document}






