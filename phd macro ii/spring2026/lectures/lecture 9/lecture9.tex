\documentclass[9pt]{beamer}
\usetheme{Boadilla}

\makeatother
\setbeamertemplate{footline}
{
	\leavevmode%
	\hbox{%
		\begin{beamercolorbox}[wd=.4\paperwidth,ht=2.25ex,dp=1ex,center]{author in head/foot}%
			\usebeamerfont{author in head/foot}\insertshortauthor
		\end{beamercolorbox}%
		\begin{beamercolorbox}[wd=.6\paperwidth,ht=2.25ex,dp=1ex,center]{title in head/foot}%
			\usebeamerfont{title in head/foot}\insertshorttitle\hspace*{3em}
			\insertframenumber{} / \inserttotalframenumber\hspace*{1ex}
	\end{beamercolorbox}}%
	\vskip0pt%
}
\makeatletter
\setbeamertemplate{navigation symbols}{}


\usepackage{lipsum}
\usepackage{appendixnumberbeamer}

\usepackage[authoryear]{natbib}
\usepackage[latin1]{inputenc}
\usepackage[T1]{fontenc}
\usepackage{caption}
\usepackage{amsmath, amssymb}
\usepackage{epstopdf}
\usepackage{graphicx}
\usepackage{lmodern}
\usepackage{xcolor}
\usepackage{xpatch}
\usepackage{multirow}

\usepackage{amsmath,theorem,amssymb,graphicx, pgfplots, tabularx, placeins}
\usepackage{dsfont}
\usepackage{caption}
%\usepackage{subcaption}
%\usepackage{subcaption}
\setbeamertemplate{caption}{\raggedright\insertcaption\par}
%\setbeamertemplate{footline}[frame number]
\usepackage{csquotes}
\usepackage{bm}
\bibliographystyle{econometrica}
\usepackage[normalem]{ulem}

\usepackage{setspace}


\definecolor{gray(x11gray)}{rgb}{0.75, 0.75, 0.75}


\newcommand{\bit}{\begin{itemize}}
	\newcommand{\eit}{\end{itemize}}
\newcommand{\ben}{\begin{enumerate}}
	\newcommand{\een}{\end{enumerate}}


\newcommand{\lb}{\label}
\newcommand{\re}{\eqref}

\newcommand{\bc}{\color{blue}}
\newcommand{\rc}{\color{red}}

\title{Macroeconomics II, Lecture IX: \\
	Diamond-Mortensen-Pissarides: Statics}
\author{Erik {\"O}berg}
\date{}

\begin{document}
	
	\begin{frame}
	\maketitle
\end{frame}

\section{Introduction}


\begin{frame}{Recap and motivation}

\bit
\setlength\itemsep{2em}

\item Last lecture: framework for studying wage dispersion in a fricitional labor market

\item Today: framework for studying unemployment in a frictional labor market

\item Unemployment dynamics in McCall followed from exogenous arrival rates of offers/separation shocks.

\item To understand the underlying forces that govern unemployment dynamics, we need to endogenize these arrival rates.


\eit

\end{frame}



\begin{frame}{The Diamond-Mortensen-Pissarides (DMP) model}

\bit
\setlength\itemsep{2em}

\item DMP emphasizes how jobs are endogeneously created and destructed


\item Key references: Diamond (JPE 1981; JPE 1982; ReStud 1982), Mortensen (AER 1982; NBER 1982) and Pissarides (ReStud 1984; AER 1985)

\item Sometimes referred to as the ``Pissarides model'' or the ``Mortensen-Pissarides model''

\item Large literature testing its assumptions/implications and proposing extensions

\eit

\end{frame}

\begin{frame}{DMP: agenda}

\ben
\setlength\itemsep{1.5em}

\item The matching function


\item Static DMP with exogenous separations


\item Static DMP with endogenous separations

%\item Efficiency (briefly)
%\bit
%\setlength\itemsep{0.5em}
%	\item Value functions
%	\item Wage determination
%	\item Equilibrium
%	\item Comparative statics
%\eit

%\item Dynamics
%\bit
%\setlength\itemsep{0.5em}
%	\item Steady-state approximation of dynamic equilibrium
%	\item Unemployment volatility puzzle
%\eit

\een

\end{frame}


\begin{frame}

\begin{center}
	\huge The matching function \normalfont
\end{center}

\end{frame}



\begin{frame}{The job-finding rate and market tightness}

\bit
\setlength\itemsep{2em}

\item Hypothesis: job search is a competitive process - it takes less time for a worker to find a job when demand is abundant and supply is low 

\item Supply $=$ the number of unemployed $U$

\item Demand $=$ the number of vacancies $V$

\item Define \emph{vacancy rate} as $v = \frac{V}{P}$, where $P$ is number of labor force participants 

\item Define market tightness:
\begin{eqnarray}
\theta \equiv \frac{V}{U} = \frac{v}{u} \nonumber
\end{eqnarray}
\bit
	\item The market is tight when demand is high relative to supply
\eit

\item {\bc Testable implication}: Job-finding rate should be positively correlated with $\theta$

\eit
\end{frame}



\begin{frame}{Corr(Job-finding rate, market tightness)}

\begin{figure}
	\centering
	\includegraphics[scale=0.5]{figures/rs_2011_fig10.pdf}
	\caption*{\footnotesize From Rogerson and Shimer (Handbook LE 2011).}
\end{figure}

\end{frame}




\begin{frame}{The aggregate matching function}

\bit
\setlength\itemsep{1.5em}

\item The {\bc aggregate matching function} endogenizes the job-finding rate via the competition analogy

\item $M=M(U,V)$ gives the \emph{flow rate of matches} as a function of current level of unemployment and vacant positions

\item \# matches in $\Delta t = M(U,V) \Delta t$  

\item $M(\cdot)$ should have two basic properties
\bit
\item increasing in both arguments

\item concave in both arguments
\eit

\item Common assumption: $M(\cdot)$ is homogeneous of degree 1
\bit
	\setlength\itemsep{0.5em}
	\item Big gain in tractability 
	\item Support in aggregate time series data (Petrongolo-Pissarides, JEL 2001)
	\item Without explicit microfoundations tied to micro evidence, it is difficult to assess this assumption
		\bit
			\item \{Carillo-Tudela\}-Gartner-Kaas (2020) use German micro data, Skandalis (2019) use French micro data, exciting research area!
		\eit
\eit

\eit
\end{frame}



\begin{frame}{Job-finding and job-filing rates}

\bit
\setlength\itemsep{2em}

\item With random matching, the aggregate matching function delivers a job-finding and job-filing rate

\item Homogeneity of degree 1 implies
\bit
\setlength\itemsep{0.5em}
\item rate at which unemployed worker meets vacant firms: $\lambda_u=\frac{M(U,V)}{U} = M(1, \frac{V}{U}) = M(1, \theta)$
\item rate at which vacant firm meets unemployed workers: $\lambda_v=\frac{M(U,V)}{V} = M(\frac{U}{V},1) = M(\theta^{-1},1)$
\eit

\item $\lambda_u=\lambda_u(\theta)$ increasing and concave, $\lambda_v=\lambda_v(\theta)$ decreasing (and often convex)

\item Note that $\lambda_u(\theta) = \theta \lambda_v(\theta)$


\eit
\end{frame}


\begin{frame}{Example: Cobb-Douglas}

\bit
\setlength\itemsep{2em}

\item Common functional form: Cobb-Douglas $M(U,V) = AU^{\alpha}V^{1-\alpha}$

\item Satisfies all assumptions: increasing, concave, homogeneous of degree 1

\item Job-finding and job-filing rates:
\begin{eqnarray}
\lambda_u(\theta) &=& A \theta^{1-\alpha} \nonumber \\
\lambda_v(\theta) &=& A \theta^{-\alpha} \nonumber
\end{eqnarray}

\item Parameteric interpretation:
\bit
	\item $A$: aggregate matching efficiency
	
	\item $\alpha$: elasticity of matches w.r.t. unemployment
\eit

\item Note that $\log \lambda_u = \log A + (1-\alpha) \log \theta$
\bit
\item Estimate $A$ and $\alpha$ by regressing the (log) job-finding rate on (log) market tightness
\eit

\eit
\end{frame}

%
%\begin{frame}{Estimating matching function parameters}
%
%\begin{figure}
%	\centering
%	\includegraphics[scale=0.5]{figures/rs_2011_fig10.pdf}
%	\caption*{\footnotesize From Rogerson and Shimer (2011).}
%\end{figure}
%
%\bit
%	\item Regressing one line on the other yields $\alpha = 0.58$
%\eit
%
%\end{frame}

\begin{frame}{Matching function implies a steady-state relationship between $v$ and $u$}

\bit
\setlength\itemsep{1.5em}

\item Unemployment dynamics:
\begin{eqnarray}
\dot{u} &=& \sigma (1-u) - \lambda_u(\theta)u \nonumber
\end{eqnarray}

\item Steady state:
\begin{eqnarray}
u &=& \frac{\sigma}{\sigma + \lambda_u(\theta)} \nonumber
\end{eqnarray}

\item With Cobb-Douglas:
\begin{eqnarray}
u &=& \frac{\sigma}{\sigma + A \left(\frac{v}{u}\right)^{1-\alpha} }\nonumber
\end{eqnarray}

\item Rewrite $v=f(u)$:
\begin{eqnarray}
v = \left( \frac{\sigma}{ A}\right)^{\frac{1}{1-\alpha}} \left(u^{-\alpha}(1-u)\right)^{\frac{1}{1-\alpha}} \nonumber
\end{eqnarray}

\item Properties of $f$: for $0<u<1$, we have $f>0$, $f'<0$, $f''>0$ {\rc (Draw graph on Whiteboard)}
\bit
	\item $f'<0 \Rightarrow$ $Corr(u,v)<0$
	%\item How does a decrease in matching efficiency $A$ look like?
\eit

\eit
\end{frame}

\begin{frame}{$Corr(u,v)$ in the data}

\begin{figure}
	\centering
	\includegraphics[scale=0.5]{figures/rs_2011_fig9.pdf}
	\caption*{\footnotesize From Rogerson and Shimer (Handbook LE 2011).}
\end{figure}

\bit
	\item Negative correlation referred to as the \bf Beveridge curve \normalfont
\eit

\end{frame}

%\begin{frame}{Beverdige curve analytics}
%
%\bit
%\setlength\itemsep{1.5em}
%
%\item Beveridge curve is usually depicted in $\{u,v\}$-space
%
%\item With Cobb-Douglas:
%\begin{eqnarray}
%u &=& \frac{\sigma}{\sigma + \lambda_u(\theta)} \nonumber \\
%&=& \frac{\sigma}{\sigma + A \left(\frac{v}{u}\right)^{1-\alpha} }\nonumber
%\end{eqnarray}
%
%\item Rewrite $v=f(u)$:
%\begin{eqnarray}
%v = \left( \frac{\sigma}{ A}\right)^{\frac{1}{1-\alpha}} \left(u^{-\alpha}(1-u)\right)^{\frac{1}{1-\alpha}} \nonumber
%\end{eqnarray}
%
%\item Properties of $f$: for $0<u<1$ (Draw picture)
%\bit
%	\item $f>0$ 
%	\item $f'<0$
%	\item $f''>0$
%\eit
%
%\item How does a decrease in matching efficiency $A$ look like? 
%
%\eit
%\end{frame}



%\begin{frame}{The Beveridge curve in $u,v$-space}
%
%\begin{center}
%\includegraphics[scale=0.2]{figures/beveridge_shimer.jpg}
%\end{center}
%
%
%\bit
%\item From Shimer's website
%\item Big literature investigating the apparent decline in US matching efficiency
%
%\eit
%\end{frame}
%
%\begin{frame}
%
%\begin{center}
%	\huge DMP Statics \normalfont
%\end{center}
%
%\end{frame}



\begin{frame}

\begin{center}
	\huge Static Equilibrium \normalfont
\end{center}

\end{frame}




\begin{frame}{DMP: basic elements}

\bit
\setlength\itemsep{1.5em}

\item Continuous time; infinitely lived agents

\item Workers: employed or unemployed

\item Firms: non-operating, vacant or filled

\item A filled firm has one employee, producing instantaneous output flow $y$

\item {\bc Exogenous job-separation rate}
\bit
	\item For endogenous separations, see you problem set
\eit

\item {\bc Endogenous job-finding rate}

\item We will abstract from any heterogeneity, closing down wage dispersion
\bit
\setlength\itemsep{0.5em}
	\item For the case with stochastic firm productivity, see Pissarides (AER 1985)
	
	\item You can also integrate DMP with Burdett-Mortensen
\eit

\eit

\end{frame}


\begin{frame}{DMP: defining elements}

\bit
\setlength\itemsep{2em}


\item \bf Matching function \normalfont
\bit
\setlength\itemsep{0.5em}
	\item Increasing, concave and homogenous of degree 1 
	\item Determines the job-finding and filing rates as function of $\theta$
\eit

\item \bf A wage setting rule \normalfont
\bit
\setlength\itemsep{0.5em}
	\item Our starting point: {\bc Nash Bargaining}
	\item Because of matching frictions, a match creates a surplus
	\item Wages are set to split the match surplus between workers and firms 
	\item Contrasts with wage posting, as in Burdett-Mortensen
\eit

\item \bf Free entry \normalfont
\bit
	\item The vacancy market is competitive: the value of opening a vacancy must be $0$ in equilibrium
\eit

\eit

\end{frame}




\begin{frame}{Worker values}

\bit
\setlength\itemsep{2em}

\item Now: static analysis (dynamics later)

\item Workers take no decisions and there is no wage distribution.

\item Worker values are simply
\begin{eqnarray}
rW &=& w  + \sigma \left(U-W \right) \nonumber \\ 
rU &=& b  + \lambda_u(\theta) \left(W-U\right) \nonumber
\end{eqnarray}

\item continuous-time structure always the same: flow value $=$ flow benefit + (flow probability of event) $\times$ (change of value from event)


\eit
\end{frame}

\begin{frame}{Firm values}

\bit
\setlength\itemsep{2em}

\item Similarly, we have firm value functions
\begin{eqnarray}
rJ &=& (y-w)  + \sigma \left(V-J\right) \nonumber \\ 
rV &=& -c  + \lambda_v(\theta) \left(J-V\right) \nonumber
\end{eqnarray}

\item $c =$ vacancy posting cost; summarizes all costs related to hiring

\item Firms take one decision: should I open a vacancy or not?

\item Here is where the free entry assumption come into play
\bit
\setlength\itemsep{0.5em}
\item Firms open vacancies as long as  $V\geq 0$
\item Competitive vacancy market, i.e., \emph{free entry}: $V=0$ in equilibrium
\eit

\eit
\end{frame}

\begin{frame}{Wage setting}

\bit
\setlength\itemsep{1.5em}

\item The model has one price: $w$

\item Without further assumptions, the equilibrium wage level $w$ is indeterminate

\item Combining the worker Bellman equations, we get worker surplus:
\begin{eqnarray}
r(W-U) &=& w-b-(\sigma+\lambda_u(\theta))\left(W-U\right) \nonumber \\
W-U &=& \frac{w-b}{r+\sigma + \lambda_u(\theta)} \nonumber
\end{eqnarray}
\item From firm job Bellman equations and $V=0$, we get firm suprlus:
\begin{eqnarray}
rJ &=& (y-w)  - \sigma J \nonumber \\
J-V &=& \frac{y-w}{r+\sigma}  \nonumber
\end{eqnarray}

\item For any wage $w\geq b$, the worker is better off employed than unemployed

\item For any wage $w \leq y$, the firm is better off operating than vacant

\item Ergo, any wage $w \in [b,y]$ is consistent with individual rationality

\item Standard assumption of wage determination in the context of DMP: \bf Nash bargaining \normalfont

\eit
\end{frame}

\begin{frame}{Nash bargaining}

\bit
\setlength\itemsep{1em}

\item When workers and firms meet, they sit down and bargain:
\begin{eqnarray}
\max_{w} \left(W(w)-U\right)^{\gamma} \left(J(w)-V\right)^{1-\gamma} \nonumber
\end{eqnarray}
\item i.e., they maximize the geometric mean of their surpluses, with bargaining weights $\gamma, 1-\gamma$

\item Nash (Ecmtra 1950): the solution to this problem is also the unqiue solution to a general bargaining problem that satisfies 4 reasonable axioms
\bit
	\item Notably, one axiom is Pareto efficiency
\eit

\item It is by no means clear that this is a reasonable approximation of the how wages are determined in the data, but it gives us a way to start thinking about it

\item First order condition:
\begin{eqnarray}
\gamma (J-V) W' + (1-\gamma) (W-U)J' =0 \nonumber
\end{eqnarray}

\item From the definition of the value functions, we have $W'(w) = -J'(w)$
\bit
	\item intuition?
\eit

\item Hence
\begin{eqnarray}
\gamma (J-V) = (1-\gamma) (W-U) \nonumber
\end{eqnarray}

\eit
\end{frame}


\begin{frame}{Nash bargaining: an alternative expression}

\bit
\setlength\itemsep{1.5em}

\item Define total match surplus $S$:
\begin{eqnarray}
S = (J-V) + (W-U) \nonumber
\end{eqnarray}

%\item Why is $S>0$?

\item Combine with
\begin{eqnarray}
\gamma (J-V) = (1-\gamma) (W-U) \nonumber
\end{eqnarray}

\item to find
\begin{eqnarray}
W -U&=& \gamma S \nonumber \\
J -V&=& (1-\gamma) S \nonumber 
\end{eqnarray}

\item Nash bargaining: contesters split the total surplus according to their bargaining power

\eit
\end{frame}

\begin{frame}{Equilibrium definition}

\bit
\setlength\itemsep{1.5em}

\item An equilibrium is a collection $\{W,U, J, V, w, \theta\}$ s.t. the following equations hold
\bit
	\item \bf Bellman equations:
		\begin{eqnarray}
		rW &=& w  + \sigma \left(U-W \right) \nonumber \\ 
		rU &=& b  + \lambda_u(\theta) \left(W-U\right) \nonumber \\
		rJ &=& (y-w)  + \sigma \left(V-J\right) \nonumber \\ 
		rV &=& -c  + \lambda_v(\theta) \left(J-V\right) \nonumber
		\end{eqnarray}

	\item Free entry:
		\begin{eqnarray}
		V &=& 0 \nonumber 
		\end{eqnarray}

	\item Wage setting rule: \normalfont
	\begin{eqnarray}
	\gamma (J-V) = (1-\gamma) (W-U) \nonumber
	\end{eqnarray}
\eit

\item Six equations, six unknowns

\item Given $\theta$, we can solve for $\{v,u\}$ from
\begin{eqnarray}
\dot{u} &=& \sigma (1-u) + \lambda_u(\theta)u \nonumber \\
u(0) &=& \underbar{u} \nonumber \\
\theta &=& \frac{v}{u} \nonumber
\end{eqnarray}
\eit
\end{frame}

\begin{frame}{Solving DMP I: job creation}

\bit
\setlength\itemsep{1em}

\item Start from firm side:
\begin{eqnarray}
rJ &=& (y-w)  + \sigma \left(V-J\right) \nonumber \\ 
rV &=& -c  + \lambda_v(\theta) \left(J-V\right) \nonumber
\end{eqnarray}

\item Free entry in vacancy value equation implies:
\begin{eqnarray}
J = \frac{c}{\lambda_v(\theta)} \nonumber
\end{eqnarray}

%\item Intuition?

%\item A producing firm has asset value $=$ flow cost of bening vacant (c) $\times$ average vacancy duration $\frac{1}{q(\theta)}$

\item Free entry in job value equation implies:
\begin{eqnarray}
J &=& \frac{y-w}{r+\sigma} \nonumber
\end{eqnarray}

\item Together:
\begin{eqnarray}
\frac{y-w}{r+\sigma} = \frac{c}{\lambda_v(\theta)} \nonumber \hspace{5mm} \text{or} \hspace{5mm} w = y - \frac{c(r+\sigma)}{\lambda_v(\theta)}
\end{eqnarray}

\item This is the {\bc job-creation curve} \normalfont in $\{w, \theta\}$-space
\bit
	\item Tells you how many vacancies per unemployed, $\theta$, firms create given the wage $w$ 
	\item Given $\lambda_v(\theta)$ decreasing, JC curve $w(\theta)$ is decreasing (and typically convex)
\eit

\eit
\end{frame}



\begin{frame}{Solving DMP II: wage curve}

\bit
\setlength\itemsep{1em}

\item Plug in firm and worker surplus into wage setting rule
\begin{eqnarray}
\gamma (J-V) &=& (1-\gamma) (W-U) \nonumber \\
\gamma \left(\frac{y-w}{r+\sigma} \right) &=& (1-\gamma) \left(\frac{w-b}{r+\sigma + \lambda_u(\theta)}  \right) \nonumber 
\end{eqnarray}

\item CRS matching: $\frac{\lambda_u(\theta)}{\lambda_v(\theta)}=\theta$ implies
\begin{eqnarray}
\gamma \left(\frac{y-w}{r+\sigma} \right) &=& (1-\gamma) \left(\frac{w-b}{r+\sigma + \theta \lambda_v(\theta)}  \right) \nonumber 
\end{eqnarray}

\item Use the job creation curve $y-w = \frac{c(r+\sigma)}{\lambda_v(\theta)}$ to substitute for $\lambda_v(\theta)$, eliminate terms and rearrange:
\begin{eqnarray}
w = (1-\gamma)b + \gamma (y +c \theta) \nonumber
\end{eqnarray}

\item This is the {\bc wage curve}

\item WC and JC: two equations in two unkowns ($\theta$ and $w$)

\eit
\end{frame}

\begin{frame}{Solving DMP III: steady state}

\bit
\setlength\itemsep{1em}

\item Up till now, we have not imposed steady state anywhere in the solution

\item In general, the level of unemployment and its dynamics is given by
\begin{eqnarray}
\theta &=& \frac{v}{u} \nonumber \\
\dot{u} &=& \sigma (1-u) - \lambda_u(\theta)u \nonumber \\
u(0) &=& \underbar{u} \nonumber
\end{eqnarray}

\item Since $\theta$ is determined without imposing steady state, the transition rates in the unemployment law of motion are constant, also outside the steady state

\item In the steady state, $\dot{u}=0$ and we have the Beveridge curve
\begin{eqnarray}
u = \frac{\sigma}{\sigma + \lambda_u(\theta)} \nonumber
\end{eqnarray}

\eit
\end{frame}


\begin{frame}{Solving DMP IV: equilibrium charachterization}

\bit
\setlength\itemsep{1em}

\item Summary: the equilibrium steady state $\{w, \theta, u, v\}$ is characterized by
\begin{eqnarray}
\text{JC:} && w = y - \frac{c(r+\sigma)}{\lambda_v(\theta)} \nonumber \\
\text{WC:} && w = (1-\gamma)b + \gamma (y +c \theta) \nonumber \\
\text{BC:} && u = \frac{\sigma}{\sigma + \lambda_u(\theta)} \nonumber \\
\text{TD:} && v = \theta u \nonumber
\end{eqnarray}

\item Note: the system is \emph{block recursive}, we can solve for $\theta$ through combining the first 2 equations
\begin{eqnarray}
\frac{c}{\lambda_v(\theta)} = \frac{(1-\gamma)(y-b)}{r + \sigma + \lambda_u(\theta) \gamma} \nonumber
\end{eqnarray} 
\bit
	\item LHS: firm cost of posting a vacancy
	
	\item RHS: firm value of a match
\eit

\item Given $\theta$, we can solve for $u$ and $v$ from the last two equations

\item To understand how the equilibrium ``work'', let's draw some grapgs {\rc(Do on whiteboard)}

%\item To solve the model explicitly, we need to impose a specific matching function
%
%\item Usually, we use Cobb-Douglas
%\begin{eqnarray}
%M(u,v) = Au^{\alpha}v^{1-\alpha} \nonumber
%\end{eqnarray}
%which implies
%\begin{eqnarray}
%p(\theta) &=& A\theta^{1-\alpha} \nonumber \\
%q(\theta) &=& A\theta^{-\alpha} \nonumber
%\end{eqnarray}

\eit
\end{frame}


\begin{frame}{Solving DMP V: graphical view of equilibrium}

\begin{center}
	\includegraphics[scale=0.8, trim=0.4cm 0.4cm 0.4cm 0.4cm, clip]{steady_state.png}
\end{center}


\end{frame}


\begin{frame}{Comparative statics}

\bit
\setlength\itemsep{1em}

\item Using our graphs, comparative statics are straightforward:
\begin{eqnarray}
\text{JC:} && w = y - \frac{c(r+\sigma)}{\lambda_v(\theta)} \nonumber \\
\text{WC:} && w = (1-\gamma)b + \gamma (y +c \theta) \nonumber \\
\text{BC:} && u = \frac{\sigma}{\sigma + \lambda_u(\theta)} \nonumber \\
\text{TD:} && v = \theta u \nonumber
\end{eqnarray}

\item $b \uparrow$ $\Rightarrow$ $w \uparrow$ and $\theta \downarrow$ $\Rightarrow$ $u \uparrow$ and $v \downarrow$
\bit
	\item Intuition? How does this compare to McCall model?
\eit

\item $\gamma \uparrow$ $\Rightarrow$ $w \uparrow$ and $\theta \downarrow$ $\Rightarrow$ $u \uparrow$ and $v \downarrow$
\bit
	\item Intuition?
\eit

\item $y \uparrow$ $\Rightarrow$ $w \uparrow$ and $\theta \uparrow$ $\Rightarrow$ $u \downarrow$ and $v \uparrow$
\bit
	\item Intuition?
\eit

\item And so on...

\item \bf A bunch of hypothesis that can be taken to the data! \normalfont
\eit
\end{frame}

\begin{frame}{Digression: equilibrium effects of extending unemployment benefits}

\bit
\setlength\itemsep{1em}

\item Did the unemployment benefit extensions in the US during the great recession contribute to the sharp increase in the unemployment rate?

\item In lecture 1, we saw that a rich micro literature has estimated the partial equilibrium effect of extending unemployment benefits on unemployment duration
\bit
	\item Estimates, however, not large enough to explain recession spike (Chetty's rule of thumb: 10 weeks of extra UI gives 1 week of extra unemployment)
\eit

\item DMP emphasises a different {\bc general equilibrium channel} through which unemployment benefits affect unemployment: {\bc vacany creation}

\item Recent research employing microeconometric methods to estimate macro effects:
\bit
	\setlength\itemsep{0.5em}
	\item US great recession: federal benefit extension program depended on state-level characteristics, such as the elevation in the state-level unemployment rate
	\item Hagedorn-Karahan-Manovskii-Mitman (2019) exploit policy discontinuity at US state borders: find big effect 
	\item Chodorow-Reich-Coglienese-Karabarbounis (QJE 2018) exploit state-level differences in duration extension due to real-time mismeasurement of unemployment rate: find no effect
	\item See also Marinescu (JPubE 2017) and Fredriksson-S�derstr�m (JPubE 2020). Ongoing debate --- more research needed!
\eit

\eit
\end{frame}


\begin{frame}

\begin{center}
	\huge Endogenous separations \normalfont
\end{center}

\end{frame}


\begin{frame}{Endogenous separations}

\bit
\setlength\itemsep{1.5em}

\item How to endogenize the separation decision?

\item Separations presumably happen when matches are not beneficial to one of the two parties

\item Consider the following twist to the our model:
\bit
\setlength\itemsep{0.5em}

\item Production is $xy$, where $y$ is aggregate productivity, and $x$ is match-specific productivity

\item When forming a match, $x=1$

\item A matched firm-worker pair draw match-specific productivity shocks $x \sim \Gamma (x)$, $x \in [0, 1]$ at arrival rate $\lambda_x$

\item If $J(x)<V$, the match is terminated

\item In each period, the firm-worker pair renegoiate the wage according to Nash bargaining

\eit

\item We focus on steady state

\item More generally, we would perhaps also think the match is terminated if $W(x)<U$, but let's start assuming the worker takes no decisions 


\eit
\end{frame}


\begin{frame}{Firm Values and the reservation productivity}

\bit
\setlength\itemsep{1em}

\item Firm Bellman equations
\begin{align*}
rJ(x) &= xy - w(x)+\lambda_x \left[ \int_{0}^{1} \max\{V, J(\epsilon)\}d\Gamma (\epsilon) -J(x) \right] \\
rV &= -c +\lambda_v(\theta)\left(J(1)-V\right)
\end{align*}


\item Suppose $J(x)$ is increasing in $x$ (will be prooved later)

\item Then, there is reservation productivity $x_R$ below which all matches are dissovled, satisfying $J(x_R)=V$

\item Rewritten Bellman equation for job values
\begin{align*}
rJ(x) &= xy - w(x)+\lambda_x \left[ \int_{0}^{1} \max\{V, J(x')\}d\Gamma (x) -J(x) \right] \\
\Rightarrow rJ(x) &= xy - w(x)+\lambda_x \left[ \int_{x_R}^{1} (J(x')-J(x)) d\Gamma (x') + \int_{0}^{x_R} (V-J(x)) d\Gamma (\epsilon)  \right] \\
\Leftrightarrow rJ(x) &= xy - w(x)+\lambda_x \left[ \int_{x_R}^{1} (J(x')-J(x)) d\Gamma (x') + (V-J(x))\Gamma (x_R)  \right]
\end{align*}

\eit
\end{frame}

\begin{frame}{Worker values}

\bit
\setlength\itemsep{1.5em}

\item Taking $x_R$ as given, worker Bellman equations are
\begin{eqnarray}
rW(x) &=& w(x)  + \lambda_x \left[ \int_{x_R}^{1} (W(x')-W(x)) d\Gamma (x') + (U-W(x))\Gamma (x_R)  \right] \nonumber \\ 
rU &=& b  + \lambda_u(\theta) \left(W(1)-U\right) \nonumber
\end{eqnarray}

\eit
\end{frame}

\begin{frame}{Nash bargaining and the wage curve}

\bit
\setlength\itemsep{1.5em}

\item With Nash rebargaining every time a match-specific shock arrives implies that
\begin{eqnarray}
\gamma (J(x)-V) = (1-\gamma) (W(x)-U) \nonumber
\end{eqnarray}
or
\begin{eqnarray*}
	W(x)-U &=& \gamma S(x) \\
	J(x)-V &=& (1-\gamma) S(x)
\end{eqnarray*}

\item Implication: $J(x)-V<0 \Rightarrow W(x)-U<0$, i.e., terminations are mutually benificial
\bit
	\item This is only true because of the rebargaining assumption (``flexible wage setting'')
\eit


\eit
\end{frame}


\begin{frame}{Equilibrium definition}

\bit
\setlength\itemsep{1em}

\item An equilibrium is a collection $\{W,U, J, V, w, x_R, \theta\}$ s.t. the following equations hold
\bit
\item \bf Bellman equations:
\begin{eqnarray}
rW(x) &=& w(x)  + \lambda_x \left[ \int_{x_R}^{1} (W(x')-W(x)) d\Gamma (x') + (U-W(x))\Gamma (x_R)  \right] \nonumber \\ 
rU &=& b  + \lambda_u(\theta) \left(W(1)-U\right) \nonumber \\
rJ(x) &=& xy - w(x)+\lambda_x \left[ \int_{x_R}^{1} (J(x')-J(x)) d\Gamma (x') + (V-J(x))\Gamma (x_R)  \right] \nonumber \\
rV &=& -c +\lambda_v(\theta)\left(J(1)-V\right) \nonumber
\end{eqnarray}

\item Separation decision:
\begin{eqnarray}
J(x_R) &=& V \nonumber 
\end{eqnarray}

\item Free entry:
\begin{eqnarray}
V &=& 0 \nonumber 
\end{eqnarray}

\item Wage setting rule: \normalfont
\begin{eqnarray}
\gamma (J(x)-V) = (1-\gamma) (W(x)-U) \nonumber
\end{eqnarray}
\eit

\item 7 equations, 7 unknowns
\eit
\end{frame}


\begin{frame}{Equilibrium unemployment}

\bit
\setlength\itemsep{1em}

\item Given $\theta, x_R$, we can solve for $\{v,u\}$ from
\begin{eqnarray}
\dot{u} &=& \sigma(x_R) (1-u) + \lambda_u(\theta)u \nonumber \\
u(0) &=& \underbar{u} \nonumber \\
\theta &=& \frac{v}{u} \nonumber
\end{eqnarray}
where $\sigma(x_R)  = \lambda_x \Gamma(x_R)$

\item In steady state:
\begin{eqnarray}
 u = \frac{\lambda_x \Gamma(x_R)}{\lambda_x \Gamma(x_R) + \lambda_u(\theta)} \nonumber
\end{eqnarray}

\eit
\end{frame}


\begin{frame}{Computing the equilibrium I: the wage curve}

\bit
\setlength\itemsep{1.5em}

\item Combine Bellman equations, the surplus splitting rule and the free entry condition $V=0$ to find the wage curve ({\rc Do at home!}):
\begin{align}
\lb{ws_rule}
w(x) &= (1-\gamma)b + \gamma (xy+c\theta) \nonumber
\end{align}

\eit
\end{frame}


\begin{frame}{Computing the equilibrium II: the surplus function}

\bit
\setlength\itemsep{1em}

\item Use the surplus splitting rule, the wage curve and the free entry condition to find the Bellman equation for total match surplus $S(x)$: ({\rc Do at home!})
\begin{align}
rS(x) =  xy-b-\frac{\gamma}{1-\gamma}c\theta + \lambda_x \left[\int_{x_R}^{1}S(x')d\Gamma(x') - S(x)\right] \nonumber
\end{align}

\item Evaluate at $x=x_R$:
\begin{align}
0 =  x_R y-b-\frac{\gamma}{1-\gamma}c\theta + \lambda_x \left[\int_{x_R}^{1}S(x')d\Gamma(x') \right] \nonumber
\end{align}

\item Take difference to get
\begin{align}
rS(x) &=  rS(x)-rS(x_R) \nonumber \\
&= (x-x_R)y -  \lambda_x S(x) \nonumber
\end{align}
or
\begin{eqnarray}
	\lb{s_solution}
	S(x) = \frac{y(x-x_R)}{r+\lambda_x} \nonumber
\end{eqnarray}

\item Interpretation?

\eit
\end{frame}


\begin{frame}{Computing the equilibrium III: the job-creation curve}

\bit
\setlength\itemsep{1em}

\item Nash bargaining:
\begin{eqnarray}
	J(x)-V &=& (1-\gamma) S(x) \nonumber \\
	&=& \frac{(1-\gamma) y(x-x_R)}{r+\lambda_x}  \nonumber
\end{eqnarray}

\item Free entry in Bellman for vacancy value:
\begin{eqnarray}
J(1)-V &=& \frac{c}{\lambda_v(\theta)} \nonumber
\end{eqnarray}

\item Together: 
\begin{eqnarray}
	\frac{c}{\lambda_v(\theta)} = (1-\gamma) y \frac{1-x_R}{r+\lambda_x} \nonumber
\end{eqnarray}

\item Let's name this the \emph{Job-creation curve} in $\theta, x_R$-space


\eit
\end{frame}


\begin{frame}{Computing the equilibrium IV: the job-destruction curve}

\bit
\setlength\itemsep{1em}

\item Bellman for Surplus:
\begin{align}
rS(x) =  xy-b-\frac{\gamma}{1-\gamma}c\theta + \lambda_x \left[\int_{x_R}^{1}S(x')d\Gamma(x') - S(x)\right] \nonumber
\end{align}
\item Evaluated at $x=x_R$:
\begin{align}
0 =  x_R y-b-\frac{\gamma}{1-\gamma}c\theta + \lambda_x \left[\int_{x_R}^{1}S(x')d\Gamma(x')\right] \nonumber
\end{align}
\item Using our solution for $S(x)$:
\begin{align}
0 =  x_R y-b-\frac{\gamma}{1-\gamma}c\theta + \lambda_x \left[\int_{x_R}^{1}\frac{ y(x-x_R)}{r+\lambda_x}d\Gamma(x')\right] \nonumber
\end{align}
or
\begin{eqnarray}
	x_Ry = b + \frac{\gamma}{1-\gamma}c\theta - y \frac{\lambda_x}{r+\lambda_x}  \int_{x_R}^{1} (x'-x_R)d\Gamma(x') \nonumber
\end{eqnarray}

\item Let's name this the \emph{Job-destruction curve} in $\theta, x_R$-space

\eit
\end{frame}



\begin{frame}{Equilibrium characterization}

\bit
\setlength\itemsep{1em}

\item Summing up, $\{x_R, \theta\}$ is solved from the job-creation and the job-destruction curve:
\begin{eqnarray}
&& \frac{c}{\lambda_v(\theta)} = (1-\gamma) y \frac{1-x_R}{r+\lambda_x} \nonumber \\
&& x_Ry = b + \frac{\gamma}{1-\gamma}c\theta - y \frac{\lambda_x}{r+\lambda_x}  \int_{x_R}^{1} (x'-x_R)d\Gamma(x') \nonumber
\end{eqnarray}

\item Given $x_R, \theta$, we can solve for $\{v,u\}$ from
\begin{eqnarray}
&& v = \theta u\nonumber \\
&& u = \frac{\lambda_x \Gamma(x_R)}{\lambda_x \Gamma(x_R) + \lambda_u(\frac{v}{u})} \nonumber
\end{eqnarray}

\item In the background, match-specific wages are given by
\begin{eqnarray}
w(x) &= (1-\gamma)b + \gamma (xy+c\theta) \nonumber 
\end{eqnarray}


\eit
\end{frame}


\begin{frame}{Graphical representation}

\begin{center}
	\includegraphics[scale=0.8, trim=0.4cm 0.4cm 0.4cm 0.4cm, clip]{steady_state_JD.png}
\end{center}
\end{frame}


\begin{frame}{Comparative statics?}

\bit
\setlength\itemsep{1em}

\item See your problem set


\eit
\end{frame}

\begin{frame}{Summary}

\bit
\setlength\itemsep{1.5em}

\item DMP: A GE theory of unemployment
\bit
	\item emphasizes vacancy creation and job destruction as key mechanisms
\eit

\item Defining elements: Matching function + Wage-setting rule + Free-entry condition

\item Can be integrated in growth and business cycle frameworks

\item Today we focused on steady state equilibrium and comparative statics

\item Next lecture: welfare and dynamics

\eit
\end{frame}





\end{document}




\begin{frame}{Efficiency}

\bit
\setlength\itemsep{1.5em}

\item Is the DMP equilibrium efficient? 
\bit
\item Clearly not (in the usual sense) 
\item An unconstrained social planner would ignore the search frictions and allocate all workers to a firm at all points in time
\eit

\item Is the DMP equilibrium \bf constrained efficient\normalfont? 
\bit
\item Constrained efficiency: would a social planner, who can pick the choices of all agents but still face the same frictions, make different choices than the agents in the decentralized equilibrium?
\eit

\item Remember: the only choice in DMP is whether to post a vacany or not

\item Reformulated question: would the creation of more vacancies increase or decrease welfare?

\item In general, there are two externalities from a firm creating an additional vacancy
\bit
\item Positive \emph{thick market externality}: workers will find jobs faster
\item Negative \emph{congestion externality}: other firms will find workers slower
\item Note: these externalities arise due to the matching function
\eit

\eit
\end{frame}


\begin{frame}{Efficiency in Steady State}

\bit
\setlength\itemsep{1.5em}

\item fff

\eit
\end{frame}

\begin{frame}{Efficiency out of Steady State}

\bit
\setlength\itemsep{1.5em}

\item fff

\eit
\end{frame}


\begin{frame}{Efficiency}

\bit
\setlength\itemsep{2em}

\item One can show that the decentralized equilibrium is efficient if and only if $\gamma = \epsilon_{m,u}(\theta)$ (Hosios, ReStud 1990), where
\begin{eqnarray}
\epsilon_{m,u} = \frac{d M(u,v)}{du} \frac{u}{M(u,v)} = 1 - \frac{\theta \lambda_u'(\theta)}{\lambda_u(\theta)} \nonumber
\end{eqnarray}
\bit
\setlength\itemsep{0.5em}
\item Show the latter equality at home!
\eit

\item If $\epsilon_{m,u}<\gamma$, worker get too large share of the surplus leading to too little vacancy creation (thick market externality dominates and unemployment is too high)

\item If $\epsilon_{m,u}>\gamma$, firms get too large share of the surplus and spend too much costly resources on creating vacancies (the congestion externality dominates and unemployment is too low)

\item If $\epsilon_{m,u} = \gamma$, the social value of creating an additional vacancy equals the private gain to firm

\eit
\end{frame}

\begin{frame}{Effiency: a comment}

\bit
\setlength\itemsep{1.5em}

\item In this model, there is no reason why we should expect $\epsilon_{m,u} = \gamma$

\item Random matching implies generically inefficient equilibria, opening the room for welfare-improving government interventations

\item Primary motivation for models with {\bc directed search}: introduces competitive force that ensures an efficient benchmark (See Moen, JPE 1997; Menzio and Shi, JPE 2011)  

\eit
\end{frame}

\begin{frame}{Summing up}

\bit
\setlength\itemsep{1.5em}

\item In this model, there is no reason why we should expect $\epsilon_{m,u} = \gamma$

\item Random matching implies generically inefficient equilibria, opening the room for welfare-improving government interventations

\item Primary motivation for models with {\bc directed search}: introduces competitive force that ensures an efficient benchmark (See Moen, JPE 1997; Menzio and Shi, JPE 2011)  

\eit
\end{frame}





%
%\appendix
%
%\begin{frame}{DMP: Bonus material I, job creation via match surplus I}
%
%\bit
%\setlength\itemsep{1em}
%
%\item Wage setting rule:
%\begin{eqnarray}
%\gamma (J-V) = (1-\gamma) (W-U) \nonumber
%\end{eqnarray}
%
%\item Note that
%\begin{eqnarray}
%W-U &=& \gamma S \nonumber \\
%J-V &=& (1-\gamma)S \nonumber
%\end{eqnarray}
%where $S=(W-U) + (J-V)$ is total match surplus
%
%\item Total match surplus does not depend on the wage:
%\begin{eqnarray}
%r S &=& r \left[ (W-U) + (J-V) \right] \nonumber \\
%&=&  y-b - \sigma \left[ (U-W) + (J-V)\right] - \lambda_u(\theta)(W-U) \nonumber \\
%&=&  y-b - \sigma S - \lambda_u(\theta) \gamma S \nonumber \\
%\Rightarrow &S& = \frac{y-b}{\rho + \sigma + \lambda_u(\theta) \gamma} \nonumber
%\end{eqnarray}
%
%\item Why does total match surplus not depend on the wage?
%
%\eit
%\end{frame}
%
%\begin{frame}{DMP: Bonus material II, job creation via match surplus II}
%
%\bit
%\setlength\itemsep{2em}
%
%\item Use previous expression together with $J = \frac{c}{\lambda_v(\theta)}$:
%\begin{eqnarray}
%\frac{c}{\lambda_v(\theta)} = (1-\gamma)S= \frac{(1-\gamma)(y-b)}{r + \sigma + \lambda_u(\theta) \gamma} \nonumber
%\end{eqnarray} 
%
%\item This is the \emph{Job creation equation} in $\{u, v\}$-space
%\bit
%\item Given that we know total match surplus, we apparently do need to know the wage to know how many jobs should be created
%\item Why?
%\eit
%
%\item LHS: firm cost of posting a vacancy
%
%\item RHS: firm value of a match
%
%\eit
%\end{frame}




\begin{frame}
\begin{center}
	\huge DMP Dynamics \normalfont
\end{center}

\end{frame}




\begin{frame}{The question}

\bit
\setlength\itemsep{2em}

\item \bf Can DMP explain the cyclicality of unemployment? \normalfont

\item This question refers to a dynamic response of unemployment rate to some underlying shock

\item So far, our analysis has only concerned the steady state

\item Potential shocks: {\bc productivity $y$}, {\rc separation rate $\sigma$}



\eit
\end{frame}




\begin{frame}{Productivity, separations and unemployment}

\begin{figure}
\centering
\includegraphics[scale=0.45]{figures/prod_urate_srate.pdf}
\caption*{\footnotesize Own calculations using hp-filtered quarterly data. Productivity: OECD Labor productivity measure downloaded from FRED. Unemployment: FRED. Separation rate: Shimer's webiste}
\end{figure}

\end{frame}


\begin{frame}{Strategy}

\bit
\setlength\itemsep{1.5em}

\item Strategy: Calibrate model primitives to match log-run moments, feed in shocks that fit detrended processes of TFP and separation rates, evalutate unemployment dynamics 
\bit
\item Just as you would evaluate an RBC or a New-Keynesian model of business cycle dynamics in output
\eit

\item A ``proper'' evaluation requires us to
\ben
\item specify a process of the aggregate (productivty) shock, e.g, they hit with an arrival rate $z$
\item describe how the value functions change with the aggregate state, e.g., 
\begin{eqnarray}
r U_{y} =  b  +  \lambda_u(\theta_{y}) \left( W_{y} - U_{y} \right) + z E_{y'|y} (U_{y'}- U_{y})  \nonumber
\end{eqnarray}
\item solve the model numerically
\een

%\item We will not discuss numerical solution methods in this class
%\bit
%\item Kurt Mitman and Kathrin Schlaffman teach \emph{Quantitative Methods} in Q1
%\eit

\item However, insights about the dynamics of DMP can be gained by simple comparative statics

\item This is because the dynamics in DMP (and the data) is fast: with realistical job-finding rates, the model converges almost instantaneously to steady state

\eit
\end{frame}


%\begin{frame}{Dynamic model}
%
%\bit
%\setlength\itemsep{1em}
%
%\item Suppose the aggregate state is indexed by $s$, and the producticity varies across these states $y=y_s$
%
%%\item When the environment is nonstationary, the value functions depend on the aggregate state $s$ at time t.
%
%\item Assume that an aggregate shock hits the economy at Poission rate $z$, in which the state $s'$ is drawn from distribution $f(s)$
%
%\item Departing from discrete time value functions, taking the limit and using l'H\^{o}pital's rule, we arrive at 
%\begin{eqnarray}
%r U_{s} =  b  +  p(\theta_{s}) \left( W_{s} - U_{s} \right) + z E_{s'|s} (U_{s'}- U_{s})  \nonumber
%\end{eqnarray}
%
%\item Note structure: Value $=$ flow utility + (prob of event) $\times$ (\bf{expected} \normalfont change of value from event), discounted by $r$
%
%\item Similarly
%		\begin{eqnarray}
%			rW_{s} &=& w_{s}  + \sigma \left(U_{s}-W_{s} \right) + z E_{s'|s} (W_{s'}- W_{s}) \nonumber \\ 
%			rJ_{s} &=& y_{s}-w_{s}  + \sigma \left(V_{s}-J_{s}\right) + z E_{s'|s} (J_{s'}- J_{s}) \nonumber \\ 
%			rV_{s} &=& -c  + q(\theta_{s}) \left(J_{s}-V_{s}\right) + z E_{s'|s} (V_{s'}- V_{s}) \nonumber
%		\end{eqnarray}
%
%\item Assume process $f(\cdot)$, solve model numerically (see, e.g., Shimer (2005))
%
%\eit
%\end{frame}


%\begin{frame}{Solving dynamic DMP without simulations?}
%
%\bit
%\setlength\itemsep{1.5em}
%
%\item We will not discuss numerical solution methods in this class
%\bit
%	\item Kurt Mitman and Kathrin Schlaffman teach \emph{Quantitative Methods} in Q1
%\eit
%
%\item However, insights about the dynamics of DMP can be gained by simple comparative statics
%
%\item This is because the dynamics in DMP (and the data) is extremely fast: the model converges almost instanteously to steady state
%
%\eit
%\end{frame}


\begin{frame}{Fast dynamics I}

\bit
\setlength\itemsep{1.5em}

\item Consider the law of motion for unemployment
\begin{eqnarray}
\dot u = \sigma (1-u_t)-\lambda_u(\theta)u_t \nonumber
\end{eqnarray}
\item This is a {\bc first order linear ordinary differential equation}

\item General solution:
\begin{eqnarray}
u_t = C e^{-(\sigma + \lambda_u(\theta))t}+\frac{\sigma}{\sigma + \lambda_u(\theta)} \nonumber 
\end{eqnarray} 
for some constant $C$

\item For any positive $\sigma, \lambda_u$: $u_t \rightarrow u_{ss} = \frac{\sigma}{\sigma + \lambda_u(\theta)}$

\item Larger $\sigma, \lambda_u$ $\rightarrow$ faster convergence

\item What is the half-life of \bf out-of-steady-state unemployment \normalfont? i.e., which $T$ solves the following equation? 
\begin{eqnarray}
\frac{u_T-u_{ss}}{u_0-u_{ss}}=\frac{1}{2} \nonumber
\end{eqnarray}

\eit
\end{frame}

\begin{frame}{Fast dynamics II}

\bit
\setlength\itemsep{1.5em}

\item Steady state unemployment:
\begin{eqnarray}
u_{ss} = \frac{\sigma}{\sigma + \lambda_u(\theta)} \nonumber 
\end{eqnarray}


\item Half life given by
\begin{eqnarray}
\frac{u_T-u_{ss}}{u_0-u_{ss}} &=& \frac{1}{2} \nonumber  \\
\frac{C e^{-(\sigma + \lambda_u(\theta))T}+\frac{\sigma}{\sigma + \lambda_u(\theta)}-\frac{\sigma}{\sigma + \lambda_u(\theta)}}{C e^{-(\sigma + \lambda_u(\theta))*0}+\frac{\sigma}{\sigma + \lambda_u(\theta)}-\frac{\sigma}{\sigma + \lambda_u(\theta)}}  &=& \frac{1}{2} \nonumber \\
e^{-(\sigma + \lambda_u(\theta))T} &=& \frac{1}{2} \nonumber
\end{eqnarray}

\item That is, $T = - \frac{1}{\sigma + \lambda_u} \log \frac{1}{2}$

\item US monthly transitition rates $\sigma \approx 0.03 , \lambda_u \approx 0.6$ $\Rightarrow$  $T=1.1$ months

\item Since convergence is so fast: $u_{ss}(\sigma_t, \lambda_{ut}) \approx u_t$


\eit
\end{frame}


\begin{frame}{$u_{ss}$ vs actual $u_t$}

\begin{figure}
\centering
\includegraphics[scale=0.45]{figures/urate_ss.pdf}
\caption*{\footnotesize Own calculcations based on quarterly data. Unemployment data: FRED. Transition rate data: Shimer's webiste}
\end{figure}

\end{frame}


\begin{frame}{Solving dynamic DMP without simulations?}

\bit
\setlength\itemsep{2em}

\item Lesson: given sufficiently high transition rates, the labor market converges to steady state very fast
%\bit
%	\item Recall that some countries do not have very dynamic labor markets
%\eit

\item Implication: Given that DMP is calibrated to match these high transition rates, the DMP equilibrium converges to steady state very fast

\item Consider a dynamic DMP model in which a simulated sequence of productivty $\{y_0, y_1, y_2,...\}$ maps into a sequence of unemployment $\{u_0, u_1, u_2,...\}$

\item Since convergence to steady state is almost immediate, we can approximate the unemployment sequence with $\{u_{ss}(y_0), u_{ss}(y_1), u_{ss}(y_2),...\}$

\item That is, we can simply do comparative statics of steady state unemployment $u$ w.r.t. to $y$!


\eit
\end{frame}

\begin{frame}{Recap: steady state equilibrium}

\bit
\setlength\itemsep{2em}

\item Recall: the steady state equilibrium is given by
\begin{eqnarray}
&& \frac{c}{\lambda_v(\theta)} = \frac{(1-\gamma)(y-b)}{r + \sigma + \lambda_u(\theta) \gamma} \nonumber \\
&& u = \frac{\sigma}{\sigma + \lambda_u(\theta)} \nonumber
\end{eqnarray} 

\item Given that the matching function is (approximately) correct, the model response of $u$ to $y$ shock matches the data if the elasticity of $\theta$ w.r.t. $y$ matches the data 

\eit
\end{frame}


\begin{frame}{Tightness-productivity elasticity in the data}

\begin{figure}
\centering
\includegraphics[scale=0.45]{figures/prod_tight.pdf}
\caption*{\footnotesize Own calculations using detrended quarterly data using hp-filter. OECD Labor productivity measure downloaded from FRED. Tightness constructed using BLS unemployment from FRED and vacancy data (Help-wanted index) from Regis Barnichon's website.}
\end{figure}

\end{frame}

\begin{frame}{Rescaled tightness-productivity elasticity in the data}

\begin{figure}
\centering
\includegraphics[scale=0.45]{figures/prod_tight_adj.pdf}
\caption*{\footnotesize Own calculations using detrended quarterly data using hp-filter. OECD Labor productivity measure downloaded from FRED. Tightness constructed using BLS unemployment from FRED and vacancy data (Help-wanted index) from Regis Barnichon's website.}
\end{figure}

\end{frame}

\begin{frame}{Can DMP explain unemployment volatility? Shimer (AER 2005)}

\bit
\setlength\itemsep{1.5em}

\item Empirical elasticity $\epsilon_{\theta, y} \approx 20$

\item DMP elasiticty $\epsilon_{\theta, y}$?

\item Job creation curve in $\{y, \theta\}$-space (using $\lambda_u(\theta)=\theta \lambda_v(\theta)$):
\begin{eqnarray}
\frac{c}{\lambda_v(\theta)} = \frac{(1-\gamma)(y-b)}{r + \sigma + \lambda_u(\theta) \gamma} \nonumber 
\end{eqnarray} 

\item Total differentiation to find {\rc (Do on whiteboard)}:
\begin{eqnarray}
\epsilon_{\theta, y} \equiv \frac{y}{\theta} \frac{\partial \theta}{\partial y} \nonumber = \frac{y}{y-b} \frac{r+ \sigma + \gamma \lambda_u(\theta)}{(r + \sigma)(1-\epsilon_{\lambda_u,\theta}) + \gamma \lambda_u(\theta)} \nonumber
\end{eqnarray}
where $\epsilon_{\lambda_u,\theta} \equiv \frac{\theta}{\lambda_u(\theta)}\lambda_u'(\theta)$ is the elasticity of $\lambda_u(\theta)$ w.r.t. $\theta$




\eit
\end{frame}



\begin{frame}{Can DMP explain unemployment volatility? Shimer (AER 2005)}

\begin{eqnarray}
\epsilon_{\theta, y} &=& \frac{y}{y-b} \frac{r+ \sigma + \gamma \lambda_u(\theta)}{(r + \sigma)(1-\epsilon_{\lambda_u,\theta}) + \gamma \lambda_u(\theta)} \nonumber
\end{eqnarray}

\bit
\setlength\itemsep{1.5em}

\item Shimer's (standard) calibration, quarterly frequency: 
\bit
\item $\sigma = 0.1, r = 0.012, y=1, b=0.4, c = 0.2$
\item Cobb-Douglass matching function with $A=1.34, \alpha = 0.72$, $\epsilon_{\lambda_u,\theta} = 1-\alpha$	
\eit

\item If $\gamma=\alpha$ (Hosios condition): $\lambda_u(\theta) = 2.13$ and
\begin{eqnarray}
\epsilon_{\theta, y} = \frac{y}{y-b} \times  \frac{r+ \sigma + \alpha \lambda_u(\theta)}{(r + \sigma)(1-\epsilon_{\lambda_u,\theta}) + \alpha \lambda_u(\theta)}  = 1.7  \nonumber
\end{eqnarray}


\item If $\gamma=0$:
\begin{eqnarray}
\epsilon_{\theta, y} = \frac{y}{y-b} \times \frac{1}{1-\epsilon_{\lambda_u,\theta}} = 3.5  \nonumber
\end{eqnarray}


\item The model comes no way near the data

\eit
\end{frame}

\begin{frame}{Shimer (AER 2005): intuition}

\bit
\setlength\itemsep{2em}

\item Why does DMP fail in generating unemployment volatility?

\item Shimer suggests intuition:
\bit
\setlength\itemsep{1em}

\item $y$ $\uparrow$ $\Rightarrow$ Value of vacancy $\uparrow$

\item Free entry and $v$ $\uparrow$ $\Rightarrow$ Tightness $\uparrow$

\item $\lambda_u(\theta)$ $\uparrow$ $\Rightarrow$ Value of unemployment $\uparrow$

\item Worker's outside option $\uparrow$ $\Rightarrow$ Wages $\uparrow$ 

\item Value of vacancy $\downarrow$ $\Rightarrow$ Equilibrium response small

\eit

\item Summarizing: The equilibrium response of wages via Nash bargaining inhibits firms' incentive to create vacancies

%\item So rigid wage setting, a plausible assumption in the context of business-cycle dynamics, should do the trick?
\eit
\end{frame}



\begin{frame}{The Shimer puzzle}

\bit
\setlength\itemsep{1.5em}

\item Result: a reasonable calibration of the basic DMP model cannot explain observed unemployment volatility
\bit
\setlength\itemsep{0.5em}
\item True both for productivity shocks and separation rate shocks
\item Separation rate shocks even worse: implies that $v$ responds countercyclically, in contrast to data --- Why?
\eit

\item Should we reject the model or are we missing something? 

\item Important papers trying the resolve the ``Shimer puzzle'':
\bit
\setlength\itemsep{1em}

\item Hall (AER 2005): Ad hoc wage rigidity

\item Hagedorn-Manovskii (AER 2008): alternative calibration

\item Hall-Milgrom (AER 2008): Alternative wage setting mechanism

\item Ljungqvist-Sargent (AER 2017): unified framework for analysing above propositions 
\eit

\eit

\end{frame}

\begin{frame}{The Shimer puzzle}

\bit
\setlength\itemsep{1.5em}

\item Result: a reasonable calibration of the basic DMP model cannot explain observed unemployment volatility
\bit
\setlength\itemsep{0.5em}
\item True both for productivity shocks and separation rate shocks
\item Separation rate shocks even worse: implies that $v$ responds countercyclically, in contrast to data --- Why?
\eit

\item Should we reject the model or are we missing something? 

\item Important papers trying the resolve the ``Shimer puzzle'':
\bit
\setlength\itemsep{1em}

\item Hall (AER 2005): Ad hoc wage rigidity

\item \bf Hagedorn-Manovskii (AER 2008): alternative calibration \normalfont

\item Hall-Milgrom (AER 2008): Alternative wage setting mechanism

\item Ljungqvist-Sargent (AER 2017): unified framework for analysing above propositions 
\eit

\eit

\end{frame}


%\begin{frame}{Hall's question}
%
%\bit
%\setlength\itemsep{2em}
%
%\item Why does DMP fail in generating unemployment volatility?
%
%\item Shimer suggests intuition:
%\bit
%\setlength\itemsep{1em}
%
%\item $y$ $\uparrow$ $\Rightarrow$ Value of vacancy $\uparrow$
%
%\item Free entry and $v$ $\uparrow$ $\Rightarrow$ Tightness $\uparrow$
%
%\item $\lambda_u(\theta)$ $\uparrow$ $\Rightarrow$ Value of unemployment $\uparrow$
%
%\item Worker's outside option $\uparrow$ $\Rightarrow$ Wages $\uparrow$ 
%
%\item Value of vacancy $\downarrow$ $\Rightarrow$ Equilibrium response small
%
%\eit
%
%\item {\bc The equilibrium response of wages via Nash bargaining inhibits firms' incentive to create vacancies} in response to shocks
%
%\item So rigid wage setting, a plausible assumption in the context of business-cycle dynamics, should do the trick?
%\eit
%\end{frame}


%\begin{frame}{Is rigid wage setting reasonable?}
%
%\bit
%\setlength\itemsep{2em}
%
%\item But isn't rigid wage setting inconsistent with individual rationaility? (Barro, 1977)
%
%\item In Walrasian markets: yes
%\bit
%\setlength\itemsep{0.5em}
%\item Suppose wages cannot adjust when you feed a positive productivity shock to the RBC model
%
%\item Workers would like to work more if wages were higher, and firms would like to hire more workers than what is available at the given wage
%
%\item If wages cannot adjust, we exclude mutually beneficial bilateral trading opportunities 
%\eit 
%
%\item In search markets: no
%\bit
%	\item A match produces a match rent - how it is divided is not theoretically clear
%\eit
%
%\item In the stationary DMP model, any wage $w \in [b,y]$ is consistent with individual rationality
%
%\eit
%\end{frame}




%\begin{frame}{Hall's argument}
%
%\bit
%\setlength\itemsep{1em}
%
%\item In his paper, Hall employs a dynamic discrete-time model 
%
%\item We focus on the intuition using steady-state elasticities within our continuous-time model
%
%\item DMP-model with Nash bargaining:
%\begin{eqnarray}
%\text{JC:} && w = y - \frac{c(r+\sigma)}{\lambda_v(\theta)} \nonumber \\
%\text{WC:} && w = (1-\gamma)b + \gamma (y +c \theta) \nonumber \\
%\text{BC:} && u = \frac{\sigma}{\sigma + \lambda_u(\theta)} \nonumber \\
%\text{TD:} && v = \theta u \nonumber
%\end{eqnarray}
%
%
%\item DMP model with fixed wage $w=\bar w$:
%\begin{eqnarray}
%\text{JC:} && \bar w = y - \frac{c(r+\sigma)}{\lambda_v(\theta)} \nonumber \\
%\text{BC:} && u = \frac{\sigma}{\sigma + \lambda_u(\theta)} \nonumber \\
%\text{TD:} && v = \theta u \nonumber
%\end{eqnarray}
%
%%\item Basic model:
%%\begin{eqnarray}
%%r U_{s} &=&  b  +  p(\theta_{s}) \left( W_{s} - U_{s} \right) + z E_{s'|s} (U_{s'}- U_{s})  \nonumber \\
%%rW_{s} &=& w_{s}  + \sigma \left(U_{s}-W_{s} \right) + z E_{s'|s} (W_{s'}- W_{s}) \nonumber \\ 
%%rJ_{s} &=& y_{s}-w_{s}  + \sigma \left(V_{s}-J_{s}\right) + z E_{s'|s} (J_{s'}- J_{s}) \nonumber \\ 
%%rV_{s} &=& -c  + q(\theta_{s}) \left(J_{s}-V_{s}\right) + z E_{s'|s} (V_{s'}- V_{s}) \nonumber \\
%%V_{s} &=& 0 \nonumber
%%\end{eqnarray}
%%
%%\item Nash bargaining solution:
%%\begin{eqnarray}
%%w_s = \arg \max_{w_s} \left(W_s-U_s\right)^{\gamma} \left(J_s-V_s\right)^{1-\gamma} \nonumber
%%\end{eqnarray}
%%
%%\item Hall's sticky wage solution:
%%\begin{eqnarray}
%%w_s = w \hspace{3mm} \forall s\nonumber
%%\end{eqnarray}
%
%
%\eit
%\end{frame}

%
%\begin{frame}{Hall's argument II}
%
%\bit
%\setlength\itemsep{1.5em}
%
%\item Job creation curve:
%\begin{eqnarray}
%\bar w = y - \frac{c(r+\sigma)}{\lambda_v(\theta)} \nonumber
%\end{eqnarray} 
%
%\item Total differentiation w.r.t to $\theta$ and $y$ yields
%\begin{eqnarray}
%\epsilon_{\theta, y} &=& \frac{y}{\theta} \frac{\partial \theta}{\partial y} \nonumber \\
%&=& \frac{y}{y- \bar w} \frac{1}{1-\epsilon_{\lambda_u,\theta}} \nonumber
%\end{eqnarray}
%
%\item Hall's result: {\bc if wages are constant and set close to its upper limit, you find a high elasticity}
%
%\eit
%\end{frame}
%
%\begin{frame}{Hall's calibration}
%
%\begin{figure}
%	\centering
%	\includegraphics[scale=0.5]{figures/hall2005_fig3.pdf}
%\end{figure}
%
%\bit
%\item Hall calibrates his dynamic model so that the fixed wage is close to its upper limit
%\eit
%
%\end{frame}



%\begin{frame}{Hall's main result}
%
%\begin{figure}
%	\centering
%	\includegraphics[scale=0.5]{figures/hall2005_fig2.pdf}
%\end{figure}
%
%\bit
%	\item As a result, his model generates a lot of unemployment volatility
%\eit
%
%\end{frame}
%
%\begin{frame}{If instead assuming Nash bargaining}
%
%\begin{figure}
%	\centering
%	\includegraphics[scale=0.5]{figures/hall2005_fig4.pdf}
%\end{figure}
%
%\end{frame}



%
%\begin{frame}{Is rigid wage setting the answer?}
%
%\bit
%\setlength\itemsep{1em}
%
%\item At the business-cycle frequency, wages on existing contracts must be rigid to some degree
%
%\item But what matters in this model is if wages are rigid for {\bc new hires}
%
%\item Whether incumbents' wages are rigid over the business cycle does not matter for firms' incentive to create additional vacancies
%
%\item Even though the overall wage level may seem rigid in relation to productivity fluctations (remember lecture I), this does not mean that wages for new hires are rigid
%
%\item Pissarides (Ecmtra 2009) summarises existing evidence and argues that wages of newly hires closely co-moves with productivity
%\bit
%	\item Recent evidence from Haefke-Sonntag-van Rens (JME 2013) and Kudlyak (JME 2014) support this conclusion
%\eit
%
%\item New research has uncovered channels from which rigidity of incumbent workers affect unemployment dynamics:
%\bit
%	\item Schoefer (2015):  financial frictions
%	\item Carlsson-Westermark (AEJmacro 2020): endogenous separations 
%\eit
%
%\eit
%\end{frame}



\begin{frame}{Hagedorn-Manovskii (2008)}

\bit
\setlength\itemsep{2em}

\item Hagedorn and Manovskii argue: the previous literature have calibrated the DMP model wrongly; if you do it right, there is no puzzle

\item Again, tightness-productivity elasticity with Nash bargaining:
\begin{eqnarray}
\epsilon_{\theta, y} &=& \frac{y}{y-b} \frac{r+ \sigma + \gamma \lambda_u(\theta)}{(r + \sigma)(1-\epsilon_{\lambda_u,\theta}) + \gamma \lambda_u(\theta)} \nonumber
\end{eqnarray}

\item If $b$ is high relative to $y$, the model can generate substantial unemployment fluctations

\item Why is $b$ the key parameter? Let's investigate

\eit
\end{frame}


\begin{frame}{Hagedorn and Manovskii's argument}

\bit
\setlength\itemsep{1.5em}

\item Intuition: To generate a strong vacancy creation response to a productivity shock, two criteria need to be satisfied
\ben
\setlength\itemsep{0.5em}
\item The productivity increase must not be fully absorbed by an increase in wages 

\item The initial level of profits need to be small 
\een

\item Why does the initial level of profits matter?
\bit
\setlength\itemsep{0.5em}
\item The elasticity of vacancy creation w.r.t. to productivity creation depends on the elasticity of profits w.r.t. productivity

\item Suppose wages are fixed and study a 1 percent productivity increase; $y_0=1 \rightarrow y_1 = 1.01$
\item Suppose $w=0.99$:
\begin{eqnarray}
\epsilon_{\pi, y} = 100*\frac{(1.01-0.99)-(1-0.99)}{(1-0.99)}= 100 \% \nonumber
\end{eqnarray}
\item Suppose $w=0.01$:
\begin{eqnarray}
\epsilon_{\pi, y} = 100*\frac{(1.01-0.01)-(1-0.01)}{1-0.01}= 1 \% \nonumber
\end{eqnarray}	
\eit

%\item Remember: Hall had not only a rigid wage setting, but also a high intitial wage level

\eit
\end{frame}

\begin{frame}{Hagedorn and Manovskii's argument II}

\bit
\setlength\itemsep{2em}

\item How to get a small (high) initial level of profits (wages) with Nash Bargaining? 

\item Look at wage curve:
\begin{eqnarray}
w &=& (1-\gamma)b + \gamma (y +c \theta) \nonumber \\
&=& b + \gamma(y-b) + \gamma c \theta \nonumber
\end{eqnarray}
Either $\gamma$ or $b$ must be high

\item Why then was it only $b$ that seemed to have a large impact on tightness-productivity elastcity?
\begin{eqnarray}
\epsilon_{\theta, y} &=& \frac{y}{y-b} \frac{r+ \sigma + \gamma \lambda_u(\theta)}{(r + \sigma)(1-\epsilon_{\lambda_u,\theta}) + \gamma \lambda_u(\theta)} \nonumber
\end{eqnarray}

\item High $\gamma$ also implies volatile wages w.r.t to $y$ $\rightarrow$ level and volatility effect on vacancy elasticity approximately cancel each other out


\eit
\end{frame}


\begin{frame}{Hagedorn and Manovskii's argument III}

\bit
\setlength\itemsep{2em}

\item Main question: is data supportive of high $b$?

\item H\&M stress two data points
\ben
\setlength\itemsep{0.5em}
\item Based on time used for hiring activities, vacancy posting costs $c$ seem small $\approx$ 3-4 percent of quarterly output per worker $y$

\item Wages are moderately procyclical, $\epsilon_{w, y} =0.45$
\een

\item Fact 1 implies that the expected firm profits from vacancy posting must be low

\item Fact 2 implies worker bargaining power $\gamma$ must be low.

\item Ergo, $b$ must be high

\eit
\end{frame}


\begin{frame}{Hagedorn and Manovskii's calibration}

\bit
\setlength\itemsep{2em}
%\item Data suggest that
%\ben
%\item Vacancy posting costs $c$ are very small, $\approx$ 3-4 percent of output $y$
%
%\item Wages are moderately procyclical, $\epsilon_{w, y} =0.45$
%\een
%
%\item Fact 1 implies $y-w$ must be small. Fact 2 implies worker bargaining power must be low.
%
%\item Ergo, $b$ must be high

\item H\&M calibrate the DMP model to match these data moments and find $\gamma = 0.05$ and $y-b= 0.05$

\item With these numbers, the tightness-productvity elasticity becomes:
\begin{eqnarray}
\epsilon_{\theta, y} &=& \frac{y}{y-b} \frac{r+ \sigma + \gamma \lambda_u(\theta)}{(r + \sigma)(1-\epsilon_{\lambda_u,\theta}) + \gamma \lambda_u(\theta)} = 20.6 \nonumber
\end{eqnarray}

\item Does this make economic sense? H\&M: yes!
\bit
\setlength\itemsep{0.5em}
\item $b$ is really the total utility flow of being unemployed:  unemployment benefit value, leisure value, home production value etc.

\item A larger model of household labor decisions will equate the marginal value of search for job with the marginal cost of giving up time for nonmarket activities.

\item If the value of unemployment is not close to that of being employed, it is difficult to explain why unemployed people don't search harder for jobs
\eit

\eit
\end{frame}


\begin{frame}{Shimer Puzzle: Taking stock}

\bit
\setlength\itemsep{1.5em}

\item Hagedorn and Manovskii's argument is controversial

\item If the welfare loss of being unemployed is so small, why do we seemingly care so much about it?

\item Hall (AER 2005): you can just specify an ad hoc fully rigid wage, high enough to make initial profits small
\bit
\setlength\itemsep{0.5em}
\item Yes, this works, but...

\item Pissarides (Ecmtra 2009) summarises existing evidence and argues that wages of newly hires closely co-moves with productivity

\item Recent evidence from Haefke-Sonntag-van Rens (JME 2013) and Kudlyak (JME 2014) support this conclusion
\eit


\item $\Rightarrow$ Shimer puzzle still a puzzle

\item Ljungqvist-Sargent (AER 2017) summarizes the litterature and provides new perspective on the underlying mechanism
\bit
\item Analysis recently questioned in Christiano-Eichenbaum-Trabant (2020)
\eit

\eit
\end{frame}


\begin{frame}{Summing up}

\bit
\setlength\itemsep{1.5em}

\item DMP: A GE theory of unemployment levels and dynamics
\bit
\item emphasizes vacancy creation as the key economic mechanism
\eit

\item Defining elements: Matching function + Wage-setting rule + Free-entry condition

\item Generates reasonable predictions, explored by growing empirical literature

\item Key question: can DMP explain unemployment volatility?

\item Maybe, but still not fully understood

\eit
\end{frame}

\begin{frame}{DMP: extensions and applications}

\bit
\setlength\itemsep{1.5em}

\item Standard extensions
\bit
\setlength\itemsep{0.5em}
\item Endogenous job destruction (Mortensen-Pissarides. ReStud 1994) - see problem set!
\item Heterogeneous match productivities (Pissarides, AER 1985)
\eit

\item Because of its tractable setup, DMP can easily be integrated in other business-cycle frameworks:
\bit
\setlength\itemsep{0.5em}
\item DMP embedded in RBC model (Merz, JME 1995)
\item DMP embedded in new-Keynesian model (Ravenna-Walsh, AEJmacro 2011)
\item DMP embedded in asset-pricing model (Hall, AER 2017) 
\eit

\item Recent applications (that I like)
\bit
\setlength\itemsep{0.5em}
\item Krusell-Mukoyama-Rogerson-Sahin (AER 2017): DMP extended with participation decision can account for the joint co-movement of unemployment, employment and participation

\item Gavazza-Mongey-Violante (AER 2018): DMP with endogeneous recruitment effort can explain procyclicality of matching efficiency

\item Coles-Kelishomi (AEJmacro 2018): Replacing free entry with stochastic entry cost implies that separation shocks can explain dynamics in job-finding rates and solve the Shimer puzzle
\eit

\eit

\end{frame}




\begin{frame}{Search models: what we have not covered}

\bit
\setlength\itemsep{1.5em}

\item The theory of money
\bit
\setlength\itemsep{0.5em}
\item Goods markets are charachterized by the absence of \emph{double coincidence of wants}
\item Search process without mediating transaction technology is very costly
\item Useless assets, such as fiat money, can have positive value beacuse it mediates trade
\item See Williamson and Wright (Handbook ME 2010)
\eit

\item Liquidity in financial markets
\bit
\setlength\itemsep{0.5em}
\item Some finanical markets are close to the Walrasian ideal. Others, such as the market for mortage-backed securities, operate with \emph{over-the-counter} trading
\item Search theory charachterize volume of trading, bid-ask spreads, how the supply of buyers might collapse during crises
\item See Lagos and Rocheteau (Ecmtra 2009)
\eit

\item Family economics
\bit
\setlength\itemsep{0.5em}
\item Marriage is a matching process under costly search
\item See Shimer and Smith (Ecmtra 2000)
\eit


\eit

\end{frame}


\begin{frame}{Search models: what we have not covered: Epidemilogy}

\bit
\setlength\itemsep{1em}

\item The baseline model of disease spread is a random matching model - the SIR model

\item Given a number of Susceptibles $S$ and a number of infected $I$, assume that the number of newly infected is given by matching function $M = \beta S I$

\item Law of motions
\begin{eqnarray}
\dot{S} &=& - \beta S I \nonumber \\
\dot{I} &=& \beta S I - \gamma I \nonumber \\
\dot{R} &=& \gamma I \nonumber
\end{eqnarray}
where $R$ is the number of dead/recovered 

\item SIR offers a way to forecast disease spread and study lockdown policies 

\item In epi literature, there is little treatment of optimizing behaviour and incentives
\bit
\item In other words, $\beta$ is taken as a parameter and not an endogeneous variable
\eit

\item Eichenbaum-Rebelo-Trabant (RFS 2021): $\beta = \beta (C_t, H_t)$, integrate SIR with standard RBC economy
\bit
\item A lot of papers currently extendning this analysis in many interesting ways
\eit


\eit

\end{frame}




