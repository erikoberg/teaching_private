\documentclass[9pt]{beamer}
\usetheme{Boadilla}

\makeatother
\setbeamertemplate{footline}
{
	\leavevmode%
	\hbox{%
		\begin{beamercolorbox}[wd=.4\paperwidth,ht=2.25ex,dp=1ex,center]{author in head/foot}%
			\usebeamerfont{author in head/foot}\insertshortauthor
		\end{beamercolorbox}%
		\begin{beamercolorbox}[wd=.6\paperwidth,ht=2.25ex,dp=1ex,center]{title in head/foot}%
			\usebeamerfont{title in head/foot}\insertshorttitle\hspace*{3em}
			\insertframenumber{} / \inserttotalframenumber\hspace*{1ex}
	\end{beamercolorbox}}%
	\vskip0pt%
}
\makeatletter
\setbeamertemplate{navigation symbols}{}


\usepackage{lipsum}
\usepackage{appendixnumberbeamer}

\usepackage[authoryear]{natbib}
\usepackage[latin1]{inputenc}
\usepackage[T1]{fontenc}
\usepackage{caption}
\usepackage{amsmath, amssymb}
\usepackage{epstopdf}
\usepackage{graphicx}
\usepackage{lmodern}
\usepackage{xcolor}
\usepackage{xpatch}
\usepackage{multirow}

\usepackage{amsmath,theorem,amssymb,graphicx, pgfplots, tabularx, placeins}
\usepackage{dsfont}
\usepackage{caption}
%\usepackage{subcaption}
%\usepackage{subcaption}
\setbeamertemplate{caption}{\raggedright\insertcaption\par}
%\setbeamertemplate{footline}[frame number]
\usepackage{csquotes}
\usepackage{bm}
\bibliographystyle{econometrica}
\usepackage[normalem]{ulem}

\usepackage{setspace}


\definecolor{gray(x11gray)}{rgb}{0.75, 0.75, 0.75}


\newcommand{\bit}{\begin{itemize}}
	\newcommand{\eit}{\end{itemize}}
\newcommand{\ben}{\begin{enumerate}}
	\newcommand{\een}{\end{enumerate}}


\newcommand{\lb}{\label}
\newcommand{\re}{\eqref}

\newcommand{\bc}{\color{blue}}
\newcommand{\rc}{\color{red}}

\title{Macroeconomics II, Lecture X: \\
	Diamond-Mortensen-Pissarides: Efficiency and Dynamics}
\author{Erik {\"O}berg}
\date{}

\begin{document}
	
	\begin{frame}
	\maketitle
\end{frame}

\section{Introduction}


\begin{frame}{Recap and motivation}

\bit
\setlength\itemsep{2em}

\item DMP: A GE theory of unemployment
\bit
\item emphasizes vacancy creation and job destruction as key mechanisms
\eit

\item Defining elements: Matching function + Wage-setting rule + Free-entry condition

\item Last lecutre: steady state equilibrium and comparative statics

\item Today: welfare and dynamics

\item We'll focus on the vanilla DMP model with exogenous separations, but all results, if not highlighted, extends to the endogenous-separation model

\eit

\end{frame}


\begin{frame}{Recap: equilibrium charachterization with exogenous separations}

\bit
\setlength\itemsep{1em}

\item Summary: the equilibrium steady state $\{w, \theta, u, v\}$ is characterized by
\begin{eqnarray}
\text{JC:} && w = y - \frac{c(r+\sigma)}{\lambda_v(\theta)} \nonumber \\
\text{WC:} && w = (1-\gamma)b + \gamma (y +c \theta) \nonumber \\
\text{BC:} && u = \frac{\sigma}{\sigma + \lambda_u(\theta)} \nonumber \\
\text{TD:} && v = \theta u \nonumber
\end{eqnarray}

\item Note: the system is \emph{block recursive}, we can solve for $\theta$ through combining the first 2 equations
\begin{eqnarray}
\frac{c}{\lambda_v(\theta)} = \frac{(1-\gamma)(y-b)}{r + \sigma + \lambda_u(\theta) \gamma} \nonumber
\end{eqnarray} 


\item Given $\theta$, we can solve for $u$ and $v$ from the last two equations

\eit
\end{frame}


\begin{frame}{Recap: graphical view of equilibrium}

\begin{center}
\includegraphics[scale=0.8, trim=0.4cm 0.4cm 0.4cm 0.4cm, clip]{steady_state.png}
\end{center}


\end{frame}



\begin{frame}{Agenda}

\ben
\setlength\itemsep{1.5em}

\item Efficiency

\item Dynamics
\bit
\setlength\itemsep{0.5em}
	\item Steady-state approximation of dynamic equilibrium
	\item Unemployment volatility puzzle
\eit

\item Summary and discussion

\een

\end{frame}


\begin{frame}

\begin{center}
	\huge Efficiency \normalfont
\end{center}

\end{frame}



\begin{frame}{Efficiency}

\bit
\setlength\itemsep{1.5em}

\item Is the DMP equilibrium efficient? 
\bit
\item Clearly not (in the usual sense) 
\item An unconstrained social planner would ignore the search frictions and allocate all workers to a firm at all points in time
\eit

\item Is the DMP equilibrium \bf constrained efficient\normalfont? 
\bit
\item Constrained efficiency: would a social planner, who can pick the choices of all agents but still face the same frictions, make different choices than the agents in the decentralized equilibrium?
\eit

\item Remember: the only choice in DMP is whether to post a vacany or not

\item Reformulated question: would the creation of more vacancies increase or decrease welfare?

\item In general, there are two externalities from a firm creating an additional vacancy
\bit
\item Positive \emph{thick market externality}: workers will find jobs faster
\item Negative \emph{congestion externality}: other firms will find workers slower
\item Note: these externalities arise due to the matching function
\eit

\eit
\end{frame}


\begin{frame}{Social planner problem}

\bit
\setlength\itemsep{1.5em}

\item To find out, we need to compare the social planner solution to the decentralized solution

\item Social planner problem (with equal Pareto weights):
\begin{eqnarray}
\max_{v} && \int_{0}^{\infty} e^{-rt}\left((1-u)y + ub- v c \right) dt \nonumber \\
\text{s.t.} &&  \dot{u} = \sigma (1-u) - \lambda_u(\frac{v}{u})u \nonumber 
\end{eqnarray}
with $u$ as the state variable and $v$ as the control

\item We can reformulate as
\begin{eqnarray}
\max_{\theta} && \int_{0}^{\infty} e^{-rt}\left((1-u)y + ub-\theta u c \right) dt \nonumber \\
\text{s.t.} &&  \dot{u} = \sigma (1-u) - \lambda_u(\theta)u \nonumber 
\end{eqnarray}
with $u$ as the state variable and $\theta$ as the control


%\bit
%	\item Why can $\theta$ jump but $u$ not?
%\eit

\eit
\end{frame}

\begin{frame}{Social planner problem: solution}

\bit
\setlength\itemsep{1em}


\item Introduce the co-state $\mu$ and form the Hamiltonian:
\begin{eqnarray}
H(u, \theta, \mu, t) = e^{-rt}\left[ (y(1-u) +bu - \theta u c ) + \mu (\sigma (1-u) - \lambda_u(\theta)u)  \right] \nonumber
\end{eqnarray}

\item From Pontryagin', we have the optimaility conditions:
\begin{eqnarray}
\frac{d H}{d u} &=& -\dot \mu \nonumber \\
\frac{d H}{d \theta} &=& 0 \nonumber
\end{eqnarray}
and thus
\begin{eqnarray}
 -y +b - \theta c - \mu (\sigma + \lambda_u(\theta)) &=& -\dot \mu \nonumber \\
-c - \mu \lambda_u'(\theta) &=& 0 \nonumber 
\end{eqnarray}

\item $\mu = $ the value of having one more unemployed. $-\mu = $ equals the expected cost of matching that unemployed with a new vacancy:
\begin{eqnarray}
-\mu = \frac{c}{\lambda_u'(\theta)} \nonumber
\end{eqnarray}

\eit
\end{frame}

\begin{frame}{Social planner problem solution II}

\bit
\setlength\itemsep{2em}

\item Evaluate in steady state $\dot \theta=0$ and work out the algebra to find
\begin{eqnarray}
\frac{c}{\lambda_v(\theta)} = \frac{(1-\epsilon_{m,u}(\theta))(y-b)}{r+\sigma + \lambda_u(\theta)\epsilon_{m,u}(\theta)} \nonumber
\end{eqnarray}
where 
\begin{eqnarray}
\epsilon_{m,u} = \frac{d M(u,v)}{du} \frac{u}{M(u,v)} = 1- \frac{\theta \lambda_u'(\theta)}{\lambda_u(\theta)} \nonumber
\end{eqnarray}

\item Compare to decentralized solution
\begin{eqnarray}
\frac{c}{\lambda_v(\theta)} = \frac{(1-\gamma)(y-b)}{r + \sigma + \lambda_u(\theta) \gamma} \nonumber
\end{eqnarray} 


\eit
\end{frame}



\begin{frame}{Is the decentralized equilibrium efficient?}

\bit
\setlength\itemsep{2em}

\item Hosios (REstud 1990): the decentralized equilibrium is efficient iff $\gamma = \epsilon_{m,u}(\theta)$!

\item If $\epsilon_{m,u}<\gamma$, worker get too large share of the surplus leading to too little vacancy creation (thick market externality dominates and unemployment is too high)

\item If $\epsilon_{m,u}>\gamma$, firms get too large share of the surplus and spend too much costly resources on creating vacancies (the congestion externality dominates and unemployment is too low)

\item If $\epsilon_{m,u} = \gamma$, the social value of creating an additional vacancy equals the private gain to firm


\eit
\end{frame}


\begin{frame}{Effiency: a comment}

\bit
\setlength\itemsep{1.5em}

\item In this model, there is no reason why we should expect $\epsilon_{m,u} = \gamma$

\item Random matching implies generically inefficient equilibria, opening the room for welfare-improving government interventions

\item Primary motivation for models with {\bc directed search}: introduces competitive force that ensures an efficient benchmark (Moen, JPE 1997; Menzio and Shi, JPE 2011)  

\eit
\end{frame}

\begin{frame}

\begin{center}
	\huge Dynamics \normalfont
\end{center}

\end{frame}




\begin{frame}{The question}

\bit
\setlength\itemsep{2em}

\item So far, our analysis has only concerned the steady state

\item \bf Can DMP explain the cyclicality of unemployment? \normalfont

\item This question refers to a dynamic response of unemployment rate to some underlying shock

\item A natural candidate: shocks to {\bc productivity $y$}



\eit
\end{frame}

\begin{frame}{Strategy}

\bit
\setlength\itemsep{1.5em}

\item Strategy: Calibrate model primitives to match log-run moments, feed in shocks that fit detrended processes of TFP and separation rates, evalutate unemployment dynamics 
\bit
\item Just as you would evaluate an RBC or a New-Keynesian model of business cycle dynamics in output
\eit

\item A ``proper'' evaluation requires us to
\ben
\item specify a process of the aggregate (productivty) shock, e.g, they hit with an arrival rate $z$
\item describe how the value functions change with the aggregate state, e.g., 
\begin{eqnarray}
r U_{y} =  b  +  \lambda_u(\theta_{y}) \left( W_{y} - U_{y} \right) + z E_{y'|y} (U_{y'}- U_{y})  \nonumber
\end{eqnarray}
\item solve the model numerically
\een

%\item We will not discuss numerical solution methods in this class
%\bit
%\item Kurt Mitman and Kathrin Schlaffman teach \emph{Quantitative Methods} in Q1
%\eit

\item However, insights about the dynamics of DMP can be gained by simple comparative statics

\item This is because the dynamics in DMP (and the data) is fast: with realistical job-finding rates, the model converges almost instantaneously to steady state

\eit
\end{frame}


\begin{frame}{Fast dynamics: a steady-state approximation}

\bit
\setlength\itemsep{1.5em}

\item Consider the formula for steady state unemployment:
\begin{eqnarray}
u_{ss} = \frac{\sigma}{\sigma + \lambda_{u,t}} \nonumber 
\end{eqnarray}

\item At any given $t$, there exist a steady state unemployment rate $s_{ss,t}$ associated with the current flow rates
\begin{eqnarray}
u_{ss, t} = \frac{\sigma_t}{\sigma_t + \lambda_{u,t}} \nonumber 
\end{eqnarray}

\item How does $u_{ss, t}$ compare with actual unemployment $u_t$?


\eit
\end{frame}


\begin{frame}{$u_{ss}$ vs actual $u_t$}

\begin{figure}
	\centering
	\includegraphics[scale=0.45]{figures/urate_ss.pdf}
	\caption*{\footnotesize Own calculcations based on quarterly data. Unemployment data: FRED. Transition rate data: Shimer's webiste}
\end{figure}

\end{frame}


\begin{frame}{Fast dynamics II}

\bit
\setlength\itemsep{1em}

\item What's going on?

\item Consider the law of motion for unemployment
\begin{eqnarray}
\dot u = \sigma (1-u_t)-\lambda_u(\theta)u_t \nonumber
\end{eqnarray}
\item This is a {\bc first order linear ordinary differential equation}

\item General solution:
\begin{eqnarray}
u_t = C e^{-(\sigma + \lambda_u(\theta))t}+\frac{\sigma}{\sigma + \lambda_u(\theta)} \nonumber 
\end{eqnarray} 
for some constant $C$

\item For any positive $\sigma, \lambda_u$: $u_t \rightarrow u_{ss} = \frac{\sigma}{\sigma + \lambda_u(\theta)}$

\item Larger $\sigma, \lambda_u$ $\rightarrow$ faster convergence

\item What is the half-life of \bf out-of-steady-state unemployment \normalfont? i.e., which $T$ solves the following equation? 
\begin{eqnarray}
\frac{u_T-u_{ss}}{u_0-u_{ss}}=\frac{1}{2} \nonumber
\end{eqnarray}

\eit
\end{frame}

\begin{frame}{Fast dynamics III}

\bit
\setlength\itemsep{1.5em}

\item Steady state unemployment:
\begin{eqnarray}
u_{ss} = \frac{\sigma}{\sigma + \lambda_u(\theta)} \nonumber 
\end{eqnarray}


\item Half life given by
\begin{eqnarray}
\frac{u_T-u_{ss}}{u_0-u_{ss}} &=& \frac{1}{2} \nonumber  \\
\frac{C e^{-(\sigma + \lambda_u(\theta))T}+\frac{\sigma}{\sigma + \lambda_u(\theta)}-\frac{\sigma}{\sigma + \lambda_u(\theta)}}{C e^{-(\sigma + \lambda_u(\theta))*0}+\frac{\sigma}{\sigma + \lambda_u(\theta)}-\frac{\sigma}{\sigma + \lambda_u(\theta)}}  &=& \frac{1}{2} \nonumber \\
e^{-(\sigma + \lambda_u(\theta))T} &=& \frac{1}{2} \nonumber
\end{eqnarray}

\item That is, $T = - \frac{1}{\sigma + \lambda_u} \log \frac{1}{2}$

\item US monthly transitition rates $\sigma \approx 0.03 , \lambda_u \approx 0.6$ $\Rightarrow$  $T=1.1$ months

\item Since convergence is so fast: $u_{ss}(\sigma_t, \lambda_{ut}) \approx u_t$


\eit
\end{frame}

\begin{frame}{Solving dynamic DMP without simulations?}

\bit
\setlength\itemsep{2em}

\item Lesson: given sufficiently high transition rates, the labor market converges to steady state very fast
%\bit
%	\item Recall that some countries do not have very dynamic labor markets
%\eit

\item Implication: Given that DMP is calibrated to match these high transition rates, the DMP equilibrium converges to steady state very fast

\item Consider a dynamic DMP model in which a simulated sequence of productivty $\{y_0, y_1, y_2,...\}$ maps into a sequence of unemployment $\{u_0, u_1, u_2,...\}$

\item Since convergence to steady state is almost immediate, we can approximate the unemployment sequence with $\{u_{ss}(y_0), u_{ss}(y_1), u_{ss}(y_2),...\}$

\item That is, we can simply do comparative statics of steady state unemployment $u$ w.r.t. to $y$!


\eit
\end{frame}

\begin{frame}{Recap: steady state equilibrium}

\bit
\setlength\itemsep{2em}

\item Recall: the steady state equilibrium is given by
\begin{eqnarray}
&& \frac{c}{\lambda_v(\theta)} = \frac{(1-\gamma)(y-b)}{r + \sigma + \lambda_u(\theta) \gamma} \nonumber \\
&& u = \frac{\sigma}{\sigma + \lambda_u(\theta)} \nonumber
\end{eqnarray} 

\item Given that the matching function is (approximately) correct, the model response of $u$ to $y$ shock matches the data if the elasticity of $\theta$ w.r.t. $y$ matches the data 

\eit
\end{frame}


\begin{frame}{Tightness-productivity elasticity in the data}

\begin{figure}
\centering
\includegraphics[scale=0.45]{figures/prod_tight.pdf}
\caption*{\footnotesize Own calculations using detrended quarterly data using hp-filter. OECD Labor productivity measure downloaded from FRED. Tightness constructed using BLS unemployment from FRED and vacancy data (Help-wanted index) from Regis Barnichon's website.}
\end{figure}

\end{frame}

\begin{frame}{Tightness-productivity elasticity in the data, rescaled}

\begin{figure}
\centering
\includegraphics[scale=0.45]{figures/prod_tight_adj.pdf}
\caption*{\footnotesize Own calculations using detrended quarterly data using hp-filter. OECD Labor productivity measure downloaded from FRED. Tightness constructed using BLS unemployment from FRED and vacancy data (Help-wanted index) from Regis Barnichon's website.}
\end{figure}

\end{frame}

\begin{frame}{Can DMP explain unemployment volatility? Shimer (AER 2005)}

\bit
\setlength\itemsep{1.5em}

\item Empirical elasticity $\epsilon_{\theta, y} \approx 20$

\item DMP elasiticty $\epsilon_{\theta, y}$?

\item Job creation curve in $\{y, \theta\}$-space (using $\lambda_u(\theta)=\theta \lambda_v(\theta)$):
\begin{eqnarray}
\frac{c}{\lambda_v(\theta)} = \frac{(1-\gamma)(y-b)}{r + \sigma + \lambda_u(\theta) \gamma} \nonumber 
\end{eqnarray} 

\item Total differentiation to find {\rc (Do on whiteboard)}:
\begin{eqnarray}
\epsilon_{\theta, y} \equiv \frac{y}{\theta} \frac{\partial \theta}{\partial y} \nonumber = \frac{y}{y-b} \frac{r+ \sigma + \gamma \lambda_u(\theta)}{(r + \sigma)(1-\epsilon_{\lambda_u,\theta}) + \gamma \lambda_u(\theta)} \nonumber
\end{eqnarray}
where $\epsilon_{\lambda_u,\theta} \equiv \frac{\theta}{\lambda_u(\theta)}\lambda_u'(\theta)$ is the elasticity of $\lambda_u(\theta)$ w.r.t. $\theta$




\eit
\end{frame}



\begin{frame}{Can DMP explain unemployment volatility? Shimer (AER 2005)}

\begin{eqnarray}
\epsilon_{\theta, y} &=& \frac{y}{y-b} \frac{r+ \sigma + \gamma \lambda_u(\theta)}{(r + \sigma)(1-\epsilon_{\lambda_u,\theta}) + \gamma \lambda_u(\theta)} \nonumber
\end{eqnarray}

\bit
\setlength\itemsep{1.5em}

\item Shimer's (standard) calibration, quarterly frequency: 
\bit
\item Normalize $y=1$
\item Take outside estimates $\sigma = 0.1, r = 0.012, b=0.4$
\item Estimate cobb-Douglass matching function with $A=1.34, \alpha = 0.72$, $\epsilon_{\lambda_u,\theta} = 1-\alpha$
\item Set $c = 0.2$	to match steady state tightness $\theta$
\eit

\item If $\gamma=\alpha$ (Hosios condition): $\lambda_u(\theta) = 2.13$ and
\begin{eqnarray}
\epsilon_{\theta, y} = \frac{y}{y-b} \times  \frac{r+ \sigma + \alpha \lambda_u(\theta)}{(r + \sigma)(1-\epsilon_{\lambda_u,\theta}) + \alpha \lambda_u(\theta)}  = 1.7  \nonumber
\end{eqnarray}


\item If $\gamma=0$:
\begin{eqnarray}
\epsilon_{\theta, y} = \frac{y}{y-b} \times \frac{1}{1-\epsilon_{\lambda_u,\theta}} = 3.5  \nonumber
\end{eqnarray}


\item The model comes no way near the data

\eit
\end{frame}

\begin{frame}{Shimer (AER 2005): intuition}

\bit
\setlength\itemsep{2em}

\item Why does DMP fail in generating unemployment volatility?

\item Shimer suggests intuition:
\bit
\setlength\itemsep{1em}

\item $y$ $\uparrow$ $\Rightarrow$ Value of vacancy $\uparrow$

\item Free entry and $v$ $\uparrow$ $\Rightarrow$ Tightness $\uparrow$

\item $\lambda_u(\theta)$ $\uparrow$ $\Rightarrow$ Value of unemployment $\uparrow$

\item Worker's outside option $\uparrow$ $\Rightarrow$ Wages $\uparrow$ 

\item Value of vacancy $\downarrow$ $\Rightarrow$ Equilibrium response small

\eit

\item Summarizing: The equilibrium response of wages via Nash bargaining inhibits firms' incentive to create vacancies

%\item So rigid wage setting, a plausible assumption in the context of business-cycle dynamics, should do the trick?
\eit
\end{frame}



\begin{frame}{The Shimer puzzle}

\bit
\setlength\itemsep{1.5em}

\item Result: a reasonable calibration of the basic DMP model cannot explain observed unemployment volatility

\item What about endogenous separations? Same problem, and actually creates another one:
\bit
\setlength\itemsep{0.5em}
	\item Consider a positive separation shock (as implied by a negative productivity shock)

	\item Higher $u$, but also lower $\theta$ 
	
	\item $\Rightarrow$ induces more vacancy creation
	
	\item $\Rightarrow$ positive correlation between vacancies and unemployment!
\eit

\item This is not quite correct, as we will see soon

\eit

\end{frame}


\begin{frame}{The Shimer puzzle: resolutions}

\bit
\setlength\itemsep{1.5em}

\item Hall (AER 2005): Ad hoc wage rigidity

\item Hagedorn-Manovskii (AER 2008): alternative calibration

\item Hall (AER 2017): discount factor shocks

\item Coles-Kelishomi (AEJmacro 2018): sluggish vacancy creation

\eit

\end{frame}


\begin{frame}{The Shimer puzzle: resolutions}

\bit
\setlength\itemsep{1.5em}

\item Hall (AER 2005): Ad hoc wage rigidity

\item \bf Hagedorn-Manovskii (AER 2008): alternative calibration \normalfont

\item Hall (AER 2017): discount factor shocks

\item \bf Coles-Kelishomi (AEJmacro 2018): separation shocks and sluggish vacancy creation \normalfont

\eit

\end{frame}


\begin{frame}{Hagedorn-Manovskii (2008)}

\bit
\setlength\itemsep{2em}

\item Hagedorn and Manovskii argue: the previous literature have calibrated the DMP model wrongly; if you do it right, there is no puzzle

\item Again, tightness-productivity elasticity with Nash bargaining:
\begin{eqnarray}
\epsilon_{\theta, y} &=& \frac{y}{y-b} \frac{r+ \sigma + \gamma \lambda_u(\theta)}{(r + \sigma)(1-\epsilon_{\lambda_u,\theta}) + \gamma \lambda_u(\theta)} \nonumber
\end{eqnarray}

\item If $b$ is high relative to $y$, the model can generate substantial unemployment fluctations

\item Why is $b$ the key parameter? Let's investigate

\eit
\end{frame}


\begin{frame}{Hagedorn and Manovskii's argument}

\bit
\setlength\itemsep{1.5em}

\item Intuition: To generate a strong vacancy creation response to a productivity shock, two criteria need to be satisfied
\ben
\setlength\itemsep{0.5em}
\item The productivity increase must not be fully absorbed by an increase in wages 

\item The initial level of profits need to be small 
\een

\item Why does the initial level of profits matter?
\bit
\setlength\itemsep{0.5em}
\item The elasticity of vacancy creation w.r.t. to productivity creation depends on the elasticity of profits w.r.t. productivity

\item Suppose wages are fixed and study a 1 percent productivity increase; $y_0=1 \rightarrow y_1 = 1.01$
\item Suppose $w=0.99$:
\begin{eqnarray}
\epsilon_{\pi, y} = 100*\frac{(1.01-0.99)-(1-0.99)}{(1-0.99)}= 100 \% \nonumber
\end{eqnarray}
\item Suppose $w=0.01$:
\begin{eqnarray}
\epsilon_{\pi, y} = 100*\frac{(1.01-0.01)-(1-0.01)}{1-0.01}= 1 \% \nonumber
\end{eqnarray}	
\eit

%\item Remember: Hall had not only a rigid wage setting, but also a high intitial wage level

\eit
\end{frame}

\begin{frame}{Hagedorn and Manovskii's argument II}

\bit
\setlength\itemsep{2em}

\item When is initial level of profits (wages) small (large) with Nash Bargaining? 

\item Look at wage curve:
\begin{eqnarray}
w &=& (1-\gamma)b + \gamma (y +c \theta) \nonumber \\
&=& b + \gamma(y-b) + \gamma c \theta \nonumber
\end{eqnarray}
Either $\gamma$ or $b$ must be high

\item Why then was it only $b$ that seemed to have a large impact on tightness-productivity elastcity?
\begin{eqnarray}
\epsilon_{\theta, y} &=& \frac{y}{y-b} \frac{r+ \sigma + \gamma \lambda_u(\theta)}{(r + \sigma)(1-\epsilon_{\lambda_u,\theta}) + \gamma \lambda_u(\theta)} \nonumber
\end{eqnarray}

\item High $\gamma$ also implies volatile wages w.r.t to $y$ $\rightarrow$ level and volatility effect on vacancy elasticity approximately cancel each other out


\eit
\end{frame}


\begin{frame}{Hagedorn and Manovskii's argument III}

\bit
\setlength\itemsep{1.5em}

\item Main question: is data supportive of high $b$?

\item Shimer did not set his parameters to be consistent with the cyclicality of wages? Instead, he just set $b=0.4$, $\gamma = \alpha$, and adjust the vacancy posting cost to match steady state tightness

\item H\&M stresses:
\bit
	\item In the data, wages are moderately procyclical, $\epsilon_{w, y} = 0.45$
	
	\item taking both labor and capital cost of vacancy posting into account, $c \approx 0.6$
\eit

\item For a given $b$ and vacancy post cost $c$, $\epsilon_{w, y}$ pins down $\gamma$

\item H\&M matches this moment, and find a much lower $\gamma = 0.05$

\item With lower $\gamma$, $b$ must be higher to match the same steady state tightness $\theta$

\item H\&M find $b=0.95$

\eit
\end{frame}


\begin{frame}{Hagedorn and Manovskii's calibration}

\bit
\setlength\itemsep{1.5em}

\item With these numbers, the tightness-productvity elasticity becomes:
\begin{eqnarray}
\epsilon_{\theta, y} &=& \frac{y}{y-b} \frac{r+ \sigma + \gamma \lambda_u(\theta)}{(r + \sigma)(1-\epsilon_{\lambda_u,\theta}) + \gamma \lambda_u(\theta)} = 20.6 \nonumber
\end{eqnarray}

\item Does this make economic sense? H\&M: yes!

\item $b$ is really the total utility flow of being unemployed:  unemployment benefit value, leisure value, home production value etc.

\item A larger model of household labor decisions will equate the marginal value of search for job with the marginal cost of giving up time for nonmarket activities.

\item If the value of unemployment is not close to that of being employed, it is difficult to explain why unemployed people don't search harder for jobs

\item This is a controversial argument, if the welfare loss of being unemployed is so small, why do we seemingly care so much about it?

\eit
\end{frame}


\begin{frame}{Comment: wage levels and wage rigidity}

\bit
\setlength\itemsep{1.5em}

\item Shimer's intuition was the failure to match unemployment volatility reflected the flexibility of wages in the model

\item H\&M showed that this intuition is partly wrong, or at least, incomplete
\bit
\setlength\itemsep{0.5em}
	\item Yes, the productivity increase must not be fully absorbed by an increase in wages in response to the shock
	
	\item but, given that it is not fully absorbed, the important thing is whether the steady state level of profits/wages is high or small relative to productivity

	\item Eloborated in Ljungqvist-Sargent (AER 2017) 
\eit

\item Nevertheless, much of the literature has focused on wage rigidity
\bit
\setlength\itemsep{0.5em}
\item Hall (AER 2005): DMP with an ad hoc fully rigid wage, shows that model matches unemployment cyclicality if steady state wage is high enough

\item Pissarides (Ecmtra 2009): it is only the wage of new hires that is allocative in the model, wage rigidity among imcumbents unimportant

\item Mixed evidence on the extent of wage rigidity for new hires in the literature, see e.g., Kudlyak (JME 2014), Grigsby-Hurst-Yildirmaz (AER 2021), Hazell-Taska (AER 2024)

\item Carlsson-Westermark (AEJmacro 2022): with endogeneous separations, rigidity among incumbents also matters
\eit

\eit
\end{frame}


\begin{frame}{Coles-Kelishomi (2018)}

\bit
\setlength\itemsep{1.5em}

\item CKs argument: Unemployment variability can be explained with separation shocks if relaxing free entry

\item Free entry is a reasonable in the long run, but in the short run, there are all kinds of barriers

\item Extend the vanilla DMP model with stochastic entry costs $\kappa$
\begin{eqnarray}
\kappa \sim H(\cdot) \nonumber
\end{eqnarray}

\item $\Rightarrow$ all firms drawing $\kappa<V_t$ create a vacancy. Vacancy creation rate: $\iota_t = H(V_t)$

\item With $H(x) = x^{\xi}$, vacancy creation $\iota_t$ has a constant elasticity w.r.t. vacancy values:
\begin{eqnarray}
\iota_t = V_t^{\xi} \nonumber
\end{eqnarray} 

\item Note: Free entry means $\xi \rightarrow \infty$

\eit
\end{frame}



\begin{frame}{Coles-Kelishomi (2018)}

\begin{figure}
	\centering
	\includegraphics[scale=0.45]{figures/CK_fig23.pdf}
\end{figure}

\end{frame}

\begin{frame}{Coles-Kelishomi (2018)}

\bit
\setlength\itemsep{1.5em}

\item With sluggish vacancy creation, newly separated deplete the current vacancy stock
\bit
\setlength\itemsep{0.5em}
	\item Negative correlation between vacancy stock and unemployment, as in the data
	
	\item $\Rightarrow$ Increase in separations depresses the job-finding rate, double effect on unemployment
	
	\item Same logic with endogenous separations and productivity shocks
\eit

\item Growing literature on implications, microfoundations and empirics:
\bit
\setlength\itemsep{0.5em}
	\item Schoefer-Mercan-Sedlacek (AEJmacro 2023): sluggish entry due to imperfect substitability

	\item Engbom (2021): sluggish entry due to compositional effect over the cycle
	
	\item Broer-Druedahl-Harmenberg-�berg (2025): implications for unemployment-risk dynamics

	\item Cederl�f-Engbom-Nybom-�berg-Tuominen (in progress): evidence from local Swedish labor markets
\eit

\eit
\end{frame}




\begin{frame}{Summing up}

\bit
\setlength\itemsep{1.5em}

\item DMP: A GE theory of unemployment levels and dynamics
\bit
\item emphasizes vacancy creation as the key economic mechanism
\eit

\item Defining elements: Matching function + Wage-setting rule + Free-entry condition

\item Generates reasonable predictions, explored by growing empirical literature

\item Key question: can DMP explain unemployment volatility?

\item Maybe, but still not fully understood

\eit
\end{frame}



\begin{frame}{Search models: what we have not covered}

\bit
\setlength\itemsep{1.5em}

\item The theory of money
\bit
\setlength\itemsep{0.5em}
\item Goods markets are charachterized by the absence of \emph{double coincidence of wants}
\item Search process without mediating transaction technology is very costly
\item Useless assets, such as fiat money, can have positive value beacuse it mediates trade
\item See Williamson and Wright (Handbook ME 2010)
\eit

\item Liquidity in financial markets
\bit
\setlength\itemsep{0.5em}
\item Some finanical markets are close to the Walrasian ideal. Others, such as the market for mortage-backed securities, operate with \emph{over-the-counter} trading
\item Search theory charachterize volume of trading, bid-ask spreads, how the supply of buyers might collapse during crises
\item See Lagos and Rocheteau (Ecmtra 2009)
\eit

\item Family economics
\bit
\setlength\itemsep{0.5em}
\item Marriage is a matching process under costly search
\item See Shimer and Smith (Ecmtra 2000)
\eit


\eit

\end{frame}


\begin{frame}{Search models: what we have not covered: Epidemilogy}

\bit
\setlength\itemsep{1em}

\item The baseline model of disease spread is a random matching model - the SIR model

\item Given a number of Susceptibles $S$ and a number of infected $I$, assume that the number of newly infected is given by matching function $M = \beta S I$

\item Law of motions
\begin{eqnarray}
\dot{S} &=& - \beta S I \nonumber \\
\dot{I} &=& \beta S I - \gamma I \nonumber \\
\dot{R} &=& \gamma I \nonumber
\end{eqnarray}
where $R$ is the number of dead/recovered 

\item SIR offers a way to forecast disease spread and study lockdown policies 

\item In epi literature, there is little treatment of optimizing behaviour and incentives
\bit
\item In other words, $\beta$ is taken as a parameter and not an endogeneous variable
\eit

\item Eichenbaum-Rebelo-Trabant (RFS 2021): $\beta = \beta (C_t, H_t)$, integrate SIR with standard RBC economy
\bit
\item A lot of papers currently extendning this analysis in many interesting ways
\eit


\eit

\end{frame}



\end{document}







