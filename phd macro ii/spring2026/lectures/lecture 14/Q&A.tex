\documentclass[9pt]{beamer}
\usetheme{Boadilla}

\makeatother
\setbeamertemplate{footline}
{
	\leavevmode%
	\hbox{%
		\begin{beamercolorbox}[wd=.4\paperwidth,ht=2.25ex,dp=1ex,center]{author in head/foot}%
			\usebeamerfont{author in head/foot}\insertshortauthor
		\end{beamercolorbox}%
		\begin{beamercolorbox}[wd=.6\paperwidth,ht=2.25ex,dp=1ex,center]{title in head/foot}%
			\usebeamerfont{title in head/foot}\insertshorttitle\hspace*{3em}
			\insertframenumber{} / \inserttotalframenumber\hspace*{1ex}
	\end{beamercolorbox}}%
	\vskip0pt%
}
\makeatletter
\setbeamertemplate{navigation symbols}{}

\usepackage{tikz}
\usetikzlibrary{positioning}

\usepackage{lipsum}
\usepackage{appendixnumberbeamer}

\usepackage[authoryear]{natbib}
\usepackage[latin1]{inputenc}
\usepackage[T1]{fontenc}
\usepackage{caption}
\usepackage{amsmath, amssymb}
\usepackage{epstopdf}
\usepackage{graphicx}
\usepackage{lmodern}
\usepackage{xcolor}
\usepackage{xpatch}
\usepackage{multirow}

\usepackage{amsmath,theorem,amssymb,graphicx, pgfplots, tabularx, placeins}
\usepackage{dsfont}
\usepackage{caption}
%\usepackage{subcaption}
%\usepackage{subcaption}
\setbeamertemplate{caption}{\raggedright\insertcaption\par}
%\setbeamertemplate{footline}[frame number]
\usepackage{csquotes}
\usepackage{bm}
\bibliographystyle{econometrica}
\usepackage[normalem]{ulem}

\usepackage{setspace}


\definecolor{gray(x11gray)}{rgb}{0.75, 0.75, 0.75}


\newcommand{\bit}{\begin{itemize}}
	\newcommand{\eit}{\end{itemize}}
\newcommand{\ben}{\begin{enumerate}}
	\newcommand{\een}{\end{enumerate}}

\newcommand{\bc}{\color{blue}}
\newcommand{\rc}{\color{red}}


\newcommand{\lb}{\label}
\newcommand{\re}{\eqref}

\title[Incomplete Markets in General Equilibrium]{Macroeconomics II, Q\&A session}
\author{Erik {\"O}berg}
\date{May 15, 2023}

\begin{document}

\maketitle

\begin{frame}{Agenda}

\ben
\setlength\itemsep{2em}

\item Clarification of some lecture note material

\item Some questions regarding problem set exercises

\item Some questions regarding old exams

\een

\end{frame}


\begin{frame}

\begin{center}
	\huge Clarification of some lecture note material \normalfont
\end{center}

\end{frame}


\begin{frame}{Question: How to use "bounded solution condition" when analyzing IRFs?}

\bit
	\setlength\itemsep{2em}
	\item Context: IRF to monetary policy shock in NK model (lecture 3)
\eit

\begin{figure}
	\centering
	\includegraphics[scale=0.5,trim= 0 200 0 200, clip]{figures/nk_monshock_9variables.pdf}
\end{figure}

\end{frame}

\begin{frame}{IRFs to monetary policy shock: mechanism}

\bit
\setlength\itemsep{1em}
\item Take as given that $\hat r_t$ increases, then
\begin{eqnarray}
\text{Intertemporal hh optimality:} && \hat c_t =  - (\hat r_t) + E_t \hat c_{t+1} \nonumber
\end{eqnarray}
implies $\Delta E_t c_{t+1}$ is positive

\item {\bf Bounded solution $\Rightarrow$ $\hat c_t<0$}

\item Market clearing $\hat c_t = \hat y_t = \hat n_t < 0$
\bit
\item $\Rightarrow$ we may think of the output drop as being caused by drop in {\bc aggregate demand}
\eit

\item How is this consistent with optimal labor supply? Intratemporal optimality condition:
\begin{eqnarray}
\text{Intratemporal hh optimality:} && \hat \omega_t = \hat c_t + \varphi \hat n_t  \nonumber
\end{eqnarray}

\item $\hat n_t<0$ only if $\hat \omega_t < \hat y_t < 0$
\bit
\setlength\itemsep{0.5em}
\item Wages need to respond a lot!
\eit

\item $\hat \omega_t <0$ $\Rightarrow$ $\hat mc_t<0$ $\Rightarrow$ $\beta E_t \pi_{t+1}-\pi_t \approx \Delta \pi_{t+1} >0$ from the Phillips curve

\item Bounded solution $\Rightarrow$ $\hat \pi_t<0$

\item Notice how the assumption of a ``bounded solution'' enters the analysis

\eit


\end{frame}


\begin{frame}{bounded solution}

\bit
\setlength\itemsep{1em}
\item ``Bounded solution'' means that no IRF diverge

\item Ignoring ``oscillation solutions'' (they are only possible with higher order AR(p) solutions - not possible in these models), the solution must converge to some steady state

\item Moreover, we also know that steady state is unique

\item Therefore, the solution must have $\hat c \to 0$ in the long run

\item With $\Delta E_t \hat c_{t+1}>0$ for all $t$, we must have $\hat c_0 <0$ 

\eit


\end{frame}



\begin{frame}

\begin{center}
	\huge Some questions regarding problem set exercises \normalfont
\end{center}

\end{frame}

\begin{frame}{Question: How to interpret IRFs in the NK model?}

\bit
\setlength\itemsep{2em}
\item Context: Problem set 2, question 1 (government spending shocks).
\eit


\end{frame}


\begin{frame}{The vanilla equilibrium system with sticky prices }

\begin{eqnarray}
\text{Intratemporal hh optimality:} && \hat \omega_t = \hat c_t + \varphi \hat n_t  \nonumber \\
\text{Intertemporal hh optimality:} && \hat c_t =  - (\hat i_t - E_t \pi_{t+1}) + E_t \hat c_{t+1}  \nonumber \\
\text{Firm optimality:} && \pi_t = \beta E_t \pi_{t+1} + \lambda \widehat{mc}_t \nonumber \\
\text{Marginal cost:} && \widehat{mc}_t = \hat \omega_t \nonumber \\
\text{Goods clearing:} && \hat c_t = \hat y_t \nonumber \\
\text{Bonds clearing:} && \hat b_t = 0 \nonumber \\
\text{Labor clearing:} && \hat y_t = \hat n_t \nonumber \\
\text{Mon policy rule:} && \hat i_t = \phi \pi_t. \nonumber 
\end{eqnarray}

\end{frame}

\begin{frame}{The equilibrium system with sticky prices and gov spending}

\begin{eqnarray}
\text{Intratemporal hh optimality:} && \hat \omega_t = \hat c_t + \varphi \hat n_t  \nonumber \\
\text{Intertemporal hh optimality:} && \hat c_t =  - (\hat i_t - E_t \pi_{t+1}) + E_t \hat c_{t+1}  \nonumber \\
\text{Firm optimality:} && \pi_t = \beta E_t \pi_{t+1} + \lambda \widehat{mc}_t \nonumber \\
\text{Marginal cost:} && \widehat{mc}_t = \hat \omega_t \nonumber \\
\text{Goods clearing:} && \bar c\hat c_t + \bar g \hat g_t = \hat y_t \nonumber \\
\text{Bonds clearing:} && \hat b_t = 0 \nonumber \\
\text{Labor clearing:} && \hat y_t = \hat n_t \nonumber \\
\text{Mon policy rule:} && \hat i_t = \phi \pi_t. \nonumber \\
\text{Spe policy rule:} && \hat g_t = \rho g_{t-1} + \epsilon^g_t. \nonumber 
\end{eqnarray}
where 
\begin{eqnarray}
\bar c = \frac{C}{Y}, \bar g = \frac{G}{Y} \nonumber
\end{eqnarray}

\end{frame}

\begin{frame}{The equilibrium system with flexible prices and gov spending}

\begin{eqnarray}
\text{Intratemporal hh optimality:} && \hat \omega_t = \hat c_t + \varphi \hat n_t  \nonumber \\
\text{Intertemporal hh optimality:} && \hat c_t =  - (\hat i_t - E_t \pi_{t+1}) + E_t \hat c_{t+1}  \nonumber \\
\text{Firm optimality:} &&  \widehat{mc}_t = 0 \nonumber \\
\text{Marginal cost:} && \widehat{mc}_t = \hat \omega_t \nonumber \\
\text{Goods clearing:} && \bar c\hat c_t + \bar g \hat g_t = \hat y_t \nonumber \\
\text{Bonds clearing:} && \hat b_t = 0 \nonumber \\
\text{Labor clearing:} && \hat y_t = \hat n_t \nonumber \\
\text{Mon policy rule:} && \hat i_t = \phi \pi_t. \nonumber \\
\text{Spe policy rule:} && \hat g_t = \rho g_{t-1} + \epsilon^g_t. \nonumber 
\end{eqnarray}
where 
\begin{eqnarray}
\bar c = \frac{C}{Y}, \bar g = \frac{G}{Y} \nonumber
\end{eqnarray}

\end{frame}


\begin{frame}{IRFs in the flex price equilibrium}

\begin{figure}
	\centering
	\includegraphics[scale=0.45,trim= 0 0 0 0, clip]{figures/nk_govshock_flexprices.pdf}
\end{figure}

\end{frame}

\begin{frame}{Interpretation}

\bit
\setlength\itemsep{1em}
\item Direct effect of government spending shock: $\hat g_t$ up $\hat c_t$ down (due to tax financing)
\bit
	\item Always start here!
	
	\item Similar to when you analyze (positive) TFP shocks, direct effect: output up.
\eit

\item Equilibrium effects:
\bit
	\item Household intratemporal optimality: $\hat c_t$ down $\rightarrow$ $\hat n_t$ up (``wealth effect'')
	
	\item Bounded solution: $\hat c_t$ down $\rightarrow$ $\Delta \hat c_{t+1} > 0$
	
	\item Household intertemporal optimality: $\Delta \hat c_{t+1} > 0$  $\Rightarrow$ $\hat i_t - E_t \pi_{t+1}$ up
	
	\item Policy rule: $\hat i_t - E_t \pi_{t+1}$ up $\rightarrow$ $i_t$ up and $\pi_t$ up
	\bit
		\item Policy rule implies that interest rates and (expected) inflation always moves in the same direction. 
	\eit
\eit

\eit


\end{frame}


\begin{frame}{IRFs in the sticky-price equilibrium}

\begin{figure}
	\centering
	\includegraphics[scale=0.45,trim= 0 0 0 0, clip]{figures/nk_govshock.pdf}
\end{figure}

\end{frame}

\begin{frame}{Interpretation}

\bit
\setlength\itemsep{1em}
\item Direct effect of government spending shock: $\hat g_t$ up $\hat c_t$ down (due to tax financing)

\item Equilibrium effects in flex-price equilibrium:
\bit
\item Household intratemporal optimality: $\hat c_t$ down $\rightarrow$ $\hat n_t$ up (``wealth effect'')

\item Bounded solution: $\hat c_t$ down $\rightarrow$ $\Delta \hat c_{t+1} > 0$

\item Household intertemporal optimality: $\Delta \hat c_{t+1} > 0$  $\Rightarrow$ $\hat i_t - E_t \pi_{t+1}$ up

\item Policy rule: $\hat i_t - E_t \pi_{t+1}$ up $\rightarrow$ $i_t$ up and $\pi_t$ up

\eit


\item Added equilibrium effects in sticky-price equilibrium:
\bit
\item With sticky prices, what changes is
\begin{eqnarray}
\text{Firm optimality:} &&  \pi_t = \beta E_t \pi_{t+1} + \lambda \widehat{mc}_t \nonumber
\end{eqnarray}

\item Bounded solution: $\pi_t$ up $\rightarrow$ $\Delta E_t \pi_{t+1}$ down

\item Firm optimality: $\Delta E_t \pi_{t+}1$ down $\rightarrow$ $\widehat{mc}_t = \hat \omega_t$ up

\item Household intratemporal optimality: $\hat \omega_t$ up $\rightarrow$ $\hat n_t$ down

\item The fall in wages offset the wealth effect in the labor-supply decision, making the equilibrium response of output smaller.
\eit

\item Note: the increase in inflation was key to this argument!
\bit
\item The central bank could have offsetted the increase in inflation if it met the shock with increasing the interest rate more, i.e., a higher coefficent $\phi$

\item In this sense, the larger government spending multiplier under sticky prices is a consequence of ``loose'' monetary policy

\item Remark: This is also why the NK model predicts that spending multipliers may be very high when monetary policy is stuck at the ZLB (Woodford AEJmacro 2011, and a very large follow-up literature)

\eit


\eit


\end{frame}


\begin{frame}

\begin{center}
	\huge Some questions regarding old exams \normalfont
\end{center}

\end{frame}


\begin{frame}{Questions}

\bit
\setlength\itemsep{1em}
\item August 2022 Q3: How are two steady states possible?

\item March 2022 Q1: How to sketch IRFs?

\item March 2022 Q2 (also March 2022 Q1): How to use guess-and-verify?

\item March 2022 Q3: How to solve for the Hosios condition? 

\item March 2022 Q4: How to solve for changes in aggregate welfare? 

\eit


\end{frame}


\end{document}






