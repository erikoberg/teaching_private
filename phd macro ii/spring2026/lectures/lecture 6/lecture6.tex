\documentclass[9pt,xcolor={dvipsnames}]{beamer}
\usetheme{Boadilla}

\makeatother
\setbeamertemplate{footline}
{
	\leavevmode%
	\hbox{%
		\begin{beamercolorbox}[wd=.4\paperwidth,ht=2.25ex,dp=1ex,center]{author in head/foot}%
			\usebeamerfont{author in head/foot}\insertshortauthor
		\end{beamercolorbox}%
		\begin{beamercolorbox}[wd=.6\paperwidth,ht=2.25ex,dp=1ex,center]{title in head/foot}%
			\usebeamerfont{title in head/foot}\insertshorttitle\hspace*{3em}
			\insertframenumber{} / \inserttotalframenumber\hspace*{1ex}
	\end{beamercolorbox}}%
	\vskip0pt%
}
\makeatletter
\setbeamertemplate{navigation symbols}{}


\usepackage{lipsum}
\usepackage{appendixnumberbeamer}

\usepackage[authoryear]{natbib}
\usepackage[latin1]{inputenc}
\usepackage[T1]{fontenc}
\usepackage{caption}
\usepackage{amsmath, amssymb}
\usepackage{epstopdf}
\usepackage{graphicx}
\usepackage{lmodern}
%\usepackage[dvipsnames]{xcolor}
\usepackage{xpatch}
\usepackage{multirow}
\usepackage{tikz}
\usepackage{subfig}

\usepackage{amsmath,theorem,amssymb,graphicx, pgfplots, tabularx, placeins}
\usepackage{dsfont}
\usepackage{caption}
%\usepackage{subcaption}
%\usepackage{subcaption}
\setbeamertemplate{caption}{\raggedright\insertcaption\par}
%\setbeamertemplate{footline}[frame number]
\usepackage{csquotes}
\usepackage{bm}
\bibliographystyle{econometrica}
\usepackage[normalem]{ulem}
\usepackage{setspace}


\definecolor{gray(x11gray)}{rgb}{0.75, 0.75, 0.75}


\newcommand{\bit}{\begin{itemize}}
	\newcommand{\eit}{\end{itemize}}
\newcommand{\ben}{\begin{enumerate}}
	\newcommand{\een}{\end{enumerate}}

\newcommand{\black}{\color{black}}
\newcommand{\bc}{\color{blue}}
\newcommand{\rc}{\color{red}}
\newcommand{\gc}{\color{ForestGreen}}
\newcommand{\wc}{\color{white}}

\newcommand{\lb}{\label}
\newcommand{\re}{\eqref}

\title[NK: Topics]{Macroeconomics II Part II, Lecture VI:\\
	  The New-Keynesian Model: Policy}
\author{Erik {\"O}berg}
\date{}

\begin{document}

\begin{frame}
\maketitle
\end{frame}

\section{Introduction}


\begin{frame}{Recap}

\bit
\setlength\itemsep{1.5em} 

\item Basic NK model = RBC model + monopolistic competition and sticky prices

\item In contrast to RBC, NK predicts inefficient fluctations, and a role for monetary policy to affect the equilibrium

\item Today: when and how should be monetary policy be used to adress fluctuations in the NK model?
\eit

\end{frame}


\begin{frame}{Agenda}

\ben
\setlength\itemsep{1.5em} 

\item Ineffiencies in the NK model

\item Optimal monetary policy with TFP shocks

\item Optimal monetary policy with Cost-push shocks
\bit
	\item Discretion
	
	\item Commitment
\eit

\item Quantitative NK models: A helicopter view (time permitting)
\een

\end{frame}



\begin{frame}{...}


\begin{center}
	\huge Ineffiencies in the NK model \normalfont
\end{center}

\end{frame}


\begin{frame}{Ineffiencies in the NK model}

\bit
\setlength\itemsep{1.5em}
\item Before we think about policy, let's think about why decentralized outcomes might be suboptimal

\item The basic NK model has two frictions
\ben
\setlength\itemsep{0.5em}
\item Monopolistic competition

\item Frictional (Calvo) price setting
\een

\item Monopolistic competition: creates a (constant) labor wedge between MRS and MRT

\item Frictional price setting: implies 
\ben
	\item time-varying labor wedge due to time-varying markups
	
	\item time varying efficiency wedge due to time-varying price dispersion
\een
\eit

\end{frame}

\begin{frame}{Planner's problem}

\bit
\setlength\itemsep{1.5em}
\item The Planner maximizes utility subject to resource constraints

\item No capital $\Rightarrow$ completely static problem
\begin{eqnarray}
\max_{C_t, N_{it}} && \log C_t - \theta \frac{N_t^{1+\varphi}}{1+\varphi} \nonumber \\
\text{s.t.} && C_t = \left(\int_{0}^{1} (A_t N_{it})^{\frac{\epsilon-1}{\epsilon}} di \right)^{\frac{\epsilon}{\epsilon-1}} \nonumber \\
&& N_t = \int_{0}^{1} N_{it} di \nonumber
\end{eqnarray}

\item Set up Lagrangian and take F.O.C.'s - you'll find:
\begin{eqnarray}
\frac{\theta N^{\varphi}}{C^{-1}} &=& A_t \nonumber \hspace{5mm} \left(\text{i.e., } MRS=-\frac{V'(N)}{U'(C)}=MRT=F'(N)  \right) \\
N_{it} &=& N_t \nonumber \\
C_{t} &=& A_t N_t \nonumber 
\end{eqnarray}

\eit

\end{frame}


\begin{frame}{Inefficiences with flexible prices}

\bit
\setlength\itemsep{1em}
\item Consider the decentralized equilibrium with flexible prices

\item With flexible prices, all firms set the same price, $P_{it}=P_t$ for all $i$, and the markup is constant and given by $M= \frac{\epsilon}{\epsilon-1}$. Household and firm optimality:
\begin{eqnarray}
\frac{\theta N^{\varphi}}{C^{-1}} &=& \frac{W_t}{P_t} \nonumber \\
P_t &=& M \frac{W_t}{A_t} \nonumber 
\end{eqnarray}

\item Therefore, in steady state,
\begin{eqnarray}
\frac{\theta N^{\varphi}}{C^{-1}} &=& \frac{A_t}{M} \nonumber 
\end{eqnarray}

\item Higher markup $\Rightarrow$ lower real wages $\Rightarrow$ lower production

\item This is a \emph{static} distorition, has nothing to do with the dynamic response to shocks

\item Going forward, we eliminate the static distortion with lump-sum financed firm subsidy $(1-\tau)$, implying firm F.O.C.
\begin{eqnarray}
(1-\tau)P_t &=& M \frac{W_t}{A_t} \nonumber 
\end{eqnarray}


\eit

\end{frame}


\begin{frame}{Inefficiency with sticky prices I: time-varying average markup}

\bit
\setlength\itemsep{1.5em}
\item Calvo-pricing implies time-varying {\bc average markup} $M_t$:
\begin{eqnarray}
M_t &\equiv& \frac{(1-\tau) P_t}{MC^{nom}_t} = \frac{(1-\tau) P_tA_t}{W_t} \nonumber
\end{eqnarray}
where $P_t$ is the price level

\item With $1-\tau=M$, we can write
\begin{eqnarray}
P_t = \frac{M_t}{M} \frac{W_t}{A_t} \nonumber
\end{eqnarray}
implying that
\begin{eqnarray}
\frac{\theta N^{\varphi}}{C^{-1}} &=& \frac{M_t}{M}A_t \nonumber 
\end{eqnarray}

\item Calvo pricing implies some firms cannot update prices, and the one that do update do not set $P_{it} = M * MC^{nom}_t$ $\Rightarrow$ fluctuations in $M_t$
\bit
	\item These fluctuations are inefficient, even if average markup is zero (as with the firm subsidy)
	
	\item Note: This inefficiency creates fluctuations in the labor wedge
\eit

\eit

\end{frame}


\begin{frame}{Inefficiency with sticky prices II: price dispersion}

\bit
\setlength\itemsep{1.5em}
\item Lecture IV: Aggregation yields
\begin{eqnarray}
Y_t = A_t D_t N_t \nonumber
\end{eqnarray}
where 
\begin{eqnarray}
D_t = \int_{0}^{1} \left(\frac{P_{it}}{P_t} \right)^{-\epsilon}di \nonumber
\end{eqnarray}

\item In response to a shock, Calvo implies $P_{it} \neq P_{jt}$ $\Rightarrow$ $D_t$ lower
	\bit
		\item $D_t$ lower, as $(\cdot)^{-\epsilon}$ is a convex function (Jensen's inequality)
	\eit
	
\item Price dispersion generates an ineffiency due to {\bc missallocation}

\item Note: missallocation induces an efficiency wedge

\eit
	
\end{frame}


\begin{frame}{...}


\begin{center}
	\huge Optimal monetary policy with TFP shocks\normalfont
\end{center}

\end{frame}



\begin{frame}{Achieving the first best}

\bit
\setlength\itemsep{1em}
\item Consider the NK model subject to TFP shocks

\item Suppose that in period $t=-1$, $P_{i} = P_{-1}$ for all $i$

\item Suppose optimal $P^*_t = P_{-1}$ for all resetters in $t=0,1,2,...$

\item Then, no price disperion, and no time-varying markup $\Rightarrow$ first-best attained

\item $\Rightarrow$ the socially optimal allocation is achieved if all price setters stay with the initial price

\item Could (in theory) be implemented with time-varying subsidy to resetters

\item Could also (in theory) be implemented by time-varying taxes of household financial decisions $\Rightarrow$ ineffiency can be conceptaulized as arising from an {\bc aggregate-demand externality} (Farhi-Werning, Ectmra 2016; see also Correia-Farhi-Nicolini-Teles, AER 2013)

\item Could it be implemented with a suitable choice of $\hat i_t$?
\eit

\end{frame}


\begin{frame}{Can monetary policy achieve the first best?}

\bit
\setlength\itemsep{1.5em}

\item Recall: Our system is
\begin{eqnarray}
\text{DIS curve:} && \tilde y_t = - (\hat r_t-\hat r^n_t) + E_t \tilde y_{t+1}  \nonumber \\
\text{Phillips curve:} && \pi_t = \beta E_t \pi_{t+1} + \kappa \tilde y_t \nonumber 
\end{eqnarray}

\item We know that the efficient solution has $\pi_t = 0$ and $\tilde y_t = 0$
\bit
\setlength\itemsep{0.5em}
\item Zero inflation - not for it's own sake, but because it eliminates markup fluctuations and price dispersion

\item Output fluctuates due to TFP, but output gap is eliminated
\eit

\item A socially optimal interest rate makes sure the real interest rate tracks the natural real interest rate

\eit

\end{frame}

\begin{frame}{How can monetary policy implement first best? part I}

\bit
\setlength\itemsep{1.5em}
\item Suppose the monetary policy sets interest rates according to the rule
\begin{eqnarray}
\hat i_t = \hat r^n_t \nonumber
\end{eqnarray}

\item In that case, the equilibrium system becomes
\begin{eqnarray}
\text{DIS curve:} && \tilde y_t = -( E_t \pi_{t+1}) + E_t \tilde y_{t+1}  \nonumber \\
\text{Phillips curve:} && \pi_t = \beta E_t \pi_{t+1} + \kappa \tilde y_t \nonumber 
\end{eqnarray}

\item One solution: $\pi_t = \tilde y_t=0$

\item But, both eigenvalues are not inside unit circle (check this at home!) $\Rightarrow$ indeterminacy

\item Ergo, this rule does not garantuee that the first best is achieved
\eit

\end{frame}

\begin{frame}{How can monetary policy implement first best? part II}

\bit
\setlength\itemsep{1.5em}
\item Suppose instead that the monetary policy rule is 
\begin{eqnarray}
\hat i_t = \hat r^n_t+ \phi \pi_t  \nonumber
\end{eqnarray}

\item In that case, the equilibrium system becomes
\begin{eqnarray}
\text{DIS curve:} && \tilde y_t = - (\phi \pi_t - E_t \pi_{t+1}) + E_t \tilde y_{t+1}  \nonumber \\
\text{Phillips curve:} && \pi_t = \beta E_t \pi_{t+1} + \kappa \tilde y_t \nonumber 
\end{eqnarray}

\item Same determinacy condition as before: $\phi>1$ acheives determinacy: $\pi_t = 0$, $\tilde y_t=0$ is the unique bounded solution

\item With TFP shocks, sound monetary policy in the NK model boils down to tracking the natural real interest rate, and making credible threats to exclude other equilibria
\eit

\end{frame}


\begin{frame}{Comments}

\bit
\setlength\itemsep{1.5em}
\item In practice: secular movements in $r^n_t$ can perhaps be computed, but $\hat r^n_t$ is unobserved
\bit
\setlength\itemsep{0.5em}
	\item However, in theory: simple policy rules, e.g., $\hat i_t = \phi \pi_t$ effectively mitigates welfare losses if $\phi$ is sufficiently high
\eit

\item TFP shocks does not generate a trade-off for monetary policy decisions
\bit
\setlength\itemsep{0.5em}
\item If you stabilize inflation, you also stablize the output gap

\item Sometimes called the {\bc divine coincide}

\item No rationale for a dual mandade
\eit

\item Other shocks do...

\eit

\end{frame}


\begin{frame}{...}


\begin{center}
	\huge Optimal monetary policy with cost-push shocks \normalfont
\end{center}

\end{frame}


\begin{frame}{The policy problem}

\bit
\setlength\itemsep{1.5em}
\item With TFP shocks, there are no trade-offs and policy analysis is easy

\item Cost-push shocks create an interesting trade-off, and also a time inconsistency problem

\item With cost-push shocks, we need some more technique to get at optimal policy

\item In general, the policy problem is to set a sequence $\{\hat i_t \}$ such to maximize some social objective function, subject to that the allocation is an equilibrium:
\begin{eqnarray}
\max_{\hat i_t} && \text{Objective function} \nonumber \\
\text{s.t. } &&  x_t = - (\hat i_t-E_t \pi_{t+1}-\hat r^e_t) + E_t x_{t+1} \text{  } \forall t \nonumber \\
&& \pi_t = \beta E_t \pi_{t+1} + \kappa x_t + u_t \text{  } \forall t \nonumber 
\end{eqnarray}
where $x_t = y_t - y^e_t$

\item Which objective function?

\eit

\end{frame}

\begin{frame}{The objective function}

\bit
\setlength\itemsep{1em}
\item If the policy maker has the same preferences as the social planner, the objective function is the welfare of the representative household: $E_0 \sum U (C_t, N_t)$

\item To put the welfare function on par with the equilibrium expressed in terms of a log-linear appoximation around steady state, we need to approximate it

\item A linear approximation does not work - why?

\item Woodford (Book 2003): Let's consider a second-order approximation
\begin{eqnarray}
U_t-U = U_c C \left(\hat y_t + \frac{1-\sigma}{2} \hat y_t^2 \right) + \frac{U_N N}{1-\alpha} \left(\hat y_t + \frac{\epsilon}{2 \Theta} \int_{0}^{1} p_{it}^2 di  + \frac{1+\varphi}{2} (\hat y_t)\right) \nonumber
\end{eqnarray}
using that
\begin{eqnarray}
N_t = \int_{0}^{1} \frac{Y_{t}}{A}\left(\frac{P_{it}}{P_t} \right)^{-\epsilon} di, \text{and  } \hat c_t = \hat y_t = \hat n_t \nonumber
\end{eqnarray}

\item With a bit of work, one can show that, in equilibrium, the welfare loss amounts to
\begin{eqnarray}
E_0 \sum_{t=0}^{\infty} \beta^t (\pi_t^2 + \alpha_x x_t^2) \nonumber
\end{eqnarray}
with $\alpha_x = \frac{\kappa}{\epsilon}$

\eit

\end{frame}


\begin{frame}{The policy problem, again}

\bit
\setlength\itemsep{1.5em}
\item The policy problem is thus
\begin{eqnarray}
\max_{\{\hat i_t\}} && E_0 \sum_{t=0}^{\infty} \beta^t (\pi_t^2 + \alpha_x x_t^2) \nonumber \\
\text{s.t. } &&  x_t = - (\hat i_t-E_t \pi_{t+1}-\hat r^e_t) + E_t x_{t+1} \text{  } \forall t \nonumber \\
&& \pi_t = \beta E_t \pi_{t+1} + \kappa x_t + u_t \text{  } \forall t \nonumber 
\end{eqnarray}

\item There is clearly a trade-off in setting the interest rate path here

\item We can split the policy problem in two
\ben
	\item Find the equilibrium allocation that maximizes welfare	
	\begin{eqnarray}
	\max_{x_t, \pi_t} && E_0 \sum_{t=0}^{\infty} \beta^t (\pi_t^2 + \alpha_x x_t^2) \nonumber \\
	\text{s.t. } &&  \pi_t = \beta E_t \pi_{t+1} + \kappa x_t + u_t \text{  } \forall t \nonumber 
	\end{eqnarray}

\item Given a solution $\{x^*_t, \pi^*_t \}$, find the interest rate path that implements this allocation by solving
\begin{eqnarray}
x^*_t = - (\hat i_t-E_t \pi^*_{t+1}-\hat r^e_t) + E_t x^*_{t+1} \text{  } \forall t \nonumber 
\end{eqnarray}	
\een

\item Note: $\hat r^e_t$ is exogenous to the policy problem

\eit

\end{frame}


\begin{frame}{Discretion vs. commitment}

\bit
\setlength\itemsep{1em}
\item Suppose the central bank cannot commit to future interest rates $\{i_{t+s}\}_{s=1}^{\infty}$ when setting the interest rate today $i_t$

\item Then, the central bank knows that tomorrow, it will set an interest rate path that solves
\begin{eqnarray}
\max_{x_{t+s}, \pi_{t+s}} && E_{t+1} \sum_{s=1}^{\infty} \beta^s (\pi_{t+s}^2 + \alpha_x x_{t+s}^2) \nonumber \\
\text{s.t. } &&  \pi_{t+s} = \beta E_t \pi_{t+s+1} + \kappa x_{t+s} + u_{t+s} \text{  } \forall s>0 \nonumber 
\end{eqnarray}
resulting in some path $\{x^*_{t+s}, \pi^*_{t+s}\}_{s=1}^{\infty}$

\item This means that when setting the interest rate today, the central bank can take the future allocation as given $\{x^*_{t+s}, \pi^*_{t+s}\}_{t=s}^{\infty}$

\item $\Rightarrow$ The maximization problem today is static!
\begin{eqnarray}
\max_{x_t, \pi_t} && \pi_t^2 + \alpha_x x_t^2 \nonumber \\
\text{s.t. } &&  \pi_t = \beta E_t \pi^*_{t+1} + \kappa x_t + u_t \nonumber 
\end{eqnarray}

\item Note: this result stems from that the equilibrium has no state variable

\eit

\end{frame}


\begin{frame}{Optimal discretionary policy}

\bit
\setlength\itemsep{1.5em}
\item The maximization problem
\begin{eqnarray}
\max_{x_t, \pi_t} && \pi_t^2 + \alpha_x x_t^2 \nonumber \\
\text{s.t. } &&  \pi_t = \beta E_t \pi^*_{t+1} + \kappa x_t + u_t \nonumber 
\end{eqnarray}

\item F.O.C.
\begin{eqnarray}
\pi_t: &&  2 \pi_t - \lambda = 0 \nonumber \\
x_t: && 2 \alpha_x x_t - \lambda \kappa = 0 \nonumber
\end{eqnarray}

\item Combine to get
\begin{eqnarray}
x_t = - \frac{\kappa}{\alpha_x} \pi_t \nonumber
\end{eqnarray}

\item Higher inflation $\Rightarrow$ optimal policy features engineering an ``inefficient'' recession 



\eit

\end{frame}


\begin{frame}{Implication and Implementation}

\bit
\setlength\itemsep{1.5em}
\item Combine with Phillips curve to get path of inflation under this policy:
\begin{eqnarray}
\pi_t = \beta E_t \pi_{t+1} - \frac{\kappa^2}{\alpha_x} \pi_t + u_t \nonumber 
\end{eqnarray}

\item $\Rightarrow$ solves for some policy function $\pi_t = \Phi_{\pi} u_t$

\item Combine with DIS curve to get path of interest rate under this policy:
\begin{eqnarray}
\frac{\kappa}{\alpha_x} \pi_t = (\hat i_t-E_t \pi_{t+1}-\hat r^e_t) + \frac{\kappa}{\alpha_x} E_t \pi_{t+1}\nonumber 
\end{eqnarray}	

\item $\Rightarrow$ solves for some policy function $i_t = \hat r^e_t + \Phi_{i} u_t$

\item Implemented as unique equilibrium with rule $i_t = \hat r^e_t + \Phi_{i} u_t + \phi (\pi_t-\Phi_{\pi} u_t)$ for large enough $\phi$


\eit

\end{frame}

\begin{frame}{Optimal policy with commitment}

\bit
\setlength\itemsep{1.5em}
\item Suppose instead the central bank can commit. The policy problem is
\begin{eqnarray}
\max_{\{x_t, \pi_t\}} && E_0 \sum_{t=0}^{\infty} \beta^t (\pi_t^2 + \alpha_x x_t^2) \nonumber \\
\text{s.t. } &&  \pi_t = \beta E_t \pi_{t+1} + \kappa x_t + u_t \text{  } \forall t \nonumber 
\end{eqnarray}

\item Let $\{\lambda_t\}$ be the sequence of Lagrange multipliers. F.O.C is
\begin{eqnarray}
\pi_t: && \pi_t + \lambda_t - \lambda_{t-1} = 0 \nonumber \\
x_t && \alpha_x x_t -  \kappa \lambda_t = 0 \nonumber
\end{eqnarray}
which gives us
\begin{eqnarray}
x_0 &=& - \frac{\kappa}{\alpha_x} \pi_0 \nonumber \\
x_t &=& x_{t-1} - \frac{\kappa}{\alpha_x} \pi_t \nonumber
\end{eqnarray}

\item Still qualitatively similar tradeoff between output and inflation, but now optimal policy features some degree of smoothing: A high $x_{t-1}$ implies a high value of $x_t$


\eit

\end{frame}


\begin{frame}{Optimal policies in response to an AR(1) cost-push shock}

\begin{figure}
	\centering
	\includegraphics[scale=0.6]{Figures/sims_optimalpolicy.pdf}
\end{figure}

\bit
\setlength\itemsep{1.5em}
\item From Eric Sims' lecture notes

\eit

\end{frame}



\begin{frame}

\begin{center}
	\huge Quantitative NK models: A helicopter view \normalfont
\end{center}

\end{frame}


\begin{frame}{Quantitative NK models}

\bit
\setlength\itemsep{1.5em}
\item The simple 3-equation model is too stylized for much quantitative analysis

\item To make serious quantitative predictions about the response to shocks and policy, we need to incorporate more frictions

\item Key questions for quantitative NK models: which extensions are most important, and how to discipline the (often many) model parameters? 
\bit
	\item Which data moments should we target?
\eit

\item Resarch program initiated by Christiano-Eichenbaum-Evans (JPE 2005) and Smets-Wouters (JEDC 2003; AER 2007)

\item Quantitative NK models are used by many central banks to make forecast and analyze policy interventions
\bit
	\item Sveriges Riksbank's model: RAMSES II
\eit


\eit

\end{frame}

\begin{frame}{Quantitative NK models:ingredients}
\bit
\setlength\itemsep{1.5em}

\item Typical ingredients:
\ben
\setlength\itemsep{0.5em}
\item Capital and investment adjustment costs

\item Variable capacity utilization

\item Financial frictions

\item Price indexation (and positive steady-state inflation)

\item Consumption habits

\item Rigid wages
\een

\item Other ingredients too, but these are the most common (I think...)

\item Let's quickly look at the last two
\eit

\end{frame}


\begin{frame}{Consumption habits}

\bit
\setlength\itemsep{1em}
\item Macro aggregates, in particular aggregate consumption, tend to respond sluggishly to shocks

\item One way to capture this: consumption habits

\item Change utility function to
\begin{eqnarray}
U(C_t, C_{t-1}) = u(C_t-bC_{t-1}) \nonumber
\end{eqnarray}
$\Rightarrow$ utility penalty from changing consumption too fast

\item Changes household F.O.C. to
\begin{eqnarray}
\lambda_t = \beta^t u'(C_t-bC_{t-1}) - \beta^{t+1} b u'(C_{t+1}-bC_{t})\nonumber
\end{eqnarray}
$\Rightarrow$ adjust Euler equation and intratemporal optimality condition accordingly

\item Is this reasonable?
\bit
\setlength\itemsep{0.5em}
	\item Some form of consumption habit formation seems plausible
	
	\item Helps explaining some asset pricing puzzles (See, e.g., Campbell-Cochrane, JPE 1999)
	
	\item Hard to square with micro evidence on consumption response to transitory income shocks (Carroll-Crawley-Slacalek-Tokuoka-White, AEJmacro 2020; Auclert-Rognlie-Straub, 2020)
	
	\item Alternative: sticky expecations (Mankiw-Reis, QJE 2002)
\eit
\eit

\end{frame}

\begin{frame}{Fagereng-Holm-Natvik (AEJmacro 2020): Consumption response to lottery gains using Norwegian administrative data}

\begin{figure}
	\centering
	\includegraphics[scale=0.4]{Figures/Fagereng_figa5.pdf}
\end{figure}

\end{frame}


\begin{frame}{Rigid wages? Aggregate data}

\begin{figure}
	\centering
	\includegraphics[scale=0.45]{figures/wages_prod.pdf}
	\caption*{\footnotesize Detrended (HP-filtered) quarterly data. OECD estimate of total labor productivity. Average hourly earnings for total private sector excluding supervisory employees, deflated with PCE. Source: FRED and own calculations.}
\end{figure}

\end{frame}

\begin{frame}{Rigid wages? Micro data}

\begin{figure}
	\centering
	\includegraphics[scale=0.5]{figures/grigsby_fig2.pdf}
\end{figure}

\bit
	\item From Grigsby-Hurst-Yildirmaz (AER 2021), using micro data from the largest US payroll processing company
\eit

\end{frame}


%\begin{frame}{Rigid wages?}
%
%\begin{figure}
%	\centering
%	\includegraphics[scale=0.45]{figures/wages_urate.pdf}
%	\caption*{\footnotesize Detrended (HP-filtered) quarterly data. Average hourly earnings for total private sector excluding supervisory employees, deflated with PCE. Source: FRED and own calculations.}
%\end{figure}
%
%\end{frame}



\begin{frame}{Rigid wages}

\bit
\setlength\itemsep{1.5em}
\item Data indicative of wage rigidities
\bit
\setlength\itemsep{0.5em}
\item Average wage rate not very volatile, although hours worked and productivity is

\item Micro data suggest: some of low aggregate volatility is due to selection, some due to rigidness
\eit

\item Wage rigidity can be modelled analogously to Calvo-style price rigidity (Erceg-Henderson-Levin, JME 2000)
\bit
\setlength\itemsep{0.5em}
\item Assumption 1: each household provides a differentiated labor service $\Rightarrow$ workers have monopoly power and set their wages accordingly

\item Assumption 2: each household belongs to a union $\Rightarrow$ consumption insurance 

\item Assumption 3: constant probability $(1-\theta_w)$ of resetting wage
\eit

\item Optimality condition to resetter's problem + aggregation yields
\begin{eqnarray}
{\bc \pi^w_t = \beta E_t  \pi^w_{t+1} - \lambda_w (\hat \omega_t- (\hat c_t + \varphi \hat n_t))} \hspace{2mm} \text{instead of } {\rc\omega_t = c_t + \varphi n_t} \nonumber
\end{eqnarray}
%where ${\bc \mu_t =  - (c_t + \varphi n_t)}$

\item A good starting point, but not very compelling assumptions

\item Broer-Harmenberg-Krusell-\"{O}berg (AER:insights 2023) offer another framework for modelling wage rigidity maintaining competitive markets
\eit

\end{frame}

\begin{frame}{How to pick parameter values?}

\bit
\setlength\itemsep{1.5em}
\item Cristiano-Eichenbaum-Evans: Set parameters values to minimize distance of model IRF to data IRF to monetary policy shocks
\ben
\setlength\itemsep{0.5em}
	\item Pick some parameter values
	
	\item Compute IRFs
	
	\item Calculate squared distance

	\item Update parameters, iterate
\een

\item Smets-Wouters: set parameters (including shock process parameters) so that model matches unconditional time series on macro aggregates
\bit
\setlength\itemsep{0.5em}
	\item Often done with {\bc Bayesian estimation techniques}
	
	\item This means specifying a prior distribution of parameter values, and then update according to Bayes law when model is confronted with data
\eit

\item Contrast this approach with RBC-style calibration

\eit

\end{frame}


\begin{frame}{Christiano-Eichenbaum-Evans (JPE 2005): model fit}

\begin{figure}
	\centering
	\subfloat[][]{\includegraphics[width=.25\textwidth]{Figures/cee_fig1_part1.pdf}}
	\subfloat[][]{\includegraphics[width=.25\textwidth]{Figures/cee_fig1_part2.pdf}}
	\subfloat[][]{\includegraphics[width=.25\textwidth]{Figures/cee_fig1_part3.pdf}}
	\subfloat[][]{\includegraphics[width=.25\textwidth]{Figures/cee_fig1_part4.pdf}}
	\caption*{$+$: empirical IRFs with CI-bands, $-$: estimated model IRFs. Y-axis: percentage points. X-axis: quarters. US data.}
\end{figure}

\bit
\setlength\itemsep{1.5em}
\item Empirical IRFs estimated using recursive ordering

\item CEE show that rigid wages is the key friction needed to match these moments
\eit


\end{frame}


\begin{frame}{Summing up}

\bit
\setlength\itemsep{1.5em}
\item NK models: RBC with monopolistic competition and sticky prices

\item Inefficient business cycles due to time-varying markups and price dispersion

\item When flex price equilibrium is optimal: policy can mitigate efficiency loss, both in terms of inflation and output, by implementing natural real interest rate

\item When shocks distort flex-price equilibrium: policy trade-off in terms of stabilizing inflation and output, and also a time-inconsistency problem

\item With a reasonable set of model extensions, NK model can match empirical IRFs to monetary policy shocks in quite some detail

\eit

\end{frame}

\begin{frame}{New-Keynesian theory: a very incomplete history}

\bit
\setlength\itemsep{1em} 

\item \bf Wicksell (1898): Interest and Prices\normalfont 
\bit
	\item Natural rate hypothesis; interest-rate gap as cause of inflation and real fluctuations
\eit

\item Keynes (1936): The General Theory

\item 40-60s: ``Neoclassical synthesis'', i.e., Keynesian equilibrium models (e.g. IS-LM) 
\bit
	\item Hicks, Samuelson, Mogiliani, Tobin etc.
\eit

\item \bf Patinkin (1956): Money, Interest and Prices \normalfont

\item Phelps (Economica 1967; JPE 1968): forward-looking Phillips curve based on firm's price setting behavior
\bit
	\item Taylor (JPE 1980): staggered price-setting; Calvo (JME 1983): random price-setting
\eit

\item 1980's: theoretical insights regarding interplay of ``aggregate demand'', price stickiness and monopoly power
\bit
\setlength\itemsep{0.5em} 
\item Akerlof-Yellen (QJE 1985); Mankiw (QJE (1985); Blanchard-Kiyotaki (AER 1987); Ball-Romer (ReStud 1990)
\eit

\item 1990's: ``New neoclassical synthesis'', i.e., the integration of sticky prices in micro-founded (RBC-style) business cycle models
\bit
\setlength\itemsep{0.5em} 
\item Yun (JME 1996); Rotemberg-Woodford (NBER 1997); Goodfriend-King (NBER 1997); Clarida-Gal\'{i}-Gertler (JEL 1999)
\eit

\item \bf Woodford (2003): Interest and Prices \normalfont

%\item 2000's: Quantitative models, e.g., Cristiano-Eichenbaum-Evans (JPE 2005), Smets-Wouters (JEDC 2003; AER 2007)
%
%\item Recent development: Heterogenous-agent New-Keynesian models, more about this later...

\eit

\end{frame}




\end{document}














