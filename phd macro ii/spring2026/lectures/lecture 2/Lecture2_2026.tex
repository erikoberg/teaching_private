\documentclass[9pt,xcolor={dvipsnames}]{beamer}
\usetheme{Boadilla}

\makeatother
\setbeamertemplate{footline}
{
	\leavevmode%
	\hbox{%
		\begin{beamercolorbox}[wd=.4\paperwidth,ht=2.25ex,dp=1ex,center]{author in head/foot}%
			\usebeamerfont{author in head/foot}\insertshortauthor
		\end{beamercolorbox}%
		\begin{beamercolorbox}[wd=.6\paperwidth,ht=2.25ex,dp=1ex,center]{title in head/foot}%
			\usebeamerfont{title in head/foot}\insertshorttitle\hspace*{3em}
			\insertframenumber{} / \inserttotalframenumber\hspace*{1ex}
	\end{beamercolorbox}}%
	\vskip0pt%
}
\makeatletter
\setbeamertemplate{navigation symbols}{}


\usepackage{lipsum}
\usepackage{appendixnumberbeamer}

\usepackage[authoryear]{natbib}
\usepackage[latin1]{inputenc}
\usepackage[T1]{fontenc}
\usepackage{caption}
\usepackage{amsmath, amssymb}
\usepackage{epstopdf}
\usepackage{graphicx}
\usepackage{lmodern}
%\usepackage[dvipsnames]{xcolor}
\usepackage{xpatch}
\usepackage{multirow}
\usepackage{tikz}

\usepackage{amsmath,theorem,amssymb,graphicx, pgfplots, tabularx, placeins}
\usepackage{dsfont}
\usepackage{caption}
%\usepackage{subcaption}
%\usepackage{subcaption}
\setbeamertemplate{caption}{\raggedright\insertcaption\par}
%\setbeamertemplate{footline}[frame number]
\usepackage{csquotes}
\usepackage{bm}
\bibliographystyle{econometrica}
\usepackage[normalem]{ulem}
\usepackage{setspace}


\definecolor{gray(x11gray)}{rgb}{0.75, 0.75, 0.75}


\newcommand{\bit}{\begin{itemize}}
	\newcommand{\eit}{\end{itemize}}
\newcommand{\ben}{\begin{enumerate}}
	\newcommand{\een}{\end{enumerate}}

\newcommand{\black}{\color{black}}
\newcommand{\bc}{\color{blue}}
\newcommand{\rc}{\color{red}}
\newcommand{\gc}{\color{ForestGreen}}
\newcommand{\wc}{\color{white}}

\newcommand{\lb}{\label}
\newcommand{\re}{\eqref}

\title[RBC accounting, measurement, extensions]{Macroeconomics II Part II, Lecture II:\\
	  RBC: accounting, measurement, extensions}
\author{Erik {\"O}berg}
\date{}
\begin{document}

\begin{frame}
\maketitle
\end{frame}

\section{Introduction}

\begin{frame}{Last time}

\bit
\setlength\itemsep{1.5em} 

\item We set up and learned how to solve the vanilla RBC model

\item The calibrated model can seemingly explain several business cycle moments...

\item .. but fails in some key aspects

\item Today, we will dig deeper into understanding the RBC mechanisms and how the model can be improved

\eit

\end{frame}


\begin{frame}{Agenda}

\ben
\setlength\itemsep{1.5em} 

\item Business cycle accounting

\item Measuring technology shocks

\item Two extensions to adress the the labor wedge
\bit
\setlength\itemsep{0.5em} 

	\item Employment lotteries
	
	\item GHH preferences
	
\eit

\een

\end{frame}



\begin{frame}{Business cycle accounting}

\bit
\setlength\itemsep{1.5em} 

\item The plain vanilla RBC model provides an efficient benchmark

\item Other BC models can be compared to this benchmark
\bit
	\item A succesful model is a model that does something better than the vanilla RBC model in terms of explaining the data
\eit

\item For this purpose, it is useful to develop methods that make it more precise what the vanilla RBC model does well and what it does not so well

\item One such method is called {\bc Business Cycle Accouting}

\item Developed by Chari-Kehoe-McGrattan (Ecmtra, 2007), further elaborated in Brinca-Chari-Kehoe-McGrattan (HB Macro, 2016)

\eit

\end{frame}

\begin{frame}{The idea}

\bit
\setlength\itemsep{1.5em} 

\item Taking the parameters as given, the RBC model describes relationships between variables that we can measure in the data

\item Thus, with given parameters and time series of these variables, we can see exactly which of these relationships fails and by how much

\item Let's see how it works

\eit

\end{frame}

\begin{frame}{RBC log-linear equilibrium}

\bit
\setlength\itemsep{1.5em} 

\item Setting TFP shocks to zero $\hat a_t=0$, our log-linear system is
\begin{eqnarray}
\hat w_t &=& \hat c_t + \varphi n_t \nonumber \\
\hat c_t &=& - \beta R^{r} E_t \hat r^{r}_{t+1} + E_t \hat {c}_{t+1} \nonumber \\
\hat y_t &=& \frac{C}{Y}\hat c_t + \frac{I}{Y} \hat i_t  \nonumber \\
\hat y_t &=& \alpha \hat k_t + (1-\alpha ) \hat n_t \nonumber \\
\hat k_{t+1} &=& (1-\delta) \hat k_t +\delta \hat i_t \nonumber\\
\hat r^{r}_t &=& -(1-\alpha)(\hat k_t - \hat n_t)  \nonumber \\
\hat w_t &=& \alpha (\hat k_t-\hat n_t) \nonumber
\end{eqnarray}
where steady state $R^{r}, C, I, Y$ are determined by the model parameters

\item Using that 
\begin{eqnarray}
\hat r^{r}_t &=& \hat y_t-\hat k_t, \nonumber \\
\hat w_t &=& \hat y_t-\hat n_t, \nonumber
\end{eqnarray}
we can eliminate $\hat r^{r}_t, \hat w_t$ from the system

\eit

\end{frame}


\begin{frame}{RBC log-linear equilibrium}

\bit
\setlength\itemsep{1.5em} 

\item The reduced system is
\begin{eqnarray}
\hat y_t &=& \hat c_t + \varphi (1+n_t) {\wc + \hat \psi^N_t} \nonumber \\
\hat c_t &=& - \beta R^{r}  E_t (\hat y_{t+1}-\hat k_{t+1}) + E_t \hat {c}_{t+1} {\wc + \hat \psi^I_t} \nonumber \\
\hat y_t &=& \frac{C}{Y}\hat c_t + \frac{I}{Y} \hat i_t {\wc + \hat \psi^G_t} \nonumber \\
\hat y_t &=& \alpha \hat k_t + (1-\alpha ) \hat n_t {\wc + \hat \psi^A_t} \nonumber 
\end{eqnarray}

\item This system describes a relationship between NIPA variables $Y,C,K,I$ and hours worked $N$ - stuff we know how to measure in the data

\item Q: By how much does these relationship break in the data?

\item {\wc Reformulated Q: How large are the {\wc wedges}}
\eit

\end{frame}


\begin{frame}{RBC log-linear equilibrium}

\bit
\setlength\itemsep{1.5em} 

\item The reduced system is
\begin{eqnarray}
\hat y_t &=& \hat c_t + \varphi (1+n_t) {\bc + \hat \psi^N_t} \nonumber \\
\hat c_t &=& - \beta R E_t (\hat y_{t+1}-\hat k_{t+1}) + E_t \hat {c}_{t+1} {\bc + \hat \psi^I_t} \nonumber \\
\hat y_t &=& \frac{C}{Y}\hat c_t + \frac{I}{Y} \hat i_t {\bc + \hat \psi^G_t} \nonumber \\
\hat y_t &=& \alpha \hat k_t + (1-\alpha ) \hat n_t {\bc + \hat \psi^A_t} \nonumber 
\end{eqnarray}

\item This system describes a relationship between NIPA variables $Y,C,K,I$ and hours worked $N$ - stuff we know how to measure in the data

\item Q: By how much does these relationship break in the data?

\item {\black Reformulated Q: How large are the {\bc wedges}?}

\eit

\end{frame}

\begin{frame}{Wedges in the data}

\begin{figure}
	\centering
	\includegraphics[scale=0.45]{Figures/sims_wedges.pdf}
	\caption*{From Eric Sims' lecture notes, using detrended data from John Fernald and same parameter values as in Lecture 1}
\end{figure}


\end{frame}


\begin{frame}{Wedges: interpretation}

\bit
\setlength\itemsep{1.5em} 

\item Efficiency wedge is the same thing as the Solow residual
\bit
	\item We treated this as an {\bc input} to the model 
\eit

\item We need large fluctuations in the labor wedge
\bit
	\item $\Rightarrow$ Indicates that something is not right about the labor market equilibrium in the model
\eit

\item We do not need large fluctuations in the investment wedge
\bit
\setlength\itemsep{0.5em} 
	\item This is surprising, the RBC theory of consumption/investment seems very rudimentary...
\eit

\item Large fluctuations in the government consumption wedge comes as no surprise
\bit
\setlength\itemsep{0.5em}
	\item We know that both exports/imports and the government expenditure fluctuate a lot over the business cycle
	
	\item Should perhaps not be interpreted as a failure, we simply didn't attempt to get this right
	
	\item For a benchmark ``international RBC model'', see Backus-Kehoe-Kydland (JPE 1992)
\eit

\eit

\end{frame}


\begin{frame}{Wedges: interpretation II}

\bit
\setlength\itemsep{1.5em} 

\item Notice that prices do not appear in this accounting method

\item The labor market wedge could stem either from
\ben
\setlength\itemsep{0.5em} 
\item $W_t \neq$ Marginal rate of substitution (MRS) between consumption and leisure, or

\item $W_t \neq$ Marginal product of labor (MPL)
\een

\item Similarly, the investment wedge could stem either from
\ben
\setlength\itemsep{0.5em} 
\item $R^{r}_t \neq$ MRS between consumption today and tomorrow, or

\item $R^{r}_t \neq$ MPK
\een

\item Nothing stops us from extending the accounting method to include prices
\bit
\setlength\itemsep{0.5em} 
	\item One attempt: Karabarbounis (RED 2014), finds that LM wedge is mostly due to $W_t \neq MRS$

	\item However, there is more uncertainty regarding measurement of prices compared to NIPA variables
	
	\item Wage fluctuations could, for example, be heavily influenced by cyclical selection
\eit

\item However, without including prices, we gain little information about which economic relationship that accounts for the wedges

\eit

\end{frame}


\begin{frame}

\begin{center}
	\huge Measuring technology shocks \normalfont
\end{center}

\end{frame}

\begin{frame}{Are Solow residuals are reasonable measure of technology shocks?}

\bit
\setlength\itemsep{1.5em} 

\item We apparently need a large fluctuations in the efficiency wedge to explain the data

\item Kydland-Prescott's original approach: interpret the efficiency wedge (or Solow residuals) as {\bc exogenous technology shocks}

\item Reasonable that technological progress fluctuates around trend, and that Solow residuals captures some of this

\item However, not reasonable that these \emph{detrended} fluctuations can be negative with $-5 \%$ or $-10 \%$ percent $\Rightarrow$ implies huge technological regress!
\bit
	\item Solow residuals implies that technological regress 40 \% of time in post-war US data (King-Rebelo, HBmacro 1999)
\eit

\item Technology concerns how factors of production are used for producing specific individual products (``blue-prints''); TFP is an abstract measurement concept
\bit
\setlength\itemsep{0.5em}
	\item TFP can capture many things
	
	\item Two examples: {\bc capacity utilization} and {\bc missallocation}
\eit

\eit

\end{frame}

\begin{frame}{Cleaning ``Solow residuals''}

\bit
\setlength\itemsep{1.5em} 

\item Capacity utilization: In reality, firms adjust the usage of their predetermined factors of production over the business cycle
\bit
\setlength\itemsep{0.5em}
\item True for both labor and capital

\item Capacity utilization often treated as a key statistic for measuring the state of the business cycle

\item Low utilization $\Rightarrow$ lower TFP

\item How to clean Solow residuals from utilization? Go and measure utilization!
\bit
\setlength\itemsep{0.5em} 
\item Burnside-Eichenbaum-Rebelo (EER 1996) uses detrended log electricity usage $z_t$ as a measure of capital utilization, measures technology shocks as
\begin{eqnarray}
a_t = y_t-(1-\alpha)n_t - \alpha (z_t + k_t) \nonumber
\end{eqnarray}

\item Basu-Fernald-Kimball (AER 2006) and Fernald (2014) uses input data to adjust sectoral solow residuals from capacity utilization, then aggregate sector-level productivity to TFP

\item Both find a technology series with 20-25 \% less volalility than original Solow residuals
\eit

\eit


\item Missallocation: Suppose the some of the factors of production $N_t, K_t$ are allocated to ``under-performing'' firms
\bit
\setlength\itemsep{0.5em}
\item This will also look like low TFP in the data

\item If missallocation moves with the business cycle, this will ``contaminate'' Solow residuals

\item Cyclical missallocation is actually a key prediction of the New-Keynesian model, introduced in Lecture III

\item I've not seen it applied in the RBC literature, but general measurement framework provided in Hsieh-Klenow (QJE 2009) 
\eit

\eit

\end{frame}



\begin{frame}{Simulation results with 25 \% lower TFP volatility (HP-filtered)}

\begin{table}[h!]
	\centering
	\begin{tabular}{|l|cc|cc|cc|cc|}
		\multicolumn{1}{c}{} &\multicolumn{2}{c}{\textbf{SD}}&\multicolumn{2}{c}{\textbf{Rel. SD}}& \multicolumn{2}{c}{\textbf{Corr $Y_t$}} & \multicolumn{2}{c}{\textbf{Autocorr}}  \\
		\hline
		& Data &Model & Data & Model & Data & Model & Data & Model \\
		\hline
		$Y_t$ & 0.017 & 0.011 & 1.00 & 1.00 & 1.00 & 1.00 & \colorbox{white}{0.85} & \colorbox{white}{0.72} \\
		\hline
		$C_t$ & 0.009 & 0.005  & \colorbox{white}{0.53} & \colorbox{white}{0.40} & \colorbox{white}{0.76} & \colorbox{white}{0.95}  & 0.79 & 0.78 \\
		\hline
		$I_t$ & 0.047 & 0.031 & \colorbox{white}{2.76} & \colorbox{white}{2.73}  & \colorbox{white}{0.79} & \colorbox{white}{0.99} & 0.87 & 0.72 \\
		\hline
		$N_t$ & 0.019 &  0.004 & \colorbox{white}{1.12} & \colorbox{white}{0.33} & \colorbox{white}{0.88} &  \colorbox{white}{0.98} & 0.90 & 0.72 \\
		\hline
		$W_t$ & 0.009 & 0.008 & 0.53 & 0.66 & \colorbox{white}{0.10} & \colorbox{white}{0.996} & 0.73 & 0.74 \\
		\hline
		$R_t$ & 0.004 & 0.011  & \colorbox{white}{0.24} & \colorbox{white}{1.0} & \colorbox{white}{0.00} & \colorbox{white}{0.97} & \colorbox{white}{0.42} & \colorbox{white}{0.71} \\
		\hline
		$A_t$ & 0.012 & 0.09 & \colorbox{white}{0.71} & \colorbox{white}{0.80}  & 0.76 & 0.999 & 0.75 & 0.72 \\
		\hline
	\end{tabular}
	\captionsetup{font=small}
	\label{sim_results}
\end{table} 

\bit
\setlength\itemsep{1em}

\item Reducing TFP volatility by 25 \% does not affect any correlations, but reduces volalitility of all variables by 25 \% - why?

\item Implication: the model needs to add more {\bc endogenous amplification} to fit the data
\bit
\item Put differently: the model cannot explain the efficiency wedge any more
\eit

\eit

\end{frame}







\begin{frame}{Taking stock}

\bit
\setlength\itemsep{1.5em}

\item A standard calibration of the vanilla RBC model with a reasonable technology shock process has quite a few problems with matching the data

\item The lack of empirical fit is manifested in large fluctuations in the labor wedge and the efficiency wedge (and also government consumption wedge)
\bit
\setlength\itemsep{0.5em}
	\item Relatedly, too little amplification and lack of persistence
	
	\item Does not produce reasonable fluctuations in hours worked
	
	\item $\Rightarrow$ we need more internal propagation, and part of it should come from larger response in hours worked
\eit

\item Also, recall the problems of getting the price moments right

\item Lets start by adressing the labor wedge and the efficiency wedge
\bit
\setlength\itemsep{0.5em}
	\item Problem set 5: endogenous capacity utilization to adress efficiency wedge
	
	\item Remainder of this lecture: adressing the labor wedge
\eit


\eit

\end{frame}




\begin{frame}

\begin{center}
	\huge Two extensions to adress the the labor wedge \normalfont
\end{center}

\end{frame}


\begin{frame}{The role of the Frish elasticity}

\bit
\setlength\itemsep{1.5em}

\item The vanilla RBC model needs more amplification, and it should come through hours worked

\item How to increase response of hours worked? Lets zoom in on the labor market equilibrium:

\item Household intratemporal first order condition describes the labor supply curve {\rc (Do diagram on whiteboard)}:
\begin{eqnarray}
\text{Labor supply:   }&& {\wc \hat w_t = \hat c_t + \varphi n_t} \nonumber \\
\text{Labor demand:   }&& {\wc \hat w_t = \hat a_t + \alpha (\hat k_t-\hat n_t)} \nonumber 
\end{eqnarray}

\item {\wc When productivity increases, we get larger response in hours either with larger subsitution effect or smaller income effect}


\eit

\end{frame}

\begin{frame}{The role of the Frish elasticity}

\bit
\setlength\itemsep{1.5em}

\item The vanilla RBC model needs more amplification, and it should come through hours worked

\item How to increase response of hours worked? Lets zoom in on the labor market equilibrium:

\item Household intratemporal first order condition describes the labor supply curve {\rc (Do diagram on whiteboard)}:
\begin{eqnarray}
\text{Labor supply:   }&& \hat w_t = \hat c_t + \varphi n_t \nonumber \\
\text{Labor demand:   }&& \hat w_t = \hat a_t + \alpha (\hat k_t-\hat n_t) \nonumber 
\end{eqnarray}

\item When productivity increases, we get larger response in hours either with larger {\bc subsitution effect} or smaller {\bc income effect}


\eit

\end{frame}


\begin{frame}{The substitution effect}

\bit
\setlength\itemsep{1.5em}

\item Holding the marginal utility of consumption constant, i.e. $c_t=0$, $\eta = 1/\varphi$ measures the elasticity of hours w.r.t. wages
\bit
\item Put differently, it measures the substitution effect of wages on labor supply
\eit

\item We call $\eta$ the {\bc Frish elasticity}

\item Our baseline calibration had $\eta = 1$

\eit

\end{frame}


\begin{frame}{Varying the Frish elasticity}

\begin{figure}
	\centering
	\includegraphics[scale=0.5, trim = 0 175 0 175, clip]{Figures/rbc_simulation_frish.pdf}
\end{figure}


\end{frame}

\begin{frame}{Is a large Frisch elasticity reasonable?}

\bit
\setlength\itemsep{1.5em}

\item Larger Frish elasticity $\Rightarrow$ more amplification through hours worked

\item Problem: most micro-level evidence suggest $\eta \in (0.0, 0.5)$
\bit
\setlength\itemsep{0.5em}
	\item Classics: MacCurdy (JPE, 1981); Ashenfelter (JME, 1984); Angrist (JE, 1991)

	\item Evidence reviewed in Chetty-Guren-Manoli-Weber (NBER annual, 2013)
	
	\item Martinez-Saez-Siegenthaler (AER, 2020) exploit tax holiday in Austria, find $\eta = 0.02$

	\item Stefanson (2019) and Sigurdsson (20222) exploit tax holiday in Iceland, find $\eta = 0.1-0.4$ 
\eit

\item Moreover, most variation in hours worked over the business cycle is along the extensive margin, our model does not speak to this

\item How to resolve?

\eit

\end{frame}


%\begin{frame}{US hours fluctuations driven by extensive margin}
%
%
%
%\begin{figure}
%	\centering
%	\includegraphics[scale=0.6]{Figures/rs_2011_fig1.pdf}
%	\caption*{\footnotesize Detrended log points. Blue solid - hours worked per person; red dashed - employment rate; green dotted - participation rate. From Rogerson-Shimer (Handbook LE 2011).}
%\end{figure}
%
%
%\end{frame}

\begin{frame}{Employment lotteries}

\bit
\setlength\itemsep{1.5em}

\item Hansen (JME 1985)/Rogerson (JME 1988) insight: {\bc indivisible labor} disconnects the aggregate from the individual Frisch elasticity
\bit
	\item In fact, with indivisibility, aggregate Frish elasticity can be \emph{infinitely large}, although micro elasticities might be very small (even 0...)	
\eit

\item Model: Suppose there exist a measure $1$ of ex-ante identical households, who have the same MacCurdy preferences:
\begin{eqnarray}
U(C_t, N_t) = \log C_t - \theta \frac{N_t^{1+\varphi}}{1+\varphi}\nonumber 
\end{eqnarray}

\item Assume choice set $N_t = \{0, \bar N\}$, but if households could pick freely, they would actually prefer some $N^*_t \in (0, \bar N)$

\item With {\bc complete markets}, these households can be made better off if they enter a lottery with other households on whether to work or not
%\bit
%	\item The fact that some households do not work in equilibrium can be interpreted as part of an optimal arrangement under such technological constraints
%\eit


\eit

\end{frame}


\begin{frame}{Employment lotteries II}

\bit
\setlength\itemsep{1.5em}

\item Denote the probability of working $\pi_t$

\item In expectation, households work $N_t = \pi_t \bar N$ hours

\item $N_t$ therefore also denotes aggregate hours worked

\item Complete markets imply $U_c(C_1, 0) = U_c(C_2, \bar N)$

\item With separable preferences, complete markets simply imply $C_1=C_2=C$

\item With approriate choice of $\pi_t$, households maximize expected utility

\eit

\end{frame}


\begin{frame}{Employment lotteries III}

\bit
\setlength\itemsep{1.5em}

\item Household per-period utility is
\begin{eqnarray}
U(C_t, N_t) &=& \log C_t - \pi_t \theta \frac{\bar N^{1+\varphi}}{1+\varphi} - (1-\pi_t) \frac{\theta \times 0}{1+\varphi} \nonumber \\
&=& \log C_t - \pi_t \theta \frac{\bar N^{1+\varphi}}{1+\varphi}\nonumber \\
&=& \log C_t - \frac{N_t}{\bar N} \theta \frac{\bar N^{1+\varphi}}{1+\varphi} \nonumber \\
&=& \log C_t - B N_t \nonumber
\end{eqnarray}
where $B = \frac{\theta}{\bar N} \frac{\bar N^{1+\varphi}}{1+\varphi}$

\item $\Rightarrow$ the lottery model is thus {\bc isomorphic} to our rep-agent model
\begin{eqnarray}
U(C_t, N_t) = \log C_t - \theta \frac{N_t^{1+\varphi}}{1+\varphi}\nonumber 
\end{eqnarray}
with $\varphi = 0$ and with $B$ calibrated to match the same average hours worked

\item Now we have that the aggregate Frish elasticity $= \infty$


\eit

\end{frame}


\begin{frame}{Employment lotteries: discussion}

\bit
\setlength\itemsep{1.5em}

\item Labor markets are obviously not organized through lotteries, but this is more of a technical trick to work anlytically with indivisible labor
\bit
\setlength\itemsep{0.5em}
	\item Indivisible labor $\Rightarrow$ discrete choice problem $\Rightarrow$ F.O.C.s are neither sufficient nor sufficient for characterizing the optimum
	
	\item The lottery convexifies the choice set $\Rightarrow$ we can proceed with our F.O.C.s as usual
\eit

\item Modern approaches instead assume a fixed cost of working and deals with the non-convexity up front (using numerical methods)
\bit
\setlength\itemsep{0.5em}
	\item Rogerson-Wallenius (JET 2009; AER 2013) show that the same discrepancy between micro-level and macro-level elasticities come out from quantitative life-cycle models 
\eit

\item At a more fundamental level, do we really believe the bulk of fluctuations in hours worked stem from households' labor supply choices? Isn't \emph{involuntary unemployment} the main reason we are concerned with recessions? 

\item Lectures 5-7 present a theory of unemployment that can be readibly integrated into RBC and other business cycle models 
\eit

\end{frame}


\begin{frame}{The income effect}

\bit
\setlength\itemsep{1.5em}

\item The income effect on labor supply is, with additively separable preferences, not separable from the elasticity of intermtemporal substiution

\item Household F.O.C.s:
\begin{eqnarray}
\hat w_t &=& \hat c_t + \varphi n_t \nonumber \\
\hat c_t &=& - \beta R^{r} E_t \hat r^{r}_{t+1} + E_t \hat {c}_{t+1} \nonumber 
\end{eqnarray}

\item If instead assuming CRRA consumption utility $U(C)=\frac{C^{1-\sigma}}{1-\sigma}$:
\begin{eqnarray}
\hat w_t &=& \sigma \hat c_t + \varphi n_t \nonumber \\
\hat c_t &=& - \beta R^{r} \frac{1}{\sigma}E_t \hat r^{r}_{t+1} + E_t \hat {c}_{t+1} \nonumber 
\end{eqnarray}

\item Higher elasticity of intermtemporal subsitution $\frac{1}{\sigma}$ $\Leftrightarrow$ smaller income effect on labor supply 

\eit

\end{frame}

\begin{frame}{What's the effect of having lower income effects? Introducing GHH preferences}

\bit
\setlength\itemsep{1.5em}

\item To admit separation, Greenwood-Hercowitz-Huffman (1988) proposed the following preference specification
\begin{eqnarray}
U(C,N) = U(C-V(N)) \nonumber
\end{eqnarray}

\item With this specification, we still have $U_c(C, N) = U'(\cdot)$ but $U_n(C, N) = - U'(\cdot)V'(N)$

\item The household's F.O.C. with respect to $N_t$ becomes
\begin{eqnarray}
-\beta^t U'(C_t-V(N_t)) V'(N) = W_t \lambda_t \nonumber
\end{eqnarray}
and thus
\begin{eqnarray}
V'(N) = W_t \nonumber
\end{eqnarray}

\item With $V(N) = \frac{N^{1+\varphi}}{1+\varphi}$, we get
\begin{eqnarray}
\hat w_t = \varphi n_t \nonumber
\end{eqnarray}

\item Now, the income effect is zero!

\eit

\end{frame}

\begin{frame}{Baseline vs. GHH}

\begin{figure}
	\centering
	\includegraphics[scale=0.5, trim = 0 175 0 175, clip]{Figures/rbc_simulation_ghh.pdf}
\end{figure}


\end{frame}

\begin{frame}{Is GHH a reasonable preference specifaction?}

\bit
\setlength\itemsep{1.5em}
	\item Cesarini-Lindqvist-Notowidigdo-�stling (AER 2017): labor supply responses to windfall lottery gains very small

	\item But if this is true, how to resolve with (almost) constant labor supply in the long run?
	
	\item Possible explanation: income effects are small in the short run, while still large in the long run
	\bit
	\setlength\itemsep{0.5em}
		\item Jaimovich-Rebelo (JPE 2009): with habits in household preferences, you can actually achieve this
	
		\item Broer-Harmenberg-Krusell-�berg (AER:insights 2023): this is what you should expect if households are subject to rigid wage contracts
	\eit
\eit

\end{frame}


\begin{frame}{Rigid-wage contracts a la Broer-Harmenberg-Krusell-�berg}

\bit
\setlength\itemsep{1em}
	\item ``Rigid'' wages = wage contracts cannot be contingent on the shock, nor can they be renegoiated once a shock happens

	\item Instead of having a competitive market for hours in exchange for fixed per-hour wages, suppose we have a competive market for rigid wage-hour contracts: $W(N)$
	
	\item If firms have the ``right to manage'', then conditional on a TFP shock, we get $A_t F'(N_t) = W'(N_t)$
	\bit
		\item Firms equate the marginal rate of transformation to the \emph{marginal wage}
	\eit
	
	\item If the asset market is complete, and aggregate shocks are small relative to idiosyncratic shocks, households marginal utility of consumption is constant when writing the contract
	\bit
		\item Ex ante, households will agree to the contract if $W'(N_t)=-\frac{V'(N_t)}{U'(C)}$ and $\mathbb E W(N_t)$ (``base pay'') is large enough 
	
		\item However, if given the opportunity to rewrite the contract after seeing the shock, households recognize that their consumption has changed, and demand $W'(N_t)=-\frac{V'(N_t)}{U'(C_t)}$.
	\eit
	
	\item $\Rightarrow$ as if GHH within the contract, as if KPR when rewriting the contract 
	\bit
		\item Frequency fo recontracting determines how short the short run is
	\eit
\eit

\end{frame}


\begin{frame}{Summing up}

\bit
\setlength\itemsep{1.5em} 

\item Busines cycle accounting (BCA): useful diagnosis tool $\Rightarrow$ we learned that vanilla RBC is associated with large labor and efficiency wedges
\bit
\setlength\itemsep{0.5em} 
	\item It is not reasonable to interpret the entire efficiency wedge as exogenous technological shocks

	\item After cleaing Solow residuals from utilization, amplification problem becomes bigger
\eit

\item Motivated by BCA, we learned that model fit is improved if making labor supply more elastic and muting income effec
\bit
	\item Highly elastic labor supply a natural implication of indivisble labor
	
	\item Small short-run income effects is a natural implication of rigid wage contracts
\eit

\item The lessons from lecture II will serve as benchmarks when we introduce models of frictional labor markets

\item But first: something about investment


\eit

\end{frame}







%
%
%\begin{frame}{Incorpating adjustment costs in the RBC model}
%
%\bit
%\setlength\itemsep{1.5em}
%
%\item Households solve
%\begin{eqnarray}
%\max_{\{C_t, N_t, I_{t}, K_{t+a}\}} && E_O \sum_{t=0}^{\infty} \beta^t \left[\log (C_t)- \theta \frac{N_t^{1+\varphi}}{1+\varphi} \right]\nonumber \\
%\text{s.t} && C_t + I_t \leq W_t N_t + R_t K_t \nonumber \\
%&& K_{t+1} \leq (1-\delta)K_t + I_t - \Phi\left(\frac{I_t}{K_t}\right)K_t \nonumber \\
%&& C_t, I_t, N_t \geq 0 \nonumber 
%\end{eqnarray}
%
%\item A concrete example that satisfies our conditions is
%\begin{eqnarray}
%\Phi\left(\frac{I_t}{K_t}\right) = \frac{\phi}{2}\left(\frac{I_t}{K_t}-\delta\right)^2 \nonumber
%\end{eqnarray}
%
%\item Notice: adjustment cost is paid with capital goods, not consumption goods
%
%\item Nothing conceptually new with this problem, just one more term to handle in our F.O.C.s
%
%\eit
%
%\end{frame}
%
%\begin{frame}{Household problem: the F.O.C.'s}
%
%\bit
%\setlength\itemsep{1em}
%
%\item Lagrangian to the household problem:
%\begin{eqnarray}
%\mathbf{L} = E_0 \Bigg(&& \sum_{t=0}^{\infty} \beta^t \left[\log (C_t)- \theta \frac{N_t^{1+\varphi}}{1+\varphi} \right] \nonumber \\
% &-& \sum_{t=0}^{\infty} \lambda_t \left[ C_t + I_{t} - W_t N_t - R_tK_t \right] \nonumber \\
% &-& \sum_{t=0}^{\infty} \mu_t \left[ K_{t+1} - (1-\delta)K_t-I_t+ \Phi\left(\frac{I_t}{K_t}\right)K_t \right]\Bigg) \nonumber
%\end{eqnarray}
%where I \bf have not \normalfont substituted the capital accumulation equation into the budget
%
%\item First order conditions
%\begin{eqnarray}
%&& \beta^t \frac{1}{C_t}- \lambda_t = 0 \nonumber \\
%&& -\beta^t \theta N_t^{\varphi} + \lambda_tW_t = 0 \nonumber \\
%&& -\lambda_t + \mu_t\left(1-\Phi'\left(\frac{I_t}{K_t}\right)\right)=0 \nonumber \\
%&& -\mu_t   + E_t\left[\lambda_{t+1} R_{t+1}+\mu_{t+1}\left((1-\delta) + \Phi'\left(\frac{I_{t+1}}{K_{t+1}}\right) \frac{I_{t+1}}{K_{t+1}} - \Phi\left(\frac{I_{t+1}}{K_{t+1}}\right) \right)  \right] = 0 \nonumber
%\end{eqnarray}
%
%\eit
%
%\end{frame}
%
%\begin{frame}{Household solution characterized}
%
%\bit
%\setlength\itemsep{1em}
%
%\item Putting things together, the solution is characterized by
%\begin{eqnarray}
%\frac{1}{C_t}W_t &=& \theta N_t^{\varphi} \nonumber \\
%q_t &=& \beta E_t\frac{C_t}{C_{t+1}} \left[ R_{t+1}+\left((1-\delta) + \Phi'\left(\frac{I_{t+1}}{K_{t+1}}\right) \frac{I_{t+1}}{K_{t+1}} - \Phi\left(\frac{I_{t+1}}{K_{t+1}}\right)  \right)  q_{t+1} \right] \nonumber \\
%q_t &=& \frac{1}{1-\Phi'\left(\frac{I_t}{K_t}\right) } \nonumber
%\end{eqnarray}
%where $q_t = \frac{\mu_t}{\lambda_t}$
%
%\item $q_t$ is called {\bc Hayashi's q}. What is the interpretation of this variable?
%
%\item What does the first and second equation tell us about the behavior of investment and asset prices?
%
%\eit
%
%\end{frame}
%
%\begin{frame}{Equilibrium characterization}
%
%\bit
%\setlength\itemsep{1em}
%
%\item Using the suggested functional form, the equilibrium is characterized by
%\begin{eqnarray}
%\frac{1}{C_t}W_t &=& \theta N_t^{\varphi} \nonumber \\
%q_t &=& \beta E_t\frac{C_t}{C_{t+1}} \left[ R_{t+1}+\left((1-\delta) + \Phi'\left(\frac{I_{t+1}}{K_{t+1}}\right) \frac{I_{t+1}}{K_{t+1}} - \Phi\left(\frac{I_{t+1}}{K_{t+1}}\right)  \right)  q_{t+1} \right] \nonumber \\
%q_t &=& \frac{1}{1-\phi\left(\frac{I_t}{K_t}-\delta\right) } \nonumber \\
%C_t + I_t &=& Y_t \nonumber \\
%Y_t &=& A_t K_t^{\alpha}N_t^{1-\alpha} \nonumber \\
%K_{t+1} &=& (1-\delta) K_t + I_t - \Phi\left(\frac{I_t}{K_t}\right)K_t \nonumber \\
%R_t &=& \alpha A_t \left(\frac{K_t}{N_t}\right)^{\alpha-1} \nonumber \\
%W_t &=& (1-\alpha) A_t \left(\frac{K_t}{N_t}\right)^{\alpha} \nonumber
%\end{eqnarray}
%
%\item One more equation, and more unknown: $q_t$
%
%\item Just to log-linearize and compute IRFs
%
%\eit
%
%\end{frame}
%
%
%\begin{frame}{IRFs when varying $\phi$}
%
%\begin{figure}
%	\centering
%	\includegraphics[scale=0.40,trim= 0 120 0 120, clip]{Figures/rbc_irf_adjcost.pdf}
%\end{figure}
%
%\end{frame}
%
%
%\begin{frame}{Investment adjustment cost: discussion}
%
%\bit
%\setlength\itemsep{1em}
%
%\item ff
%
%\eit
%
%\end{frame}



\end{document}

\begin{frame}{RBC: what have we not covered?}

\bit
\setlength\itemsep{1.5em}
\item Analysis of investment and financial frictions
\bit
\setlength\itemsep{0.5em}
\item Very large literature

\item I have lecture notes on this, let me know if you are interested
\eit

\item Other shocks
\bit
\setlength\itemsep{0.5em}
\item Investment-specific shocks: see, e.g., Greenwood-Hersowitz-Krusell (EER 2000)

\item Uncertainty shocks: see, e.g., Bloom (Ecmtra 2007)

\item News shocks: see, e.g., Beaudry-Portier (AER 2006; Koslyk 2023)
\eit

\item International business cycle models
\bit
\setlength\itemsep{0.5em}
\item see, e.g., Backus-Kehoe-Kydland (JPE 1992); Baxter (HBintecon 1995); Schmitt-Grohe-Uribe (Book 2017)
\eit

\eit

\end{frame}


\begin{center}
\huge Extension 2: Capacity Utilization \normalfont
\end{center}

\end{frame}


\begin{frame}{Capacity utilization}

\bit
\setlength\itemsep{1.5em}

\item We argued that volatility of our TFP shock process should be lowered due to variation in capacity utilization
\bit
\item As a consequence, the model has a bigger problem matching output fluctuations
\eit

\item Solution: put capacity utilization into the model!

\item Reminder: this will affect the level of output holding production factors constant $\Rightarrow$ speaks directly to the effiency wedge
\bit
\item But will also have implications for other variables
\eit

\item In our basic RBC model, the capital stock is predetermined, but labor can be freely adjusted
\bit
\setlength\itemsep{0.5em}
\item Put differently, capital is a {\bc state variable}, and labor is a {\bc jump variable}
\eit

\item We will therefore focus on adjustment of capital utilization
\bit
\item For an alternative model specification with {\bc labor hoarding}, see Burnside-Eichenbaum (AER 1996)
\eit

\item How to model utilization?
\bit
\setlength\itemsep{0.5em}
\item Should be choice subject to some cost function

\item The fact that firms do not always operate at full capacity indicates that marginal cost is increasing in utilization
\eit


\eit

\end{frame}


\begin{frame}{Capacity utilization: model}

\bit
\setlength\itemsep{1.5em}

\item In our setup, households own and rent out the capital stock

\item We could just as well assume that firms own the capital stock, but let's stick with the current setup

\item Following King-Rebelo (HB Macro, 1999), we assume
\ben
\setlength\itemsep{0.5em}

\item Households choose utilization level $U_t\in [0,\infty]$, renting out capital services $K^*_t = U_t K_t$

\item Cost of increasing $U_t$ is that depreciation increases:
\begin{eqnarray}
\delta = \delta (U_t) \nonumber
\end{eqnarray}
where $\delta (\cdot)$ is increasing and convex

\item To make things concrete, we assume a quadratic function
\begin{eqnarray}
\delta (U_t) = \delta_0 + \eta_1 (U_t-1) + \frac{\eta_2}{2}(U_t-1)^2 \nonumber
\end{eqnarray}
Note: at $U_t=1$, the model looks the same as before
\een

\eit

\end{frame}

\begin{frame}{Household problem}

\bit
\setlength\itemsep{1.5em}

\item Households solve
\begin{eqnarray}
\max_{\{C_t, N_t, I_{t}, U_t\}} && E_O \sum_{t=0}^{\infty} \beta^t \left[\log (C_t)- \theta \frac{N_t^{1+\varphi}}{1+\varphi} \right]\nonumber \\
\text{s.t} && C_t + I_t \leq W_t N_t + R^{r}_t K^*_t \nonumber \\
&& K^*_t = U_t K_t \nonumber \\
&& K_{t+1} = (1-\delta(U_t))K_t + I_t  \nonumber \\
&& \delta (U_t) = \delta_0 + \eta_1 (U_t-1) + \frac{\eta_2}{2}(U_t-1)^2 \nonumber \\
&& C_t, I_t, N_t, U_t \geq 0 \nonumber 
\end{eqnarray}

\item Nothing conceptually new, just need to take one more F.O.C.

\eit

\end{frame}

\begin{frame}{Household problem, F.O.C.}

\bit
\setlength\itemsep{1.5em}

\item After some substitutions,
\begin{eqnarray}
&& E_O \sum_{t=0}^{\infty} \beta^t \left[\log (C_t)- \theta \frac{N_t^{1+\varphi}}{1+\varphi} \right]\nonumber \\
\text{s.t} && C_t + K_{t+1} \leq W_t N_t + \left(R^{r}_t  U_t + (1-\delta (U_t) )  \right)  K_t  \nonumber 
\end{eqnarray}

\item Our two known friends from before (one in slightly new outfit)
\begin{eqnarray}
\frac{1}{C_t}W_t &=& \theta N_t^{\varphi} \nonumber \\
\frac{1}{C_t}&=& \beta E_t\left[(U_{t+1}R^{r}_{t+1}+(1-\delta(U_{t+1}))) \frac{1}{C_{t+1}}\right] \nonumber 
\end{eqnarray}

\item The new F.O.C. 
\begin{eqnarray}
R^r_t &=& \delta'(U_t) \nonumber \\
&=& \eta_1 + \eta_2 (U_t-1) \nonumber
\end{eqnarray}

\item When the rental rate is higher, the household is willing to increase utilization paying the higher marginal depreciation rate
\bit
\item With procyclical marginal productivity of capital, utilization is procyclical
\eit

\eit

\end{frame}

\begin{frame}{Utilization and the efficiency wedge}

\bit
\setlength\itemsep{1.5em}

\item All other parts of the model are identical, except that firm production is now $F(N_t, K^*_t)$, instead of $F(N_t, K_t)$

\item With Cobb-Douglas, firm production is thus
\begin{eqnarray}
Y_t &=& A_t (K^*_t)^{\alpha}N_t^{1-\alpha} \nonumber \\
&=& A_t (U_t K_t)^{\alpha}N_t^{1-\alpha} \nonumber \\
&=& A_t U_t^{\alpha} K_t^{\alpha}N_t^{1-\alpha} \nonumber
\end{eqnarray}

\item $\Rightarrow$ Fluctuations in $U_t$ shows up as fluctuations in the efficiency wedge
\eit

\end{frame}

\begin{frame}{Calibration}

\bit
\setlength\itemsep{1.5em}

\item We can make a calibration that normalize steady-state utilization to $1$: $U=1$
\bit
\setlength\itemsep{0.5em}
\item Determines $\delta_0$, as $\delta(1)=\delta_0$

\item set $\delta_0$ to the same value as before, $10$ \% yearly
\eit

\item Utilization F.O.C. in steady state implies
\begin{eqnarray}
R^{r} &=& \eta_1 \nonumber
\end{eqnarray}
which pins down $\eta_1$, given that we target $R^{r}$

\item Nothing in the steady state informs us about $\eta_2$, calibration of this parameter needs to be dynamic

\item $\eta_2 \to \infty$ $\Rightarrow$ model is {\bc isomorphic} to vanilla RBC

\item Can target moments of, e.g., volatility in utilization rates

\item Below, I just show IRFs using som different values for $\eta_2$

\eit

\end{frame}

\begin{frame}{IRFs when varying $\eta_2$}

\begin{figure}
\centering
\includegraphics[scale=0.40,trim= 0 120 0 120, clip]{Figures/rbc_irf_utilization.pdf}
\end{figure}

\end{frame}

\begin{frame}{Comments}

\bit
\setlength\itemsep{1.5em} 

\item More variation in capacity utilization $\Rightarrow$ more amplification

\item Also, smaller fluctuations in real rental rate - why?
\bit
\item Recall, one problem of RBC: too large fluctuations in rental/real interest rate
\eit
\eit

\end{frame}




Common intuition: investment is gradual, and a decreasing finite function of $R_t$
\bit
\item If making investment less elastic, we will presumably also increase the interal persistence of the RBC model
\eit

\item Whatever model of capital formation, Tobin (JMCB 1969) suggested an intuitive rule for firm investment behavior: If $Q$ is higher, invest more  
\begin{eqnarray}
Q &=& \frac{\text{Market value of firm capital}}{\text{Replacement cost of capital}} \nonumber 
\end{eqnarray}

\item Tobin formulated his rule as an {\bc average Q}, but optimizing firms should think in terms of the {\bc marginal Q}  

\item This idea has guided a large empircal literature on firm investment

\item In the neoclassical theory of investment, Marginal $Q=1$ always:
\begin{eqnarray}
\text{Marginal value of firm capital} &=& \lambda_{t+1}(R_{t+1} + (1-\delta))  \nonumber \\
\text{Replacement cost} &=& \lambda_{t}  \nonumber
\end{eqnarray}
where $\lambda_{t}$ is the {\bc Shadow value of consumption}

















