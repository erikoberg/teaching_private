\documentclass[9pt]{beamer}
\usetheme{Boadilla}

\makeatother
\setbeamertemplate{footline}
{
	\leavevmode%
	\hbox{%
		\begin{beamercolorbox}[wd=.4\paperwidth,ht=2.25ex,dp=1ex,center]{author in head/foot}%
			\usebeamerfont{author in head/foot}\insertshortauthor
		\end{beamercolorbox}%
		\begin{beamercolorbox}[wd=.6\paperwidth,ht=2.25ex,dp=1ex,center]{title in head/foot}%
			\usebeamerfont{title in head/foot}\insertshorttitle\hspace*{3em}
			\insertframenumber{} / \inserttotalframenumber\hspace*{1ex}
	\end{beamercolorbox}}%
	\vskip0pt%
}
\makeatletter
\setbeamertemplate{navigation symbols}{}


\usepackage{lipsum}
\usepackage{appendixnumberbeamer}

\usepackage[authoryear]{natbib}
\usepackage[latin1]{inputenc}
\usepackage[T1]{fontenc}
\usepackage{caption}
\usepackage{amsmath, amssymb}
\usepackage{epstopdf}
\usepackage{graphicx}
\usepackage{lmodern}
\usepackage{xcolor}
\usepackage{xpatch}
\usepackage{multirow}

\usepackage{amsmath,theorem,amssymb,graphicx, pgfplots, tabularx, placeins}
\usepackage{dsfont}
\usepackage{caption}
%\usepackage{subcaption}
%\usepackage{subcaption}
\setbeamertemplate{caption}{\raggedright\insertcaption\par}
%\setbeamertemplate{footline}[frame number]
\usepackage{csquotes}
\usepackage{bm}
\bibliographystyle{econometrica}
\usepackage[normalem]{ulem}

\usepackage{setspace}


\definecolor{gray(x11gray)}{rgb}{0.75, 0.75, 0.75}


\newcommand{\bit}{\begin{itemize}}
	\newcommand{\eit}{\end{itemize}}
\newcommand{\ben}{\begin{enumerate}}
	\newcommand{\een}{\end{enumerate}}


\newcommand{\lb}{\label}
\newcommand{\re}{\eqref}

\newcommand{\bc}{\color{blue}}
\newcommand{\rc}{\color{red}}

\title[Job ladders and Wage Dispersion]{Macroeconomics II, Lecture VIII:\\
	 Burdett-Mortensen}
\author{Erik {\"O}berg}
\date{}

\begin{document}

\begin{frame}
\maketitle
\end{frame}

\section{Introduction}



\begin{frame}{Recap}

\bit
	\setlength\itemsep{2em}
	
	\item Search models offers a theory of unemployment and wage dispersion
	
	\item Today we will dig deeper into understanding wage dispersion

	\item The McCall model: search frictions + offer distribution $\Rightarrow$ reservation wage strategy

	\item Generates a theory of residual wage dispersion that makes sense

	
	\item But, the model had no chance in coming close to empirical measures of wage dispersion
	
	\item And, how is an exogenous wage dispersion consistent with firm wage posting in the first place (Rothschild critique/Diamond paradox)?
\eit

\end{frame}

\begin{frame}{Today}

\bit
\setlength\itemsep{1.5em}

\item Today, we will introduce a class of models that speaks to both these problems
\bit
	\item In contrast to McCall, we will explicitly model firm behavior and study \emph{equilibrium} wage formation
\eit

\item Core idea: {\bc job ladders}

\item Employed workers can search while on the job and, as a consequence, they will sequentially move from worse to better jobs

\item Realistic feature of most labor markets

\item Generates more residual wage dispserion beacuse ex-ante identical workers are different not only because initial offers are different, but also because some have climbed further up the job-ladder than others

\item Solves the Diamond paradox, as firms face a trade-off when posting wages
\bit
	\item Higher wage attracts more workers
	\item Higher wage reduces per-worker profit
\eit

\item But first: something more on the nature of wage dispersion in the data

\eit

\end{frame}


\begin{frame}{Agenda}

\ben
\setlength\itemsep{2em}

\item The Abowd-Kramarz-Margolis regression

\item The Burdett-Mortensen model
\bit
	\item Setup
	\item Solution
	\item Predictions
\eit

\item Discussion and extensions

\een

\end{frame}

\begin{frame}

\begin{center}
	\huge The AKM regression \normalfont
\end{center}

\end{frame}



\begin{frame}{The AKM regression}

\bit
\setlength\itemsep{1em}

\item Remember Mincer regression
\begin{eqnarray}
\log y_{it} = \mu_y + \beta \mathbf{X}_{it} + \epsilon_{it} \nonumber
\end{eqnarray}
where, typically, $R^2<0.3$

\item Abowd-Kramarz-Margolis (Ecmtra 1999) were first to use (French) matched employer-employee data to estimtate
\begin{eqnarray}
\log y_{it} = \mu_y + \beta \mathbf{X}_{it} + \theta_i + \Psi_{j(i,t)} +  \epsilon_{it} \nonumber
\end{eqnarray}
for individual $i$ working for firm $j(i,t)$ at time t

\item Reduced-form empirical framwork for studying labor market sorting and inequality
\bit
 	\item Card-Heining-Kline (QJE 2013); Bloom-Guvenen-Price-Song-von Wachter (QJE 2019); Fredrikson-Hensvik-Skans (AER 2018); Eliason-Hensvik-Kramarz-Skans (JE 2023);
\eit

\item $\theta_i$ is identified from variation in worker earnings within firms; $\Psi_{J(i,t)}$ is identified from variation in worker earnings when switching jobs

\item AKM find:
\bit
	\item No fixed effects: $R^2 \approx 0.3$
	\item Firm fixed effects only: $R^2 \approx 0.55$
	\item Person fixed effects only: $R^2 \approx 0.75$
	\item Both fixed effects: $R^2 \approx 0.85$
\eit

\eit

\end{frame}


\begin{frame}{The AKM regression}

\bit
\setlength\itemsep{2em}

\item How to interpret these findings?

\item Considerable variation is due to firm and individual fixed effects
\bit
	\setlength\itemsep{0.5em}
	
	\item Some workers are persistently paid better across all their jobs compared to other, otherwise similar, workers
	
	\item Some firms are persistently paying more to all their employees although they are similar to employees at other firms

	\item Does this mean that unobserved heterogeneity, as opposed to frictions, explain most of residual wage dispersion?
\eit
%\item Individual fixed effects more important than firm fixed effects
%\bit
%	\item Is there more unobserved heterogeneity among workers than among firms?
%\eit

\item Theory will guide our interpretation...


\eit

\end{frame}


\begin{frame}

\begin{center}
	\huge The Burdett-Mortensen (IER, 1998) model\normalfont
\end{center}

\end{frame}



\begin{frame}{Overview}

\bit
\setlength\itemsep{1em}

\item Continuous time; infinite horizon

\item Measure $1$ of ex-ante identical firms, measure $1$ of ex-ante identical households

\item One-to-many matching: each firm can employ many workers
\bit
	\item Each match produces $y$ output flow
\eit

\item Separation rate $\sigma$, discount rate $r$, unemployment utility $b$

\item Firm decide on which wage $w$ to post, producing offer distribution $F(w)$
%\bit
%	\item \emph{Wage posting} as opposed to \emph{wage bargaining}
%\eit 

\item Unemployed workers search for jobs at rate $\lambda_u$, drawing opportunities from offer distribution $F(w)$

\item Employed workers search for jobs at rate $\lambda_e$, drawing opportunities from the same offer distribution $F(w)$

\item Problem of a worker is to accept or reject offer

\item We focus on steady state

\eit

\end{frame}

\begin{frame}{Worker value functions}

\bit
\setlength\itemsep{2em}

\item Unemployed:
\begin{eqnarray}
rU  &=&  b + \lambda_u  \int_{\mathds{W}} \max\{W(w)-U, 0\} dF(w) \nonumber
\end{eqnarray}

\item Employed:
\begin{eqnarray}
rW(w)  &=&  w + \lambda_e  \int_{\mathds{W}} \max\{W(w')-W(w), 0\} dF(w') + \sigma(U-W(w)) \nonumber
\end{eqnarray}

\item What are the optimal acceptance strategies?

\eit

\end{frame}


\begin{frame}{Worker strategies}

\bit
\setlength\itemsep{2em}

\item For employed, optimal acceptance rule is to accept all wages $w' \geq w$:
\begin{eqnarray}
rW(w)  &=&  w + \lambda_e  \int_{w' \geq w} (W(w') - W(w)) dF(w') + \sigma(U-W(w)) \nonumber
\end{eqnarray}

\item $W(w)$ is increasing in $w$ (see below) $\Rightarrow$ optimal acceptance rule of unemployed is a reservation wage $w_R$ (as in McCall):
\begin{eqnarray}
rU  &=&  b + \lambda_u  \int_{w\geq w_R} (W(w)-U) dF(w) \nonumber
\end{eqnarray}

\item Characterizing worker behaviour boils down to finding $w_R$, given $F, b, \lambda_e, \lambda_u, \sigma$. 

\item In doing so, we assume that $F$ has bounded support - we will later prove that this follows from firm optimization.

\eit

\end{frame}


\begin{frame}{Finding the reservation wage I}

\bit
\setlength\itemsep{1.5em}

\item Reservation wage satisfies
\begin{eqnarray}
W(w_R) = U \nonumber
\end{eqnarray}

\item From employed worker's value function equation:
\begin{eqnarray}
rW(w_{R})  &=&  w_R + \lambda_e  \int_{w'\geq w_R} \left(W(w')-W(w_R)\right) dF(w') + \sigma(U-W(w_R)) \nonumber \\
&=&  w_R + \lambda_e  \int_{w'\geq w_R} \left(W(w')-W(w_R)\right) dF(w')\nonumber
\end{eqnarray}

\item Using $rW(w_R) = rU$ and the unemployed worker's value function equation:
\begin{eqnarray}
&& w_R + \lambda_e  \int_{w \geq w_R} (W(w) - U) dF(w) = b + \lambda_u  \int_{w \geq w_R} (W(w)-U) dF(w)   \nonumber \\
\lb{reservation_wage}
\Leftrightarrow && w_R-b = (\lambda_u-\lambda_e) \int_{w \geq w_R} (W(w)-U) dF(w) \nonumber  
\end{eqnarray}


\eit

\end{frame}

\begin{frame}{Finding the reservation wage II}

\bit
\setlength\itemsep{2em}

\item We have found our {\bc reservation wage equation}:
\begin{eqnarray}
\lb{bm_reservation_wage}
w_R-b = (\lambda_u-\lambda_e) \int_{w \geq w_R} (W(w)-U) dF(w)
\end{eqnarray}

\item Interpretation
\bit
\setlength\itemsep{0.5em}

\item $\lambda_u>\lambda_e \Rightarrow w_R>b$: workers find better opportunties while unemployed $\Rightarrow$ unemployed workers demand compensation for giving up the option value of searching

\item $\lambda_u<\lambda_e \Rightarrow w_R<b$: workers find better opportunties while employed $\Rightarrow$ unemployed workers accept a wage lower than $b$ in anticipation than they will get better offers faster as employed
\eit
%
%\item Still contains an endogenous function $W(w)$

\item To solve for $w_R$, we need to evaluate $\int_{w \geq w_R} (W(w)-U) dF(w)$. Two steps
\ben
\setlength\itemsep{0.5em}
	\item Compute $W'(w)$
	
	\item Use intergration by parts
\een

\eit

\end{frame}

\begin{frame}{Finding the reservation wage III, computing $W'(w)$}

\bit
\setlength\itemsep{1.5em}

\item Rewrite employer value:
\begin{eqnarray}
(r+\sigma)W(w)  &=&  w + \lambda_e  \int_{w' \geq w} (W(w') - W(w)) dF(w') + \sigma U \nonumber
\end{eqnarray}

\item Apply Leibniz's rule (using bounded support):
\begin{eqnarray}
(r+\sigma)W'(w) &=& 1 + \lambda_e(W(w_{max}) - W(w)) \frac{\partial F(w_{max})}{\partial w}  \nonumber \\
&& - \lambda_e(W(w) - W(w)) \frac{\partial F(w)}{\partial w} + \lambda_e  \int_{w' \geq w} - W'(w) dF(w') \nonumber \\
&=& 1 -\lambda_e W'(w) \int_{w' \geq w} dF(w') \nonumber \\
&=& 1 -\lambda_e W'(w) (1-F(w)) \nonumber
\end{eqnarray}

\item Hence,
\begin{eqnarray}
\lb{e_derivative}
W'(w) &=& \frac{1}{r+\sigma + \lambda_e(1-F(w))}
\end{eqnarray}

\item Btw, we have now also shown that $W(w)$ is increasing in $w$ 

\eit

\end{frame}

\begin{frame}{Finding the reservation wage IV, integration by parts}


\footnotesize
\begin{eqnarray}
\int_{w \geq w_R} (W(w)-U) dF(w)   &=& (W(w)-U) F(w)|_{w_R}^{w_{max}} - \int_{w \geq w_R} F(w) d(W(w)-U) \nonumber \\
\text{(use differentiability)} &=& (W(w)-U) F(w)|_{w_R}^{w_{max}} - \int_{w \geq w_R} F(w) W'(w)dw    \nonumber \\ 
\text{(add/subtract)} &=& (W(w)-U) F(w)|_{w_R}^{w_{max}}- \int_{w \geq w_R} F(w) W'(w)dw    \nonumber \\ 
&& + (W(w)-U) |_{w_R}^{w_{max}} - (W(w)-U) |_{w_R}^{w_{max}}\nonumber \\
\text{(use $W(w_R) = U$, $F(w_R)=0$, $F(w_{max})=1$)} &=&  - \int_{w \geq w_R} F(w) W'(w)dw + (W(w)-W(w_R)) |_{w_R}^{w_{max}} \nonumber \\
\text{(evaluate second term)} &=&  - \int_{w \geq w_R} F(w) W'(w)dw + \int_{w \geq w_R} W'(w)dw \nonumber \\
\lb{integration}
\text{(rearrange)} &=& \int_{w \geq w_R} (1-F(w)) W'(w)dw 
\end{eqnarray}
\normalsize

\end{frame}


\begin{frame}{Finding the reservation wage V}

\bit
\setlength\itemsep{2em}

\item Plug \eqref{e_derivative} and \eqref{integration} into reservation wage equation \eqref{bm_reservation_wage}:
\begin{eqnarray}
w_R-b &=& (\lambda_u-\lambda_e) \int_{w \geq w_R} (W(w)-U) dF(w) \nonumber \\ 
 &=& (\lambda_u-\lambda_e)  \int_{w \geq w_R} (1-F(w)) W'(w)dw  \nonumber \\ 
\lb{reservation_wage_final}
 &=& (\lambda_u-\lambda_e) \int_{w \geq w_R}\frac{1-F(w)}{r+\sigma + \lambda_e(1-F(w))} dw 
\end{eqnarray}

\item LHS increasing

\item Integrand is positive: RHS decreasing

\item $\Rightarrow$ \eqref{reservation_wage_final} implicitly determines $w_R$ given $F(w)$
\eit

\end{frame}


\begin{frame}{Worker distribution I}

\bit
\setlength\itemsep{1.5em}

\item We have now characterized worker behaviour, taking offer distribution $F(w)$ as given 

\item Before going to firm problem (this is where $F(w)$ comes from), we will derive the worker distribution $G(w)$, taking $F(w)$ as given 

\item Unemployment law of motion;
\begin{eqnarray}
\dot u = \sigma (1-u(t)) - \lambda_u (1-F(w_R)) u(t) \nonumber
\end{eqnarray}

\item Steady state
\begin{eqnarray}
u &=& \frac{\sigma }{\sigma + \lambda_u (1-F(w_R))} \nonumber \\
\label{u_ss}
&=& \frac{1}{1+ k_u (1-F(w_R))}
\end{eqnarray}
where $k_u = \frac{\lambda_u}{\sigma}$

\eit

\end{frame}

\begin{frame}{Worker distribution II}

\bit
\setlength\itemsep{1.5em}

\item $G(w) =$ the share of workers with wage $\leq w$

\item $G(w) (1-u(t)) =$ the \emph{mass} of workers with wage $\leq w$

\item Law of motion for $G(w, t) (1-u(t))$: 
\begin{eqnarray}
\frac{d}{dt}G(w, t) (1-u(t)) &=& \lambda_u \max \left\{F(w)-F(w_R),0\right\}u(t) \nonumber \\
&& - \sigma G(w,t)(1-u(t)) \nonumber \\
&& - \lambda_e(1-F(w))G(w,t)(1-u(t)) \nonumber
\end{eqnarray}

\bit
	\item inflow: unemployed households that find job wage in interval $[w_R, w]$ 
	\item outflow 1: employed households with wage $\leq w$ that separate at rate $\sigma$
	\item outflow 2: employed households with wage $\leq w$ that find job with wage $>w$
\eit

\item Using steady state condition $\frac{d}{dt}G(w, t) (1-u(t)) = 0$ and Eq. \eqref{u_ss}, we solve for $G(w)$:
\begin{eqnarray}
\lb{worker_distribution}
G(w) = \left\{ \begin{array}{cc}
0 & \text{if } w < w_R \\
\frac{\left[F(w)-F(w_R)\right]/(1-F(w_R))}{1+k_e(1-F(w))} & \text{if } w \geq w_R
\end{array} \right. 
\end{eqnarray}
where $k_e = \frac{\lambda_e}{\sigma}$

\eit

\end{frame}

\begin{frame}{Illustration of worker distribution in steady state}

\begin{center}
	\includegraphics[width=\textwidth]{code/bm_density.pdf}
\end{center}

\bit
	\item Example: assume offer distribution $F(w)$ is normal, compute $w_R$ from \eqref{reservation_wage_final}, then $G(w)$ from \eqref{worker_distribution}
	\item Why is pdf $g(w)$ ``to the right of'' pdf $f(w)$?
\eit

\end{frame}


\begin{frame}{Firm size distribution I}

\bit
\setlength\itemsep{1.5em}

\item Given $G$ and $F$, we can also characterize firm size distribution

\item Limit argument: determine who many firms offer jobs and how many workers have jobs on interval $[w-\epsilon,w]$ and let $\epsilon \rightarrow 0$

\item Workers on interval $[w-\epsilon,w]$: $\left[G(w)-G(w-\epsilon)\right]\times (1-u)$

\item Firms on inteval $[w-\epsilon,w]$: $\left[F(w)-F(w-\epsilon)\right] \times 1$

\item Average number of workers per firm on interval $[w-\epsilon,w]$:
\begin{eqnarray}
\frac{G(w)-G(w-\epsilon)}{F(w)-F(w-\epsilon)}(1-u) \nonumber
\end{eqnarray}

\item Average firm size at point $w$:
\begin{eqnarray}
\lb{firm_size_dist}
I(w) = \lim\limits_{\epsilon \rightarrow 0}\frac{G(w)-G(w-\epsilon)}{F(w)-F(w-\epsilon)}(1-u)
\end{eqnarray}

\eit

\end{frame}

\begin{frame}{Firm size distribution II}

\bit
\setlength\itemsep{1.5em}

\item Given our solution of $u$ and $G$, we can solve for $I$ 

\item We have not assumed that $F$ is continuous:
\bit
	\item We don't know if $\lim\limits_{\epsilon \rightarrow 0}F(w-\epsilon) = \lim\limits_{\epsilon \rightarrow 0}F(w+\epsilon)$, i.e., we don't know if $F$ has mass points
\eit

\item Denote $F(w^{-}) = \lim\limits_{\epsilon \rightarrow 0}F(w-\epsilon)$

\item Plug in solution of $G,u$ into firm size distribution \eqref{firm_size_dist}, and do the algebra:
\begin{eqnarray}
\lb{firm_size}
I(w) = \left\{ \begin{array}{cc}
0 & \text{if } w < w_R \\
\frac{k_u (1+k_e(1-F(w_R)))/(1+k_u(1-F(w_R)))}{(1+k_e(1-F(w)))(1+k_e(1-F(w^{-})))} & \text{if } w \geq w_R
\end{array} \right.
\end{eqnarray}

\item $I(w)$ is increasing in $w$ - why?

\item $I(w)$ is continuous everywhere where $F$ is continuous

\eit

\end{frame}

\begin{frame}{Firm problem}

\bit
\setlength\itemsep{2em}

\item Firms maximize the discounted sum of expected profits

\item For simplicity, we assume that $r \rightarrow 0$ $\Rightarrow$ firms only care about the steady state value of profits

\item Firms post wage $w$ to maximize
\begin{eqnarray}
\lb{firm_problem}
\pi = \max_{w} (y-w) I(w)
\end{eqnarray}
where $I(w)$ is the firm size distribution, taking the behavior of workers ($w_R$) and other firms ($F$) as given

\item Equilibrium must have $\pi(w)=\pi(w')$ for all $w,w'\in F$

\item Why don't all firms set $w=w_R$ (as in the wage posting equilibrium of McCall model)?

\eit

\end{frame}

\begin{frame}{Equilibrium definition}

\bit
\setlength\itemsep{1.5em}

\item An equilibrium is a triple $\{w_R, \pi, F\}$ such that
\ben
	\setlength\itemsep{1em}
	\item Given $F$, each worker behaves optimally: $w_R$ solves reservation wage equation \eqref{reservation_wage_final} 
	
	\item Given $F$ and $w_R$, each firm behaves optimally:  $\pi$ follows from \eqref{firm_problem}
	
	\item $F, \mathbb{W}_F$ are such that there is no arbitrage:
	\begin{eqnarray}
	(y-w)I(w) \left\{ \begin{array}{cc}
	< \pi & \text{if } w \notin \mathbb{W}_F \\
	= \pi & \text{if } w \in \mathbb{W}_F
	\end{array} \right. \nonumber
	\end{eqnarray}
	where $\mathbb{W}_F$ is the support of $F$
\een

\eit

\end{frame}


\begin{frame}{Solution strategy}

\bit
\setlength\itemsep{1.5em}

\item Given the progress we made on characterizing worker behaviour and firm size distribution, the equilibrium can be solved with the following 5 steps:
\ben
\setlength\itemsep{1em}
\item Establish that $F$ has bounded support $[\underbar w, \bar w]$

\item Establish that $F$ is continuous

\item Establish $\underbar w = w_R$

\item Solve for $F$

\item Solve for $w_R, \pi$
\een


\eit

\end{frame}


\begin{frame}{Step 1: bounded support}

\bit
\setlength\itemsep{2em}

\item No worker accept a wage $w< w_R \Rightarrow$ $\pi(w<w_R) = 0$

\item Also, $\pi(w>y) < 0$

\item Ergo, F has bounded support $[\underbar w, \bar w] \subset [w_R, y]$

\eit

\end{frame}

\begin{frame}{Step 2: continuity}

\bit
\setlength\itemsep{1em}

\item Proof strategy: If $F$ has a mass point $\hat w$, you make excess profits by offering $\hat w + \epsilon$

\item Assume $F$ has a mass point at $\hat w$: {\rc (Draw graph on whiteboard)}
\begin{eqnarray}
F(\hat w) = F(\hat w^{-}) + v_1(\hat w) \text{  where $v_1(\hat w)>0$ }\nonumber
\end{eqnarray}

\item Then, average firm size $I$ is discontinuous at $\hat w$: {\rc (Do on whiteboard)}
\begin{eqnarray}
I(\hat w^{+}) = I(\hat w) + v_2(\hat w) \text{  where $v_2(\hat w)>0$ } \nonumber
\end{eqnarray}

\item Then,
\begin{eqnarray}
\lim\limits_{\epsilon \rightarrow 0} \pi(\hat w+\epsilon)-\pi(\hat w) &=& \lim\limits_{\epsilon \rightarrow 0} (y-\hat w-\epsilon) I(\hat{w}+\epsilon) - (y-\hat w) I(\hat{w}) \nonumber \\
&=& \lim\limits_{\epsilon \rightarrow 0} (y-\hat w)(I(\hat{w}+\epsilon)-I(\hat{w})) -\epsilon I(\hat{w}+\epsilon) \nonumber \\
&=& \lim\limits_{\epsilon \rightarrow 0}  (y-\hat w)(I(\hat{w}+\epsilon)-I(\hat{w})) \nonumber \\ 
&=& (y-\hat w)v_2(\hat w)  \nonumber \\ 
&>& 0 \nonumber 
\end{eqnarray}

\item You can make excess profits by offering wage contract $\hat{w}+\epsilon$ because
\bit
	\item per worker profit decrease continuously
	\item firm competition decrease discretely 
\eit

\eit

\end{frame}



\begin{frame}{Step 3: $\underbar w = w_R$}

\bit
\setlength\itemsep{1em}

\item Step 1: $\mathbb{W}_F =[\underbar w, \bar w] \subset [w_R, y]$ 

\item Step 2: $F$ is continuous:
\begin{eqnarray}
I(w) &=&  \frac{k_u (1+k_e(1-F(W_R)))/(1+k_u(1-F(w_R)))}{(1+k_e(1-F(w)))(1+k_1(1-F(w^{-})))} \nonumber \\
&=&  \frac{k_u (1+k_e(1-F(w_R)))/(1+k_u(1-F(w_R)))}{(1+k_e(1-F(w)))^2} \nonumber
\end{eqnarray}
for $w \geq w_R$. 

\item Moreover $F(\underbar w) = 0$, which in turn implies $F(w_R)=0$

\item Hence, 
\begin{eqnarray}
I(\underbar w) =  \frac{k_u (1+k_e)/(1+k_u)}{(1+k_e)^2} =  \frac{k_u /(1+k_u)}{(1+k_e)} \nonumber
\end{eqnarray}

\item Expected firm size at $\underbar w$ is independent of the particular value of $\underbar w$!

\item Therefore, $\underbar w = w_R$, since if $\underbar w > w_R$, a firm could offer $w_R$ and get the same amount of workers and higher per worker profits $\Rightarrow$ excess profits
%
%\item Notice: $\pi=\pi(w_R)=(p-w_R)I(\underbar w)>0$. Why?

\eit

\end{frame}


\begin{frame}{Step 4: solve for F}

\bit
\setlength\itemsep{1.5em}

\item In equilibrium, $\pi(w) = \pi(w')$ for all $w, w' \in \mathbb{W}_F$

\item In particular, $\pi(w) = \pi(w_R)$ for all $w \in \mathbb{W}_F$

\item Solve $\pi(w) = \pi(w_R)$, using our solutions for $I(w), I(w_R)$:
\begin{eqnarray}
(y-w)I(w) &=& (y-w_R)I(w_R) \Leftrightarrow \nonumber \\
(y-w)\frac{k_u (1+k_e)/(1+k_u)}{(1+k_e(1-F(w)))^2} &=& (y-w_R) \frac{k_u /(1+k_u)}{(1+k_e)} \nonumber
\end{eqnarray}
\item Do the algebra to find
\begin{eqnarray}
F(w) = \frac{1+k_e}{k_e} \left[1-\left(\frac{y-w}{y-w_R}\right)^{1/2}\right]
\end{eqnarray}

\item Using that $F(\bar w) = 1$, we see that $\bar w<y$:
\begin{eqnarray}
\bar w = y - \frac{y-w_R}{(1+k_e)^2}<y
\end{eqnarray}

\eit

\end{frame}


\begin{frame}{Step 5: solve for $w_R$}

\bit
\setlength\itemsep{1.5em}

\item Our reservation wage equation \eqref{reservation_wage_final}:
\begin{eqnarray}
w_R-b = (\lambda_u-\lambda_e) \int_{w \geq w_R}\frac{1-F(w)}{r+\sigma + \lambda_e(1-F(w))} dw \nonumber
\end{eqnarray}
\item For simplicity, we use the assumption $r \rightarrow 0$ again: 
\begin{eqnarray}
w_R-b = (k_u-k_e) \int_{w \geq w_R}\frac{1-F(w)}{1 + k_e(1-F(w))} dw \nonumber
\end{eqnarray}
\item Plug in the solution of $F$:
%\begin{eqnarray}
%w_R-b = \frac{(k_u-k_e)}{k_e} \int_{w_R}^{\bar w}\frac{k_e \left(\frac{y-w}{y-w_R}\right)^{1/2}-1}{k_e\left(\frac{y-w}{y-w_R}\right)^{1/2} + 1} dw \nonumber
%\end{eqnarray}
\begin{eqnarray}
w_R-b = \frac{(k_u-k_e)}{k_e} \int_{w_R}^{\bar w} \left[1-\frac{1}{1+k_e} \left( \frac{y-w}{y-w_R}\right)^{-\frac{1}{2}}\right]dw \nonumber
\end{eqnarray}
\item Solve the integral and use the solution to $\bar w$ to find:
\begin{eqnarray}
w_R = \frac{(1+k_e)^2b + (k_u-k_e)k_ey}{(1+k_e)^2+(k_u-k_e)k_e}
\end{eqnarray}

\item Summary: a continuous offer distribution with $b<\underbar w = w_R< \bar w<y$!
\eit


\end{frame}


\begin{frame}{Model summary}

\bit
\setlength\itemsep{2em}

\item Model components: McCall + on-the-job search + fixed number of optimizing firms

\item Importantly, all firms and households are ex ante identical

\item Key result: a {\bc non-degenerate continuous wage distribution}
\bit 
	\item Not possible in simple wage-posting model without on-the-job search 
\eit

\item Intuition?

%\item Primitives: $b, \lambda_u, \lambda_e, \sigma, r$ - only one new parameter compared to McCall
%\bit
%	\setlength\itemsep{0.5em}
%	\item $\lambda_e$ can be identified from the frequency of job-to-job transitions
%	\item An employed worker with wage $w$ switches job with probability $\lambda_e(1-F(w))$ --- The mass of workers that switches at every instant: $ \xi = \int_{w_R}^{\bar w} \lambda_e (1-F(w)) d G(w)$
%	\item One can show that $\xi = h(\sigma, \lambda_e)$	
%	\item $\lambda_u$ is directly identified from the job-finding rate, as unemployed worker accept all offers in this model (Why?)
%\eit

\eit


\end{frame}


\begin{frame}{Taking the model to the data}

\bit
\setlength\itemsep{2em}

\item Primitives: $b, \lambda_u, \lambda_e, \sigma, r$ - only one new parameter compared to McCall (but also one less!)

\item $\lambda_e$ can be identified from the frequency of job-to-job transitions

\item An employed worker with wage $w$ switches job at rate  $\lambda_e(1-F(w))$ 

\item Total job switch rate: $ \xi = \int_{w_R}^{\bar w} \lambda_e (1-F(w)) d G(w)$


\item One can show that, in equilibrium, $\xi = \xi(\sigma, \lambda_e)$	(take-home exericse!)


\item $\lambda_u$ is directly identified from the unemployed workers' job-finding rate, as unemployed worker accept all offers in this model


\eit


\end{frame}


\begin{frame}{Predicitions I: wage dispersion}

\bit
\setlength\itemsep{1.5em}

\item Wage dispersion. As in the McCall model, a formula for the mean-to-min ratio can be derived:
\begin{eqnarray}
Mm_{BM} \approx \frac{\frac{\lambda_u - \lambda_e}{r+\sigma+\lambda_e} +1}{\frac{\lambda_u-\lambda_e}{r+\sigma + \lambda_e}+\rho}   \hspace{10mm} Mm_{McCall} = \frac{\frac{\lambda}{r+\sigma}+1}{\frac{\lambda}{r+\sigma}+\rho} \nonumber
\end{eqnarray}
\item $\approx$ comes from assuming $r \rightarrow 0$

\item The BM model generates additional dispersion, as workers do not only differ in initial offers, but also in that some have climbed further up the job-ladder than others

\item Hornstein-Krusell-Violante (AER, 2011) reports monthly EE rate = 3.2 percent $\Rightarrow$ $\lambda_e = 0.135$

\item {\bc$Mm_{BM} = 1.22$}, $Mm_{McCall} = 1.05$, $Mm_{data} > 1.8$

\item Not there yet, but a big step in the right direction
\bit
	\item How does this relate to the results from the AKM regression?
\eit

\eit


\end{frame}

\begin{frame}{Predicitions II: wage correlations}

\bit
\setlength\itemsep{2em}

\item Worker with wage $w$ finds new job at rate:
\begin{eqnarray}
\lb{ee_rate}
\lambda_e^* = \lambda_e(1-F(w))
\end{eqnarray}

\item Tenure is positively correlated with wage
\bit
	\item From \eqref{ee_rate}: higher $w$ $\Rightarrow$ lower job-finding rate
\eit


\item Challenges interpretation of returns-to-tenure as reflecting productivity

\item Firm size is positvely correlated with wage
\bit
	\item From \eqref{firm_size}: $I(w)$ is increasing in $w$, beacuse workers continuously climb the job-ladder 
\eit

\item Both true in the data (Mortensen, Book 2003)
%
%\item Corollary: Firm size is negatively correlated with quit rate


\eit


\end{frame}



\begin{frame}{Taking stock}

\bit
\setlength\itemsep{1.5em}

\item Benchmark model for studying frictional wage dispersion, easily taken to the data

\item New predictions: search frictions can rationalize
\bit
\setlength\itemsep{0.5em}
	\item wage-tenure correlation
	\item wage-firm size correlation
\eit

\item We came closer in quantitatively accounting for unobserved wage dispersion

\item Also, BM offers an interpretation of the AKM regression results
	\bit
	\setlength\itemsep{0.5em}
		\item all wage dispersion stems from firm heterogeneity, but all firms are identical in terms of productivity
		\item firm fixed effects in AKM should not necessarily be interpreted as firm productivity differentials
	\eit


\item But, no role for search frictions in explaining
\bit
\setlength\itemsep{0.5em}
	\item Within-firm wage dispersion and, therefore, individual-fixed effects in AKM	
	
	\item Downward wage mobility
\eit
%
%\item And, what about person fixed effects in residual wages?
%\bit
%	\item Should we interpret the individual fixed effects in AKM as solely reflecting unobserved heterogeneity and having nothing to do with frictions?
%\eit

\eit


\end{frame}


\begin{frame}{Extensions}

\bit
\setlength\itemsep{1.5em}

\item The basic BM model is too stylized to deal with many empirical phenomena

\item Because of its parsimony, the model can be extended in various directions: firm heterogeneity, worker heterogeneity etc.

\item For quantitative assessments, see Jolivet-\{Postel-Vinay\}-Robin (EER 2006), Mortensen (Book 2003), Barlevy (ReStud 2008)


\eit


\end{frame}


\begin{frame}{Recent applications}


\bit
\setlength\itemsep{1em}
\item Sorkin (QJE 2018) estimate a BM model for US to quantify the role of compensating differentials in residual wage dispersion
\bit
\setlength\itemsep{0.5em}
\item Firms offer utility bundles rather than wage contracts. 

\item Worker transitions from higher to lower paying firms can be used to infer value of compensating differentials 

\item compensating differentials account for over half of the firm component of the variance of earnings
\eit

\item Gottfries-Jarosch (2023) estimate a BM model for US to quantify how much non-compete practices can surpress wages
\bit
\setlength\itemsep{0.5em}
\item A BM model with finite number of firms and DRS productions technology - natural laboratory for studying the effects of monopsony power

\item Model ``non-competes'' as wage offers which does not allow job-to-job transitions

\item Banning ``non-competes'' raises average wages by 2-15\% depending on labor market characteristics
\eit

\item Engbom-Moser (AER 2023) estimate a BM model for Brazil 1996-2012 following a reform that increased the mandated minimum wage:
\bit
\setlength\itemsep{0.5em}
\item Key fact: the entire wage distribution became more compressed after the reform, not just the lower end

\item BM theory: Because firms set wage offer strategically in relation to $F(w)$, higher minimum wage affects entire distribution

\item EM finds that 70 percent of the drop in earnings inequality during this period can be attributed to the increased mandated minimum wage
\eit
\eit



\end{frame}

\begin{frame}{Engbom-Moser (AER 2023)}

\begin{center}
	\includegraphics[width=\textwidth]{figures/em_fig1.pdf}
\end{center}

\end{frame}


\begin{frame}{Is wage posting reasonable benchmark?}

\bit
\setlength\itemsep{1.5em}

\item Key question: is wage posting a reasonable approximation for the applications considered?
\bit	
\setlength\itemsep{0.5em}
\item Bilateral bargaining?
\bit
	\item See next lecture
\eit

\item Shouldn't contracts be allowed to be made contingent on observables?
\bit
	\item Stevens (ReStud 2004); Burdett-Coles (Ecmtra 2003) note that firms can reduce turnover and raise profits by offering wage-tenure contracts
\eit

\item Shouldn't firms be allowed to respond to employees' outside offers?
\eit

\item \{Postel-Vinay\}-Robin (Ecmtra 2002) construct job-ladder model extended with
\bit
	\setlength\itemsep{0.5em}
	\item firm and worker heterogeneity
	\item allowing firms to make counteroffers
	\item rationalizes within-firm wage dispersion among similar workers as well as downward wage mobility
	\item offers a way to quantitatively separate the role of unobserved firm heterogeneity, unobserved worker heterogeneity and luck in residual wage dispersion
\eit
\eit

\end{frame}


\begin{frame}{Summary}

\bit
\setlength\itemsep{1.5em}

\item AKM: Reduced-form model for quantifying sources of residual wage dispersion

\item Burdett-Mortensen: Benchmark model for studying frictional wage dispersion

\item Key assumptions: on the job-search + firm wage posting

\item Key result: non-degenerate distrubition of wages across identical firms due to trade-off in wage-possting decision
\bit
\item Higher wage attracts more workers
\item Higher wage reduces per-worker profit
\eit

\item We endogenized firm wage setting behavior, but not the contact rate

\item Next up: endogenzing the contract rate (and thinking about its implications for unemployment)

\eit


\end{frame}

%\begin{frame}
%
%\begin{center}
%	\huge The \{Postel-Vinay\}-Robin model \normalfont
%\end{center}
%
%\end{frame}
%
%
%
%\begin{frame}{Setup}
%
%\bit
%\setlength\itemsep{1.5em}
%
%\item Basic BM structure: 
%\bit
%		\setlength\itemsep{0.5em}
%	\item unit mass of workers, unit mass of firms
%	\item job-finding rates $\lambda_u, \lambda_e$, matching is random
%	\item separation rate $\sigma$
%	\item worker problem: accept or reject offer
%\eit
%
%\item Firm productivity heterogeneity: $y \sim F$
%
%\item Worker productivity heterogeneity: $\epsilon \sim H$ 
%
%\item A succesful match results in production $\epsilon y$
%
%\item workers unemployment utility is $\epsilon b$ (scaled with $\epsilon$ for tractability)
%
%\item Firms offer optimal long-term wage contracts as a function of worker observables
%\bit
%\item Firms observe employment status, worker productivity $\epsilon$ and wage on current job $w$ 
%%\item Offers are \emph{take-it-or-leave-it}; no bargaining with the worker
%\eit
%
%\item Incumbent employers can counteroffer new job offers
%
%\eit
%
%\end{frame}
%
%\begin{frame}{Wage setting mechanism I}
%
%\bit
%\setlength\itemsep{2em}
%
%\item Unemployed value function $U(\epsilon)$
%
%\item Employed value function $W(\epsilon, w, y)$
%
%\item If firm meets unemployed: offer reservation wage $w_R$, implicitly defined by
%\begin{eqnarray}
%U(\epsilon) = W(\epsilon, w_R(\epsilon,y),y) \nonumber
%\end{eqnarray}	
%
%\item Notice: $w_R$ depends on which firm you meet
%\bit
%	\item You are willing to accept lower starting wage if the firm is very productive (counterintuitive?), as that means that you can raise your wage a lot if you get an outside offer
%\eit
%
%\eit
%
%\end{frame}
%
%\begin{frame}{Wage setting mechanism II}
%
%\bit
%\setlength\itemsep{1.5em}
%
%\item If employed worker with value $W(\epsilon, w_0, y)$ meets firm $y'$, the two firms enter Bertrand competition (sequential auction)
%
%\item Firm $y$ is willing to bid at maximum $\epsilon y$
%
%\item Firm $y'$ is willing to bid at maximum $\epsilon y'$
%
%\item Two cases:
%\ben
%	\item If $y'>y$
%		\bit
%			\item Firm $y'$ will win the auction
%			
%			\item Firm $y'$ offers the wage $w(\epsilon, y, y')$ required to get the worker to switch jobs, implicitly defined by
%			\begin{eqnarray}
%			W(\epsilon, \epsilon y, y) = W(\epsilon, w(\epsilon,y, y'), y') \nonumber
%			\end{eqnarray}
%		\eit
%	\item If $y' \leq y$
%		\bit
%			\item Firm $y$ will win the auction
%
%			\item Firm $y$ offers the wage $w(\epsilon, y', y)$ required to get the worker to stay:
%			\begin{eqnarray}
%			 	w(\epsilon, y', y) = \max\{w_0, w^*(\epsilon, y', y)\} \nonumber
%			\end{eqnarray}
%			where $w^*(\epsilon, y', y)$ is implictly defined by
%			\begin{eqnarray}
%			W(\epsilon, \epsilon y', y') = W(\epsilon, w^*(\epsilon,y', y), y) \nonumber
%			\end{eqnarray}
%\eit
%\een
%
%\eit
%
%\end{frame}


%\begin{frame}{Predictions}
%
%\bit
%\setlength\itemsep{1em}
%
%\item The steady state wage distribution needs to be solved numerically
%\bit
%	\item However, the worker-firm allocation is independent of the wage distribution, since workers always switch jobs if $y'>y$ 
%\eit	
%
%\item Unemployed workers will accept offers with $w<b$ $\Rightarrow$ lower tail of wage distribution more stretched 
%\bit
%	\item Using PVR's calibration, Papp (RED 2013) finds mean-to-min ratio $\approx 2$, close to empirical estimates 
%\eit
%
%\item There is \emph{within-firm} wage dispersion because workers enter firms with
%\ben
%	\item different worker productivies
%	\item different match history: some workers have been luckier than others, and will get a higher wage offer 
%\een
%
%\item Second point challenges simple interpretation of AKM regression: individual fixed effects might be due to frictions, not only worker heterogeneity
%\bit
%	\item PVR estimate their model on french matched employer-employee data -- find that worker heterogeneity contribute substantially to wage dispersion among high-skilled, but nothing among low-skilled
%\eit
%
%\item There is \emph{downwards mobility}
%\bit
%	\item If encountering a very productive firm, you might be willing to accept a wage cut, beacuse you know that this firm can outbid incoming offers in the future
%\eit
%
%\eit
%
%\end{frame}



%\begin{frame}{Applications}
%
%\bit
%\setlength\itemsep{1.5em}
%
%\item Bagger-Fontaine-\{Postel-Vinay\}-Robin (AER 2014) 
%\bit
%\setlength\itemsep{0.5em}
%	\item Estimate sequential auction model with human capital accumulation on Danish register data
%	\item Decompose wage dispersion into human capital component and search component
%	\item Find that search component is more important early in careers
%\eit
%
%\item Moscarini-\{Postel-Vinay\} (2022)
%\bit
%\setlength\itemsep{0.5em}
%	\item Integrate PVR into New-Keynesian business cycle model
%	
%	\item Offers new way to think about the Phillips curve
%	
%	\item In their model, unemployment is a poor measure of slack, as productivity also depends on placement in the job-ladder 
%	
%	\item A better measure is the number of job-to-job transitions, which predicts inflation well
%\eit
%
%\item Bilal-Engbom-Mongey-Violante (Ecmtra 2023)
%\bit
%\setlength\itemsep{0.5em}
%	\item Integrate PVR with competitive firm-dynamics model (Hopenhayn, Ecmtra 1992)
%	
%	\item Predicts that net poaching depends on firm size and age, consistent with micro data
%	
%	\item Predicts that job-to-job transitions fall as firm entry declines, consistent with macro data
%\eit
%
%\eit
%
%\end{frame}



%\begin{frame}{PVR: some recent developments}
%
%\bit
%\setlength\itemsep{1.5em}
%
%\item Bagger, Fontaine, Postel-Vinay and Robin (2011) 
%\bit
%	\item estimate sequential auction model with human capital accumulation on Danish data
%	\item decompose wage dispersion into human capital component and search component
%	\item finds that search component is more important early in careers
%\eit
%
%\item Bagger, Fontaine, Postel-Vinay and Robin (2011) 
%\bit
%\item estimate sequential auction model with human capital accumulation on Danish data
%\item decompose wage dispersion into human capital component and search component
%\item finds that search component is more important early in careers
%\eit
%
%
%\eit
%
%\end{frame}


%\begin{frame}
%
%\begin{center}
%	\huge Non-labor applications of search models \normalfont
%\end{center}
%
%\end{frame}


%\begin{frame}{Search models}
%
%\bit
%\setlength\itemsep{1.5em}
%
%\item Search models have wider applications than discussed in this class
%
%\item The theory of money
%\bit
%	\item Goods markets are charachterized by the absence of \emph{double coincidence of wants}
%	\item Search process without mediating transaction technology is very costly
%	\item Useless assets, such as fiat money, can have positive value beacuse it mediates trade
%	\item Microfounded theory of the price of money, as opposed to ad hoc cash-in-advance constraints
%	\item See Williamson and Wright (Handbook of Monetary Economics, 2010)
%\eit
%
%\item Liquidity in financial markets
%\bit
%	\item Some finanical markets are close to the Walrasian ideal
%	\item Others, such as the market for mortage-backed securities, operate with \emph{over-the-counter} trading
%	\item Search theory charachterize volume of trading, bid-ask spreads, how the supply of buyers might collapse during crises
%	\item See Lagos and Rocheteau (2009)
%\eit
%
%\item Family economics
%\bit
%	\item Marriage is a matching process under costly search
%	\item See Shimer and Smith (2000)
%\eit
%
%\eit
%
%\end{frame}


\end{document}





















