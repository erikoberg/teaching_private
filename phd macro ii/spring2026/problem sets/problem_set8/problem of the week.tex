             \documentclass{article}[11pt]
\linespread{1.5}
\usepackage{fullpage}
\usepackage{amsmath,theorem,amssymb,graphicx, pgfplots, tabularx, placeins}
\usepackage[semicolon,authoryear]{natbib}
\usepackage{caption}
\usepackage{subcaption}
\usepackage{csquotes}
\usepackage{epstopdf}

\usepackage[semicolon,authoryear]{natbib}
\usepackage{bibentry}
\nobibliography*

\newcommand{\lb}{\label}
\newtheorem{thm}{Theorem}
\newtheorem{prop}{Proposition}
\newtheorem{definition}{Definition}


\newcommand{\bit}{\begin{itemize}}
	\newcommand{\eit}{\end{itemize}}
\newcommand{\ben}{\begin{enumerate}}
	\newcommand{\een}{\end{enumerate}}
\newcommand\setItemnumber[1]{\setcounter{enumi}{\numexpr#1-1\relax}}

\title{Macro II Problem of the week: Wage dispersion in McCall with stochastic wage growth}
\author{Erik \"{O}berg}
\date{}

\begin{document}
\maketitle

Consider the following version of the McCall model. Time is continuous. Workers' utility is linear and the discount rate is $r$. Workers can be employed or unemployed. Unemployed workers recieve flow utility $b$ and get job offers at rate $\lambda_u$. When recieving a job offer, the wage of that offer is drawn from a distribution with cdf $F(w)$ with finite support $\mathcal{W} = [w_{min}, w_{max}]$, and the unemployed household decides whether to accept or reject. While employed, workers experience wage changes at rate $\lambda_w>0$. When the shock is realized, an employed worker with current wage $w$ draws an $\epsilon$ from a distribution with cdf $G(\epsilon)$ with finite support $\mathcal{E} = [0, \epsilon_{max}]$ and mean value $\bar \epsilon$. The new wage $w'$ is then $w'=w(1+\epsilon)$. The workers separatate at the exogenous rate $\sigma$.

\ben
\item Denote the employment value with $W(w)$. Write the Bellman equation for an employed worker.

\item Guess that the employment value $W(w)$ is linear in $w$: $W(w)=kw+m$. What are the values of $k$ and $m$ if the Bellman equation of the employed worker is to be satisfied for all wages $w$? 

\item Write the new Bellman equation for an unemployed worker.

\item Find the reservation wage equation. Is the reservation wage in this model higher or lower compared to the model without stochastic wage growth?

\item For the standard model without stochastic wage growth, Hornstein-Krusell-Violante (AER 2011) showed that the mean-min ratio of observed wages is given by
\begin{eqnarray}
Mm = \frac{\frac{\lambda}{r+\sigma}+1}{\frac{\lambda}{r+\sigma}+\rho} \nonumber
\end{eqnarray}
where $\lambda$ is the job-finding rate and $\rho$ is the average replacement rate ($\rho$ is defined by $b=\rho \bar w$, where $\bar w$ is the average observed wage). If matched to the same data on the discount rate, separation rate, job-finding rate and average replacement rate, would the model with stochastic wage growth predict more or less residual wage dispersion compared to the standard model with constant wages on the job? Would it make a big difference?
\een


\end{document}