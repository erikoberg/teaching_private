             \documentclass{article}[11pt]
\linespread{1.5}
\usepackage{fullpage}
\usepackage{amsmath,theorem,amssymb,graphicx, pgfplots, tabularx, placeins}
\usepackage[semicolon,authoryear]{natbib}
\usepackage{caption}
\usepackage{subcaption}
\usepackage{csquotes}
\usepackage{epstopdf}

\usepackage[semicolon,authoryear]{natbib}
\usepackage{bibentry}
\nobibliography*

\newcommand{\lb}{\label}
\newtheorem{thm}{Theorem}
\newtheorem{prop}{Proposition}
\newtheorem{definition}{Definition}


\newcommand{\bit}{\begin{itemize}}
	\newcommand{\eit}{\end{itemize}}
\newcommand{\ben}{\begin{enumerate}}
	\newcommand{\een}{\end{enumerate}}
\newcommand\setItemnumber[1]{\setcounter{enumi}{\numexpr#1-1\relax}}

\title{Macro II Problem of the week: Identification of Taylor Rules}
\author{Erik \"{O}berg}
\date{}

\begin{document}
\maketitle

A big literature has, in various ways, estimated time-series regressions of nominal interest rates on inflation (and other variables), with the purpose of retrieving the coefficients of the Taylor rule, i.e., regressions of the form:
\begin{eqnarray}
i_t = \alpha + \beta \pi_t + \zeta_t
\end{eqnarray}
The perhaps most famous example is Clarida-Gali-Gertler (QJE 2000). Let's explore how we can interpret such regressions, focusing on the case with flexible price setting.

The flexible-price model studied in the class is summarized by
\begin{eqnarray}
\text{DIS curve:} &&  i_t = r + E_t \pi_{t+1}  \nonumber \\
\text{Policy rule:} && i_t = r + \phi \pi_t + \nu_t \nonumber 
\end{eqnarray}
where $r$ is the log of the steady state real interest rate and the monetary policy shock is assumed to follow $\nu_t = \rho \nu_{t-1} + \epsilon_t$, where $\epsilon_t$ is an i.i.d. disturbance.

\ben
	\item Assume that $\phi>1$. Show that the unique bounded solution is given by
	\begin{eqnarray}
	\pi_t = - \frac{\nu_t}{\phi-\rho} \nonumber
	\end{eqnarray}
	
	\item Show that, in equilibrium, inflation and interest rates are related according to
	\begin{eqnarray}
	i_t = r  +  \rho \pi_{t}.
	\end{eqnarray}
	
	\item Why does this equilibrium condition (2) not contradict the assumed policy rule $i_t = r + \phi \pi_t + \nu_t$?
	
	\item Would you interpret the time-series estimate of $\beta$ in Equation (1) as the structural Taylor rule coefficient $\phi$? Explain in the words what the identification problem is.
	
	\item Bonus question (not graded): Show that the same identification problem arises also with sticky price setting.
\een


	
\end{document}
