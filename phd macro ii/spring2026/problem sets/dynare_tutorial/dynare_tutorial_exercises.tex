\documentclass[11pt]{article}
\linespread{1.5}
\usepackage{fullpage}
\usepackage{amsmath,theorem,amssymb,graphicx, pgfplots, tabularx, placeins}
\pgfplotsset{compat=1.18}
\usepackage{caption}
\usepackage{subcaption}
\usepackage{csquotes}
\usepackage{epstopdf}

\newcommand{\lb}{\label}
\newtheorem{thm}{Theorem}
\newtheorem{prop}{Proposition}
\newtheorem{definition}{Definition}


\newcommand{\bit}{\begin{itemize}}
	\newcommand{\eit}{\end{itemize}}
\newcommand{\ben}{\begin{enumerate}}
	\newcommand{\een}{\end{enumerate}}
\newcommand\setItemnumber[1]{\setcounter{enumi}{\numexpr#1-1\relax}}

\title{Macroeconomics II: Dynare Tutorial Exercise}
\author{Erik \"{O}berg}
\date{}

\begin{document}
\maketitle

Consider the vanilla RBC model discussed in Lecture 1. Its equilibrium and log-linear equilibrium are characterized by 
\begin{eqnarray}
\frac{1}{C_t}W_t = \theta N_t^{\varphi} &\Rightarrow& \hat w_t = \hat c_t + \varphi n_t \nonumber \\
\frac{1}{C_t} = \beta E_t\left[(R^{r}_{t+1}+(1-\delta)) \frac{1}{C_{t+1}} \right] &\Rightarrow& \hat c_t = - \beta R^{r} E_t \hat r^{r}_{t+1} + E_t \hat {c}_{t+1} \nonumber \\
C_t + I_t = Y_t &\Rightarrow& \frac{C}{Y}\hat c_t + \frac{I}{Y} \hat i_t = \hat y_t \nonumber \\
Y_t = A_t K_t^{\alpha}N_t^{1-\alpha} &\Rightarrow& \hat y_t = \hat a_t + \alpha \hat k_t + (1-\alpha ) \hat n_t \nonumber \\
K_{t+1} = (1-\delta) K_t + I_t &\Rightarrow& \nonumber \hat k_{t+1} = (1-\delta) \hat k_t +\delta \hat i_t \nonumber\\
R^{r}_t = \alpha A_t \left(\frac{K_t}{N_t}\right)^{\alpha-1}  &\Rightarrow& \hat r^{r}_t = \hat a_t -(1-\alpha)(\hat k_t - \hat n_t)  \nonumber \\
W_t = (1-\alpha) A_t \left(\frac{K_t}{N_t}\right)^{\alpha} &\Rightarrow& \hat w_t = \hat a_t + \alpha (\hat k_t-\hat n_t) \nonumber \\
A_t = A_{t-1}^{\rho_a}exp(\epsilon_t) &\Rightarrow& \hat a_t = \rho_a \hat a_{t-1} + \epsilon_{t} \nonumber
\end{eqnarray}
and its steady state is given by
\begin{eqnarray}
R^{r} &=& \frac{1}{\beta}-(1-\delta) \nonumber \\
W &=& (1-\alpha)\left(\frac{R^{r}}{\alpha}\right)^{-\frac{\alpha}{1-\alpha}} \nonumber \\
N &=& \left[\frac{1}{\theta} \frac{W}{\frac{R^{r}}{\alpha}-\delta}\left(\frac{R^{r}}{\alpha
}\right)^{\frac{1}{1-\alpha}}   \right]^{\frac{1}{1+\varphi}} \nonumber \\
K &=& \left(\frac{R^{r}}{\alpha
}\right)^{-\frac{1}{1-\alpha}}N \nonumber \\
Y &=& \frac{R^{r}K}{\alpha} \nonumber \\
I &=& \delta K \nonumber \\
C &=& (\frac{R^{r}}{\alpha}-\delta) K \nonumber
\end{eqnarray}

\section*{Exercise 1: Calibration}
Calibrate the model on a \emph{quarterly frequency}. Set $\varphi=1$, and
\bit
	\item Pick $\delta$ to match NIPA estimates of average \emph{yearly} capital depreciation rate $\sim 10 \%$
	\item Pick $\beta$ to match a \emph{yearly} real return on capital of $4$ percent, net of depreciation
	\item Pick $\alpha$ to match long-run labor share $\sim 2/3$
	\item Pick $\theta$ to match average hours worked $\sim 0.7$
\eit
With these parameter values, what are the values of $\frac{C}{Y}$ and $\frac{I}{Y}$?

\section*{Exercise 2: Simulation using Dynare}
Going forward to analyze dynamics, let's set $\rho_a=0.979$ and $\sigma_{\epsilon}=0.009$.
\ben
	\item Plug in the log-linear system into Dynare and simulate a first-order perturbation
	
	\item Compare your estimated IRFs with those we retrieved in class, are they the same?
	
	\item Look at the simulation moments, are they the same as those we looked at in class?
	
	\item Set the HP-filter parameter to 1600 (standard for quarterly data), do the moments change? Are they the same as in class?
\een

\section*{Exercise 3: Making your own graphs using the IRFs}
\ben
\item Construct a figure of the IRFs of output and TFP in the same graph

\item Construct a figure of the IRFs of Investment and consumption in the same graph
\een

\section*{Exercise 4: From IRFs to simulation plots}
\ben
	\item Generate a sequence of $T=200$ random draws of the shocks
	
	\item Produce a time series of output, consumption and investment based on these draws and your estimated IRFs, similar to the graph in Lecture 1
\een



	
\end{document}