             \documentclass{article}[11pt]
\linespread{1.5}
\usepackage{fullpage}
\usepackage{amsmath,theorem,amssymb,graphicx, pgfplots, tabularx, placeins}
\usepackage[semicolon,authoryear]{natbib}
\usepackage{caption}
\usepackage{subcaption}
\usepackage{csquotes}
\usepackage{epstopdf}

\usepackage[semicolon,authoryear]{natbib}
\usepackage{bibentry}
\nobibliography*

\newcommand{\lb}{\label}
\newtheorem{thm}{Theorem}
\newtheorem{prop}{Proposition}
\newtheorem{definition}{Definition}


\newcommand{\bit}{\begin{itemize}}
	\newcommand{\eit}{\end{itemize}}
\newcommand{\ben}{\begin{enumerate}}
	\newcommand{\een}{\end{enumerate}}
\newcommand\setItemnumber[1]{\setcounter{enumi}{\numexpr#1-1\relax}}

\title{Macroeconomics II: Problem set 7}
\author{Erik \"{O}berg}

\begin{document}
\maketitle

Send your solutions to Andrii by \bf April 10, 12.00 \normalfont at the latest.


\section*{Exercise 1: Government spending multipliers}
Since the days of Keynes, there's been the hypothesis that you can use government spending to stimulate the economy. Let's study this in the vanilla NK model. 

Assume the government finances wasteful government spending $G_t$ (by ``wasteful'', we mean that $G_t$ does not enter anyone's utility nor any production function) by non-distortionary lump-sum taxes $TAX_t$ levied on households. We assume the government budget is balanced, so $G_t=TAX_t$ in all periods. The household problem is still
\begin{eqnarray}
\max_{\{C_t, N_t, B_{t+1}\}} && E_O \sum_{t=0}^{\infty} \beta^t \left[\log C- \theta \frac{N^{1+\varphi}}{1+\varphi}\right]\nonumber \\
\text{s.t} && P_t C_t + Q_t B_{t+1} \leq W_t N_t + B_{t} + T_t \nonumber \\
&& C_t, N_t, B_{t+1}  \geq 0 \nonumber
\end{eqnarray}
but now, $T_t=D_t-TAX_t$ is the sum of firm dividends and taxes.

\ben
\item In our baseline model with $G_t=0$, the log-linear equilibrium was described by
\begin{eqnarray}
\text{Intratemporal hh optimality:} && \hat \omega_t = \hat c_t + \varphi \hat n_t  \nonumber \\
\text{Intertemporal hh optimality:} && \hat c_t =  - (\hat i_t - E_t \pi_{t+1}) + E_t \hat c_{t+1}  \nonumber \\
\text{Firm optimality:} && \pi_t = \beta E_t \pi_{t+1} + \lambda \widehat{mc}_t \nonumber \\
\text{Marginal cost:} && \widehat{mc}_t = \hat \omega_t \nonumber \\
\text{Goods clearing:} && \hat c_t = \hat y_t \nonumber \\
\text{Bonds clearing:} && \hat b_t = 0 \nonumber \\
\text{Labor clearing:} && \hat y_t = \hat n_t \nonumber \\
\text{Policy rule:} && \hat i_t = \phi \pi_t + \nu_t. \nonumber 
\end{eqnarray}
Argue that introducing positive and time-varying government spending only changes one of these equations. Which, and how?

\item Suppose government spending follows the process
\begin{eqnarray}
g_t = \rho_g g_{t-1} + \epsilon^g_t \nonumber
\end{eqnarray}
with $\rho_g=0.5$ and the government spending share of output is 30 percent in steady state. Construct graphs of the response of output, inflation and interest rates to 1 percent positive government spending shock.

\item To start interpreting these results, let's first start with constructing the same graphs, but with flexible prices. With flexible prices, the firm optimality condition changes to $\widehat{mc}_t=0$, as discussed in class. You should find that output increases in response to the shock. What is the mechanism?

\item Let's go back to sticky prices. With sticky prices, you should have found a larger response of output to the same shock. What is the intuition for this? To uncover the mechanism, you might find it helpful to investigate how the response of output changes when you vary the monetary policy response parameter $\phi$.
\een

Note: If you are interested in digging deeper, check out Woodford (AEJmacro, 2011).



\section*{Exercise 2: Forward Guidance}
Optimal policy under commitment typically feature what is called \emph{Forward Guidance}. This tool was forcefully used by the Fed and several other central banks in the aftermath of the Financial Crisis 2007-2008, during which the zero lower bound became binding. The idea is that although you're not able to affect current interest (due to the zero lower bound), you can still affect aggregate demand by promising to keep future interest rates low. 
\ben
\item Consider the linearized Euler equation in the vanilla NK model
\begin{eqnarray}
\hat c_t &=& - (\hat i_t-E_t \pi_{t+1}) + E_t \hat c_{t+1}. \nonumber
\end{eqnarray}
Iterate on this condition, and use the boundary condition $\lim\limits_{T \to \infty} \hat c_T=0$ to show that
\begin{eqnarray}
\hat c_t = - E_t \sum_{s=0}^{\infty} (\hat i_{t+s}-\pi_{t+s+1}) \nonumber	
\end{eqnarray}

\item Suppose the central bank can commit to a policy that lowers the future real interest rate $E_t[\hat i_{t+s}-\pi_{t+s+1}]$ by 1 percent for some $s>0$. How will that affect consumption demand in period $t$ in comparison to changing the current real interest rate $\hat i_{t}-E_t\pi_{t+1}$ by 1 percent? Does it matter if $s=1,10,1000$?

\item Explain the intuition to this result.

\item Suppose the central bank makes a commitment today to keep real rates low for the next ten years. Is it credible? Explain your reasoning.

\item Do you think this result is a reasonable prediction of the vanilla NK model?

\item Gabaix (AER 2020) shows that if assuming bounded rationality in the form of ``cognitive discounting''- the idea that agents understand current economic conditions better than future ones -  the linearized Euler equation changes to
\begin{eqnarray}
\hat c_t &=& - (\hat i_t-E_t \pi_{t+1}) + \delta E_t \hat c_{t+1}. \nonumber
\end{eqnarray} 
where $\delta \in [0,1)$ is the discounting parameter. In this model, does it now matter if the central bank changes the interest rates in period $s=1,10$ or $s=1000$?
\een


\section*{Exercise 3: Identification of the Phillips curve}
During the period of elevated inflation in the EU and the US 2021-2024, several academics and policy commentators argued that the data does not support a clear tradeoff between real economic activity and inflation (=a Phillips curve). An exmaple is the Claudia Sahm's opinion piece in the Financial Times on Jul 27, 2022, which contained Figure 1 and argued that since the 70's, US data show little empirical support for a positive comovement between unemployment/output and inflation. The author concluded that the concept of a Phillips curve is dead. 

\begin{figure}[!]
	\centering
	\includegraphics[scale=0.25]{phillips.jpeg}
	\caption{From FT, Jul 27 2022}
\end{figure}

Let's examine the argument from the point of view of the New Keynesian model. Consider the dynamics of the following version of the linearized NK model:
\begin{eqnarray}
\text{DIS curve:} && \hat y_t = - (\hat i_t-E_t \pi_{t+1}) + E_t \hat y_{t+1} + \nu_t \nonumber \\
\text{Phillips curve:} && \pi_t = \beta E_t \pi_{t+1} + \kappa \hat y_t  + \xi_t \nonumber \\
\text{Policy rule:} && \hat i_t = \phi E_t \pi_{t+1} \nonumber 
\end{eqnarray}
where $\nu_t$ is a ``demand shock'' and $\xi_t$ is a ``supply shock''. There are no ``policy shocks'', and to simplify some algebra, we assume that monetary policy responds to expected inflation rather than current inflation.

\ben
\item Suppose there are no supply shocks, $\xi_t=0$, but only AR(1) demand shocks: $\nu_t = \rho \nu_{t-1} + \epsilon_t$, where $\epsilon_t$ is exogenous. Guess that in a solution to a positive demand shock $\epsilon_t$, output and inflation responds linearly to $\nu_t$:
\begin{eqnarray}
&& \hat y_t = \theta_y \nu_t \nonumber \\
&& \pi_t = \theta_{\pi} \nu_t \nonumber
\end{eqnarray}
By using this guess, solve for $\theta_y, \theta_{\pi}$ in terms of the model parameters (and thereby verify that the guess is correct). Is there a positive correlation between $\hat y_t$ and $\pi_t$?

\item Now, suppose there are no demand shocks, $\nu_t=0$, but only AR(1) supply shocks: $\xi_t = \rho \xi_{t-1} + \epsilon_t$, where $\epsilon_t$ is exogenous. Analogously, guess that in the solution to a positive supply shock $\epsilon_t$:
\begin{eqnarray}
&& \hat y_t = \theta_y \xi_t \nonumber \\
&& \pi_t = \theta_{\pi} \xi_t \nonumber
\end{eqnarray}
Solve for $\theta_y, \theta_{\pi}$ in this case too. Is there a positive correlation between $\hat y_t$ and $\pi_t$?

\item Much research have showed in the late 70's, US monetary policy shifted from being ``passive'' to ``active''. Put differently, since the late 70's, the Fed has tried to raise the real interest rate in response to inflation rising above target (and lowering rates in response to inflation below target). We can interpret this in our model as if $\phi$ shifted from $\phi<1$ to $\phi>1$. Ignoring determinacy issues, how does this parameter shift affect the correlation between inflation and output in response to demand and supply shocks? 

\item In general, do you think that we should expect a positive correlation between inflation and real activity in the data? Can we retrieve the slope of the Phillips curve from the reduced-form relationship in Figure 1? Use your findings from subquestions 1-3 to motivate your answer. 
\een 
Note: If you are interested in digging deeper, check out McLeay-Tenreyro (NBER Macro Annual 2020) and Rognlie (NBER Macro Annual 2020).


\section*{Exercise 4: Rigid wages instead of rigid prices in the NK model}
Let's consider the basic NK model discussed in class, but instead of rigid prices, let's assume rigid wages (together with fully flexible price setting). Specifically, let's be a bit extreme and assume that while resting in the steady state, the representative household and the firms have agreed to honor a labor contract with the following features: 
\bit
\item If a transitory shocks occurs, the nominal wage will not change. 

\item Firms are always free to choose how much labor services to hire at the going wage 

\item Households are committed to supply whatever hours worked is demanded by the firms
\eit
In contrast to the NK model discussed in class, we assume that the intermedidate goods market is fully competitive (meaning that the final goods producer operates a linear production function) and also that intermediate goods production function is DRS:
\begin{eqnarray}
Y_{it} = N_{it}^{1-\alpha} \nonumber
\end{eqnarray}
where TFP $A=1$.

\ben
\item Show that in this model, optimal firm behavior prescribes that the real wage equals 
\begin{eqnarray}
\frac{W_t}{P_t} = (1-\alpha) Y_{t}^{-\frac{\alpha}{1-\alpha}} \nonumber
\end{eqnarray}
where, $Y_t$ is the production of final goods.

\item Denote log deviations in the real wage with $\hat \omega_t = \hat w_t- p_t$. Argue that the growth rate in log real wages must equal the negative of the inflation rate, i.e., $\Delta\hat{\omega}_t = -\pi_t$,

\item Show that combining the results from question 1 and 2 yields the following Phillips curve
\begin{eqnarray}
\pi_t = \frac{\alpha}{1-\alpha}\Delta \hat y_t \nonumber
\end{eqnarray}  

\item Interpret the previous equation, why does firm optimality imply a positive relationship between inflation and output growth in this model?

\item Show that household optimality together with goods market clearing imply the following IS curve:
\begin{eqnarray}
\Delta E_t \hat y_{t+1} = \hat i_t - E_t \pi_{t+1} \nonumber
\end{eqnarray}
You don't need to set up the entire household problem, but can depart from log-linearized optimality condition directly.

\item Suppose the central bank policy rule responds to inflation expectations rather than realized inflation (this is argubly pretty close to how actual monetary policy is conducted):
\begin{eqnarray}
\hat i_t  = \phi E_t \pi_{t+1} + \nu_t \nonumber
\end{eqnarray}
Show that if $\phi \leq \frac{1}{\alpha}$, then a positive monetary policy shock will result in positive expected output growth $\Delta E_t \hat y_{t+1}>0$.

\item One can show that a unique bounded solution is garuanteed if also $\phi > 1$. Argue that, given a unique bounded solution and $\phi < \frac{1}{\alpha}$, output today (in the period in which the shock occurs) must be negative: $\hat y_t<0$

\item Summing up, explain the mechanism, step-by-step, whereby a positive monetary contraction results in a negative output response in this model.

\item Finally, compare the welfare properties in this model to the rigid-price model discussed in this class. Specifically, indicate in what manner the allocation in this model is inefficient, and compare that to the inefficiencies in the rigid-price model discussed in class.

\een

	
\end{document}


