\documentclass[11pt]{article}
\usepackage[utf8]{inputenc}
\usepackage[T1]{fontenc}
\usepackage{amsmath}
\usepackage{amsfonts}
\usepackage{amssymb}
\usepackage[version=4]{mhchem}
\usepackage{stmaryrd}

\begin{document}
\section*{Part 1}


\subsection*{The Romer (1990) model of expanding varieties (20 points)}


Take the Romer model of expanding varieties as seen in lectures, but with some modifications. Consumer preferences are $\int_{0}^{\infty} e^{-\rho t}[\log C(t)] d t$. The final good production function is $Y(t)=\frac{1}{1-\beta} L_{E}(t)^{\beta} \int_{0}^{N(t)} x(\nu, t)^{1-\beta} d \nu$, and machines can be produced at a marginal cost $\psi \equiv 1-\beta$. New ideas are produced according to $\dot{N}(t)=\eta N(t) L_{R}(t)$, where $L_{R}(t)$ is the number of workers engaged in research. The remaining workers ($L_E(t)$) are employed in the production of the final good and earn a wage $w(t)$. Workers are perfectly mobile between the two activities. Let $V(\nu, t)$ be the value of a machine patent and let $w(t)$ be the wage paid by final good firms. Suppose that profits from selling machines $\pi(\nu, t)$ are subsidized at a rate $s>0$.\\

1. Explain why $\eta N(t) V(\nu, t)=w(t)$ must hold in an equilibrium in which both production and research take place. [2 points]\\

2. Explain why $r(t) V(\nu, t)=(1+s) \pi(\nu, t)+\dot{V}(\nu, t)$ must hold in equilibrium. [2 points]\\

3. It can be shown that $\pi(\nu, t)=\beta L_{E}(t)$ and $w(t)=\frac{\beta}{1-\beta} N(t)$. Use these and prior results to show that $r(t)=\eta(1+s)(1-\beta) L_{E}(t)$. [4 points]\\

4. The equilibrium of the model is a balanced growth path on which output, consumption, and ideas grow at a constant rate $g$. Let $L$ denote the size of the labor force. Show that

$$
g=\frac{\eta(1+s)(1-\beta) L-\rho}{\eta(1+s)(1-\beta)+1}
$$

Hint: Consumer optimization requires that $\dot{C}(t) / C(t)=r(t)-\rho$. [4 points]\\

5. A social planner (SP) would allocate resources such that $g_{S P}=\eta L-\rho$. Let $s^{*}$ denote the value of the subsidy that ensures $g=g_{S P}$. Derive $s^{*}$. [2 points]\\

6. TRUE or FALSE? "If $s=s^{*}$, welfare is maximized. That is, welfare is the same as in the SP solution." Briefly explain your answer. [2 points]\\

7. In the quality ladders model, growth in the decentralized economy may either be too high or too low relative to the social planner solution. On the other hand, in the expanding varieties model growth in the decentralized economy is always too low. Explain briefly how this differences arises. [4 points]


\end{document}