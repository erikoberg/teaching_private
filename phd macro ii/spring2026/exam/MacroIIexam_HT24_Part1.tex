\documentclass[11pt]{article}
\usepackage[utf8]{inputenc}
\usepackage[T1]{fontenc}
\usepackage{amsmath}
\usepackage{amsfonts}
\usepackage{amssymb}
\usepackage[version=4]{mhchem}
\usepackage{stmaryrd}

\begin{document}
\section*{Part 1}


\subsection*{The Romer (1990) model of expanding varieties (20 points)}
Suppose that final output $Y(t)$ is produced according to\\ $Y(t)=\frac{1}{1-\beta} L_{E}(t)^{\beta} \int_{0}^{N(t)} x(\nu, t)^{1-\beta} d \nu$, where $L_{E}(t)$ is the number of workers engaged in production, $x(\nu, t)$ is the number of machines of variety $\nu$ being used, and $N(t)$ is the number of existing varieties, all at time $t$. Suppose the final good sector is perfectly competitive, but each machine variety is supplied by a monopolist who faces a unit cost of making machines equal to $\psi$. For convenience, we assume $\psi=1-\beta$.\\

1. Show that profit maximization by final good firms yields the demand curve $x(\nu, t)=p(\nu, t)^{-1 / \beta} L_{E}(t)$. Given this, show that $p(\nu, t)=1$, and show that therefore $Y(t)=\frac{1}{1-\beta} N(t) L_{E}(t)$. Finally, show that the profits of machine producers are $\pi(t)=\beta L_{E}(t)$.\\

Ideas for new varieties are produced according to $\dot{N}(t)=\eta N(t) L_{R}(t)$, where $L_{R}(t)$ is the number of workers doing research at time $t$. A firm producing the new variety receives a patent and acts as a monopolist. Workers are perfectly mobile across the production and research sectors. Let $w(t)$ be the wage earned in both sectors, and let $V(t)$ be the value of a patent.\\

2. State and explain the free-entry condition ensuring that workers are indifferent between the two activities.\\

3. Infinitely lived consumers own Arrow securities paying a return $r(t)$. State and explain the no-arbitrage condition ensuring that traders in financial markets are indifferent between holding Arrow securities and patents.\\

It can be shown that the wage equals $w(t)=\frac{\beta}{1-\beta} N(t)$ (not needed to derive this result). Finally, assume population is constant and consumers have log utility, so that the growth rate of consumption equals $g_{C}(t)=r(t)-\rho$.\\

4. Define the BGP equilibrium of the economy.\\

5. Take as given that the defined BGP is the unique equilibrium outcome of the model and solve for the BGP $r$, $g$ (growth rate of output pc), and $L_{E}$. What makes this economy grow sustainably?\\

6. Assume now that the population grows at a constant rate $n$ and that you want to remove the scale effect property of the model. How would you modify the R\&D technology? What is the BGP growth rate in this economy?\\

7. Suppose that the policy maker wants to promote the BGP growth and introduces a subsidy to R\&D cost by paying each research firm a fraction $s^R$ of the wage bill, i.e. the effective wage paid to researchers is now $w^R(t)=(1-s^R)w(t)$. Will the policy have the intended growth effect? How about the welfare effect?




\end{document}