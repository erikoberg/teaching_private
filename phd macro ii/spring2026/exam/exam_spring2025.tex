\documentclass{article}[11pt]
\linespread{1.5}
\usepackage{fullpage}
\usepackage{amsmath,theorem,amssymb,graphicx, pgfplots, tabularx, placeins}
\usepackage[semicolon,authoryear]{natbib}
\usepackage{caption}
\usepackage{subcaption}
\usepackage{csquotes}
\usepackage{epstopdf}

\usepackage[semicolon,authoryear]{natbib}
\usepackage{bibentry}
\nobibliography*

\newcommand{\lb}{\label}
\newtheorem{thm}{Theorem}
\newtheorem{prop}{Proposition}
\newtheorem{definition}{Definition}


\newcommand{\bit}{\begin{itemize}}
	\newcommand{\eit}{\end{itemize}}
\newcommand{\ben}{\begin{enumerate}}
	\newcommand{\een}{\end{enumerate}}
\newcommand\setItemnumber[1]{\setcounter{enumi}{\numexpr#1-1\relax}}

\newcommand{\bc}{\color{blue}}
\newcommand{\rc}{\color{red}}

\title{Exam Ph.D. Macroeconomics II \\
Department of Economics, Uppsala University\\
June 2, 2025}
\date{}

\begin{document}
\maketitle

\section*{Instructions}
\bit
	\item Writing time: 5 hours.
	
	\item The exam is closed book. 
	
	\item The exam has 70 points in total 
	
	\item A passing grade requires a) at least 30 points on  the exam, and b) 50 points in total for the course (incl the points you have from your problem sets).
	
	\item Start each question on a new paper. Write your anonymous code on all answer pages.
	
	\item You may write your solutions by pen or pencil; use your best handwriting.
	
	\item Answers shall be given in English.

	\item Motivate your answers carefully; if you think you need to make additional assumptions to answer the questions, state them.
	
	\item If you have any questions during the exam, you may call Erik Öberg (+46 730 606 796) or Teodora Borota (+46 739 262 330) at any time between 3 PM and 5 PM.
\eit

\newpage

\section{The Romer (1990) model of expanding varieties (20 points)}
Suppose that final output $Y(t)$ is produced according to\\ $Y(t)=\frac{1}{1-\beta} L_{E}(t)^{\beta} \int_{0}^{N(t)} x(\nu, t)^{1-\beta} d \nu$, where $L_{E}(t)$ is the number of workers engaged in production, $x(\nu, t)$ is the number of machines of variety $\nu$ being used, and $N(t)$ is the number of existing varieties, all at time $t$. Suppose the final good sector is perfectly competitive, but each machine variety is supplied by a monopolist who faces a unit cost of making machines equal to $\psi$. For convenience, we assume $\psi=1-\beta$.\\

1. Show that profit maximization by final good firms yields the demand curve $x(\nu, t)=p(\nu, t)^{-1 / \beta} L_{E}(t)$. Given this, show that $p(\nu, t)=1$, and show that therefore $Y(t)=\frac{1}{1-\beta} N(t) L_{E}(t)$. Finally, show that the profits of machine producers are $\pi(t)=\beta L_{E}(t)$.\\

Ideas for new varieties are produced according to $\dot{N}(t)=\eta N(t) L_{R}(t)$, where $L_{R}(t)$ is the number of workers doing research at time $t$. A firm producing the new variety receives a patent and acts as a monopolist. Workers are perfectly mobile across the production and research sectors. Let $w(t)$ be the wage earned in both sectors, and let $V(t)$ be the value of a patent.\\

2. State and explain the free-entry condition ensuring that workers are indifferent between the two activities.\\

3. Infinitely lived consumers own Arrow securities paying a return $r(t)$. State and explain the no-arbitrage condition ensuring that traders in financial markets are indifferent between holding Arrow securities and patents.\\

It can be shown that the wage equals $w(t)=\frac{\beta}{1-\beta} N(t)$ (not needed to derive this result). Finally, assume population is constant and consumers have log utility, so that the growth rate of consumption equals $g_{C}(t)=r(t)-\rho$.\\

4. Define the BGP equilibrium of the economy.\\

5. Take as given that the defined BGP is the unique equilibrium outcome of the model and solve for the BGP $r$, $g$ (growth rate of output pc), and $L_{E}$. What makes this economy grow sustainably?\\

6. Assume now that the population grows at a constant rate $n$ and that you want to remove the scale effect property of the model. How would you modify the R\&D technology? What is the BGP growth rate in this economy?\\

7. Suppose that the policy maker wants to promote the BGP growth and introduces a subsidy to R\&D cost by paying each research firm a fraction $s^R$ of the wage bill, i.e. the effective wage paid to researchers is now $w^R(t)=(1-s^R)w(t)$. Will the policy have the intended growth effect? How about the welfare effect?


\section{Demand shocks in the New-Keynesian model (20 points)}
Consider the vanilla New-Keynesian model in class, in which we allow for shocks to the household discount factor (``demand shocks''). The log-linear equilibrium is described by the following set of equations
\begin{eqnarray}
\text{Intratemporal hh optimality:} && \hat \omega_t = \hat c_t + \varphi \hat n_t  \nonumber \\
\text{Intertemporal hh optimality:} && \hat c_t =  - (\hat i_t - E_t \pi_{t+1} ) + E_t \hat c_{t+1} + \xi_t  \nonumber \\
\text{Firm optimality:} && \pi_t = \beta E_t \pi_{t+1} + \lambda \widehat{mc}_t \nonumber \\
\text{Marginal cost:} && \widehat{mc}_t = \hat \omega_t \nonumber \\
\text{Goods market clearing:} && \hat c_t = \hat y_t \nonumber \\
\text{Labor market clearing:} && \hat y_t = \hat n_t \nonumber \\
\text{Policy rule:} && \hat i_t = \phi \pi_t + \nu_t \nonumber \\
\text{Shock process:} && \xi_t = \rho_{\xi} \xi_{t-1} + \epsilon_{\xi, t} \nonumber
\end{eqnarray}
where $\xi_t = -\log \beta_t$ is the negative of the log of the household discount factor, $\hat \omega_t = \hat w_t-p_t$ is log deivations in the real wage, and $\lambda = \frac{(1-\theta)(1-\theta \beta)}{\theta}$.

Figure 1 contains the IRFs to a positive demand shock with $\rho = 0.5$. The other parameters take the same value as in class: The Frisch elasticity $\varphi = 1$, the price-resetting probability $1-\theta = 1/3$, the policy rule coefficent $\phi = 1.5$, and the steady state value of the household discount factor $\beta = 0.99$.

\ben
	\item Explain the sign of the responses for all variables displayed in Figure 1.
	
	\item Figure 2 contains the IRFs to the same shock, but with $1-\theta = 0.999$. Explain why, in this case, we see close to no response in output. 
	
	\item Consider again the benchmark case where $1-\theta = 1/3$. How would the response of output look like if monetary policy were set to maximize social welfare?
\een

\begin{figure}
	\centering
	\includegraphics[scale=0.75,trim= 0 200 0 200, clip]{figures/nk_cpshock_9variables.pdf}
	\caption{IRFs to a demand shock in the vanilla NK model}
\end{figure}

\begin{figure}
	\centering
	\includegraphics[scale=0.75,trim= 0 200 0 200, clip]{figures/nk_cpshock_flexprice_9variables.pdf}
	\caption{IRFs to a demand shock in the NK model with $1-\theta=0.999$}
\end{figure}


\section*{Hiring subsidies to reduce unemployment (15 points)} 
Consider the standard DMP model of frictional unemployment with Nash bargaining and exogenous separations. A legislator is considering giving a constant subsidy $s$ to any firm that employs a worker in order to promote job creation and reduce the economy’s unemployment rate. With the subsidy, the profit flow of a producing firm is $y+s-w$, where $y$ is output and $w$ is the wage level. The legislator is considering two options to finance this subsidy under a balanced budget: 1) by levying a labor income tax $\tau_e$ on currently employed households, or 2) by levying an income tax $\tau_u$ on currently unemployed households (the second financing option may also be interpreted as a reduction in unemployment benefits).

 \ben
 \item Will the hiring subsidy be succesful in reducing the unemployment rate? Does it depend on how it is financed?
 
 \item Suppose the policy is succesful in reducing the economy’s unemployment rate. Is it clear that the policy have also raised social welfare?
 \een


\section*{A Huggett model with government debt (15 points)}
Consider an infinite-horizon economy with a continuum (measure 1) of ex-ante identical households each having efficiency units of labor $\epsilon_{it}$, drawn from distribution $F$ with finite support $[\epsilon_{min}, \epsilon_{max}]$ and mean $1$, i.i.d. across households and time. Consumers can trade a risk-free bond subject to a credit constraint. Each household $i$ solves
\begin{eqnarray}
\max_{a_{it+1}, c_{it}} && E_0 \sum_{t=0}^{\infty} \beta^t \left[u(c_{it}) + v(G_t) \right] \nonumber \\
\text{s.t.} && c_{it} + a_{it+1} \leq \epsilon_{it} w_t - T_t + (1+r_t) a_{it} \nonumber \\
&& a_{it+1} \geq - \underbar{a} \nonumber
\end{eqnarray}
where $u$ and $v$ satisfy standard conditions, $\underbar{a}>0$ is some exogenous debt limit, $G_t$ is government consumption and $T_t$ is lump-sum tax. A representative firm hires labor in a competitive labor market and employs the production function $Y_t=L_t$, where $L_t$ is the aggregate labor endowment. We denote total bond demand with $A_{t+1} = \int_{i=0}^{1} a_{it+1} di$.

The bonds are issued by the government. The government sets the consumption level $G_t$ and the tax rate $T_t$ exogenously. Bond supply $B_{t+1}$ are resiudally determined from the government budget constraint:
\begin{eqnarray}
T_t+B_{t+1} = R_t B_t + G_t \nonumber
\end{eqnarray}

\ben
\item Define a competitive equilibrium.

\item Consider a steady state, in which government surplus is some positive constant: $T-G>0$. Derive an equation for bond supply, and draw it together with bond demand in an ``Ayiagari diagram''.

\item What happens to the steady state interest rate if the steady state government suprlus increases?
\een


	
\end{document}